\documentclass{article}
\usepackage{amsmath}
\usepackage{color,pxfonts,fix-cm}
\usepackage{latexsym}
\usepackage[mathletters]{ucs}
\DeclareUnicodeCharacter{34}{\textquotedbl}
\DeclareUnicodeCharacter{46}{\textperiodcentered}
\DeclareUnicodeCharacter{8220}{\textquotedblleft}
\DeclareUnicodeCharacter{8594}{$\rightarrow$}
\DeclareUnicodeCharacter{58}{$\colon$}
\DeclareUnicodeCharacter{8221}{\textquotedblright}
\DeclareUnicodeCharacter{8226}{$\bullet$}
\DeclareUnicodeCharacter{8216}{\textquoteleft}
\DeclareUnicodeCharacter{8230}{$\ldots$}
\DeclareUnicodeCharacter{124}{\textbar}
\DeclareUnicodeCharacter{126}{\textasciitilde}
\DeclareUnicodeCharacter{60}{\textless}
\DeclareUnicodeCharacter{8217}{\textquoteright}
\DeclareUnicodeCharacter{62}{\textgreater}
\DeclareUnicodeCharacter{32}{$\ $}
\usepackage[T1]{fontenc}
\usepackage[utf8x]{inputenc}
\usepackage{pict2e}
\usepackage{wasysym}
\usepackage[english]{babel}
\usepackage{tikz}
\pagestyle{empty}
\usepackage[margin=0in,paperwidth=595pt,paperheight=841pt]{geometry}
\begin{document}
\definecolor{color_29791}{rgb}{0,0,0}
\definecolor{color_217499}{rgb}{0.74902,0.505882,0.619608}
\definecolor{color_29919}{rgb}{0,0,0.501961}
\definecolor{color_258292}{rgb}{0.917647,0.458824,0}
\definecolor{color_35081}{rgb}{0,0.662745,0.2}
\definecolor{color_282751}{rgb}{1,1,0}
\definecolor{color_224005}{rgb}{0.788235,0.129412,0.117647}
\definecolor{color_54166}{rgb}{0.082353,0.517647,0.4}
\definecolor{color_144481}{rgb}{0.447059,0.623529,0.811765}
\definecolor{color_119557}{rgb}{0.34902,0.513726,0.690196}
\begin{tikzpicture}[overlay]\path(0pt,0pt);\end{tikzpicture}
\begin{picture}(-5,0)(2.5,0)
\put(41.8,-104.211){\fontsize{19.6}{1}\usefont{T1}{cmr}{b}{n}\selectfont\color{color_29791}Funzionamento del sistema GNU/Linux}
\put(41.8,-140.811){\fontsize{17.5}{1}\usefont{T1}{cmr}{b}{n}\selectfont\color{color_29791}Funzionamento di base}
\put(41.8,-170.711){\fontsize{14.1}{1}\usefont{T1}{cmr}{b}{n}\selectfont\color{color_29791}Moduli per la gestione hardware}
\put(41.8,-204.711){\fontsize{12}{1}\usefont{T1}{cmr}{m}{n}\selectfont\color{color_29791}Il sistema operativo svolge una molteplicità di ruoli, astraendo le risorse fisiche (dischi, porte, }
\put(41.8,-218.511){\fontsize{12}{1}\usefont{T1}{cmr}{m}{n}\selectfont\color{color_29791}schede di rete) e logiche (filesystem, stack di rete) e controllandone l’accesso. In Linux queste }
\put(41.8,-232.311){\fontsize{12}{1}\usefont{T1}{cmr}{m}{n}\selectfont\color{color_29791}funzioni sono realizzate mediante un framework che permette di attivarle e disattivarle }
\put(41.8,-246.111){\fontsize{12}{1}\usefont{T1}{cmr}{m}{n}\selectfont\color{color_29791}modularmente, in particolare l’accesso all’hardware è astratto a moduli chiamati device driver.}
\put(41.8,-273.711){\fontsize{12}{1}\usefont{T1}{cmr}{m}{n}\selectfont\color{color_29791}Alcuni sono parte del kernel, ma la maggior parte viene caricata dinamicamente dalla directory }
\put(41.8,-287.511){\fontsize{12}{1}\usefont{T1}{cmr}{m}{n}\selectfont\color{color_29791}/lib/modules. Più in dettaglio, a seguito della rilevazione di un dispositivo fisico il controller di I/O }
\put(41.8,-301.311){\fontsize{12}{1}\usefont{T1}{cmr}{m}{n}\selectfont\color{color_29791}invia un interrupt alla CPU, che eseguendo un handler posta un evento su dbus (canale pub-sub per }
\put(41.8,-315.111){\fontsize{12}{1}\usefont{T1}{cmr}{m}{n}\selectfont\color{color_29791}eventi di sistema). Successivamente, udev riceve l’evento e consulta }
\put(41.8,-328.911){\fontsize{12}{1}\usefont{T1}{cmr}{m}{n}\selectfont\color{color_29791}/lib/modules/<kernel\_vers>/modules.alias dove individua il device driver adatto al dispositivo.}
\put(41.8,-356.511){\fontsize{12}{1}\usefont{T1}{cmr}{m}{n}\selectfont\color{color_29791}Ogni modulo definisce in che modo vanno implementate le system call previste per i dispositivi:}
\put(59.8,-370.311){\fontsize{12}{1}\usefont{T1}{cmr}{m}{n}\selectfont\color{color_29791}•Dispositivi a blocchi: usano buffer e cache per ottimizzare il trasferimento di blocchi di una }
\put(77.8,-384.111){\fontsize{12}{1}\usefont{T1}{cmr}{m}{n}\selectfont\color{color_29791}certa dimensione. Si comportano come file conservando un elenco di byte singolarmente }
\put(77.8,-397.911){\fontsize{12}{1}\usefont{T1}{cmr}{m}{n}\selectfont\color{color_29791}indirizzabili (es. dischi)}
\put(59.8,-411.711){\fontsize{12}{1}\usefont{T1}{cmr}{m}{n}\selectfont\color{color_29791}•Dispositivi a caratteri: trasferiscono dati un carattere alla volta, possono consumarli per farci}
\put(77.8,-425.511){\fontsize{12}{1}\usefont{T1}{cmr}{m}{n}\selectfont\color{color_29791}qualcosa (es. mostrarli su un terminale) o fornirli se ne hanno disponibili, lasciando il }
\put(77.8,-439.311){\fontsize{12}{1}\usefont{T1}{cmr}{m}{n}\selectfont\color{color_29791}consumatore in attesa se non ne hanno (es. tastiera)}
\put(59.8,-453.111){\fontsize{12}{1}\usefont{T1}{cmr}{m}{n}\selectfont\color{color_29791}•Dispositivi vari (misc)}
\put(41.8,-480.711){\fontsize{12}{1}\usefont{T1}{cmr}{m}{n}\selectfont\color{color_29791}Le operazioni definite nei device drivers sono attuate usando le system call tipiche dei file su alcuni }
\put(41.8,-494.511){\fontsize{12}{1}\usefont{T1}{cmr}{m}{n}\selectfont\color{color_29791}file speciali detti Device Files. Questi file si trovano in /dev e hanno degli attributi particolari:}
\put(77.3,-508.311){\fontsize{12}{1}\usefont{T1}{cmr}{m}{n}\selectfont\color{color_29791}es. brw-rw---- 1 rootdisk8,1 Mar 20 11:14 /dev/sda1}
\put(59.8,-522.111){\fontsize{12}{1}\usefont{T1}{cmr}{m}{n}\selectfont\color{color_29791}•Un bit speciale per identificare se sono a blocchi (b) o a caratteri (c)}
\put(59.8,-535.911){\fontsize{12}{1}\usefont{T1}{cmr}{m}{n}\selectfont\color{color_29791}•Un major number (es. 8) per identificare quale device driver usare}
\put(59.8,-549.711){\fontsize{12}{1}\usefont{T1}{cmr}{m}{n}\selectfont\color{color_29791}•Un minor number (es. 1) per indicare l’istanza di dispositivo di quella classe}
\put(41.8,-577.311){\fontsize{12}{1}\usefont{T1}{cmr}{m}{n}\selectfont\color{color_29791}Alcuni device files non rappresentano vere periferiche, essendo implementati dal kernel per utilità }
\put(41.8,-591.111){\fontsize{12}{1}\usefont{T1}{cmr}{m}{n}\selectfont\color{color_29791}nel lavoro coi processi:}
\put(59.8,-604.911){\fontsize{12}{1}\usefont{T1}{cmr}{m}{n}\selectfont\color{color_29791}•/dev/zero produce uno stream infinito di zeri binari}
\put(59.8,-618.711){\fontsize{12}{1}\usefont{T1}{cmr}{m}{n}\selectfont\color{color_29791}•/dev/null restituisce EOF a ogni read e ogni write viene scartata}
\put(59.8,-632.511){\fontsize{12}{1}\usefont{T1}{cmr}{m}{n}\selectfont\color{color_29791}•/dev/random produce byte casuali ad alta entropia, bloccante ove non ce ne sia abbastanza}
\put(59.8,-646.311){\fontsize{12}{1}\usefont{T1}{cmr}{m}{n}\selectfont\color{color_29791}•/dev/urandom produce stream pseudocasuale infinito}
\put(41.8,-660.111){\fontsize{12}{1}\usefont{T1}{cmr}{m}{n}\selectfont\color{color_29791}Altri device files notevoli relativi a vere periferiche }
\put(59.8,-673.911){\fontsize{12}{1}\usefont{T1}{cmr}{m}{n}\selectfont\color{color_29791}•/dev/tty* terminali fisici del sistema}
\put(59.8,-687.711){\fontsize{12}{1}\usefont{T1}{cmr}{m}{n}\selectfont\color{color_29791}•/dev/pts/* pseudo-terminali (finestre grafiche)}
\put(59.8,-701.511){\fontsize{12}{1}\usefont{T1}{cmr}{m}{n}\selectfont\color{color_29791}•/dev/sd* dischi e partizioni}
\put(41.8,-738.111){\fontsize{14.1}{1}\usefont{T1}{cmr}{b}{n}\selectfont\color{color_29791}Il sistema di storage}
\end{picture}
\newpage
\begin{tikzpicture}[overlay]\path(0pt,0pt);\end{tikzpicture}
\begin{picture}(-5,0)(2.5,0)
\put(41.8,-85.01099){\fontsize{12}{1}\usefont{T1}{cmr}{m}{n}\selectfont\color{color_29791}I dispositivi di storage sono tipicamente dispositivi a blocchi, essi possono essere utilizzati come un }
\put(41.8,-98.81097){\fontsize{12}{1}\usefont{T1}{cmr}{m}{n}\selectfont\color{color_29791}semplice elenco di blocchi numerati. Per sfruttarli in maniera più organizzata è necessario compiere }
\put(41.8,-112.611){\fontsize{12}{1}\usefont{T1}{cmr}{m}{n}\selectfont\color{color_29791}tre operazioni:}
\put(59.8,-126.411){\fontsize{12}{1}\usefont{T1}{cmr}{m}{n}\selectfont\color{color_29791}•Partizionamento: suddivisione in sottoinsiemi di blocchi. Non è obbligatoria ma è sempre }
\put(77.8,-140.211){\fontsize{12}{1}\usefont{T1}{cmr}{m}{n}\selectfont\color{color_29791}utilizzato perché permette di usare ogni partizione come un dispositivo indipendente, utile }
\put(77.8,-154.011){\fontsize{12}{1}\usefont{T1}{cmr}{m}{n}\selectfont\color{color_29791}per separare spazi con esigenze diverse (filesystem diversi, dati di sistema e dati utente). Un }
\put(77.8,-167.811){\fontsize{12}{1}\usefont{T1}{cmr}{m}{n}\selectfont\color{color_29791}esempio utile può essere una partizione di swap, che, seppur oggi il calo di prestazioni sia }
\put(77.8,-181.611){\fontsize{12}{1}\usefont{T1}{cmr}{m}{n}\selectfont\color{color_29791}notevole per usarla realmente come memoria virtuale, viene spesso usata per caricare la }
\put(77.8,-195.411){\fontsize{12}{1}\usefont{T1}{cmr}{m}{n}\selectfont\color{color_29791}RAM andando in ibernazione. }
\put(77.8,-209.211){\fontsize{12}{1}\usefont{T1}{cmr}{m}{n}\selectfont\color{color_29791}I due standard più diffusi sono }
\put(77.8,-223.011){\fontsize{12}{1}\usefont{T1}{cmr}{m}{n}\selectfont\color{color_29791}◦MBR (Master Boot Record) in cui si ha una tabella principale con massimo 4 partizioni }
\put(95.8,-236.811){\fontsize{12}{1}\usefont{T1}{cmr}{m}{n}\selectfont\color{color_29791}primarie (delimitate da blocco di inizio e blocco di fine), e una di queste può essere }
\put(95.8,-250.611){\fontsize{12}{1}\usefont{T1}{cmr}{m}{n}\selectfont\color{color_29791}segnata come partizione estesa contenente fino a 12 unità logiche (utilizzabili come }
\put(95.8,-264.411){\fontsize{12}{1}\usefont{T1}{cmr}{m}{n}\selectfont\color{color_29791}fossero una partizione primaria)}
\put(77.8,-278.211){\fontsize{12}{1}\usefont{T1}{cmr}{m}{n}\selectfont\color{color_29791}◦GPT (GUID Partition Table Scheme) usato specialmente con UEFI, permette di avere un}
\put(95.8,-292.011){\fontsize{12}{1}\usefont{T1}{cmr}{m}{n}\selectfont\color{color_29791}massimo di 128 partizioni da 8 ZB invece delle 15 da 2 TB offerte da MBR}
\put(77.3,-319.611){\fontsize{12}{1}\usefont{T1}{cmr}{m}{n}\selectfont\color{color_29791}I device file per dischi e partizioni seguono il formato /dev/sdXXNN dove XX è una o più }
\put(77.3,-333.411){\fontsize{12}{1}\usefont{T1}{cmr}{m}{n}\selectfont\color{color_29791}lettere che identifica il disco e NN è il numero di partizione. Per i dischi NVMe/M.2 il }
\put(77.3,-347.211){\fontsize{12}{1}\usefont{T1}{cmr}{m}{n}\selectfont\color{color_29791}formato è /dev/nvmeXnYpZ dove X identifica il disco, Y il namespace e Z la partizione. Gli }
\put(77.3,-361.011){\fontsize{12}{1}\usefont{T1}{cmr}{m}{n}\selectfont\color{color_29791}identificatori di disco non sono fissi ma variano in base all’ordine in cui essi vengono }
\put(77.3,-374.811){\fontsize{12}{1}\usefont{T1}{cmr}{m}{n}\selectfont\color{color_29791}rilevati al boot.}
\put(59.8,-388.611){\fontsize{12}{1}\usefont{T1}{cmr}{m}{n}\selectfont\color{color_29791}•Formattazione: creazione del filesystem per organizzare lo spazio in modo comprensibile. }
\put(77.8,-402.411){\fontsize{12}{1}\usefont{T1}{cmr}{m}{n}\selectfont\color{color_29791}La formattazione richiede spazio contiguo, potendo crescere e ridursi ma senza avere }
\put(77.8,-416.211){\fontsize{12}{1}\usefont{T1}{cmr}{m}{n}\selectfont\color{color_29791}“buchi”. Permette l’accesso ai file secondo il modello del VFS (Virtual File System) di }
\put(77.8,-430.011){\fontsize{12}{1}\usefont{T1}{cmr}{m}{n}\selectfont\color{color_29791}Linux, che astrae dall’organizzazione dati specifica di vari FS. }
\put(77.8,-443.811){\fontsize{12}{1}\usefont{T1}{cmr}{m}{n}\selectfont\color{color_29791}Il FS più comune oggi in Linux è ext4, nato per estendere ext3 nei limiti di dimensione file, }
\put(77.8,-457.611){\fontsize{12}{1}\usefont{T1}{cmr}{m}{n}\selectfont\color{color_29791}filesystem e numero di subdirectory e per introdurre il journaling nativo, offrendo }
\put(77.8,-471.411){\fontsize{12}{1}\usefont{T1}{cmr}{m}{n}\selectfont\color{color_29791}prestazioni superiori rispetto a quello aggiunto da ext3 su ext2; aggiungendo anche la }
\put(77.8,-485.211){\fontsize{12}{1}\usefont{T1}{cmr}{m}{n}\selectfont\color{color_29791}possibilità di usare un checksum per la consistenza del journal invece di un commit a due }
\put(77.8,-499.011){\fontsize{12}{1}\usefont{T1}{cmr}{m}{n}\selectfont\color{color_29791}fasi. Il journal è disattivabile per evitare l’eccesso di scritture in pochi blocchi per i dischi }
\put(77.8,-512.811){\fontsize{12}{1}\usefont{T1}{cmr}{m}{n}\selectfont\color{color_29791}SSD. Esso introduce il concetto di extent invece di allocazione indiretta, permettendo a file }
\put(77.8,-526.611){\fontsize{12}{1}\usefont{T1}{cmr}{m}{n}\selectfont\color{color_29791}di grandi dimensioni di essere allocati in modo contiguo.}
\put(59.8,-540.411){\fontsize{12}{1}\usefont{T1}{cmr}{m}{n}\selectfont\color{color_29791}•Mount: innestamento della gerarchia locale creata dal partizionamento nella gerarchia }
\put(77.8,-554.211){\fontsize{12}{1}\usefont{T1}{cmr}{m}{n}\selectfont\color{color_29791}globale. A seguito del mount, il nome assoluto di un file diventa la combinazione tra la }
\put(77.8,-568.011){\fontsize{12}{1}\usefont{T1}{cmr}{m}{n}\selectfont\color{color_29791}posizione relativa nel device di appartenenza e il punto di mount di questo, ad esempio}
\put(77.8,-581.811){\fontsize{12}{1}\usefont{T1}{cmr}{m}{n}\selectfont\color{color_29791}◦1) mount /dev/sda1 / }
\put(77.8,-595.611){\fontsize{12}{1}\usefont{T1}{cmr}{m}{n}\selectfont\color{color_29791}◦2) mount /dev/sda5 /home}
\put(77.3,-609.411){\fontsize{12}{1}\usefont{T1}{cmr}{m}{n}\selectfont\color{color_29791}Essendo /home nella root di sda1, montato su /, }
\put(77.3,-623.211){\fontsize{12}{1}\usefont{T1}{cmr}{m}{n}\selectfont\color{color_29791}il file mary presente nella root di sda5 avrà nome assoluto /home (punto di mount di }
\put(77.3,-637.011){\fontsize{12}{1}\usefont{T1}{cmr}{m}{n}\selectfont\color{color_29791}sda5) + /mary (posizione relativa nel device di appartenenza) = /home/mary}
\put(41.8,-673.611){\fontsize{14.1}{1}\usefont{T1}{cmr}{b}{n}\selectfont\color{color_29791}Collocazione delle risorse}
\put(41.8,-693.811){\fontsize{12}{1}\usefont{T1}{cmr}{m}{n}\selectfont\color{color_29791}La struttura del filesystem UNIX viene definita da FHS (Filesystem Hierarchy Standard), per }
\put(41.8,-707.611){\fontsize{12}{1}\usefont{T1}{cmr}{m}{n}\selectfont\color{color_29791}rendere più facile l’individuazione delle risorse e la condivisione di parti del filesystem. Le }
\put(41.8,-721.411){\fontsize{12}{1}\usefont{T1}{cmr}{m}{n}\selectfont\color{color_29791}distinzioni base per la corretta collocazione dei dati sono 2:}
\put(59.8,-735.211){\fontsize{12}{1}\usefont{T1}{cmr}{m}{n}\selectfont\color{color_29791}•Dati statici / variabili (i dati statici possono essere salvati su supporti read-only e non }
\put(77.8,-749.011){\fontsize{12}{1}\usefont{T1}{cmr}{m}{n}\selectfont\color{color_29791}richiedono backup con la stessa frequenza di quelli variabili)}
\end{picture}
\newpage
\begin{tikzpicture}[overlay]\path(0pt,0pt);\end{tikzpicture}
\begin{picture}(-5,0)(2.5,0)
\put(59.8,-85.01099){\fontsize{12}{1}\usefont{T1}{cmr}{m}{n}\selectfont\color{color_29791}•Dati condivisibili / non condivisibili (I dati condivisibili possono stare in un solo host ed }
\put(77.8,-98.81097){\fontsize{12}{1}\usefont{T1}{cmr}{m}{n}\selectfont\color{color_29791}essere usati in molteplici, alcuni sono obbligatoriamente host-only)}
\put(41.8,-126.411){\fontsize{12}{1}\usefont{T1}{cmr}{m}{n}\selectfont\color{color_29791}La radice della gerarchia, /, deve contenere tutti i dati necessari all’avvio del sistema ma essere il }
\put(41.8,-140.211){\fontsize{12}{1}\usefont{T1}{cmr}{m}{n}\selectfont\color{color_29791}più compatto possibile per ridurre i rischi di corruzione e poter risiedere in media a bassa capacità.}
\put(41.8,-167.811){\fontsize{12}{1}\usefont{T1}{cmr}{m}{n}\selectfont\color{color_29791}In un sistema Linux si possono osservare diversi filesystem virtuali, non corrispondenti a nessun }
\put(41.8,-181.611){\fontsize{12}{1}\usefont{T1}{cmr}{m}{n}\selectfont\color{color_29791}dispositivo fisico. Notiamo proc e sys, che permettono accesso diretto ai dati del kernel come aree }
\put(41.8,-195.411){\fontsize{12}{1}\usefont{T1}{cmr}{m}{n}\selectfont\color{color_29791}di memoria, parametri di processi e di configurazione moduli; udev, che permette la generazione }
\put(41.8,-209.211){\fontsize{12}{1}\usefont{T1}{cmr}{m}{n}\selectfont\color{color_29791}automatica dei device files da parte dei device drivers, tmpfs per mappare in memoria anziché su }
\put(41.8,-223.011){\fontsize{12}{1}\usefont{T1}{cmr}{m}{n}\selectfont\color{color_29791}disco dati volatili.}
\put(41.8,-250.611){\fontsize{12}{1}\usefont{T1}{cmr}{m}{n}\selectfont\color{color_29791}FUSE è un driver che permette di accedere a filesystem diversi da quelli definiti nel kernel }
\put(41.8,-264.411){\fontsize{12}{1}\usefont{T1}{cmr}{m}{n}\selectfont\color{color_29791}mediante un processo user, è utile ad accedere a filesystem che non necessitano di vero hardware }
\put(41.8,-278.211){\fontsize{12}{1}\usefont{T1}{cmr}{m}{n}\selectfont\color{color_29791}come quelli di rete. }
\put(41.8,-321.011){\fontsize{17.5}{1}\usefont{T1}{cmr}{b}{n}\selectfont\color{color_29791}Gestione dei pacchetti software}
\put(41.8,-355.711){\fontsize{12}{1}\usefont{T1}{cmr}{m}{n}\selectfont\color{color_29791}Ogni sistema ha bisogno di un’oculata gestione del software installato, dal momento della sua }
\put(41.8,-369.511){\fontsize{12}{1}\usefont{T1}{cmr}{m}{n}\selectfont\color{color_29791}installazione fino a quello della disinstallazione: generalmente parlando è sempre opportuno }
\put(41.8,-383.311){\fontsize{12}{1}\usefont{T1}{cmr}{m}{n}\selectfont\color{color_29791}disinstallare software che non ci serve più. La fase dell’aggiornamento dei pacchetti è }
\put(41.8,-397.111){\fontsize{12}{1}\usefont{T1}{cmr}{m}{n}\selectfont\color{color_29791}particolarmente critica, in quanto nuove versioni potrebbero interrompere il supporto ad alcune }
\put(41.8,-410.911){\fontsize{12}{1}\usefont{T1}{cmr}{m}{n}\selectfont\color{color_29791}componenti hardware o software del sistema, o causare problemi con altri pacchetti con cui c’è }
\put(41.8,-424.711){\fontsize{12}{1}\usefont{T1}{cmr}{m}{n}\selectfont\color{color_29791}dipendenza. Generalmente, se un software funziona non è importante avere l’ultima versione, può }
\put(41.8,-438.511){\fontsize{12}{1}\usefont{T1}{cmr}{m}{n}\selectfont\color{color_29791}però essere necessario applicare eventuali patch di sicurezza ad hoc, nel caso in cui non si possa }
\put(41.8,-452.311){\fontsize{12}{1}\usefont{T1}{cmr}{m}{n}\selectfont\color{color_29791}passare ad una versione più recente che includa i fix necessari. La configurazione dei software è }
\put(41.8,-466.111){\fontsize{12}{1}\usefont{T1}{cmr}{m}{n}\selectfont\color{color_29791}altrettanto importante in quanto deve tenere conto delle politiche di utilizzo ed eventualmente }
\put(41.8,-479.911){\fontsize{12}{1}\usefont{T1}{cmr}{m}{n}\selectfont\color{color_29791}prevedere particolari casi d’uso o caratteristiche del sistema.}
\put(41.8,-516.511){\fontsize{14.1}{1}\usefont{T1}{cmr}{b}{n}\selectfont\color{color_29791}Installazione manuale}
\put(41.8,-536.711){\fontsize{12}{1}\usefont{T1}{cmr}{m}{n}\selectfont\color{color_29791}L’installazione manuale può passare da binari precompilati, che necessitano soltanto di essere }
\put(41.8,-550.511){\fontsize{12}{1}\usefont{T1}{cmr}{m}{n}\selectfont\color{color_29791}copiati nelle posizioni di interesse, o da file sorgente, che necessitano di compilazione ma sono }
\put(41.8,-564.311){\fontsize{12}{1}\usefont{T1}{cmr}{m}{n}\selectfont\color{color_29791}indipendenti dall’architettura e offrono più flessibilità nel soddisfacimento delle dipendenze. Questo}
\put(41.8,-578.111){\fontsize{12}{1}\usefont{T1}{cmr}{m}{n}\selectfont\color{color_29791}perché i sorgente spesso offrono interfacce usate per adeguare l’installazione al sistema, tenendo }
\put(41.8,-591.911){\fontsize{12}{1}\usefont{T1}{cmr}{m}{n}\selectfont\color{color_29791}conto ad esempio di componenti già presenti.}
\put(41.8,-612.711){\fontsize{12}{1}\usefont{T1}{cmr}{m}{n}\selectfont\color{color_29791}Nella maggior parte dei casi il software è scritto in C e viene distribuito con archivi tar.gz, }
\put(41.8,-626.511){\fontsize{12}{1}\usefont{T1}{cmr}{m}{n}\selectfont\color{color_29791}dotandolo di un’autoconf che verifica la soddisfazione dei prerequisiti, rilevando la presenza dei }
\put(41.8,-640.311){\fontsize{12}{1}\usefont{T1}{cmr}{m}{n}\selectfont\color{color_29791}pacchetti necessari e la loro versione, permette all’utente di specificare configurazioni particolari e }
\put(41.8,-654.111){\fontsize{12}{1}\usefont{T1}{cmr}{m}{n}\selectfont\color{color_29791}infine genera i Makefile tenendo conto delle specificità del sistema e delle scelte utente.}
\put(41.8,-674.911){\fontsize{12}{1}\usefont{T1}{cmr}{m}{n}\selectfont\color{color_29791}Per un’installazione manuale da sorgenti, i passi tipici sono:}
\put(59.8,-695.711){\fontsize{12}{1}\usefont{T1}{cmr}{m}{n}\selectfont\color{color_29791}•reperimento software: }
\put(77.8,-716.511){\fontsize{12}{1}\usefont{T1}{cmr}{m}{n}\selectfont\color{color_29791}◦E’ importante autenticare i pacchetti usati, solitamente mediante una chiave pubblica }
\put(95.8,-730.311){\fontsize{12}{1}\usefont{T1}{cmr}{m}{n}\selectfont\color{color_29791}fidata (comando gpg) o come minimo verificando il fingerprint del file (es. comando }
\put(95.8,-744.111){\fontsize{12}{1}\usefont{T1}{cmr}{m}{n}\selectfont\color{color_29791}sha256, si possono avere varie hash)}
\put(59.8,-764.911){\fontsize{12}{1}\usefont{T1}{cmr}{m}{n}\selectfont\color{color_29791}•estrazione pacchetto}
\end{picture}
\newpage
\begin{tikzpicture}[overlay]\path(0pt,0pt);\end{tikzpicture}
\begin{picture}(-5,0)(2.5,0)
\put(77.8,-85.01099){\fontsize{12}{1}\usefont{T1}{cmr}{m}{n}\selectfont\color{color_29791}◦I file vengono tipicamente prima compattati in un solo tar, e poi compressi. E’ buona }
\put(95.8,-98.81097){\fontsize{12}{1}\usefont{T1}{cmr}{m}{n}\selectfont\color{color_29791}prassi individuare una posizione sensata per l’estrazione, e testare l’archivio prima }
\put(95.8,-112.611){\fontsize{12}{1}\usefont{T1}{cmr}{m}{n}\selectfont\color{color_29791}dell’estrazione per verificare la gerarchia di directory che sarà generata}
\put(59.8,-133.411){\fontsize{12}{1}\usefont{T1}{cmr}{m}{n}\selectfont\color{color_29791}•esame delle scelte offerte}
\put(77.8,-154.211){\fontsize{12}{1}\usefont{T1}{cmr}{m}{n}\selectfont\color{color_29791}◦Nella directory data dall’estrazione è opportuno leggere eventuali README e }
\put(95.8,-168.011){\fontsize{12}{1}\usefont{T1}{cmr}{m}{n}\selectfont\color{color_29791}INSTALL. Se si ha un eseguibile configure, si lancia con –help per avere una lista dei }
\put(95.8,-181.811){\fontsize{12}{1}\usefont{T1}{cmr}{m}{n}\selectfont\color{color_29791}parametri di configurazione; ad esempio collocazione del software, attivazione di }
\put(95.8,-195.611){\fontsize{12}{1}\usefont{T1}{cmr}{m}{n}\selectfont\color{color_29791}componenti, caricamento dinamico di moduli invece che integrazione statica. Opzioni }
\put(95.8,-209.411){\fontsize{12}{1}\usefont{T1}{cmr}{m}{n}\selectfont\color{color_29791}Without oppure with per scegliere componenti previsti di default/per escluderli}
\put(59.8,-230.211){\fontsize{12}{1}\usefont{T1}{cmr}{m}{n}\selectfont\color{color_29791}•configurazione sorgenti}
\put(77.8,-251.011){\fontsize{12}{1}\usefont{T1}{cmr}{m}{n}\selectfont\color{color_29791}◦Si lancia configure con i parametri scelti, risolvendo eventuali problemi evidenziati (di }
\put(95.8,-264.811){\fontsize{12}{1}\usefont{T1}{cmr}{m}{n}\selectfont\color{color_29791}solito mancanza di pacchetti necessari, può essere necessaria un’indagine manuale). }
\put(95.8,-278.611){\fontsize{12}{1}\usefont{T1}{cmr}{m}{n}\selectfont\color{color_29791}Viene generato un file config.log}
\put(59.8,-299.411){\fontsize{12}{1}\usefont{T1}{cmr}{m}{n}\selectfont\color{color_29791}•compilazione}
\put(77.8,-320.211){\fontsize{12}{1}\usefont{T1}{cmr}{m}{n}\selectfont\color{color_29791}◦Si lancia make o si seguono le indicazioni del config.log}
\put(59.8,-341.011){\fontsize{12}{1}\usefont{T1}{cmr}{m}{n}\selectfont\color{color_29791}•installazione, usando sudo make install. È importante che solo questo passo venga eseguito }
\put(77.8,-354.811){\fontsize{12}{1}\usefont{T1}{cmr}{m}{n}\selectfont\color{color_29791}come root essendoci stati in passato malware che sfruttano privilegi eccessivi nei passi }
\put(77.8,-368.611){\fontsize{12}{1}\usefont{T1}{cmr}{m}{n}\selectfont\color{color_29791}precedenti.}
\put(41.8,-389.411){\fontsize{12}{1}\usefont{T1}{cmr}{m}{n}\selectfont\color{color_29791}L’installazione di sorgenti offre la possibilità di verificare il codice, tuttavia è più difficile da }
\put(41.8,-403.211){\fontsize{12}{1}\usefont{T1}{cmr}{m}{n}\selectfont\color{color_29791}mantenere richiedendo una verifica manuale di dipendenze e configurazioni. Richiede inoltre molti }
\put(41.8,-417.011){\fontsize{12}{1}\usefont{T1}{cmr}{m}{n}\selectfont\color{color_29791}componenti ausiliari (header, librerie, compilatori) che possono facilitare un attaccante. }
\put(41.8,-437.811){\fontsize{12}{1}\usefont{T1}{cmr}{m}{n}\selectfont\color{color_29791}Le distribuzioni e i pacchetti puntano ad alleviare questi contro, fornendo una gestione automatica }
\put(41.8,-451.611){\fontsize{12}{1}\usefont{T1}{cmr}{m}{n}\selectfont\color{color_29791}delle dipendenze e garantendo compatibilità tra tutti gli elementi del set di pacchetti.}
\put(41.8,-481.411){\fontsize{14.1}{1}\usefont{T1}{cmr}{b}{n}\selectfont\color{color_29791}Installazione assistita}
\put(41.8,-501.611){\fontsize{12}{1}\usefont{T1}{cmr}{m}{n}\selectfont\color{color_29791}L’installazione assistita si effettua tipicamente usando un package manager, nelle distribuzioni }
\put(41.8,-515.411){\fontsize{12}{1}\usefont{T1}{cmr}{m}{n}\selectfont\color{color_29791}Linux. Il tool si fa carico di verificare le dipendenze, scorrendo il grafo delle dipendenze di un dato }
\put(41.8,-529.211){\fontsize{12}{1}\usefont{T1}{cmr}{m}{n}\selectfont\color{color_29791}pacchetto individuando tutti i componenti di cui ha bisogno.}
\put(41.8,-550.011){\fontsize{12}{1}\usefont{T1}{cmr}{m}{n}\selectfont\color{color_29791}Un pacchetto tipicamente si presenta come un singolo file che contiene sia il software precompilato }
\put(41.8,-563.811){\fontsize{12}{1}\usefont{T1}{cmr}{m}{n}\selectfont\color{color_29791}che i criteri per la verifica di compatibilità e prerequisiti. Chiaramente la garanzia di compatibilità }
\put(41.8,-577.611){\fontsize{12}{1}\usefont{T1}{cmr}{m}{n}\selectfont\color{color_29791}può esserci solo vincolando l’architettura di esecuzione, la versione della distribuzione e la versione}
\put(41.8,-591.411){\fontsize{12}{1}\usefont{T1}{cmr}{m}{n}\selectfont\color{color_29791}del software.}
\put(41.8,-621.211){\fontsize{14.1}{1}\usefont{T1}{cmr}{b}{n}\selectfont\color{color_29791}Distribuzioni}
\put(41.8,-641.411){\fontsize{12}{1}\usefont{T1}{cmr}{m}{n}\selectfont\color{color_29791}Nella scelta di una distribuzione per un particolare applicazione bisognare tenere conto di diversi }
\put(41.8,-655.211){\fontsize{12}{1}\usefont{T1}{cmr}{m}{n}\selectfont\color{color_29791}fattori:}
\put(59.8,-676.011){\fontsize{12}{1}\usefont{T1}{cmr}{m}{n}\selectfont\color{color_29791}•Architetture supportate: tutte le distribuzioni supportano Intel 32bit, alcune 64bit, altre sono }
\put(77.8,-689.811){\fontsize{12}{1}\usefont{T1}{cmr}{m}{n}\selectfont\color{color_29791}disponibili per la grande varietà di processori su cui è stato portato il kernel, ma è bene }
\put(77.8,-703.611){\fontsize{12}{1}\usefont{T1}{cmr}{m}{n}\selectfont\color{color_29791}ricordare che il supporto a pacchetti di terze parti non è garantito per architetture esotiche}
\put(59.8,-724.411){\fontsize{12}{1}\usefont{T1}{cmr}{m}{n}\selectfont\color{color_29791}•Stabilità vs aggiornamento: in GNU/Linux il processo di rilascio frequente di software porta}
\put(77.8,-738.211){\fontsize{12}{1}\usefont{T1}{cmr}{m}{n}\selectfont\color{color_29791}a dover confrontare i benefici dell’avere versioni più recenti del software, con più }
\put(77.8,-752.011){\fontsize{12}{1}\usefont{T1}{cmr}{m}{n}\selectfont\color{color_29791}funzionalità, con la stabilità di versioni già ampiamente testate; e la stabilità è importante su }
\put(77.8,-765.811){\fontsize{12}{1}\usefont{T1}{cmr}{m}{n}\selectfont\color{color_29791}un server in produzione}
\end{picture}
\newpage
\begin{tikzpicture}[overlay]\path(0pt,0pt);\end{tikzpicture}
\begin{picture}(-5,0)(2.5,0)
\put(59.8,-85.01099){\fontsize{12}{1}\usefont{T1}{cmr}{m}{n}\selectfont\color{color_29791}•Version vs rolling: Le distribuzioni con versione forniscono solo aggiornamenti correttivi }
\put(77.8,-98.81097){\fontsize{12}{1}\usefont{T1}{cmr}{m}{n}\selectfont\color{color_29791}durante tutto il loro ciclo di vita. Eventuali novità vengono testate e accumulate per una }
\put(77.8,-112.611){\fontsize{12}{1}\usefont{T1}{cmr}{m}{n}\selectfont\color{color_29791}nuova versione, che dovrà essere installata sovrascrivendo la precedente. Per contro in una }
\put(77.8,-126.411){\fontsize{12}{1}\usefont{T1}{cmr}{m}{n}\selectfont\color{color_29791}distribuzione rolling ogni novità viene testata e distribuita, quindi si ha sempre a }
\put(77.8,-140.211){\fontsize{12}{1}\usefont{T1}{cmr}{m}{n}\selectfont\color{color_29791}disposizione la versione più recente. Le distribuzioni rolling per loro natura non sono molto }
\put(77.8,-154.011){\fontsize{12}{1}\usefont{T1}{cmr}{m}{n}\selectfont\color{color_29791}adatte all’uso su un server}
\put(59.8,-174.811){\fontsize{12}{1}\usefont{T1}{cmr}{m}{n}\selectfont\color{color_29791}•Supporto e durata: Il supporto garantito è tipico delle distribuzioni commerciali, per quelle }
\put(77.8,-188.611){\fontsize{12}{1}\usefont{T1}{cmr}{m}{n}\selectfont\color{color_29791}gratuite dipende dalla dimensione della comunità di utenti. Per installazioni server esistono }
\put(77.8,-202.411){\fontsize{12}{1}\usefont{T1}{cmr}{m}{n}\selectfont\color{color_29791}distribuzioni LTS (Long Term Support) per le quali i gestori assicurano per 5/7 anni che le }
\put(77.8,-216.211){\fontsize{12}{1}\usefont{T1}{cmr}{m}{n}\selectfont\color{color_29791}API restino le stesse, garantendo il backporting dei security fix}
\put(41.8,-237.011){\fontsize{12}{1}\usefont{T1}{cmr}{m}{n}\selectfont\color{color_29791}Le due distribuzioni capostipite da cui derivano le varianti più diffuse sono Debian e Red Hat. }
\put(41.8,-250.811){\fontsize{12}{1}\usefont{T1}{cmr}{m}{n}\selectfont\color{color_29791}Offrono sistemi di gestione pacchetti che si assomigliano, con tool di basso livello per la gestione }
\put(41.8,-264.611){\fontsize{12}{1}\usefont{T1}{cmr}{m}{n}\selectfont\color{color_29791}dei singoli pacchetti, tool intermedi per la gestione coordinata delle dipendenze e tool per reperire }
\put(41.8,-278.411){\fontsize{12}{1}\usefont{T1}{cmr}{m}{n}\selectfont\color{color_29791}automaticamente i pacchetti necessari dai repository (depositi).}
\put(41.8,-308.211){\fontsize{14.1}{1}\usefont{T1}{cmr}{b}{n}\selectfont\color{color_29791}Pacchetti}
\put(41.8,-328.411){\fontsize{12}{1}\usefont{T1}{cmr}{m}{n}\selectfont\color{color_29791}Il nome dei pacchetti segue tipicamente il formato nome-versioneSoftware-versionePacchetto-}
\put(41.8,-342.211){\fontsize{12}{1}\usefont{T1}{cmr}{m}{n}\selectfont\color{color_29791}architettura.estensione, quest’ultima per Debian e derivate è .deb, per Red Hat e derivate in .rpm. La}
\put(41.8,-356.011){\fontsize{12}{1}\usefont{T1}{cmr}{m}{n}\selectfont\color{color_29791}versione del pacchetto permette di patchare eventuali bug senza installare una versione del software }
\put(41.8,-369.811){\fontsize{12}{1}\usefont{T1}{cmr}{m}{n}\selectfont\color{color_29791}superiore, che potrebbe si includere nuove funzionalità ma anche interrompere il supporto ad altre }
\put(41.8,-383.611){\fontsize{12}{1}\usefont{T1}{cmr}{m}{n}\selectfont\color{color_29791}utilizzate dalle dipendenze. }
\put(41.8,-404.411){\fontsize{12}{1}\usefont{T1}{cmr}{m}{n}\selectfont\color{color_29791}Si distingue tra pacchetti base (dotati solo del necessario per eseguire funzioni di libreria, ovvero }
\put(41.8,-418.211){\fontsize{12}{1}\usefont{T1}{cmr}{m}{n}\selectfont\color{color_29791}codice oggetto in formato adatto al linking dinamico, per Linux .so Shared Object) e pacchetti }
\put(41.8,-432.011){\fontsize{12}{1}\usefont{T1}{cmr}{m}{n}\selectfont\color{color_29791}development (-dev per Debian e -devel per RedHat), che dispongo anche di codice oggetto in }
\put(41.8,-445.811){\fontsize{12}{1}\usefont{T1}{cmr}{m}{n}\selectfont\color{color_29791}formato adatto al linking statico e prototipi per il compilatore. }
\put(41.8,-466.611){\fontsize{12}{1}\usefont{T1}{cmr}{m}{n}\selectfont\color{color_29791}Verifica dipendenze dinamiche: ldd percorsopacchetto}
\put(41.8,-487.411){\fontsize{12}{1}\usefont{T1}{cmr}{m}{n}\selectfont\color{color_29791}I pacchetti possono essere scaricati e gestiti singolarmente, ma di solito si utilizzano dei repository, }
\put(41.8,-501.211){\fontsize{12}{1}\usefont{T1}{cmr}{m}{n}\selectfont\color{color_29791}ovvero raccolte indicizzate di pacchetti, che possono essere sia online che locali. I package manager}
\put(41.8,-515.011){\fontsize{12}{1}\usefont{T1}{cmr}{m}{n}\selectfont\color{color_29791}leggono per ogni repo l’indice e i metadati dei pacchetti e riconoscono quali versioni sono }
\put(41.8,-528.811){\fontsize{12}{1}\usefont{T1}{cmr}{m}{n}\selectfont\color{color_29791}disponibili, oltre a conoscere le dipendenze come risolverle. Le liste di repo si trovano in }
\put(41.8,-542.611){\fontsize{12}{1}\usefont{T1}{cmr}{m}{n}\selectfont\color{color_29791}/etc/apt/sources.list in deb.}
\put(41.8,-563.411){\fontsize{12}{1}\usefont{T1}{cmr}{m}{n}\selectfont\color{color_29791}Per i pacchetti .deb il package manager è tipicamente apt ad alto livello e dpkg a basso livello, per }
\put(41.8,-577.211){\fontsize{12}{1}\usefont{T1}{cmr}{m}{n}\selectfont\color{color_29791}quelli .rpm corrispondono yum e rpm. Si aggiungono poi i sottocomandi update per aggiornare le }
\put(41.8,-591.011){\fontsize{12}{1}\usefont{T1}{cmr}{m}{n}\selectfont\color{color_29791}liste pacchetti e upgrade per passare alla versione più recente, install e remove per ovviamente }
\put(41.8,-604.811){\fontsize{12}{1}\usefont{T1}{cmr}{m}{n}\selectfont\color{color_29791}installare e rimuovere.}
\put(41.8,-625.611){\fontsize{12}{1}\usefont{T1}{cmr}{m}{n}\selectfont\color{color_29791}La firma dei pacchetti viene gestita centralmente, i mantainer di una distribuzione forniscono le }
\put(41.8,-639.411){\fontsize{12}{1}\usefont{T1}{cmr}{m}{n}\selectfont\color{color_29791}chiavi di verifica nei media di installazione o sui repo online.}
\put(41.8,-660.211){\fontsize{12}{1}\usefont{T1}{cmr}{m}{n}\selectfont\color{color_29791}Per utilizzare pacchetti ben supportati ma non inclusi nei canali ufficiali della distribuzione è }
\put(41.8,-674.011){\fontsize{12}{1}\usefont{T1}{cmr}{m}{n}\selectfont\color{color_29791}sufficiente aggiungere il repo all’elenco. Bisogna però fare attenzione, perché avere un pacchetto }
\put(41.8,-687.811){\fontsize{12}{1}\usefont{T1}{cmr}{m}{n}\selectfont\color{color_29791}con lo stesso nome con versione diverse in più repo può causare confusione, in quanto i package }
\put(41.8,-701.611){\fontsize{12}{1}\usefont{T1}{cmr}{m}{n}\selectfont\color{color_29791}manager scelgono sempre la versione più avanzata a default. Si pensi a un repo semisconosciuto in }
\put(41.8,-715.411){\fontsize{12}{1}\usefont{T1}{cmr}{m}{n}\selectfont\color{color_29791}cui viene caricato un pacchetto “core” con versione più avanzata di quella ufficiale, realizzando una}
\put(41.8,-729.211){\fontsize{12}{1}\usefont{T1}{cmr}{m}{n}\selectfont\color{color_29791}software injection.}
\put(41.8,-750.011){\fontsize{12}{1}\usefont{T1}{cmr}{m}{n}\selectfont\color{color_29791}E’ sconsigliabile mischiare installazioni manuali con pacchetti. Quando possibile è utile }
\put(41.8,-763.811){\fontsize{12}{1}\usefont{T1}{cmr}{m}{n}\selectfont\color{color_29791}pacchettizzare le proprie applicazioni, seguendo il processo per costruire un proprio repo, in RPM.}
\end{picture}
\newpage
\begin{tikzpicture}[overlay]\path(0pt,0pt);\end{tikzpicture}
\begin{picture}(-5,0)(2.5,0)
\put(41.8,-85.01099){\fontsize{12}{1}\usefont{T1}{cmr}{m}{n}\selectfont\color{color_29791}Per aggiornare un un pacchetto già in uso bisogna tenere conto di problemi derivanti da}
\put(59.8,-105.811){\fontsize{12}{1}\usefont{T1}{cmr}{m}{n}\selectfont\color{color_29791}•prerequisiti: pacchetti necessari perché il candidato funzioni, come per l’installazione}
\put(59.8,-126.611){\fontsize{12}{1}\usefont{T1}{cmr}{m}{n}\selectfont\color{color_29791}•configurazione: si pensi a modifiche al formato delle configurazioni già messe a punto per la}
\put(77.8,-140.411){\fontsize{12}{1}\usefont{T1}{cmr}{m}{n}\selectfont\color{color_29791}versione funzionante}
\put(59.8,-161.211){\fontsize{12}{1}\usefont{T1}{cmr}{m}{n}\selectfont\color{color_29791}•dipendenze di altri software: modifiche alle interfacce mostrate potrebbero influire sul }
\put(77.8,-175.011){\fontsize{12}{1}\usefont{T1}{cmr}{m}{n}\selectfont\color{color_29791}funzionamento di altri software. Se il software interagisce mediante interfacce standard, }
\put(77.8,-188.811){\fontsize{12}{1}\usefont{T1}{cmr}{m}{n}\selectfont\color{color_29791}potrebbero esserci problemi di trasparenza qualora si volesse usare una configurazione di }
\put(77.8,-202.611){\fontsize{12}{1}\usefont{T1}{cmr}{m}{n}\selectfont\color{color_29791}test per provare una nuova versione (legando le due versioni a socket/porte/IP diversi }
\put(41.8,-223.411){\fontsize{12}{1}\usefont{T1}{cmr}{m}{n}\selectfont\color{color_29791}La disinstallazione di un pacchetto pone problemi simili all’aggiornamento in termini di eventuali }
\put(41.8,-237.211){\fontsize{12}{1}\usefont{T1}{cmr}{m}{n}\selectfont\color{color_29791}dipendenze di altri software da quello che si vuole rimuovere, in entrambi i casi può essere difficile }
\put(41.8,-251.011){\fontsize{12}{1}\usefont{T1}{cmr}{m}{n}\selectfont\color{color_29791}valutare gli effetti sul sistema con una gestione manuale. }
\put(41.8,-271.811){\fontsize{12}{1}\usefont{T1}{cmr}{m}{n}\selectfont\color{color_29791}Il grafo delle dipendenze rappresenta il valore aggiunto più importante offerto dai package manager.}
\put(41.8,-307.811){\fontsize{17.5}{1}\usefont{T1}{cmr}{b}{n}\selectfont\color{color_217499}Crittografia e SSH}
\put(41.8,-328.711){\fontsize{12}{1}\usefont{T1}{cmr}{m}{n}\selectfont\color{color_217499}La sicurezza delle informazioni necessita di }
\put(59.8,-349.511){\fontsize{12}{1}\usefont{T1}{cmr}{m}{n}\selectfont\color{color_29791}•riservatezza: evitare che entità non autorizzate vengano messe a conoscenza di }
\put(77.8,-363.311){\fontsize{12}{1}\usefont{T1}{cmr}{m}{n}\selectfont\color{color_217499}informazioni}
\put(59.8,-384.111){\fontsize{12}{1}\usefont{T1}{cmr}{m}{n}\selectfont\color{color_29791}•integrità (comprende autenticità): riconoscere con certezza se un dato è come era stato }
\put(77.8,-397.911){\fontsize{12}{1}\usefont{T1}{cmr}{m}{n}\selectfont\color{color_217499}prodotto. L'autenticità consiste nella corretta attribuzione dell'autore ad un'informazione, per}
\put(77.8,-411.711){\fontsize{12}{1}\usefont{T1}{cmr}{m}{n}\selectfont\color{color_217499}evitare che appaia prodotta da qualcun altro}
\put(59.8,-432.511){\fontsize{12}{1}\usefont{T1}{cmr}{m}{n}\selectfont\color{color_29791}•disponibilità: non può essere difesa crittograficamente, se l'informazione diventa }
\put(77.8,-446.311){\fontsize{12}{1}\usefont{T1}{cmr}{m}{n}\selectfont\color{color_217499}inaccessibile lo resta, a prescindere da come fosse stata cifrata.}
\put(41.8,-467.111){\fontsize{12}{1}\usefont{T1}{cmr}{m}{n}\selectfont\color{color_217499}Attacchi a riservatezza e integrità possono essere attuati da intrusi posti tra mittente e destinatario; }
\put(41.8,-480.911){\fontsize{12}{1}\usefont{T1}{cmr}{m}{n}\selectfont\color{color_217499}la soluzione è la crittografia; ovvero un'elaborazione matematica e algoritmica della codifica delle }
\put(41.8,-494.711){\fontsize{12}{1}\usefont{T1}{cmr}{m}{n}\selectfont\color{color_217499}informazioni. Violazioni alla riservatezza sono prevenute alterando il codice in modo che risulti }
\put(41.8,-508.511){\fontsize{12}{1}\usefont{T1}{cmr}{m}{n}\selectfont\color{color_217499}incomprensibile a chi non ha diritto di conoscere l'informazione (l'operazione di cifratura deve }
\put(41.8,-522.311){\fontsize{12}{1}\usefont{T1}{cmr}{m}{n}\selectfont\color{color_217499}essere fatta per intervenire in maniera preventiva e far si che l'attacco non abbia successo, un }
\put(41.8,-536.111){\fontsize{12}{1}\usefont{T1}{cmr}{m}{n}\selectfont\color{color_217499}approccio a posteriori non ha senso nel caso di riservatezza perché ciò che doveva restare segreto }
\put(41.8,-549.911){\fontsize{12}{1}\usefont{T1}{cmr}{m}{n}\selectfont\color{color_217499}non lo è più), violazioni all'integrità possono essere solo rilevate a posteriori (impossibile }
\put(41.8,-563.711){\fontsize{12}{1}\usefont{T1}{cmr}{m}{n}\selectfont\color{color_217499}prevenirle) mediante strumenti in grado di dire se un dato è integro o meno; per prendere una }
\put(41.8,-577.511){\fontsize{12}{1}\usefont{T1}{cmr}{m}{n}\selectfont\color{color_217499}decisione razionale in cui si rifiuta l'informazione finché non passa i controlli.}
\put(41.8,-598.311){\fontsize{12}{1}\usefont{T1}{cmr}{m}{n}\selectfont\color{color_217499}I cifrari per la sicurezza prevedono due azioni, cifratura (per passare da testo in chiaro a testo }
\put(41.8,-612.111){\fontsize{12}{1}\usefont{T1}{cmr}{m}{n}\selectfont\color{color_217499}cifrato, Encrypt) e decifrazione (viceversa). }
\put(41.8,-632.911){\fontsize{12}{1}\usefont{T1}{cmr}{m}{n}\selectfont\color{color_217499}La funzione D è critica: potrebbe esistere uno schema in cui chiunque può svolgere E (se è una }
\put(41.8,-646.711){\fontsize{12}{1}\usefont{T1}{cmr}{m}{n}\selectfont\color{color_217499}funzione pubblica), la cosa importante è che la funzione D sia segreta (conosciuta solo al legittimo }
\put(41.8,-660.511){\fontsize{12}{1}\usefont{T1}{cmr}{m}{n}\selectfont\color{color_217499}destinatario). E può essere pubblica ma deve essere robusta}
\end{picture}
\begin{tikzpicture}[overlay]
\path(0pt,0pt);
\draw[color_217499,line width=0.7pt]
(291.4pt, -661.611pt) -- (326.7pt, -661.611pt)
;
\end{tikzpicture}
\begin{picture}(-5,0)(2.5,0)
\put(326.7,-660.511){\fontsize{12}{1}\usefont{T1}{cmr}{m}{n}\selectfont\color{color_217499} (ovvero trasforma testo in chiaro in }
\put(41.8,-674.311){\fontsize{12}{1}\usefont{T1}{cmr}{m}{n}\selectfont\color{color_217499}cifrato in modo che sia obbligatorio}
\end{picture}
\begin{tikzpicture}[overlay]
\path(0pt,0pt);
\draw[color_217499,line width=0.7pt]
(154.8pt, -675.411pt) -- (213.4pt, -675.411pt)
;
\end{tikzpicture}
\begin{picture}(-5,0)(2.5,0)
\put(213.4,-674.311){\fontsize{12}{1}\usefont{T1}{cmr}{m}{n}\selectfont\color{color_217499} avere D per decifrarlo), D invece deve essere segreta }
\put(41.8,-688.111){\fontsize{12}{1}\usefont{T1}{cmr}{m}{n}\selectfont\color{color_217499}necessariamente.}
\put(41.8,-708.911){\fontsize{12}{1}\usefont{T1}{cmr}{m}{n}\selectfont\color{color_217499}Alla base dei sistemi crittografici sono i principi di Kerckhoffs:}
\put(59.8,-729.711){\fontsize{12}{1}\usefont{T1}{cmr}{m}{n}\selectfont\color{color_217499}•1) Il sistema di cifratura deve essere materialmente, se non matematicamente, indecifrabile }
\put(77.8,-743.511){\fontsize{12}{1}\usefont{T1}{cmr}{m}{n}\selectfont\color{color_217499}(sicurezza computazionale o assoluta)}
\end{picture}
\newpage
\begin{tikzpicture}[overlay]\path(0pt,0pt);\end{tikzpicture}
\begin{picture}(-5,0)(2.5,0)
\put(59.8,-85.01099){\fontsize{12}{1}\usefont{T1}{cmr}{m}{n}\selectfont\color{color_29791}•2) Il sistema non deve basarsi sul fatto che il segreto, se cade nelle mani nemiche, causi un }
\put(77.8,-98.81097){\fontsize{12}{1}\usefont{T1}{cmr}{m}{n}\selectfont\color{color_217499}grave problema. Il segreto deve diventare solo un parametro della funzione: non è più la }
\put(77.8,-112.611){\fontsize{12}{1}\usefont{T1}{cmr}{m}{n}\selectfont\color{color_217499}funzione ad essere il segreto. segreto=chiave; NON segreto=algoritmo}
\put(59.8,-133.411){\fontsize{12}{1}\usefont{T1}{cmr}{m}{n}\selectfont\color{color_29791}•3) La chiave deve essere qualcosa di semplice, possibilmente memorizzabile senza supporti. }
\put(77.8,-147.211){\fontsize{12}{1}\usefont{T1}{cmr}{m}{n}\selectfont\color{color_217499}Dover ricordare un segreto semplice permette lo scambio di molti segreti di dimensioni }
\put(77.8,-161.011){\fontsize{12}{1}\usefont{T1}{cmr}{m}{n}\selectfont\color{color_217499}arbitrarie }
\put(41.8,-181.811){\fontsize{12}{1}\usefont{T1}{cmr}{m}{n}\selectfont\color{color_217499}Un crittoanalista di fronte ad un testo cifrato con algoritmo noto può analizzare le proprietà }
\put(41.8,-195.611){\fontsize{12}{1}\usefont{T1}{cmr}{m}{n}\selectfont\color{color_217499}statistiche del testo per rilevare (si evidenzia l'importanza della robustezza di E, ovvero la sua }
\put(41.8,-209.411){\fontsize{12}{1}\usefont{T1}{cmr}{m}{n}\selectfont\color{color_217499}capacità di occultare le proprietà del testo in chiaro) oppure cercare la chiave tra tutte quelle }
\put(41.8,-223.211){\fontsize{12}{1}\usefont{T1}{cmr}{m}{n}\selectfont\color{color_217499}possibili (entra in gioco il primo principio: con la sicurezza assoluta, la chiave corretta è }
\put(41.8,-237.011){\fontsize{12}{1}\usefont{T1}{cmr}{m}{n}\selectfont\color{color_217499}indistinguibile dalle altre – es. con analisi si ottengono 800K testi insensati e 300K sensati: anche di}
\put(41.8,-250.811){\fontsize{12}{1}\usefont{T1}{cmr}{m}{n}\selectfont\color{color_217499}fronte ad attacco a forza bruta il crittoanalista non può essere certo del messaggio corretto- ; con la }
\put(41.8,-264.611){\fontsize{12}{1}\usefont{T1}{cmr}{m}{n}\selectfont\color{color_217499}sicurezza computazionale si rende troppo oneroso cercare la chiave – es. con analisi si trovano }
\put(41.8,-278.411){\fontsize{12}{1}\usefont{T1}{cmr}{m}{n}\selectfont\color{color_217499}999K testi incomprensibili e 1 comprensibile; analista ha certezza di aver trovato la chiave, ma è }
\put(41.8,-292.211){\fontsize{12}{1}\usefont{T1}{cmr}{m}{n}\selectfont\color{color_217499}oneroso scoprirla).}
\put(41.8,-313.011){\fontsize{12}{1}\usefont{T1}{cmr}{m}{n}\selectfont\color{color_217499}Alla base della robustezza ci sono le proprietà di confusione (l'analisi del testo cifrato non }
\put(41.8,-326.811){\fontsize{12}{1}\usefont{T1}{cmr}{m}{n}\selectfont\color{color_217499}restituisce informazioni sulla chiave, modifica ad un singolo elemento della chiave si riflette sul }
\put(41.8,-340.611){\fontsize{12}{1}\usefont{T1}{cmr}{m}{n}\selectfont\color{color_217499}50\% del cifrato) e di diffusione (l'analisi del testo cifrato non restituisce informazioni sul testo in }
\put(41.8,-354.411){\fontsize{12}{1}\usefont{T1}{cmr}{m}{n}\selectfont\color{color_217499}chiaro, modifica di un singolo elemento del testo in chiaro altera il 50\% del cifrato).}
\put(41.8,-375.211){\fontsize{12}{1}\usefont{T1}{cmr}{m}{n}\selectfont\color{color_217499}Un esempio di cifratura con scarsa robustezza è la sostituzione monoalfabetica (ogni elemento }
\put(41.8,-389.011){\fontsize{12}{1}\usefont{T1}{cmr}{m}{n}\selectfont\color{color_217499}dell'alfabeto viene sostituito con un altro, con tabella di corrispondenza). La chiave ha uno spazio di}
\put(41.8,-402.811){\fontsize{12}{1}\usefont{T1}{cmr}{m}{n}\selectfont\color{color_217499}26! (Avendo 26 lettere con cui sostituire la A, una volta sostituita la A, mi restano 25 lettere con cui }
\put(41.8,-416.611){\fontsize{12}{1}\usefont{T1}{cmr}{m}{n}\selectfont\color{color_217499}sostituire la B…) quindi 2\^88 chiavi: non è male, si ha resistenza alla forza bruta ma manca la }
\put(41.8,-430.411){\fontsize{12}{1}\usefont{T1}{cmr}{m}{n}\selectfont\color{color_217499}robustezza: le proprietà statistiche del testo in chiaro sono mantenute nel testo cifrato (per ogni }
\put(41.8,-444.211){\fontsize{12}{1}\usefont{T1}{cmr}{m}{n}\selectfont\color{color_217499}lettere si ha una lettera). C'è scarsa diffusione: gli attacchi alla sostituzione si basano sulla frequenza}
\put(41.8,-458.011){\fontsize{12}{1}\usefont{T1}{cmr}{m}{n}\selectfont\color{color_217499}statistica dei caratteri, in una data lingua naturale (es. in inglese, il 12\% dei caratteri è 'e'); ma in }
\put(41.8,-471.811){\fontsize{12}{1}\usefont{T1}{cmr}{m}{n}\selectfont\color{color_217499}binario ogni "lettera" può essere un lungo blocco di bit: se le lettere non sono più 26 ma 256, il }
\put(41.8,-485.611){\fontsize{12}{1}\usefont{T1}{cmr}{m}{n}\selectfont\color{color_217499}valor medio di ogni colonna è molto più basso. Inoltre la compressione che si può attuare sul }
\put(41.8,-499.411){\fontsize{12}{1}\usefont{T1}{cmr}{m}{n}\selectfont\color{color_217499}binario cifrato fa sì che la ridondanza venga rimossa e quindi la differenza statistica tra le lettere si }
\put(41.8,-513.211){\fontsize{12}{1}\usefont{T1}{cmr}{m}{n}\selectfont\color{color_217499}riduce ancora di più.}
\put(41.8,-534.011){\fontsize{12}{1}\usefont{T1}{cmr}{m}{n}\selectfont\color{color_217499}La diffusione viene introdotta in modo semplice dalla trasposizione, che si basa su una tabella in cui}
\put(41.8,-547.811){\fontsize{12}{1}\usefont{T1}{cmr}{m}{n}\selectfont\color{color_217499}il testo viene scritto per colonne e letto per righe. La ricerca della chiave prevede quindi di scoprire }
\put(41.8,-561.611){\fontsize{12}{1}\usefont{T1}{cmr}{m}{n}\selectfont\color{color_217499}la dimensione della tabella e l'ordine di lettura delle righe, la minaccia alla robustezza sta nel fatto }
\put(41.8,-575.411){\fontsize{12}{1}\usefont{T1}{cmr}{m}{n}\selectfont\color{color_217499}che digrammi e trigrammi potrebbero presentarsi a distanze simili, permettendo di scoprire di }
\put(41.8,-589.211){\fontsize{12}{1}\usefont{T1}{cmr}{m}{n}\selectfont\color{color_217499}quanto sono stati trasposti i caratteri. Applicando la trasposizione più volte rende difficile questa }
\put(41.8,-603.011){\fontsize{12}{1}\usefont{T1}{cmr}{m}{n}\selectfont\color{color_217499}analisi perché i passi successivi partono da un testo i cui digrammi non sono quelli della lingua }
\put(41.8,-616.811){\fontsize{12}{1}\usefont{T1}{cmr}{m}{n}\selectfont\color{color_217499}originale.}
\put(41.8,-637.611){\fontsize{12}{1}\usefont{T1}{cmr}{m}{n}\selectfont\color{color_217499}La sostituzione polialfabetica prevede la somma della chiave al testo, modulo 26 (A=0, … Z=25). Il}
\put(41.8,-651.411){\fontsize{12}{1}\usefont{T1}{cmr}{m}{n}\selectfont\color{color_217499}risultato è che la frequenza di un carattere in chiaro sia sparsa su più caratteri cifrati (stesso char }
\put(41.8,-665.211){\fontsize{12}{1}\usefont{T1}{cmr}{m}{n}\selectfont\color{color_217499}chiaro diventa diversi char cifrati a seconda di quale parte della chiave viene sommata); e la }
\put(41.8,-679.011){\fontsize{12}{1}\usefont{T1}{cmr}{m}{n}\selectfont\color{color_217499}frequenza di un carattere cifrato deriva da contributi di più caratteri in chiaro (stesso char cifrato }
\put(41.8,-692.811){\fontsize{12}{1}\usefont{T1}{cmr}{m}{n}\selectfont\color{color_217499}può essere risultato di diversi char chiaro in base alla parte di chiave sommata). L'analisi statistica }
\put(41.8,-706.611){\fontsize{12}{1}\usefont{T1}{cmr}{m}{n}\selectfont\color{color_217499}diventa difficile, ma è attaccabile osservando il ripetersi delle sostituzioni (stessi char chiari }
\put(41.8,-720.411){\fontsize{12}{1}\usefont{T1}{cmr}{m}{n}\selectfont\color{color_217499}diventano stesso char cifrato ogni X sostituzioni) oppure conoscendo parte del messaggio (se si può }
\put(41.8,-734.211){\fontsize{12}{1}\usefont{T1}{cmr}{m}{n}\selectfont\color{color_217499}sapere che tutti i messaggi iniziano con "domani" o che i bollettini meteo inizino con l'indicazione }
\put(41.8,-748.011){\fontsize{12}{1}\usefont{T1}{cmr}{m}{n}\selectfont\color{color_217499}della stazione da cui provengono). }
\end{picture}
\newpage
\begin{tikzpicture}[overlay]\path(0pt,0pt);\end{tikzpicture}
\begin{picture}(-5,0)(2.5,0)
\put(41.8,-85.01099){\fontsize{12}{1}\usefont{T1}{cmr}{m}{n}\selectfont\color{color_217499}Dall'idea di base che la polialfabetica sia debole perché la chiave si ripete, nasce il one-time pad: }
\put(41.8,-98.81097){\fontsize{12}{1}\usefont{T1}{cmr}{m}{n}\selectfont\color{color_217499}anziché generare la chiave in modo deterministico, ogni simbolo è scelto casualmente per avere una}
\put(41.8,-112.611){\fontsize{12}{1}\usefont{T1}{cmr}{m}{n}\selectfont\color{color_217499}chiave con la stessa lunghezza del messaggio. Questo fornisce sicurezza assoluta (non cambia }
\put(41.8,-126.411){\fontsize{12}{1}\usefont{T1}{cmr}{m}{n}\selectfont\color{color_217499}l'incertezza sul contenuto del messaggio se l'attaccante fa analisi statistica prima o dopo il tentativo }
\put(41.8,-140.211){\fontsize{12}{1}\usefont{T1}{cmr}{m}{n}\selectfont\color{color_217499}a forza bruta: in entrambi i casi lo spazio è quello di tutte le parole della lingua considerata). È }
\put(41.8,-154.011){\fontsize{12}{1}\usefont{T1}{cmr}{m}{n}\selectfont\color{color_217499}difficile da implementare perché c'è rischio di perdere la sincronia tra mittente e ricevente su quale }
\put(41.8,-167.811){\fontsize{12}{1}\usefont{T1}{cmr}{m}{n}\selectfont\color{color_217499}sia il carattere usato per cifrarne uno, e c'è il costo di tenere in memoria tutte le chiavi possibili.}
\put(41.8,-188.611){\fontsize{12}{1}\usefont{T1}{cmr}{m}{n}\selectfont\color{color_217499}I cifrari simmetrici moderni operano basandosi sugli stessi principi di confusione, ripetendo }
\put(41.8,-202.411){\fontsize{12}{1}\usefont{T1}{cmr}{m}{n}\selectfont\color{color_217499}sostituzioni e trasposizioni. Sono studiati per essere computazionalmente sicuri, questa sicurezza }
\put(41.8,-216.211){\fontsize{12}{1}\usefont{T1}{cmr}{m}{n}\selectfont\color{color_217499}risiede nella lunghezza della chiave: lo standard attuale, AES, usa chiavi di oltre 64 bit, già da 128 }
\put(41.8,-230.011){\fontsize{12}{1}\usefont{T1}{cmr}{m}{n}\selectfont\color{color_217499}sono introvabili in tempi umani con le tecnologie attuali. }
\put(41.8,-250.811){\fontsize{12}{1}\usefont{T1}{cmr}{m}{n}\selectfont\color{color_217499}Gli stessi principi si possono usare senza chiave per ottenere un'"impronta digitale" compatta di }
\put(41.8,-264.611){\fontsize{12}{1}\usefont{T1}{cmr}{m}{n}\selectfont\color{color_217499}documenti di dimensione arbitraria. I fingerprint sono il risultato di dimensione fissa di funzioni }
\put(41.8,-278.411){\fontsize{12}{1}\usefont{T1}{cmr}{m}{n}\selectfont\color{color_217499}pubbliche senza chiave (funzioni di hash), avendo dimensione fissa non permettono una }
\put(41.8,-292.211){\fontsize{12}{1}\usefont{T1}{cmr}{m}{n}\selectfont\color{color_217499}corrispondenza biunivoca tra fingerprint e documento. Le funzioni hash sono robuste se non si }
\put(41.8,-306.011){\fontsize{12}{1}\usefont{T1}{cmr}{m}{n}\selectfont\color{color_217499}riesce a trovare un documento fornendo il fingerprint (unidirezionalità) e non si possono trovare due}
\put(41.8,-319.811){\fontsize{12}{1}\usefont{T1}{cmr}{m}{n}\selectfont\color{color_217499}documenti con lo stesso fingerprint (assenza di collisioni). La difficoltà di invertire la funzione è }
\put(41.8,-333.611){\fontsize{12}{1}\usefont{T1}{cmr}{m}{n}\selectfont\color{color_217499}fondamentale, si realizza mediante molte operazioni in aritmetica modulare (come risultato di }
\put(41.8,-347.411){\fontsize{12}{1}\usefont{T1}{cmr}{m}{n}\selectfont\color{color_217499}un'operazione si prende il resto della divisione per un modulo fisso). Queste operazioni rendono }
\put(41.8,-361.211){\fontsize{12}{1}\usefont{T1}{cmr}{m}{n}\selectfont\color{color_217499}difficile la ricerca efficiente delle radici.}
\put(41.8,-382.011){\fontsize{12}{1}\usefont{T1}{cmr}{m}{n}\selectfont\color{color_217499}La crittografia asimmetrica si basa sulla generazione di chiavi partendo da due numeri primi p e q : }
\put(41.8,-395.811){\fontsize{12}{1}\usefont{T1}{cmr}{m}{n}\selectfont\color{color_217499}viene calcolato il modulo n = p . q (p e q devono essere abbastanza grandi che da n non si possa }
\put(41.8,-409.611){\fontsize{12}{1}\usefont{T1}{cmr}{m}{n}\selectfont\color{color_217499}risalire ad essi in modo efficiente), dopo aver scelto a caso un numero d si calcola e tale che e . d }
\put(41.8,-423.411){\fontsize{12}{1}\usefont{T1}{cmr}{m}{it}\selectfont\color{color_217499}mod (p-1)(q-1) = 1. Calcolare e è facile soltanto conoscendo p e q, che dopo la generazione delle }
\put(41.8,-437.211){\fontsize{12}{1}\usefont{T1}{cmr}{m}{n}\selectfont\color{color_217499}chiavi vengono dimenticati: la chiave pubblica sarà (e,n), quella privata (d,n). La cifratura di un }
\put(41.8,-451.011){\fontsize{12}{1}\usefont{T1}{cmr}{m}{n}\selectfont\color{color_217499}valore seguirà c = m\^e mod n ; la decifrazione m = c\^d mod n (l'elevamento a potenza è }
\put(41.8,-464.811){\fontsize{12}{1}\usefont{T1}{cmr}{m}{n}\selectfont\color{color_217499}un'operazione facile; l'estrazione della radice è difficile: esiste ma è grande da calcolare).}
\put(41.8,-485.611){\fontsize{12}{1}\usefont{T1}{cmr}{m}{n}\selectfont\color{color_217499}La chiave pubblica può essere distribuita (essendo impossibile usarla per trovare la corrispondente }
\put(41.8,-499.411){\fontsize{12}{1}\usefont{T1}{cmr}{m}{n}\selectfont\color{color_217499}privata) e chiunque può usarla per cifrare, mentre solo la chiave privata corrispondente può }
\put(41.8,-513.211){\fontsize{12}{1}\usefont{T1}{cmr}{m}{n}\selectfont\color{color_217499}decifrare (garantendo la riservatezza). La chiave privata, essendo specifica di un utente, può essere }
\put(41.8,-527.011){\fontsize{12}{1}\usefont{T1}{cmr}{m}{n}\selectfont\color{color_217499}utile anche per autenticare (guardando a integrità e autenticità): si prende il documento, si passa }
\put(41.8,-540.811){\fontsize{12}{1}\usefont{T1}{cmr}{m}{n}\selectfont\color{color_217499}attraverso l'hash, poi l'hash viene cifrato con chiave privata. Si ottiene così un piccolo dato chiamato}
\put(41.8,-554.611){\fontsize{12}{1}\usefont{T1}{cmr}{m}{n}\selectfont\color{color_217499}firma, allegato al documento originale: il destinatario fa hash del documento e verifica che il }
\put(41.8,-568.411){\fontsize{12}{1}\usefont{T1}{cmr}{m}{n}\selectfont\color{color_217499}fingerprint sia uguale alla decifrazione della firma usando la chiave pubblica corrispondente alla }
\put(41.8,-582.211){\fontsize{12}{1}\usefont{T1}{cmr}{m}{n}\selectfont\color{color_217499}chiave privata che ha prodotto la firma. L'integrità viene garantita dal non cambiare dell'hash, }
\put(41.8,-596.011){\fontsize{12}{1}\usefont{T1}{cmr}{m}{n}\selectfont\color{color_217499}l'autenticità dal fatto che il confronto ha successo solo se la firma è stata prodotta dalla privata }
\put(41.8,-609.811){\fontsize{12}{1}\usefont{T1}{cmr}{m}{n}\selectfont\color{color_217499}corrispondente (si assume che la pubblica usata per decifrare la firma sia effettivamente del }
\put(41.8,-623.611){\fontsize{12}{1}\usefont{T1}{cmr}{m}{n}\selectfont\color{color_217499}mittente...).}
\put(41.8,-644.411){\fontsize{12}{1}\usefont{T1}{cmr}{m}{n}\selectfont\color{color_217499}I vantaggi della crittografia asimmetrica risiedono nella possibilità di distribuzione delle chiavi e }
\put(41.8,-658.211){\fontsize{12}{1}\usefont{T1}{cmr}{m}{n}\selectfont\color{color_217499}nell'utilità per tutte le proprietà di sicurezza (riservatezza, integrità, autenticità); i punti deboli sono}
\put(59.8,-679.011){\fontsize{12}{1}\usefont{T1}{cmr}{m}{n}\selectfont\color{color_29791}•le prestazioni (5-10 volte più lento di AES, si ovvia usando sistemi ibridi: si genera una }
\put(77.8,-692.811){\fontsize{12}{1}\usefont{T1}{cmr}{m}{n}\selectfont\color{color_217499}chiave casuale e si usa con AES (molto efficiente) per cifrare il messaggio grande. Questa }
\put(77.8,-706.611){\fontsize{12}{1}\usefont{T1}{cmr}{m}{n}\selectfont\color{color_217499}chiave si cifra con la chiave pubblica del destinatario, che userà la propria chiave privata per}
\put(77.8,-720.411){\fontsize{12}{1}\usefont{T1}{cmr}{m}{n}\selectfont\color{color_217499}decifrarla e poi la chiave simmetrica ottenuta per riavere il messaggio originale)}
\put(59.8,-741.211){\fontsize{12}{1}\usefont{T1}{cmr}{m}{n}\selectfont\color{color_29791}•attacchi specifici (known plaintext: es. referendum dove il messaggio è sempre SI o NO, se }
\put(77.8,-755.011){\fontsize{12}{1}\usefont{T1}{cmr}{m}{n}\selectfont\color{color_217499}cifrato con chiave pubblica, anche l'attaccante può verificare qual è il risultato. Si usano }
\end{picture}
\newpage
\begin{tikzpicture}[overlay]\path(0pt,0pt);\end{tikzpicture}
\begin{picture}(-5,0)(2.5,0)
\put(77.8,-85.01099){\fontsize{12}{1}\usefont{T1}{cmr}{m}{n}\selectfont\color{color_217499}algoritmi per garantire che plaintext troppo corti o semplici vengano trasformati in numeri di}
\put(77.8,-98.81097){\fontsize{12}{1}\usefont{T1}{cmr}{m}{n}\selectfont\color{color_217499}complessità maggiore).}
\put(41.8,-119.611){\fontsize{12}{1}\usefont{T1}{cmr}{m}{n}\selectfont\color{color_217499}E' possibile usare la crittografia asimmetrica per l'autenticazione attiva. Nell'auth passiva (es. }
\put(41.8,-133.411){\fontsize{12}{1}\usefont{T1}{cmr}{m}{n}\selectfont\color{color_217499}password) il segreto viene svelato a chi verifica ed è sempre uguale (intercettare una password }
\put(41.8,-147.211){\fontsize{12}{1}\usefont{T1}{cmr}{m}{n}\selectfont\color{color_217499}permette un replay attack, riutilizzandola). Con auth attiva un prover dimostra a un verifier di avere }
\put(41.8,-161.011){\fontsize{12}{1}\usefont{T1}{cmr}{m}{n}\selectfont\color{color_217499}il possesso di una chiave privata (ovvero la capacità di decifrare un grande numero R random scelto}
\put(41.8,-174.811){\fontsize{12}{1}\usefont{T1}{cmr}{m}{n}\selectfont\color{color_217499}dal verifier e cifrato con la chiave pubblica del prover). Si superano entrambi i problemi dell'auth }
\put(41.8,-188.611){\fontsize{12}{1}\usefont{T1}{cmr}{m}{n}\selectfont\color{color_217499}passiva: il segreto non viene svelato (il prover invia soltanto il risultato della decifrazione e non la }
\put(41.8,-202.411){\fontsize{12}{1}\usefont{T1}{cmr}{m}{n}\selectfont\color{color_217499}chiave privata), e R non viene mai riutilizzato quindi intercettarlo e rigiocarlo è inutile. Tuttavia si }
\put(41.8,-216.211){\fontsize{12}{1}\usefont{T1}{cmr}{m}{n}\selectfont\color{color_217499}ha lo stesso problema della verifica di autenticità di una firma, in quanto non c'è garanzia che il }
\put(41.8,-230.011){\fontsize{12}{1}\usefont{T1}{cmr}{m}{n}\selectfont\color{color_217499}verifier detenga la chiave pubblica effettivamente del prover e non di un impostore.}
\put(41.8,-250.811){\fontsize{12}{1}\usefont{T1}{cmr}{m}{n}\selectfont\color{color_217499}Il metodo più efficiente di attacco alla crittografia asimmetrica è la fattorizzazione del modulo, per }
\put(41.8,-264.611){\fontsize{12}{1}\usefont{T1}{cmr}{m}{n}\selectfont\color{color_217499}questo la lunghezza del modulo consigliata è di 2048 bit e oltre.}
\put(41.8,-300.611){\fontsize{17.5}{1}\usefont{T1}{cmr}{b}{n}\selectfont\color{color_29791}Vagrant e SSH}
\end{picture}
\begin{tikzpicture}[overlay]
\path(0pt,0pt);
\draw[color_29791,line width=0.9pt]
(41.8pt, -295.611pt) -- (161.4pt, -295.611pt)
;
\end{tikzpicture}
\begin{picture}(-5,0)(2.5,0)
\put(41.8,-330.511){\fontsize{14.1}{1}\usefont{T1}{cmr}{b}{n}\selectfont\color{color_29791}SSH}
\put(41.8,-350.711){\fontsize{12}{1}\usefont{T1}{cmr}{m}{n}\selectfont\color{color_29791}Per poter avere amministrazione remota è stato sviluppato TELNET, che però non offriva nessuna }
\put(41.8,-364.511){\fontsize{12}{1}\usefont{T1}{cmr}{m}{n}\selectfont\color{color_29791}confidenzialità del canale né autenticazione dell’host, vi era soltanto autenticazione passiva }
\put(41.8,-378.311){\fontsize{12}{1}\usefont{T1}{cmr}{m}{n}\selectfont\color{color_29791}dell’utente. Secure Shell (SSH) è l’evoluzione nata per offrire maggiore sicurezza, basandosi }
\put(41.8,-392.111){\fontsize{12}{1}\usefont{T1}{cmr}{m}{n}\selectfont\color{color_29791}sull’autenticazione a chiave pubblica. Questa si basa si basa sulla crittografia asimmetrica, il segreto}
\put(41.8,-405.911){\fontsize{12}{1}\usefont{T1}{cmr}{m}{n}\selectfont\color{color_29791}è una chiave privata posseduta soltanto da chi vuole autenticarsi, che riceve una challenge di }
\put(41.8,-419.711){\fontsize{12}{1}\usefont{T1}{cmr}{m}{n}\selectfont\color{color_29791}autenticazione dal verifier e usa la chiave privata per fare response; il verificatore possiede una }
\put(41.8,-433.511){\fontsize{12}{1}\usefont{T1}{cmr}{m}{n}\selectfont\color{color_29791}chiave pubblica legata matematicamente a quella privata (ma che non permette di calcolarla) con }
\put(41.8,-447.311){\fontsize{12}{1}\usefont{T1}{cmr}{m}{n}\selectfont\color{color_29791}cui può controllare se la risposta è corretta. Il collegamento SSH tra client (ssh) e server (sshd) }
\put(41.8,-461.111){\fontsize{12}{1}\usefont{T1}{cmr}{m}{n}\selectfont\color{color_29791}percorre alcuni passi:}
\put(59.8,-481.911){\fontsize{12}{1}\usefont{T1}{cmr}{m}{n}\selectfont\color{color_29791}•Negoziazione dei cifrari disponibili}
\put(59.8,-502.711){\fontsize{12}{1}\usefont{T1}{cmr}{m}{n}\selectfont\color{color_29791}•Autenticazione host remoto per mezzo della sua chiave pubblica}
\put(59.8,-523.511){\fontsize{12}{1}\usefont{T1}{cmr}{m}{n}\selectfont\color{color_29791}•Inizializzazione canale comunicazione cifrata}
\put(59.8,-544.311){\fontsize{12}{1}\usefont{T1}{cmr}{m}{n}\selectfont\color{color_29791}•Negoziazione dei metodi disponibili per l’autenticazione utente}
\put(59.8,-565.111){\fontsize{12}{1}\usefont{T1}{cmr}{m}{n}\selectfont\color{color_29791}•Autenticazione utente}
\put(41.8,-585.911){\fontsize{12}{1}\usefont{T1}{cmr}{m}{n}\selectfont\color{color_29791}Autenticare l’host remoto è importante per evitare attacchi MITM che potrebbero catturare la }
\put(41.8,-599.711){\fontsize{12}{1}\usefont{T1}{cmr}{m}{n}\selectfont\color{color_29791}password dell’amministratore spacciandosi per l’host, ma non è previsto un sistema centrale per la }
\put(41.8,-613.511){\fontsize{12}{1}\usefont{T1}{cmr}{m}{n}\selectfont\color{color_29791}verifica della chiave dell’host; quindi alla prima connessione è l’amministratore a dover usare un }
\put(41.8,-627.311){\fontsize{12}{1}\usefont{T1}{cmr}{m}{n}\selectfont\color{color_29791}metodo out-of-band (verifica esterna ad ssh) per verificare la chiave pubblica presentata dall’host. }
\put(41.8,-641.111){\fontsize{12}{1}\usefont{T1}{cmr}{m}{n}\selectfont\color{color_29791}Le chiavi pubbliche vengono salvate in known\_hosts nella directory .ssh nella home client, }
\put(41.8,-654.911){\fontsize{12}{1}\usefont{T1}{cmr}{m}{n}\selectfont\color{color_29791}permettendo autenticazione attiva alle connessioni successive. }
\put(41.8,-675.711){\fontsize{12}{1}\usefont{T1}{cmr}{m}{n}\selectfont\color{color_29791}Per autenticare l’utente si può avere autenticazione passiva (tradizionale con username e password }
\put(41.8,-689.511){\fontsize{12}{1}\usefont{T1}{cmr}{m}{n}\selectfont\color{color_29791}trasmessi su canale cifrato) o attiva (usando un protocollo challenge-response a chiave pubblica, }
\put(41.8,-703.311){\fontsize{12}{1}\usefont{T1}{cmr}{m}{n}\selectfont\color{color_29791}supponendo che l’utente abbia una coppia di chiavi e che installi correttamente sull’host remoto la }
\put(41.8,-717.111){\fontsize{12}{1}\usefont{T1}{cmr}{m}{n}\selectfont\color{color_29791}chiave pubblica. In entrambi i casi l’identità dell’utente che vuole fare login sull’host remoto può }
\put(41.8,-730.911){\fontsize{12}{1}\usefont{T1}{cmr}{m}{n}\selectfont\color{color_29791}essere selezionata, a default viene usata lo stesso nome utente che sta lavorando sul client (es. user }
\put(41.8,-744.711){\fontsize{12}{1}\usefont{T1}{cmr}{m}{n}\selectfont\color{color_29791}esegue “ssh rmserver” e si dovra autenticare come user, user esegue “ssh marco@rmserver}
\end{picture}
\begin{tikzpicture}[overlay]
\path(0pt,0pt);
\draw[color_29919,line width=0.7pt]
(394.2pt, -745.811pt) -- (477.9pt, -745.811pt)
;
\end{tikzpicture}
\begin{picture}(-5,0)(2.5,0)
\put(477.9,-744.711){\fontsize{12}{1}\usefont{T1}{cmr}{m}{n}\selectfont\color{color_29791}” e dovrà}
\put(41.8,-758.511){\fontsize{12}{1}\usefont{T1}{cmr}{m}{n}\selectfont\color{color_29791}autenticarsi come marco). }
\end{picture}
\newpage
\begin{tikzpicture}[overlay]\path(0pt,0pt);\end{tikzpicture}
\begin{picture}(-5,0)(2.5,0)
\put(41.8,-85.01099){\fontsize{12}{1}\usefont{T1}{cmr}{m}{n}\selectfont\color{color_29791}Per fare l’autenticazione attiva un utente deve}
\put(59.8,-105.811){\fontsize{12}{1}\usefont{T1}{cmr}{m}{n}\selectfont\color{color_29791}•generare una coppia di chiavi asimmetriche (comando ssh-keygen -t rsa -b 2048)}
\put(59.8,-126.611){\fontsize{12}{1}\usefont{T1}{cmr}{m}{n}\selectfont\color{color_29791}•installare sull’host remoto la chiave pubblica (file locale .ssh/is\_rsa.pub. copia su host }
\put(77.8,-140.411){\fontsize{12}{1}\usefont{T1}{cmr}{m}{n}\selectfont\color{color_29791}remoto (scp .ssh/id\_rsa.pub user@remote}
\end{picture}
\begin{tikzpicture}[overlay]
\path(0pt,0pt);
\draw[color_29919,line width=0.7pt]
(214.2pt, -141.511pt) -- (278.8pt, -141.511pt)
;
\end{tikzpicture}
\begin{picture}(-5,0)(2.5,0)
\put(278.8,-140.411){\fontsize{12}{1}\usefont{T1}{cmr}{m}{it}\selectfont\color{color_29791}: ) )}
\put(59.8,-161.211){\fontsize{12}{1}\usefont{T1}{cmr}{m}{n}\selectfont\color{color_29791}•append alla lista di utenti autorizzati (su remote, cat id\_rsa.pub >> .ssh\_authorized\_keys)}
\put(41.8,-182.011){\fontsize{12}{1}\usefont{T1}{cmr}{m}{n}\selectfont\color{color_29791}In questo caso la segretezza della password viene sostituita dalla riservatezza del file della chiave }
\put(41.8,-195.811){\fontsize{12}{1}\usefont{T1}{cmr}{m}{n}\selectfont\color{color_29791}privata sul client, in pratica è necessario avere grande cura dei permessi di file e directory (es. il }
\put(41.8,-209.611){\fontsize{12}{1}\usefont{T1}{cmr}{m}{n}\selectfont\color{color_29791}passwordless login può non funzionare se l’host remoto ha permessi troppo larghi sulla dir .ssh, il }
\put(41.8,-223.411){\fontsize{12}{1}\usefont{T1}{cmr}{m}{n}\selectfont\color{color_29791}server sshd non si fida dell’integrità del contenuto). Se proteggiamo il file chiave privata con una }
\put(41.8,-237.211){\fontsize{12}{1}\usefont{T1}{cmr}{m}{n}\selectfont\color{color_29791}password perdiamo la possibilità di passwordless login ma abbiamo più sicurezza del digitare }
\put(41.8,-251.011){\fontsize{12}{1}\usefont{T1}{cmr}{m}{n}\selectfont\color{color_29791}direttamente la password dell’account remoto, ed è più pratico se si gestiscono molti host remoti.}
\put(41.8,-271.811){\fontsize{12}{1}\usefont{T1}{cmr}{m}{n}\selectfont\color{color_29791}Con ssh utente@host}
\end{picture}
\begin{tikzpicture}[overlay]
\path(0pt,0pt);
\draw[color_29919,line width=0.7pt]
(83.2pt, -272.9109pt) -- (143.5pt, -272.9109pt)
;
\end{tikzpicture}
\begin{picture}(-5,0)(2.5,0)
\put(143.5,-271.811){\fontsize{12}{1}\usefont{T1}{cmr}{m}{it}\selectfont\color{color_29791} si ottiene un terminale remoto interattivo, aggiungendo un ulteriore parametro}
\put(41.8,-285.611){\fontsize{12}{1}\usefont{T1}{cmr}{m}{n}\selectfont\color{color_29791}viene interpretato come comando da eseguire sull’host remoto al posto della shell interattiva (es. ssh}
\put(41.8,-299.411){\fontsize{12}{1}\usefont{T1}{cmr}{m}{n}\selectfont\color{color_29919}root@server}
\end{picture}
\begin{tikzpicture}[overlay]
\path(0pt,0pt);
\draw[color_29919,line width=0.7pt]
(41.8pt, -300.511pt) -- (101.5pt, -300.511pt)
;
\end{tikzpicture}
\begin{picture}(-5,0)(2.5,0)
\put(101.5,-299.411){\fontsize{12}{1}\usefont{T1}{cmr}{m}{n}\selectfont\color{color_217499} "grep pattern") , gli stream di I/O vengono riportati attraverso il canale cifrato sul }
\put(41.8,-313.211){\fontsize{12}{1}\usefont{T1}{cmr}{m}{n}\selectfont\color{color_29791}client (STDIN del processo ssh sul client passa dati a STDIN del processo invocato sul server, }
\put(41.8,-327.011){\fontsize{12}{1}\usefont{T1}{cmr}{m}{n}\selectfont\color{color_29791}STDOUT e STDERR del processo remoto fuoriescono dagli analoghi stream dal processo ssh sul }
\put(41.8,-340.811){\fontsize{12}{1}\usefont{T1}{cmr}{m}{n}\selectfont\color{color_29791}client).}
\put(41.8,-370.611){\fontsize{14.1}{1}\usefont{T1}{cmr}{b}{n}\selectfont\color{color_29791}Vagrant}
\end{picture}
\begin{tikzpicture}[overlay]
\path(0pt,0pt);
\draw[color_29791,line width=0.8pt]
(41.8pt, -366.511pt) -- (93.5pt, -366.511pt)
;
\end{tikzpicture}
\begin{picture}(-5,0)(2.5,0)
\put(41.8,-390.811){\fontsize{12}{1}\usefont{T1}{cmr}{m}{n}\selectfont\color{color_29791}Vagrant è una libreria per la gestione di macchine virtuali che permette di rendere portabile la }
\end{picture}
\begin{tikzpicture}[overlay]
\path(0pt,0pt);
\draw[color_29791,line width=0.7pt]
(41.8pt, -387.411pt) -- (493.6pt, -387.411pt)
;
\end{tikzpicture}
\begin{picture}(-5,0)(2.5,0)
\put(41.8,-404.611){\fontsize{12}{1}\usefont{T1}{cmr}{m}{n}\selectfont\color{color_29791}configurazione rispetto a diversi virtualizzatori. L’immagine della VM di base (box) può essere }
\end{picture}
\begin{tikzpicture}[overlay]
\path(0pt,0pt);
\draw[color_29791,line width=0.7pt]
(41.8pt, -401.211pt) -- (501.6pt, -401.211pt)
;
\end{tikzpicture}
\begin{picture}(-5,0)(2.5,0)
\put(41.8,-418.411){\fontsize{12}{1}\usefont{T1}{cmr}{m}{n}\selectfont\color{color_29791}depositata e facilmente importata da un deposito (VagrantCloud), i parametri per istanziare una VM }
\end{picture}
\begin{tikzpicture}[overlay]
\path(0pt,0pt);
\draw[color_29791,line width=0.7pt]
(41.8pt, -415.011pt) -- (523pt, -415.011pt)
;
\end{tikzpicture}
\begin{picture}(-5,0)(2.5,0)
\put(41.8,-432.211){\fontsize{12}{1}\usefont{T1}{cmr}{m}{n}\selectfont\color{color_29791}sono definiti in un file (VagrantFile). Un V.File contiene informazioni sulla box di base, sui vari tipi }
\end{picture}
\begin{tikzpicture}[overlay]
\path(0pt,0pt);
\draw[color_29791,line width=0.7pt]
(41.8pt, -428.811pt) -- (523.2pt, -428.811pt)
;
\end{tikzpicture}
\begin{picture}(-5,0)(2.5,0)
\put(41.8,-446.011){\fontsize{12}{1}\usefont{T1}{cmr}{m}{n}\selectfont\color{color_29791}di interfacce di rete, la mappatura di cartelle condivise host-guest, il provider del virtualizzatore e la}
\end{picture}
\begin{tikzpicture}[overlay]
\path(0pt,0pt);
\draw[color_29791,line width=0.7pt]
(41.8pt, -442.611pt) -- (521.2pt, -442.611pt)
;
\end{tikzpicture}
\begin{picture}(-5,0)(2.5,0)
\put(41.8,-459.811){\fontsize{12}{1}\usefont{T1}{cmr}{m}{n}\selectfont\color{color_29791}configurazione dell’ HW virtuale, oltre che la configurazione d’ambiente, il provisioning: una serie }
\end{picture}
\begin{tikzpicture}[overlay]
\path(0pt,0pt);
\draw[color_29791,line width=0.7pt]
(41.8pt, -456.411pt) -- (520.2pt, -456.411pt)
;
\end{tikzpicture}
\begin{picture}(-5,0)(2.5,0)
\put(41.8,-473.611){\fontsize{12}{1}\usefont{T1}{cmr}{m}{n}\selectfont\color{color_29791}di comandi eseguiti da vagrant nella shell del guest al primo avvio (se fallisce in modo plateale è }
\end{picture}
\begin{tikzpicture}[overlay]
\path(0pt,0pt);
\draw[color_29791,line width=0.7pt]
(41.8pt, -470.211pt) -- (508.9pt, -470.211pt)
;
\end{tikzpicture}
\begin{picture}(-5,0)(2.5,0)
\put(41.8,-487.411){\fontsize{12}{1}\usefont{T1}{cmr}{m}{n}\selectfont\color{color_29791}vagrant a ripeterlo, altrimenti tocca all’amministratore).}
\end{picture}
\begin{tikzpicture}[overlay]
\path(0pt,0pt);
\draw[color_29791,line width=0.7pt]
(41.8pt, -484.011pt) -- (309.3pt, -484.011pt)
;
\end{tikzpicture}
\begin{picture}(-5,0)(2.5,0)
\put(41.8,-508.211){\fontsize{12}{1}\usefont{T1}{cmr}{m}{n}\selectfont\color{color_29791}Per installare una box si può ricorrere al repo ufficiale VagrantCloud, scegliendo una box tra queste }
\end{picture}
\begin{tikzpicture}[overlay]
\path(0pt,0pt);
\draw[color_29791,line width=0.7pt]
(41.8pt, -504.811pt) -- (521.5pt, -504.811pt)
;
\end{tikzpicture}
\begin{picture}(-5,0)(2.5,0)
\put(41.8,-522.011){\fontsize{12}{1}\usefont{T1}{cmr}{m}{n}\selectfont\color{color_29791}l’installazione è automatica al primo avvio ma si può scaricare localmente per evitare l’attesa del }
\end{picture}
\begin{tikzpicture}[overlay]
\path(0pt,0pt);
\draw[color_29791,line width=0.7pt]
(41.8pt, -518.611pt) -- (509.9pt, -518.611pt)
;
\end{tikzpicture}
\begin{picture}(-5,0)(2.5,0)
\put(41.8,-535.811){\fontsize{12}{1}\usefont{T1}{cmr}{m}{n}\selectfont\color{color_29791}download all’avvio VM o per scegliere fonte diversa (}
\end{picture}
\begin{tikzpicture}[overlay]
\path(0pt,0pt);
\draw[color_29791,line width=0.7pt]
(41.8pt, -532.411pt) -- (302.1pt, -532.411pt)
;
\end{tikzpicture}
\begin{picture}(-5,0)(2.5,0)
\put(302.2,-535.811){\fontsize{12}{1}\usefont{T1}{cmr}{m}{it}\selectfont\color{color_29791}vagrant box add autore/version [url]}
\end{picture}
\begin{tikzpicture}[overlay]
\path(0pt,0pt);
\draw[color_29791,line width=0.7pt]
(302.2pt, -532.411pt) -- (479.7pt, -532.411pt)
;
\end{tikzpicture}
\begin{picture}(-5,0)(2.5,0)
\put(479.7,-535.811){\fontsize{12}{1}\usefont{T1}{cmr}{m}{n}\selectfont\color{color_29791}). La box}
\end{picture}
\begin{tikzpicture}[overlay]
\path(0pt,0pt);
\draw[color_29791,line width=0.7pt]
(479.7pt, -532.411pt) -- (523.3pt, -532.411pt)
;
\end{tikzpicture}
\begin{picture}(-5,0)(2.5,0)
\put(41.8,-549.611){\fontsize{12}{1}\usefont{T1}{cmr}{m}{n}\selectfont\color{color_29791}è un template che viene salvato in una directory; per crearne diverse istanze VM tipicamente si crea }
\end{picture}
\begin{tikzpicture}[overlay]
\path(0pt,0pt);
\draw[color_29791,line width=0.7pt]
(41.8pt, -546.211pt) -- (523pt, -546.211pt)
;
\end{tikzpicture}
\begin{picture}(-5,0)(2.5,0)
\put(41.8,-563.411){\fontsize{12}{1}\usefont{T1}{cmr}{m}{n}\selectfont\color{color_29791}una directory per ogni istanza, ci si pone lì e si lancia }
\end{picture}
\begin{tikzpicture}[overlay]
\path(0pt,0pt);
\draw[color_29791,line width=0.7pt]
(41.8pt, -560.011pt) -- (300pt, -560.011pt)
;
\end{tikzpicture}
\begin{picture}(-5,0)(2.5,0)
\put(300.1,-563.411){\fontsize{12}{1}\usefont{T1}{cmr}{m}{it}\selectfont\color{color_29791}vagrant init autore/version, }
\end{picture}
\begin{tikzpicture}[overlay]
\path(0pt,0pt);
\draw[color_29791,line width=0.7pt]
(300.1pt, -560.011pt) -- (434.9pt, -560.011pt)
;
\end{tikzpicture}
\begin{picture}(-5,0)(2.5,0)
\put(434.9,-563.411){\fontsize{12}{1}\usefont{T1}{cmr}{m}{n}\selectfont\color{color_29791}così il V.File }
\end{picture}
\begin{tikzpicture}[overlay]
\path(0pt,0pt);
\draw[color_29791,line width=0.7pt]
(434.9pt, -560.011pt) -- (498.4pt, -560.011pt)
;
\end{tikzpicture}
\begin{picture}(-5,0)(2.5,0)
\put(41.8,-577.211){\fontsize{12}{1}\usefont{T1}{cmr}{m}{n}\selectfont\color{color_29791}generato avrà il riferimento a quella VM in un luogo separato. Con la init viene creato in quella dir }
\end{picture}
\begin{tikzpicture}[overlay]
\path(0pt,0pt);
\draw[color_29791,line width=0.7pt]
(41.8pt, -573.811pt) -- (518.7pt, -573.811pt)
;
\end{tikzpicture}
\begin{picture}(-5,0)(2.5,0)
\put(41.8,-591.011){\fontsize{12}{1}\usefont{T1}{cmr}{m}{n}\selectfont\color{color_29791}un V.File con i default della box, modificabile; i comandi }
\end{picture}
\begin{tikzpicture}[overlay]
\path(0pt,0pt);
\draw[color_29791,line width=0.7pt]
(41.8pt, -587.611pt) -- (319.3pt, -587.611pt)
;
\end{tikzpicture}
\begin{picture}(-5,0)(2.5,0)
\put(319.3,-591.011){\fontsize{12}{1}\usefont{T1}{cmr}{m}{it}\selectfont\color{color_29791}vagrant \{up|status|halt|destroy\} }
\end{picture}
\begin{tikzpicture}[overlay]
\path(0pt,0pt);
\draw[color_29791,line width=0.7pt]
(319.3pt, -587.611pt) -- (475.6pt, -587.611pt)
;
\end{tikzpicture}
\begin{picture}(-5,0)(2.5,0)
\put(475.7,-591.011){\fontsize{12}{1}\usefont{T1}{cmr}{m}{n}\selectfont\color{color_29791}cercano }
\end{picture}
\begin{tikzpicture}[overlay]
\path(0pt,0pt);
\draw[color_29791,line width=0.7pt]
(475.7pt, -587.611pt) -- (516pt, -587.611pt)
;
\end{tikzpicture}
\begin{picture}(-5,0)(2.5,0)
\put(41.8,-604.811){\fontsize{12}{1}\usefont{T1}{cmr}{m}{n}\selectfont\color{color_29791}un V.File nella directory per lanciarla/vederne lo stato/fermarla/distruggerla. }
\end{picture}
\begin{tikzpicture}[overlay]
\path(0pt,0pt);
\draw[color_29791,line width=0.7pt]
(41.8pt, -601.411pt) -- (410.5pt, -601.411pt)
;
\end{tikzpicture}
\begin{picture}(-5,0)(2.5,0)
\put(41.8,-634.611){\fontsize{14.1}{1}\usefont{T1}{cmr}{b}{n}\selectfont\color{color_29791}Vagrant + SSH }
\end{picture}
\begin{tikzpicture}[overlay]
\path(0pt,0pt);
\draw[color_29791,line width=0.8pt]
(41.8pt, -630.511pt) -- (142.5pt, -630.511pt)
;
\end{tikzpicture}
\begin{picture}(-5,0)(2.5,0)
\put(41.8,-654.811){\fontsize{12}{1}\usefont{T1}{cmr}{m}{it}\selectfont\color{color_29791}vagrant up }
\end{picture}
\begin{tikzpicture}[overlay]
\path(0pt,0pt);
\draw[color_29791,line width=0.7pt]
(41.8pt, -651.411pt) -- (97.1pt, -651.411pt)
;
\end{tikzpicture}
\begin{picture}(-5,0)(2.5,0)
\put(97.2,-654.811){\fontsize{12}{1}\usefont{T1}{cmr}{m}{n}\selectfont\color{color_29791}crea automaticamente una chiave privata sull’host, nella directory in cui viene fatta init, }
\end{picture}
\begin{tikzpicture}[overlay]
\path(0pt,0pt);
\draw[color_29791,line width=0.7pt]
(97.2pt, -651.411pt) -- (521.3pt, -651.411pt)
;
\end{tikzpicture}
\begin{picture}(-5,0)(2.5,0)
\put(41.8,-668.611){\fontsize{12}{1}\usefont{T1}{cmr}{m}{n}\selectfont\color{color_29791}e installa la chiave pubblica nella VM, predisponendo una mappatura di rete da una porta host alla }
\end{picture}
\begin{tikzpicture}[overlay]
\path(0pt,0pt);
\draw[color_29791,line width=0.7pt]
(41.8pt, -665.211pt) -- (516pt, -665.211pt)
;
\end{tikzpicture}
\begin{picture}(-5,0)(2.5,0)
\put(41.8,-682.411){\fontsize{12}{1}\usefont{T1}{cmr}{m}{n}\selectfont\color{color_29791}22 guest. v}
\end{picture}
\begin{tikzpicture}[overlay]
\path(0pt,0pt);
\draw[color_29791,line width=0.7pt]
(41.8pt, -679.011pt) -- (94.1pt, -679.011pt)
;
\end{tikzpicture}
\begin{picture}(-5,0)(2.5,0)
\put(94.2,-682.411){\fontsize{12}{1}\usefont{T1}{cmr}{m}{it}\selectfont\color{color_29791}agrant ssh }
\end{picture}
\begin{tikzpicture}[overlay]
\path(0pt,0pt);
\draw[color_29791,line width=0.7pt]
(94.2pt, -679.011pt) -- (147.5pt, -679.011pt)
;
\end{tikzpicture}
\begin{picture}(-5,0)(2.5,0)
\put(147.5,-682.411){\fontsize{12}{1}\usefont{T1}{cmr}{m}{n}\selectfont\color{color_29791}usa in automatico queste impostazioni per connettersi alla VM come user }
\end{picture}
\begin{tikzpicture}[overlay]
\path(0pt,0pt);
\draw[color_29791,line width=0.7pt]
(147.5pt, -679.011pt) -- (502.1pt, -679.011pt)
;
\end{tikzpicture}
\begin{picture}(-5,0)(2.5,0)
\put(41.8,-696.211){\fontsize{12}{1}\usefont{T1}{cmr}{m}{n}\selectfont\color{color_29791}vagrant, senza password, abilitato a sudo. v}
\end{picture}
\begin{tikzpicture}[overlay]
\path(0pt,0pt);
\draw[color_29791,line width=0.7pt]
(41.8pt, -692.811pt) -- (250pt, -692.811pt)
;
\end{tikzpicture}
\begin{picture}(-5,0)(2.5,0)
\put(250.1,-696.211){\fontsize{12}{1}\usefont{T1}{cmr}{m}{it}\selectfont\color{color_29791}agrant ssh-config }
\end{picture}
\begin{tikzpicture}[overlay]
\path(0pt,0pt);
\draw[color_29791,line width=0.7pt]
(250.1pt, -692.811pt) -- (337.4pt, -692.811pt)
;
\end{tikzpicture}
\begin{picture}(-5,0)(2.5,0)
\put(337.4,-696.211){\fontsize{12}{1}\usefont{T1}{cmr}{m}{n}\selectfont\color{color_29791}mostra come configurare il client }
\end{picture}
\begin{tikzpicture}[overlay]
\path(0pt,0pt);
\draw[color_29791,line width=0.7pt]
(337.4pt, -692.811pt) -- (499.6pt, -692.811pt)
;
\end{tikzpicture}
\begin{picture}(-5,0)(2.5,0)
\put(499.7,-696.211){\fontsize{12}{1}\usefont{T1}{cmr}{m}{it}\selectfont\color{color_29791}ssh}
\end{picture}
\begin{tikzpicture}[overlay]
\path(0pt,0pt);
\draw[color_29791,line width=0.7pt]
(499.7pt, -692.811pt) -- (515pt, -692.811pt)
;
\end{tikzpicture}
\begin{picture}(-5,0)(2.5,0)
\put(515,-696.211){\fontsize{12}{1}\usefont{T1}{cmr}{m}{n}\selectfont\color{color_29791} }
\end{picture}
\begin{tikzpicture}[overlay]
\path(0pt,0pt);
\draw[color_29791,line width=0.7pt]
(515pt, -692.811pt) -- (518pt, -692.811pt)
;
\end{tikzpicture}
\begin{picture}(-5,0)(2.5,0)
\put(41.8,-710.011){\fontsize{12}{1}\usefont{T1}{cmr}{m}{n}\selectfont\color{color_29791}per comportarsi come }
\end{picture}
\begin{tikzpicture}[overlay]
\path(0pt,0pt);
\draw[color_29791,line width=0.7pt]
(41.8pt, -706.611pt) -- (149.4pt, -706.611pt)
;
\end{tikzpicture}
\begin{picture}(-5,0)(2.5,0)
\put(149.5,-710.011){\fontsize{12}{1}\usefont{T1}{cmr}{m}{it}\selectfont\color{color_29791}vagrant ssh}
\end{picture}
\begin{tikzpicture}[overlay]
\path(0pt,0pt);
\draw[color_29791,line width=0.7pt]
(149.5pt, -706.611pt) -- (205.1pt, -706.611pt)
;
\end{tikzpicture}
\begin{picture}(-5,0)(2.5,0)
\put(205.1,-710.011){\fontsize{12}{1}\usefont{T1}{cmr}{m}{n}\selectfont\color{color_29791}, in modo da poter usare }
\end{picture}
\begin{tikzpicture}[overlay]
\path(0pt,0pt);
\draw[color_29791,line width=0.7pt]
(205.1pt, -706.611pt) -- (324pt, -706.611pt)
;
\end{tikzpicture}
\begin{picture}(-5,0)(2.5,0)
\put(324.1,-710.011){\fontsize{12}{1}\usefont{T1}{cmr}{m}{it}\selectfont\color{color_29791}ssh }
\end{picture}
\begin{tikzpicture}[overlay]
\path(0pt,0pt);
\draw[color_29791,line width=0.7pt]
(324.1pt, -706.611pt) -- (342.4pt, -706.611pt)
;
\end{tikzpicture}
\begin{picture}(-5,0)(2.5,0)
\put(342.4,-710.011){\fontsize{12}{1}\usefont{T1}{cmr}{m}{n}\selectfont\color{color_29791}originale per operazioni non }
\end{picture}
\begin{tikzpicture}[overlay]
\path(0pt,0pt);
\draw[color_29791,line width=0.7pt]
(342.4pt, -706.611pt) -- (481pt, -706.611pt)
;
\end{tikzpicture}
\begin{picture}(-5,0)(2.5,0)
\put(41.8,-723.811){\fontsize{12}{1}\usefont{T1}{cmr}{m}{n}\selectfont\color{color_29791}supportate da }
\end{picture}
\begin{tikzpicture}[overlay]
\path(0pt,0pt);
\draw[color_29791,line width=0.7pt]
(41.8pt, -720.411pt) -- (109.1pt, -720.411pt)
;
\end{tikzpicture}
\begin{picture}(-5,0)(2.5,0)
\put(109.1,-723.811){\fontsize{12}{1}\usefont{T1}{cmr}{m}{it}\selectfont\color{color_29791}vagrant ssh, }
\end{picture}
\begin{tikzpicture}[overlay]
\path(0pt,0pt);
\draw[color_29791,line width=0.7pt]
(109.1pt, -720.411pt) -- (170.7pt, -720.411pt)
;
\end{tikzpicture}
\begin{picture}(-5,0)(2.5,0)
\put(170.8,-723.811){\fontsize{12}{1}\usefont{T1}{cmr}{m}{n}\selectfont\color{color_29791}come }
\end{picture}
\begin{tikzpicture}[overlay]
\path(0pt,0pt);
\draw[color_29791,line width=0.7pt]
(170.8pt, -720.411pt) -- (199.8pt, -720.411pt)
;
\end{tikzpicture}
\begin{picture}(-5,0)(2.5,0)
\put(199.8,-723.811){\fontsize{12}{1}\usefont{T1}{cmr}{m}{it}\selectfont\color{color_29791}scp}
\end{picture}
\begin{tikzpicture}[overlay]
\path(0pt,0pt);
\draw[color_29791,line width=0.7pt]
(199.8pt, -720.411pt) -- (215.8pt, -720.411pt)
;
\end{tikzpicture}
\begin{picture}(-5,0)(2.5,0)
\put(215.8,-723.811){\fontsize{12}{1}\usefont{T1}{cmr}{m}{n}\selectfont\color{color_29791}. }
\end{picture}
\begin{tikzpicture}[overlay]
\path(0pt,0pt);
\draw[color_29791,line width=0.7pt]
(215.8pt, -720.411pt) -- (221.8pt, -720.411pt)
;
\end{tikzpicture}
\begin{picture}(-5,0)(2.5,0)
\put(41.8,-744.611){\fontsize{12}{1}\usefont{T1}{cmr}{m}{it}\selectfont\color{color_29791}vagrant ssh-config > ssh.conf ; ssh -F ssh.conf default }
\end{picture}
\begin{tikzpicture}[overlay]
\path(0pt,0pt);
\draw[color_29791,line width=0.7pt]
(41.8pt, -741.211pt) -- (305.5pt, -741.211pt)
;
\end{tikzpicture}
\begin{picture}(-5,0)(2.5,0)
\put(305.5,-744.611){\fontsize{12}{1}\usefont{T1}{cmr}{m}{it}\selectfont\color{color_29791}      }
\end{picture}
\begin{tikzpicture}[overlay]
\path(0pt,0pt);
\draw[color_29791,line width=0.7pt]
(305.5pt, -741.211pt) -- (325.4pt, -741.211pt)
;
\end{tikzpicture}
\begin{picture}(-5,0)(2.5,0)
\put(325.4,-744.611){\fontsize{12}{1}\usefont{T1}{cmr}{m}{n}\selectfont\color{color_29791}Connessione}
\end{picture}
\begin{tikzpicture}[overlay]
\path(0pt,0pt);
\draw[color_29791,line width=0.7pt]
(325.4pt, -741.211pt) -- (386.7pt, -741.211pt)
;
\end{tikzpicture}
\begin{picture}(-5,0)(2.5,0)
\put(41.7,-765.411){\fontsize{12}{1}\usefont{T1}{cmr}{m}{it}\selectfont\color{color_29791}           }
\end{picture}
\begin{tikzpicture}[overlay]
\path(0pt,0pt);
\draw[color_29791,line width=0.7pt]
(41.7pt, -762.011pt) -- (77.2pt, -762.011pt)
;
\end{tikzpicture}
\begin{picture}(-5,0)(2.5,0)
\put(77.3,-765.411){\fontsize{12}{1}\usefont{T1}{cmr}{m}{n}\selectfont\color{color_29791}Entrambi da host}
\end{picture}
\begin{tikzpicture}[overlay]
\path(0pt,0pt);
\draw[color_29791,line width=0.7pt]
(77.3pt, -762.011pt) -- (159.3pt, -762.011pt)
;
\end{tikzpicture}
\begin{picture}(-5,0)(2.5,0)
\put(159.2,-765.411){\fontsize{12}{1}\usefont{T1}{cmr}{m}{it}\selectfont\color{color_29791}        }
\end{picture}
\begin{tikzpicture}[overlay]
\path(0pt,0pt);
\draw[color_29791,line width=0.7pt]
(159.2pt, -762.011pt) -- (183.6pt, -762.011pt)
;
\end{tikzpicture}
\begin{picture}(-5,0)(2.5,0)
\put(183.6,-765.411){\fontsize{12}{1}\usefont{T1}{cmr}{m}{it}\selectfont\color{color_29791}; scp -F ssh.conf default:/path pathlocale }
\end{picture}
\begin{tikzpicture}[overlay]
\path(0pt,0pt);
\draw[color_29791,line width=0.7pt]
(183.6pt, -762.011pt) -- (384.5pt, -762.011pt)
;
\end{tikzpicture}
\begin{picture}(-5,0)(2.5,0)
\put(384.5,-765.411){\fontsize{12}{1}\usefont{T1}{cmr}{m}{it}\selectfont\color{color_29791}   }
\end{picture}
\begin{tikzpicture}[overlay]
\path(0pt,0pt);
\draw[color_29791,line width=0.7pt]
(384.5pt, -762.011pt) -- (396.3pt, -762.011pt)
;
\end{tikzpicture}
\begin{picture}(-5,0)(2.5,0)
\put(396.3,-765.411){\fontsize{12}{1}\usefont{T1}{cmr}{m}{n}\selectfont\color{color_29791}copia da guest}
\end{picture}
\begin{tikzpicture}[overlay]
\path(0pt,0pt);
\draw[color_29791,line width=0.7pt]
(396.3pt, -762.011pt) -- (464.9pt, -762.011pt)
;
\end{tikzpicture}
\newpage
\begin{tikzpicture}[overlay]\path(0pt,0pt);\end{tikzpicture}
\begin{picture}(-5,0)(2.5,0)
\put(41.7,-85.01099){\fontsize{12}{1}\usefont{T1}{cmr}{m}{n}\selectfont\color{color_29791}           }
\end{picture}
\begin{tikzpicture}[overlay]
\path(0pt,0pt);
\draw[color_29791,line width=0.7pt]
(41.7pt, -81.61096pt) -- (77.2pt, -81.61096pt)
;
\end{tikzpicture}
\begin{picture}(-5,0)(2.5,0)
\put(77.2,-85.01099){\fontsize{12}{1}\usefont{T1}{cmr}{m}{n}\selectfont\color{color_29791}           }
\end{picture}
\begin{tikzpicture}[overlay]
\path(0pt,0pt);
\draw[color_29791,line width=0.7pt]
(77.2pt, -81.61096pt) -- (112.7pt, -81.61096pt)
;
\end{tikzpicture}
\begin{picture}(-5,0)(2.5,0)
\put(112.6,-85.01099){\fontsize{12}{1}\usefont{T1}{cmr}{m}{n}\selectfont\color{color_29791}           }
\end{picture}
\begin{tikzpicture}[overlay]
\path(0pt,0pt);
\draw[color_29791,line width=0.7pt]
(112.6pt, -81.61096pt) -- (148.1pt, -81.61096pt)
;
\end{tikzpicture}
\begin{picture}(-5,0)(2.5,0)
\put(148.1,-85.01099){\fontsize{12}{1}\usefont{T1}{cmr}{m}{n}\selectfont\color{color_29791}           }
\end{picture}
\begin{tikzpicture}[overlay]
\path(0pt,0pt);
\draw[color_29791,line width=0.7pt]
(148.1pt, -81.61096pt) -- (183.6pt, -81.61096pt)
;
\end{tikzpicture}
\begin{picture}(-5,0)(2.5,0)
\put(183.6,-85.01099){\fontsize{12}{1}\usefont{T1}{cmr}{m}{n}\selectfont\color{color_29791}; }
\end{picture}
\begin{tikzpicture}[overlay]
\path(0pt,0pt);
\draw[color_29791,line width=0.7pt]
(183.6pt, -81.61096pt) -- (189.9pt, -81.61096pt)
;
\end{tikzpicture}
\begin{picture}(-5,0)(2.5,0)
\put(190,-85.01099){\fontsize{12}{1}\usefont{T1}{cmr}{m}{it}\selectfont\color{color_29791}scp -F ssh.conf fileLocale default:/pathremoto }
\end{picture}
\begin{tikzpicture}[overlay]
\path(0pt,0pt);
\draw[color_29791,line width=0.7pt]
(190pt, -81.61096pt) -- (414.8pt, -81.61096pt)
;
\end{tikzpicture}
\begin{picture}(-5,0)(2.5,0)
\put(414.7,-85.01099){\fontsize{12}{1}\usefont{T1}{cmr}{m}{it}\selectfont\color{color_29791}     }
\end{picture}
\begin{tikzpicture}[overlay]
\path(0pt,0pt);
\draw[color_29791,line width=0.7pt]
(414.7pt, -81.61096pt) -- (431.7pt, -81.61096pt)
;
\end{tikzpicture}
\begin{picture}(-5,0)(2.5,0)
\put(431.8,-85.01099){\fontsize{12}{1}\usefont{T1}{cmr}{m}{n}\selectfont\color{color_29791}copia verso guest}
\end{picture}
\begin{tikzpicture}[overlay]
\path(0pt,0pt);
\draw[color_29791,line width=0.7pt]
(431.8pt, -81.61096pt) -- (515.1pt, -81.61096pt)
;
\end{tikzpicture}
\begin{picture}(-5,0)(2.5,0)
\put(41.8,-105.811){\fontsize{12}{1}\usefont{T1}{cmr}{m}{n}\selectfont\color{color_29791}Vagrant permette di modificare hostname della VM con direttiva }
\end{picture}
\begin{tikzpicture}[overlay]
\path(0pt,0pt);
\draw[color_29791,line width=0.7pt]
(41.8pt, -102.4109pt) -- (355.1pt, -102.4109pt)
;
\end{tikzpicture}
\begin{picture}(-5,0)(2.5,0)
\put(355.2,-105.811){\fontsize{12}{1}\usefont{T1}{cmr}{m}{it}\selectfont\color{color_29791}config.vm.hostname = "prova" , }
\end{picture}
\begin{tikzpicture}[overlay]
\path(0pt,0pt);
\draw[color_29791,line width=0.7pt]
(355.2pt, -102.4109pt) -- (511.9pt, -102.4109pt)
;
\end{tikzpicture}
\begin{picture}(-5,0)(2.5,0)
\put(41.8,-119.611){\fontsize{12}{1}\usefont{T1}{cmr}{m}{n}\selectfont\color{color_29791}inoltre permette di utilizzare linked-clone mediante direttiva }
\end{picture}
\begin{tikzpicture}[overlay]
\path(0pt,0pt);
\draw[color_29791,line width=0.7pt]
(41.8pt, -116.211pt) -- (334pt, -116.211pt)
;
\end{tikzpicture}
\begin{picture}(-5,0)(2.5,0)
\put(41.7,-140.411){\fontsize{12}{1}\usefont{T1}{cmr}{m}{n}\selectfont\color{color_29791}           }
\end{picture}
\begin{tikzpicture}[overlay]
\path(0pt,0pt);
\draw[color_29791,line width=0.7pt]
(41.7pt, -137.011pt) -- (77.2pt, -137.011pt)
;
\end{tikzpicture}
\begin{picture}(-5,0)(2.5,0)
\put(77.3,-140.411){\fontsize{12}{1}\usefont{T1}{cmr}{m}{it}\selectfont\color{color_29791}config.vm.provider "virtualbox" do |vb|}
\end{picture}
\begin{tikzpicture}[overlay]
\path(0pt,0pt);
\draw[color_29791,line width=0.7pt]
(77.3pt, -137.011pt) -- (266.5pt, -137.011pt)
;
\end{tikzpicture}
\begin{picture}(-5,0)(2.5,0)
\put(41.8,-161.211){\fontsize{12}{1}\usefont{T1}{cmr}{m}{it}\selectfont\color{color_29791}       }
\end{picture}
\begin{tikzpicture}[overlay]
\path(0pt,0pt);
\draw[color_29791,line width=0.7pt]
(41.8pt, -157.811pt) -- (62.8pt, -157.811pt)
;
\end{tikzpicture}
\begin{picture}(-5,0)(2.5,0)
\put(62.7,-161.211){\fontsize{12}{1}\usefont{T1}{cmr}{m}{it}\selectfont\color{color_29791}    }
\end{picture}
\begin{tikzpicture}[overlay]
\path(0pt,0pt);
\draw[color_29791,line width=0.7pt]
(62.7pt, -157.811pt) -- (77.2pt, -157.811pt)
;
\end{tikzpicture}
\begin{picture}(-5,0)(2.5,0)
\put(77.3,-161.211){\fontsize{12}{1}\usefont{T1}{cmr}{m}{it}\selectfont\color{color_29791}  \# Display the VirtualBox GUI when booting the machine}
\end{picture}
\begin{tikzpicture}[overlay]
\path(0pt,0pt);
\draw[color_29791,line width=0.7pt]
(77.3pt, -157.811pt) -- (355.7pt, -157.811pt)
;
\end{tikzpicture}
\begin{picture}(-5,0)(2.5,0)
\put(41.8,-182.011){\fontsize{12}{1}\usefont{T1}{cmr}{m}{it}\selectfont\color{color_29791}    }
\end{picture}
\begin{tikzpicture}[overlay]
\path(0pt,0pt);
\draw[color_29791,line width=0.7pt]
(41.8pt, -178.611pt) -- (53.8pt, -178.611pt)
;
\end{tikzpicture}
\begin{picture}(-5,0)(2.5,0)
\put(53.7,-182.011){\fontsize{12}{1}\usefont{T1}{cmr}{m}{it}\selectfont\color{color_29791}       }
\end{picture}
\begin{tikzpicture}[overlay]
\path(0pt,0pt);
\draw[color_29791,line width=0.7pt]
(53.7pt, -178.611pt) -- (77.2pt, -178.611pt)
;
\end{tikzpicture}
\begin{picture}(-5,0)(2.5,0)
\put(77.3,-182.011){\fontsize{12}{1}\usefont{T1}{cmr}{m}{it}\selectfont\color{color_29791}  vb.linked\_clone = true}
\end{picture}
\begin{tikzpicture}[overlay]
\path(0pt,0pt);
\draw[color_29791,line width=0.7pt]
(77.3pt, -178.611pt) -- (192.3pt, -178.611pt)
;
\end{tikzpicture}
\begin{picture}(-5,0)(2.5,0)
\put(41.7,-202.811){\fontsize{12}{1}\usefont{T1}{cmr}{m}{it}\selectfont\color{color_29791}           }
\end{picture}
\begin{tikzpicture}[overlay]
\path(0pt,0pt);
\draw[color_29791,line width=0.7pt]
(41.7pt, -199.4109pt) -- (77.2pt, -199.4109pt)
;
\end{tikzpicture}
\begin{picture}(-5,0)(2.5,0)
\put(77.3,-202.811){\fontsize{12}{1}\usefont{T1}{cmr}{m}{it}\selectfont\color{color_29791}end}
\end{picture}
\begin{tikzpicture}[overlay]
\path(0pt,0pt);
\draw[color_29791,line width=0.7pt]
(77.3pt, -199.4109pt) -- (94.60001pt, -199.4109pt)
;
\end{tikzpicture}
\begin{picture}(-5,0)(2.5,0)
\put(41.8,-259.611){\fontsize{17.5}{1}\usefont{T1}{cmr}{b}{n}\selectfont\color{color_29791}Gestione servizi (demoni)}
\put(41.8,-280.511){\fontsize{12}{1}\usefont{T1}{cmr}{m}{n}\selectfont\color{color_29791}/sbin/init è il processo che fa partire tutte le attività di servizio in background (di sistema), detti }
\put(41.8,-294.311){\fontsize{12}{1}\usefont{T1}{cmr}{m}{n}\selectfont\color{color_29791}demoni in Unix. Non ci si accorge della loro presenza essendo disconnessi da qualsiasi terminale e }
\put(41.8,-308.111){\fontsize{12}{1}\usefont{T1}{cmr}{m}{n}\selectfont\color{color_29791}eseguiti a nome di utenti specifici senza accesso a shell interattiva, gestiscono in maniera }
\put(41.8,-321.911){\fontsize{12}{1}\usefont{T1}{cmr}{m}{n}\selectfont\color{color_29791}automatica il sistema.}
\put(41.8,-342.711){\fontsize{12}{1}\usefont{T1}{cmr}{m}{n}\selectfont\color{color_29791}Per gestire correttamente i processi è utile sapere che origine hanno, come terminarli evitando che }
\put(41.8,-356.511){\fontsize{12}{1}\usefont{T1}{cmr}{m}{n}\selectfont\color{color_29791}ricompaiano: processi inutili consumano risorse e offrono opportunità di attacco. Ps, top e kill sono}
\put(41.8,-370.311){\fontsize{12}{1}\usefont{T1}{cmr}{m}{n}\selectfont\color{color_29791}efficaci per individuare e risolvere problemi istantanei ma non per evitare che riappaiano. Ci sono }
\put(41.8,-384.111){\fontsize{12}{1}\usefont{T1}{cmr}{m}{n}\selectfont\color{color_29791}tre fonti primarie di processi: pianificatori periodici e sporadici, demoni di gestione eventi, }
\put(41.8,-397.911){\fontsize{12}{1}\usefont{T1}{cmr}{m}{n}\selectfont\color{color_29791}procedure di avvio del sistema.}
\put(41.8,-418.711){\fontsize{12}{1}\usefont{T1}{cmr}{m}{n}\selectfont\color{color_29791}La più semplice delle 3 fonti è l’esecuzione pianificata, che può essere periodica o sporadica.}
\put(41.8,-439.511){\fontsize{12}{1}\usefont{T1}{cmr}{b}{it}\selectfont\color{color_29791}crond è il demone che si occupa dell’esecuzione periodica. }
\put(59.8,-460.311){\fontsize{12}{1}\usefont{T1}{cmr}{m}{n}\selectfont\color{color_29791}•Ogni utente ha la propria crontab (cron table), non importa sapere dove si trovano perché }
\put(77.8,-474.111){\fontsize{12}{1}\usefont{T1}{cmr}{m}{n}\selectfont\color{color_29791}ognuno può configurarle usando crontab -e che apre editor. Opzione -l elenca le task. Se }
\put(77.8,-487.911){\fontsize{12}{1}\usefont{T1}{cmr}{m}{n}\selectfont\color{color_29791}root lancia crontab -e edita il suo cron table, con -u edita quello di altri utenti. }
\put(59.8,-508.711){\fontsize{12}{1}\usefont{T1}{cmr}{m}{n}\selectfont\color{color_29791}•I task di sistema si trovano in /etc/crontab, questo editabile solo da root e solo direttamente; }
\put(77.8,-522.511){\fontsize{12}{1}\usefont{T1}{cmr}{m}{n}\selectfont\color{color_29791}a differenza di crontab contiene un campo in più che indica a nome di quale utente dovrà }
\put(77.8,-536.311){\fontsize{12}{1}\usefont{T1}{cmr}{m}{n}\selectfont\color{color_29791}essere eseguito un certo comando, perché se root vuole pianificare task per altro utente non }
\put(77.8,-550.111){\fontsize{12}{1}\usefont{T1}{cmr}{m}{n}\selectfont\color{color_29791}va a modificare il crontab di quell'utente (che potrebbe anche cambiarlo) ma modifica }
\put(77.8,-563.911){\fontsize{12}{1}\usefont{T1}{cmr}{m}{n}\selectfont\color{color_29791}/etc/crontab. Contiene già forme “cron.frequenza” standard agganciate a tutto ciò che trova }
\put(77.8,-577.711){\fontsize{12}{1}\usefont{T1}{cmr}{m}{n}\selectfont\color{color_29791}nelle directory /etc/cron.hourly daily weekly monthly, queste forme standard rendono più }
\put(77.8,-591.511){\fontsize{12}{1}\usefont{T1}{cmr}{m}{n}\selectfont\color{color_29791}facile settare task periodici senza far danni ad altri task del crontab generale.}
\put(59.8,-612.311){\fontsize{12}{1}\usefont{T1}{cmr}{m}{n}\selectfont\color{color_29791}•Ogni crontab ha un elenco di direttive nella forma }
\put(77.8,-633.111){\fontsize{12}{1}\usefont{T1}{cmr}{m}{n}\selectfont\color{color_29791}MINUTO ORA G.MESE MESE G.SETTIMANA <comando>}
\put(77.8,-653.911){\fontsize{12}{1}\usefont{T1}{cmr}{m}{n}\selectfont\color{color_29791}L’azione è eseguita quando l’ora corrente corrisponde a tutti i selettori di una riga (campi in }
\put(77.8,-667.711){\fontsize{12}{1}\usefont{T1}{cmr}{m}{n}\selectfont\color{color_29791}AND logico), tranne se c'è sia giorno mese che giorno settimana (in quel caso OR)}
\put(41.8,-688.511){\fontsize{12}{1}\usefont{T1}{cmr}{b}{it}\selectfont\color{color_29791}atd è il demone che si occupa di gestire code di compiti da svolgere in momenti prefissati, offre }
\put(41.8,-702.311){\fontsize{12}{1}\usefont{T1}{cmr}{m}{n}\selectfont\color{color_29791}scheduling differito di un servizio, con possibilità di disinnesco.}
\put(41.8,-723.111){\fontsize{12}{1}\usefont{T1}{cmr}{m}{n}\selectfont\color{color_29791}L’interfaccia ad atd si basa sui comandi }
\put(59.8,-743.911){\fontsize{12}{1}\usefont{T1}{cmr}{m}{n}\selectfont\color{color_29791}•at [-V] [-q queue] [-f file] [-mldbv] TIME prende come parametro quando fare un job, }
\put(77.8,-757.711){\fontsize{12}{1}\usefont{T1}{cmr}{m}{n}\selectfont\color{color_29791}quando si fa invio poi mostra altro prompt, perché si aspetta l'inserimento dei comandi da }
\end{picture}
\newpage
\begin{tikzpicture}[overlay]\path(0pt,0pt);\end{tikzpicture}
\begin{picture}(-5,0)(2.5,0)
\put(77.8,-85.01099){\fontsize{12}{1}\usefont{T1}{cmr}{m}{n}\selectfont\color{color_29791}eseguire POI sullo stdin. at non è silente, restituisce il numero del job. Questo si può usare }
\put(77.8,-98.81097){\fontsize{12}{1}\usefont{T1}{cmr}{m}{n}\selectfont\color{color_29791}con atrm per rimuoverlo. }
\put(59.8,-119.611){\fontsize{12}{1}\usefont{T1}{cmr}{m}{n}\selectfont\color{color_29791}•atq [-V] [-q queue] [-v] elenca i comandi in coda, formato }
\put(77.8,-140.411){\fontsize{12}{1}\usefont{T1}{cmr}{m}{it}\selectfont\color{color_29791}9Sat Apr  2 16:04:00 2022 a vagrant}
\put(59.8,-161.211){\fontsize{12}{1}\usefont{T1}{cmr}{m}{n}\selectfont\color{color_29791}•atrm [-V] job [job…] rimuove comandi dalla coda.}
\put(59.8,-182.011){\fontsize{12}{1}\usefont{T1}{cmr}{m}{n}\selectfont\color{color_29791}•AT E CRON SONO DEMONI, gli stream vanno obbligatoriamente ridiretti, se non ci }
\put(77.8,-195.811){\fontsize{12}{1}\usefont{T1}{cmr}{m}{n}\selectfont\color{color_29791}interessano si ridirigono a /dev/null. Se non li ridirigiamo non sappiamo che fine fanno, }
\put(77.8,-209.611){\fontsize{12}{1}\usefont{T1}{cmr}{m}{n}\selectfont\color{color_29791}IMPORTANTE usare path assoluti dentro file crontab o dentro comando at perché non è }
\put(77.8,-223.411){\fontsize{12}{1}\usefont{T1}{cmr}{m}{n}\selectfont\color{color_29791}detto che abbiano lo stesso environment della shell interattiva.}
\put(77.8,-244.211){\fontsize{12}{1}\usefont{T1}{cmr}{m}{n}\selectfont\color{color_258292}◦PATHCOMANDO="\$(readlink -f "\$\{BASH\_SOURCE\}")"salva nella variabile il }
\put(95.8,-258.011){\fontsize{12}{1}\usefont{T1}{cmr}{m}{n}\selectfont\color{color_258292}percorso assoluto dello script in cui è contenuto}
\put(41.8,-278.811){\fontsize{12}{1}\usefont{T1}{cmr}{m}{n}\selectfont\color{color_29791}Esempio esercizio: }
\put(77.3,-299.611){\fontsize{12}{1}\usefont{T1}{cmr}{b}{it}\selectfont\color{color_29791}ATJOB=\$(mktemp)preparo file temp per salvare il job id}
\put(77.3,-320.411){\fontsize{12}{1}\usefont{T1}{cmr}{b}{it}\selectfont\color{color_29791}echo "/bin/date > /tmp/prova2.las" | at now + 30 minutes 2>\&1 | grep '\^job ' | cut -f2 -d' ' }
\put(41.8,-334.211){\fontsize{12}{1}\usefont{T1}{cmr}{b}{it}\selectfont\color{color_29791}>"\$ATJOB" linea perfettamente non interattiva adatta per script:}
\put(77.3,-355.011){\fontsize{12}{1}\usefont{T1}{cmr}{m}{n}\selectfont\color{color_29791}con echo diamo input ad at per sapere cosa fare, con grep prendiamo la linea che ci }
\put(41.8,-368.811){\fontsize{12}{1}\usefont{T1}{cmr}{m}{n}\selectfont\color{color_29791}interessa, con cut prendiamo il numero che ci interessa, con redirezione finale scriviamo sul file.}
\put(77.3,-389.611){\fontsize{12}{1}\usefont{T1}{cmr}{m}{n}\selectfont\color{color_29791}poi con cat \$ATJOB possiamo recuperare il jobid (e possiamo usare atrm \$(cat }
\put(41.8,-403.411){\fontsize{12}{1}\usefont{T1}{cmr}{b}{it}\selectfont\color{color_29791}\$ATJOB)  per de-schedulare prima dell'esecuzione)}
\put(41.8,-424.211){\fontsize{12}{1}\usefont{T1}{cmr}{m}{n}\selectfont\color{color_29791}Esempio uso watchdog:esempio applicato sta nel book Gestione di servizi e }
\put(41.8,-438.011){\fontsize{12}{1}\usefont{T1}{cmr}{m}{n}\selectfont\color{color_29791}monitoraggio}
\put(41.8,-458.811){\fontsize{12}{1}\usefont{T1}{cmr}{b}{n}\selectfont\color{color_29791}Watchdog: Utile se si vuole porre un limite al tempo di esecuzione di un task, senza ricorrere a una }
\put(41.8,-472.611){\fontsize{12}{1}\usefont{T1}{cmr}{m}{n}\selectfont\color{color_29791}busy wait }
\put(112.7,-493.411){\fontsize{12}{1}\usefont{T1}{cmr}{m}{it}\selectfont\color{color_29791}start\_long\_task \& PID=\$!}
\put(112.7,-514.211){\fontsize{12}{1}\usefont{T1}{cmr}{m}{n}\selectfont\color{color_29791}WD=\$(mktemp)}
\put(112.7,-535.011){\fontsize{12}{1}\usefont{T1}{cmr}{m}{n}\selectfont\color{color_29791}echo "/bin/kill \$PID" | at now + \$LIMITE minutes 2>\&1 | grep '\^job ' | cut -f2 -d' ' > }
\put(41.8,-548.811){\fontsize{12}{1}\usefont{T1}{cmr}{m}{n}\selectfont\color{color_29791}\$WD}
\put(41.8,-569.611){\fontsize{12}{1}\usefont{T1}{cmr}{m}{n}\selectfont\color{color_29791}Opportuno fare un array dove i PID sono indici e i valori sono i nomi dei comandi, così quando è il }
\put(41.8,-583.411){\fontsize{12}{1}\usefont{T1}{cmr}{m}{n}\selectfont\color{color_29791}momento di lanciare kill, si può testare se al pid è ancora associato quel particolare comando. (Vedi }
\put(41.8,-597.211){\fontsize{12}{1}\usefont{T1}{cmr}{m}{n}\selectfont\color{color_29791}parallenne.sh)}
\put(41.8,-638.811){\fontsize{12}{1}\usefont{T1}{cmr}{m}{n}\selectfont\color{color_29791}Init è il primo processo avviato dal kernel, si occupa di gestire le sequenze di eventi per raggiungere}
\put(41.8,-652.611){\fontsize{12}{1}\usefont{T1}{cmr}{m}{n}\selectfont\color{color_29791}un dato runlevel: questi sono degli stati definiti del sistema, in cui sono stati sicuramente avviati }
\put(41.8,-666.411){\fontsize{12}{1}\usefont{T1}{cmr}{m}{n}\selectfont\color{color_29791}alcuni processi e sicuramente fermati altri. La variante di interesse per noi è Systemd. (dettagli su }
\put(41.8,-680.211){\fontsize{12}{1}\usefont{T1}{cmr}{m}{n}\selectfont\color{color_29791}sysvinit e upstart, nel pdf 3\_gestione\_servizi, c'è anche cheat-sheet per comandi)}
\put(41.8,-710.011){\fontsize{14.1}{1}\usefont{T1}{cmr}{b}{n}\selectfont\color{color_29791}Systemd}
\put(41.8,-730.211){\fontsize{12}{1}\usefont{T1}{cmr}{b}{n}\selectfont\color{color_29791}Systemd è il demone che si occupa di }
\put(59.8,-751.011){\fontsize{12}{1}\usefont{T1}{cmr}{m}{n}\selectfont\color{color_29791}•gestire le dipendenze tra i servizi: ha senso avviare alcuni servizi soltanto se sono già avviati}
\put(77.8,-764.811){\fontsize{12}{1}\usefont{T1}{cmr}{m}{n}\selectfont\color{color_29791}i servizi su cui si basano (es. non ha senso chiedere login a utenti se non si è riusciti a }
\end{picture}
\newpage
\begin{tikzpicture}[overlay]\path(0pt,0pt);\end{tikzpicture}
\begin{picture}(-5,0)(2.5,0)
\put(77.8,-85.01099){\fontsize{12}{1}\usefont{T1}{cmr}{m}{n}\selectfont\color{color_29791}montare la loro home) così come quando si installa software ha senso installare un }
\put(77.8,-98.81097){\fontsize{12}{1}\usefont{T1}{cmr}{m}{n}\selectfont\color{color_29791}pacchettoB solo se esiste già pacchettoA da cui dipende}
\put(59.8,-119.611){\fontsize{12}{1}\usefont{T1}{cmr}{m}{n}\selectfont\color{color_29791}•avviare servizi a richiesta: se in una sequenza di passi c’è dipendenza ad es. del passo 5 dal }
\put(77.8,-133.411){\fontsize{12}{1}\usefont{T1}{cmr}{m}{n}\selectfont\color{color_29791}4, ma non dei passi dal 6 al 18; ho la possibilità di lanciare questi in parallelo invece di }
\put(77.8,-147.211){\fontsize{12}{1}\usefont{T1}{cmr}{m}{n}\selectfont\color{color_29791}attendere la fine del 5}
\put(59.8,-168.011){\fontsize{12}{1}\usefont{T1}{cmr}{m}{n}\selectfont\color{color_29791}•logging precoce: il sistema di log è un demone, ma anche se lo lancio presto, tutto ciò che è }
\put(77.8,-181.811){\fontsize{12}{1}\usefont{T1}{cmr}{m}{n}\selectfont\color{color_29791}partito prima di lui non può essere registrato. systemd può registrare diagnostica di qualsiasi }
\put(77.8,-195.611){\fontsize{12}{1}\usefont{T1}{cmr}{m}{n}\selectfont\color{color_29791}servizio anche se non è ancora partito il log di quel particolare servizio}
\put(41.8,-216.411){\fontsize{12}{1}\usefont{T1}{cmr}{m}{n}\selectfont\color{color_29791}Si propone di sostituire molti moduli (init, udev, crond/atd...) ma in realtà, come da modello UNIX, }
\put(41.8,-230.211){\fontsize{12}{1}\usefont{T1}{cmr}{m}{n}\selectfont\color{color_29791}è un framework che delega compiti a sottocomponenti. Questi sono detti control unit. I cui nomi }
\put(41.8,-244.011){\fontsize{12}{1}\usefont{T1}{cmr}{m}{n}\selectfont\color{color_29791}seguono la convenzione di name.type. type indica lo scopo dell’unit, alcune unit gestiscono i }
\put(41.8,-257.811){\fontsize{12}{1}\usefont{T1}{cmr}{m}{n}\selectfont\color{color_29791}demoni (Service), altre le comunicazioni tra processi (Socket) , altre i target che sostituiscono il }
\put(41.8,-271.611){\fontsize{12}{1}\usefont{T1}{cmr}{m}{n}\selectfont\color{color_29791}concetto di runlevel (Target). }
\put(41.8,-292.411){\fontsize{12}{1}\usefont{T1}{cmr}{m}{n}\selectfont\color{color_29791}Le unit sono file di testo, quelli tipici di riferimento (anche da copiare) sono in /lib/systemd/system/.}
\put(41.8,-306.211){\fontsize{12}{1}\usefont{T1}{cmr}{m}{n}\selectfont\color{color_29791}Quando si installano pacchetti, vengono forniti anche i file necessari alla configurazione dei }
\put(41.8,-320.011){\fontsize{12}{1}\usefont{T1}{cmr}{m}{n}\selectfont\color{color_29791}demoni, posti in /usr/lib/systemd/system/ (tipicamente sono link a quelli di riferimento). I file con le}
\put(41.8,-333.811){\fontsize{12}{1}\usefont{T1}{cmr}{m}{n}\selectfont\color{color_29791}personalizzazioni sono in /etc/systemd/system e sono sempre prioritari rispetto alle definizioni di }
\put(41.8,-347.611){\fontsize{12}{1}\usefont{T1}{cmr}{m}{n}\selectfont\color{color_29791}sistema.}
\put(41.8,-368.411){\fontsize{12}{1}\usefont{T1}{cmr}{b}{n}\selectfont\color{color_29791}systemctl è il comando con cui si controlla il funzionamento del sistema (interfaccia a systemd). }
\put(41.8,-382.211){\fontsize{12}{1}\usefont{T1}{cmr}{m}{n}\selectfont\color{color_29791}Nasce per gestire gerarchie di servizi rispettando le dipendenze tra uno e l'altro. Segue formato }
\put(41.8,-396.011){\fontsize{12}{1}\usefont{T1}{cmr}{b}{it}\selectfont\color{color_29791}systemctl COMANDO nomeservizio COMANDO è \{start|stop|status|restart|reload\}}
\put(41.8,-416.811){\fontsize{12}{1}\usefont{T1}{cmr}{m}{n}\selectfont\color{color_29791}con opzione -H [hostname] si connette a host remoto via ssh}
\put(41.8,-437.611){\fontsize{12}{1}\usefont{T1}{cmr}{m}{n}\selectfont\color{color_29791}I comandi da eseguire per ogni opzione sono definiti nelle unit con parametri (es. ExecStart), in }
\put(41.8,-451.411){\fontsize{12}{1}\usefont{T1}{cmr}{m}{n}\selectfont\color{color_29791}generale systemd tiene traccia dei processi avviati con start, in modo che i loro PID possano essere }
\put(41.8,-465.211){\fontsize{12}{1}\usefont{T1}{cmr}{m}{n}\selectfont\color{color_29791}usati come parametri nei comandi reload/stop. Inviare un segnale per fermare un processo in modo }
\put(41.8,-479.011){\fontsize{12}{1}\usefont{T1}{cmr}{m}{n}\selectfont\color{color_29791}diretto generalmente è una cattiva idea, si usa systemctl stop per avere una corretta gestione }
\put(41.8,-492.811){\fontsize{12}{1}\usefont{T1}{cmr}{m}{n}\selectfont\color{color_29791}(SIGTERM, poi SIGKILL dopo timeout: per demoni definiti da noi, è opportuno implementare }
\put(41.8,-506.611){\fontsize{12}{1}\usefont{T1}{cmr}{m}{n}\selectfont\color{color_29791}handler per SIGTERM per garantire gestione di systemctl stop).}
\put(41.8,-527.411){\fontsize{12}{1}\usefont{T1}{cmr}{m}{n}\selectfont\color{color_29791}Le operazioni di start, stop, status, restart sono volatili, ovvero non cambiano nulla nella }
\put(41.8,-541.211){\fontsize{12}{1}\usefont{T1}{cmr}{m}{n}\selectfont\color{color_29791}configurazione del sistema. I comandi di \{enable|disable|mask|unmask\} viceversa hanno effetto }
\put(41.8,-555.011){\fontsize{12}{1}\usefont{T1}{cmr}{m}{n}\selectfont\color{color_29791}persistente sulla configurazione e nessun effetto immediato, si usano per automatizzare avvio al }
\put(41.8,-568.811){\fontsize{12}{1}\usefont{T1}{cmr}{m}{n}\selectfont\color{color_29791}boot e arresto allo shutdown di servizi. Se un processo dà fastidio serve sia stop per fermarlo subito }
\put(41.8,-582.611){\fontsize{12}{1}\usefont{T1}{cmr}{m}{n}\selectfont\color{color_29791}che disable per evitare che riparta. disable lascia disponibile la possibilità di usare manualmente }
\put(41.8,-596.411){\fontsize{12}{1}\usefont{T1}{cmr}{m}{it}\selectfont\color{color_29791}start sul servizio, mask neutralizza l’intera definizione della unit, impedendo anche il controllo }
\put(41.8,-610.211){\fontsize{12}{1}\usefont{T1}{cmr}{m}{n}\selectfont\color{color_29791}manuale. Qualche esempio di altri comandi:}
\put(59.8,-631.011){\fontsize{12}{1}\usefont{T1}{cmr}{m}{n}\selectfont\color{color_29791}•systemctl list-units mostra tutte le unit gestite di tutti i tipi elencati}
\put(59.8,-651.811){\fontsize{12}{1}\usefont{T1}{cmr}{m}{n}\selectfont\color{color_29791}•systemctl -t TIPO mostra tutte le unit attive}
\end{picture}
\begin{tikzpicture}[overlay]
\path(0pt,0pt);
\draw[color_29791,line width=0.7pt]
(277.5pt, -652.911pt) -- (304.1pt, -652.911pt)
;
\end{tikzpicture}
\begin{picture}(-5,0)(2.5,0)
\put(304.1,-651.811){\fontsize{12}{1}\usefont{T1}{cmr}{m}{n}\selectfont\color{color_29791} del tipo specificato (es. systemctl -t timers)}
\put(59.8,-672.611){\fontsize{12}{1}\usefont{T1}{cmr}{m}{n}\selectfont\color{color_29791}•systemctl list-unit-files [-t TIPO] mostra tutte le unit installate}
\end{picture}
\begin{tikzpicture}[overlay]
\path(0pt,0pt);
\draw[color_29791,line width=0.7pt]
(348.4pt, -673.711pt) -- (391.7pt, -673.711pt)
;
\end{tikzpicture}
\begin{picture}(-5,0)(2.5,0)
\put(391.7,-672.611){\fontsize{12}{1}\usefont{T1}{cmr}{m}{n}\selectfont\color{color_29791} del tipo specificato }
\put(77.8,-686.411){\fontsize{12}{1}\usefont{T1}{cmr}{m}{n}\selectfont\color{color_29791}(esempio utile systemctl list-unit-files -t services tutti i servizi installati; aggiungendo --}
\put(77.8,-700.211){\fontsize{12}{1}\usefont{T1}{cmr}{b}{it}\selectfont\color{color_29791}state=enabled elenca i servizi che partono al boot)}
\put(59.8,-721.011){\fontsize{12}{1}\usefont{T1}{cmr}{m}{n}\selectfont\color{color_29791}•systemctl --state STATOmostra tutte le unit che si trovano nello stato specificato }
\put(77.8,-734.811){\fontsize{12}{1}\usefont{T1}{cmr}{m}{n}\selectfont\color{color_29791}(esempio utile systemctl –state failed tutte le unit che systemctl ha cercato di avviare senza }
\put(77.8,-748.611){\fontsize{12}{1}\usefont{T1}{cmr}{m}{n}\selectfont\color{color_29791}riuscire)}
\end{picture}
\newpage
\begin{tikzpicture}[overlay]\path(0pt,0pt);\end{tikzpicture}
\begin{picture}(-5,0)(2.5,0)
\put(41.8,-85.01099){\fontsize{12}{1}\usefont{T1}{cmr}{m}{n}\selectfont\color{color_29791}Con systemd i runlevel sono rimpiazzati dai target, systemctl get-default restituisce il target di }
\put(41.8,-98.81097){\fontsize{12}{1}\usefont{T1}{cmr}{m}{n}\selectfont\color{color_29791}default (systemctl set-default [target] per modificarlo). I target gestiscono le dipendenze tra le unità,}
\put(41.8,-112.611){\fontsize{12}{1}\usefont{T1}{cmr}{m}{n}\selectfont\color{color_29791}queste partono appena sono soddisfatti i vincoli espressi dalle direttive. Una unit può avere direttive}
\put(41.8,-126.411){\fontsize{12}{1}\usefont{T1}{cmr}{m}{n}\selectfont\color{color_29791}del tipo:}
\put(59.8,-147.211){\fontsize{12}{1}\usefont{T1}{cmr}{m}{n}\selectfont\color{color_29791}•Requires elenco di altre unit di cui questa necessita: se l’avvio di queste fallisce, questa}
\put(77.8,-161.011){\fontsize{12}{1}\usefont{T1}{cmr}{m}{n}\selectfont\color{color_29791}viene arrestata (configurabile relazione temporale dopo, prima, simultaneamente)}
\put(59.8,-181.811){\fontsize{12}{1}\usefont{T1}{cmr}{m}{n}\selectfont\color{color_29791}•Wantsdipendenze di questa meno strette di Requires, viene tentato l’avvio delle }
\put(77.8,-195.611){\fontsize{12}{1}\usefont{T1}{cmr}{m}{n}\selectfont\color{color_29791}altre unit ma se fallisce, questa viene avviata comunque}
\put(59.8,-216.411){\fontsize{12}{1}\usefont{T1}{cmr}{m}{n}\selectfont\color{color_29791}•Conflicts vincolo negativo per rendere unit mutualmente esclusive}
\put(59.8,-237.211){\fontsize{12}{1}\usefont{T1}{cmr}{m}{n}\selectfont\color{color_29791}•OnFailure unit da avviare quando questa fallisce}
\put(59.8,-258.011){\fontsize{12}{1}\usefont{T1}{cmr}{m}{n}\selectfont\color{color_29791}•RequiredBy / WantedByspecifica dipendenza inversa: quando si installa questa unit }
\put(77.8,-271.811){\fontsize{12}{1}\usefont{T1}{cmr}{m}{n}\selectfont\color{color_29791}vengono informate le altre che hanno bisogno di lei che esiste; aggiungendo }
\put(77.8,-285.611){\fontsize{12}{1}\usefont{T1}{cmr}{m}{n}\selectfont\color{color_29791}automaticamente entry Requires / Wants }
\put(59.8,-306.411){\fontsize{12}{1}\usefont{T1}{cmr}{m}{n}\selectfont\color{color_29791}•Restart riavvia il servizio in caso di terminazione; è l’equivalente del respawn di init }
\put(77.8,-320.211){\fontsize{12}{1}\usefont{T1}{cmr}{m}{n}\selectfont\color{color_29791}che si occupava di garantire il riavvio dei servizi fondamentali (es. servizio per gestione }
\put(77.8,-334.011){\fontsize{12}{1}\usefont{T1}{cmr}{m}{n}\selectfont\color{color_29791}terminale)}
\put(41.8,-354.811){\fontsize{12}{1}\usefont{T1}{cmr}{m}{n}\selectfont\color{color_29791}Alcune unit speciali (principalmente target) hanno nomi fissi e funzioni fondamentali, usate come }
\put(41.8,-368.611){\fontsize{12}{1}\usefont{T1}{cmr}{m}{n}\selectfont\color{color_29791}punti di controllo nella sequenza di boot (vedi systemd.special(7) e bootup(7)).}
\put(41.8,-398.411){\fontsize{14.1}{1}\usefont{T1}{cmr}{b}{n}\selectfont\color{color_29791}Scrittura di unit file}
\put(41.8,-418.611){\fontsize{12}{1}\usefont{T1}{cmr}{m}{n}\selectfont\color{color_29791}Volendo scrivere un unit file, le configurazioni sono molte (vedi man 5 systemd.service o }
\put(41.8,-432.411){\fontsize{12}{1}\usefont{T1}{cmr}{m}{n}\selectfont\color{color_29919}https://www.freedesktop.org/software/systemd/man/systemd.service.html}
\end{picture}
\begin{tikzpicture}[overlay]
\path(0pt,0pt);
\draw[color_29919,line width=0.7pt]
(41.8pt, -433.511pt) -- (394pt, -433.511pt)
;
\end{tikzpicture}
\begin{picture}(-5,0)(2.5,0)
\put(394.1,-432.411){\fontsize{12}{1}\usefont{T1}{cmr}{m}{n}\selectfont\color{color_29791} ) ma gli elementi davvero }
\put(41.8,-446.211){\fontsize{12}{1}\usefont{T1}{cmr}{m}{n}\selectfont\color{color_29791}indispensabili sono: }
\put(59.8,-467.011){\fontsize{12}{1}\usefont{T1}{cmr}{m}{n}\selectfont\color{color_29791}•[Unit]}
\put(77.8,-487.811){\fontsize{12}{1}\usefont{T1}{cmr}{m}{n}\selectfont\color{color_29791}◦Description= Descrizione}
\put(77.8,-508.611){\fontsize{12}{1}\usefont{T1}{cmr}{m}{n}\selectfont\color{color_29791}◦Requires/Wants= da chi dipende questa unit}
\put(77.8,-529.411){\fontsize{12}{1}\usefont{T1}{cmr}{m}{n}\selectfont\color{color_29791}◦Documentation= inserire indicazione su dove trovare documentazione, agevola la }
\put(95.8,-543.211){\fontsize{12}{1}\usefont{T1}{cmr}{m}{n}\selectfont\color{color_29791}vita al sistemista}
\put(59.8,-564.011){\fontsize{12}{1}\usefont{T1}{cmr}{m}{n}\selectfont\color{color_29791}•[Service]}
\put(77.8,-584.811){\fontsize{12}{1}\usefont{T1}{cmr}{m}{n}\selectfont\color{color_29791}◦Type= tipo di avvio}
\put(77.8,-605.611){\fontsize{12}{1}\usefont{T1}{cmr}{m}{n}\selectfont\color{color_29791}◦ExecStart= comando da lanciare all’avvio }
\put(77.8,-626.411){\fontsize{12}{1}\usefont{T1}{cmr}{m}{n}\selectfont\color{color_29791}◦Exec[Stop/Reload]= opzionali, comandi per stop e reload}
\put(77.8,-647.211){\fontsize{12}{1}\usefont{T1}{cmr}{m}{n}\selectfont\color{color_29791}◦Restart= opzionale, reazione alla terminazione}
\put(59.8,-668.011){\fontsize{12}{1}\usefont{T1}{cmr}{m}{n}\selectfont\color{color_29791}•[Install]}
\put(77.8,-688.811){\fontsize{12}{1}\usefont{T1}{cmr}{m}{n}\selectfont\color{color_29791}◦WantedBy= chi dipende da questa unit}
\put(77.8,-709.611){\fontsize{12}{1}\usefont{T1}{cmr}{m}{n}\selectfont\color{color_29791}◦Alias= nome con cui la unit è nota a sistemd}
\put(41.8,-730.411){\fontsize{12}{1}\usefont{T1}{cmr}{m}{n}\selectfont\color{color_29791}Le dipendenze vanno risolte direttamente in fase di design degli unit file; ad esempio se A non solo }
\put(41.8,-744.211){\fontsize{12}{1}\usefont{T1}{cmr}{m}{n}\selectfont\color{color_29791}ha bisogno di B, ma deve anche attendere che sia partito, si aggiunge Requires=B e After=B alla }
\put(41.8,-758.011){\fontsize{12}{1}\usefont{T1}{cmr}{m}{n}\selectfont\color{color_29791}sezione [Unit] di A (senza vincolo After, la semantica sarà la partenza in parallelo delle Unit con }
\end{picture}
\newpage
\begin{tikzpicture}[overlay]\path(0pt,0pt);\end{tikzpicture}
\begin{picture}(-5,0)(2.5,0)
\put(41.8,-85.01099){\fontsize{12}{1}\usefont{T1}{cmr}{m}{n}\selectfont\color{color_29791}verifica a posteriori del vincolo). Le dipendenze sono tipicamente inserite nei servizi e non nei }
\put(41.8,-98.81097){\fontsize{12}{1}\usefont{T1}{cmr}{m}{n}\selectfont\color{color_29791}Target.}
\put(41.8,-119.611){\fontsize{12}{1}\usefont{T1}{cmr}{m}{n}\selectfont\color{color_29791}Il parametro Type nella sezione [Service] specifica il tipo di start-up da considerare. }
\put(59.8,-140.411){\fontsize{12}{1}\usefont{T1}{cmr}{m}{n}\selectfont\color{color_29791}•Type=simple default, systemd considera il servizio avviato con successo appena ha }
\put(77.8,-154.211){\fontsize{12}{1}\usefont{T1}{cmr}{m}{n}\selectfont\color{color_29791}forkato un figlio per eseguire il comando ExecStart (se la exec fallisce, non ce ne }
\put(77.8,-168.011){\fontsize{12}{1}\usefont{T1}{cmr}{m}{n}\selectfont\color{color_29791}accorgiamo)}
\put(77.8,-188.811){\fontsize{12}{1}\usefont{T1}{cmr}{m}{n}\selectfont\color{color_29791}◦Non è adatto ad un servizio che deve ordinarne altri}
\put(59.8,-209.611){\fontsize{12}{1}\usefont{T1}{cmr}{m}{n}\selectfont\color{color_29791}•Type=forking fa eseguire un processo in background, con un pid che si può catturare}
\put(77.8,-223.411){\fontsize{12}{1}\usefont{T1}{cmr}{m}{n}\selectfont\color{color_29791}e mettere in un file.  Ricordiamo che quando si forka un figlio, il pid viene messo da parte }
\put(77.8,-237.211){\fontsize{12}{1}\usefont{T1}{cmr}{m}{n}\selectfont\color{color_29791}perché si possa recuperare senza usare ps cercandolo con il nome, che potrebbe anche essere}
\put(77.8,-251.011){\fontsize{12}{1}\usefont{T1}{cmr}{m}{n}\selectfont\color{color_29791}ambiguo}
\put(77.8,-271.811){\fontsize{12}{1}\usefont{T1}{cmr}{m}{n}\selectfont\color{color_29791}◦systemd considera il servizio partito quando il processo avviato con ExecStart esegue }
\put(95.8,-285.611){\fontsize{12}{1}\usefont{T1}{cmr}{m}{n}\selectfont\color{color_29791}una sua fork e il genitore esce). Tipicamente usato per riutilizzare un classico demone }
\put(95.8,-299.411){\fontsize{12}{1}\usefont{T1}{cmr}{m}{n}\selectfont\color{color_29791}UNIX, utile opzione PIDFile= per tracciare il processo principale.}
\put(59.8,-320.211){\fontsize{12}{1}\usefont{T1}{cmr}{m}{n}\selectfont\color{color_29791}•Type=oneshot per lanciare un processo una volta e poi uscire, es. gestione interfaccia}
\put(77.8,-334.011){\fontsize{12}{1}\usefont{T1}{cmr}{m}{n}\selectfont\color{color_29791}di rete fa ciò che deve una sola volta e poi esce.}
\put(77.8,-354.811){\fontsize{12}{1}\usefont{T1}{cmr}{m}{n}\selectfont\color{color_29791}◦Si può settare RemainAfterExit=yes così che systemd possa considerare il servizio come }
\put(95.8,-368.611){\fontsize{12}{1}\usefont{T1}{cmr}{m}{n}\selectfont\color{color_29791}attivo dopo la sua uscita.}
\put(41.8,-389.411){\fontsize{12}{1}\usefont{T1}{cmr}{m}{n}\selectfont\color{color_29791}Tutti i servizi usano ExecStart per lanciare un comando, se è una riga di shell andrà eseguita con sh }
\put(41.8,-403.211){\fontsize{12}{1}\usefont{T1}{cmr}{m}{it}\selectfont\color{color_29791}-c 'comandi'. Per supportare il comportamento di reload (sarebbe diverso da restart che è stop+start)}
\put(41.8,-417.011){\fontsize{12}{1}\usefont{T1}{cmr}{m}{n}\selectfont\color{color_29791}è necessario specificare ExecReload per gestire il segnale lanciato da systemd. Nella configurazione}
\put(41.8,-430.811){\fontsize{12}{1}\usefont{T1}{cmr}{m}{n}\selectfont\color{color_29791}avremo ad esempio ExecReload=kill -HUP \$MAINPID}
\put(41.8,-451.611){\fontsize{12}{1}\usefont{T1}{cmr}{m}{n}\selectfont\color{color_29791}Per un servizio persistente (tipo simple) è comune non specificare ExecStop. I servizi di tipo }
\put(41.8,-465.411){\fontsize{12}{1}\usefont{T1}{cmr}{m}{n}\selectfont\color{color_29791}oneshot (es. gestione interfaccia di rete) tipicamente quando avviati configurano qualche aspetto del}
\put(41.8,-479.211){\fontsize{12}{1}\usefont{T1}{cmr}{m}{n}\selectfont\color{color_29791}sistema, e quando arrestati ripristinano lo stato precedente. Avremo quindi ad esempio }
\put(41.8,-493.011){\fontsize{12}{1}\usefont{T1}{cmr}{b}{it}\selectfont\color{color_29791}ExecStart=/usr/bin/mio-configuratore ExecStop=/usr/bin/mio-de-configuratore}
\put(41.8,-513.811){\fontsize{12}{1}\usefont{T1}{cmr}{m}{n}\selectfont\color{color_29791}Con opzione Restart si può delegare systemd a monitorare il processo avviato. Se il processo }
\put(41.8,-527.611){\fontsize{12}{1}\usefont{T1}{cmr}{m}{n}\selectfont\color{color_29791}termina per cause diverse da systemctl (diverse da systemctl stop altrimenti loop infinito, è }
\put(41.8,-541.411){\fontsize{12}{1}\usefont{T1}{cmr}{m}{n}\selectfont\color{color_29791}necessario un watchdog per fare questo tipo di controllo), systemd lo riavvierà o meno a seconda }
\put(41.8,-555.211){\fontsize{12}{1}\usefont{T1}{cmr}{m}{n}\selectfont\color{color_29791}della combinazione tra il valore di Restart e la causa di malfunzionamento rilevata. I timeout dei }
\put(41.8,-569.011){\fontsize{12}{1}\usefont{T1}{cmr}{m}{n}\selectfont\color{color_29791}watchdog vanno gestiti separatamente con le loro unit (un watchdog è un processo indipendente che}
\put(41.8,-582.811){\fontsize{12}{1}\usefont{T1}{cmr}{m}{n}\selectfont\color{color_29791}ogni tanto va a controllare se un processo risponde). }
\put(41.8,-603.611){\fontsize{12}{1}\usefont{T1}{cmr}{m}{n}\selectfont\color{color_29791}Le configurazioni default delle unit di sistema (si trovano in /lib/systemd/system) sono molto utili, }
\put(41.8,-617.411){\fontsize{12}{1}\usefont{T1}{cmr}{m}{n}\selectfont\color{color_29791}anche da copiare quando dovremo realizzare servizi che si comportano in un certo modo. Esempio }
\put(41.8,-631.211){\fontsize{12}{1}\usefont{T1}{cmr}{m}{n}\selectfont\color{color_29791}oneshot: lancia comandi di configurazione con start e termina ripristinando lo stato precedente con }
\put(41.8,-645.011){\fontsize{12}{1}\usefont{T1}{cmr}{m}{n}\selectfont\color{color_29791}stop, forking: fare qualcosa e restare in background}
\put(59.8,-665.811){\fontsize{12}{1}\usefont{T1}{cmr}{m}{n}\selectfont\color{color_29791}•oneshot:/lib/systemd/system/networking.service}
\put(59.8,-686.611){\fontsize{12}{1}\usefont{T1}{cmr}{m}{n}\selectfont\color{color_29791}•forking:/lib/systemd/system/dnsmasq.service}
\put(59.8,-707.411){\fontsize{12}{1}\usefont{T1}{cmr}{m}{n}\selectfont\color{color_29791}•con opzione Restart:}
\put(77.8,-728.211){\fontsize{12}{1}\usefont{T1}{cmr}{m}{n}\selectfont\color{color_29791}◦/lib/systemd/system/atd.servicenon ha type perché è di tipo simple, viene lanciato e}
\put(95.8,-742.011){\fontsize{12}{1}\usefont{T1}{cmr}{m}{n}\selectfont\color{color_29791}sorvegliato automaticamente: basta fare il file service e poi fare enable perché sia }
\put(95.8,-755.811){\fontsize{12}{1}\usefont{T1}{cmr}{m}{n}\selectfont\color{color_29791}monitorato}
\end{picture}
\newpage
\begin{tikzpicture}[overlay]\path(0pt,0pt);\end{tikzpicture}
\begin{picture}(-5,0)(2.5,0)
\put(77.8,-85.01099){\fontsize{12}{1}\usefont{T1}{cmr}{m}{n}\selectfont\color{color_29791}◦/lib/systemd/system/cron.serviceRestart=on-failure  significa che se viene terminato }
\put(95.8,-98.81097){\fontsize{12}{1}\usefont{T1}{cmr}{m}{n}\selectfont\color{color_29791}in modo non pulito, ci penserà systemd a riavviarlo automaticamente}
\put(77.8,-119.611){\fontsize{12}{1}\usefont{T1}{cmr}{m}{n}\selectfont\color{color_29791}◦/lib/systemd/system/rsyslog.service}
\put(41.8,-170.211){\fontsize{14.1}{1}\usefont{T1}{cmr}{b}{n}\selectfont\color{color_29791}Monitoraggio}
\put(41.8,-190.411){\fontsize{12}{1}\usefont{T1}{cmr}{m}{n}\selectfont\color{color_29791}Il logging (che tiene una traccia dettagliata dell'attività dei demoni) e la diagnostica istantanea }
\put(41.8,-204.211){\fontsize{12}{1}\usefont{T1}{cmr}{m}{n}\selectfont\color{color_29791}(comandi per sondare lo stato corrente delle risorse del sistema) sono necessari per tracciare se sono}
\put(41.8,-218.011){\fontsize{12}{1}\usefont{T1}{cmr}{m}{n}\selectfont\color{color_29791}avvenuti eventi necessari al corretto funzionamento dei servizi, o diagnosticare i problemi successi.}
\put(41.8,-238.811){\fontsize{12}{1}\usefont{T1}{cmr}{m}{n}\selectfont\color{color_29791}I log (diari) tenuti dal sistema sono indispensabili per la diagnostica, anche per rilevare attività }
\put(41.8,-252.611){\fontsize{12}{1}\usefont{T1}{cmr}{m}{n}\selectfont\color{color_29791}malevole o sospette. Vanno replicati su macchine remote sia per garantirne la sicurezza, sia perché }
\put(41.8,-266.411){\fontsize{12}{1}\usefont{T1}{cmr}{m}{n}\selectfont\color{color_29791}se abbiamo più macchine, possiamo guardarli tutti in uno stesso posto e non cercarli in più punti.}
\put(41.8,-287.211){\fontsize{12}{1}\usefont{T1}{cmr}{b}{it}\selectfont\color{color_29791}systemd ha integrato il logging di Linux, tenuto in un journal: questo è binario, quindi va prima }
\put(41.8,-301.011){\fontsize{12}{1}\usefont{T1}{cmr}{m}{n}\selectfont\color{color_29791}passato dal comando journalctl per renderlo visualizzabile in un dato formato. }
\put(41.8,-321.811){\fontsize{12}{1}\usefont{T1}{cmr}{b}{it}\selectfont\color{color_29791}syslog è la versione originale, i cui principi di base sono mantenuti dalle evoluzioni:}
\put(59.8,-342.611){\fontsize{12}{1}\usefont{T1}{cmr}{m}{n}\selectfont\color{color_29791}•I messaggi dai processi vengono serializzati, poi}
\put(59.8,-363.411){\fontsize{12}{1}\usefont{T1}{cmr}{m}{n}\selectfont\color{color_29791}•viene posto un timestamp: importante per stabilire relazione causale tra due eventi, quando }
\put(77.8,-377.211){\fontsize{12}{1}\usefont{T1}{cmr}{m}{n}\selectfont\color{color_29791}si diagnostica un problema, poi}
\put(59.8,-398.011){\fontsize{12}{1}\usefont{T1}{cmr}{m}{n}\selectfont\color{color_29791}•vengono classificati: questo permette di gestire dove finiscono i messaggi (magari alcuni di }
\put(77.8,-411.811){\fontsize{12}{1}\usefont{T1}{cmr}{m}{n}\selectfont\color{color_29791}uso e alcuni di diagnostica) inviandoli a diverse destinazioni}
\put(41.8,-432.611){\fontsize{12}{1}\usefont{T1}{cmr}{m}{n}\selectfont\color{color_29791}Ogni messaggio è etichettato con una coppia <facility>.<priority>, ovvero argomento (auth, }
\put(41.8,-446.411){\fontsize{12}{1}\usefont{T1}{cmr}{m}{it}\selectfont\color{color_29791}authpriv, cron, daemon, ftp, kern, lpr, mail, news, syslog, user, uucp, local0...local7. Da local0 a}
\put(41.8,-460.211){\fontsize{12}{1}\usefont{T1}{cmr}{m}{n}\selectfont\color{color_29791}local7 utilizzabili a piacere, se una delle nostre diagnostiche non rientra nella categorie precedenti) }
\put(41.8,-474.011){\fontsize{12}{1}\usefont{T1}{cmr}{m}{n}\selectfont\color{color_29791}e importanza (in ordine decrescente, dal più importante emerg, alert, crit, err, warning, notice, info, }
\put(41.8,-487.811){\fontsize{12}{1}\usefont{T1}{cmr}{m}{it}\selectfont\color{color_29791}debug). Le destinazioni possibili sono}
\put(59.8,-508.611){\fontsize{12}{1}\usefont{T1}{cmr}{m}{n}\selectfont\color{color_29791}•File indicato con nome assoluto}
\put(59.8,-529.411){\fontsize{12}{1}\usefont{T1}{cmr}{m}{n}\selectfont\color{color_29791}•STDIN di un processo, identificato da pipe verso il programma da lanciare}
\put(59.8,-550.211){\fontsize{12}{1}\usefont{T1}{cmr}{m}{n}\selectfont\color{color_29791}•Utenti collegati username, o * per tutti}
\put(59.8,-571.011){\fontsize{12}{1}\usefont{T1}{cmr}{m}{n}\selectfont\color{color_29791}•Server di syslog remoto  }
\end{picture}
\begin{tikzpicture}[overlay]
\path(0pt,0pt);
\draw[color_29791,line width=0.7pt]
(77.7pt, -572.111pt) -- (192pt, -572.111pt)
;
\end{tikzpicture}
\begin{picture}(-5,0)(2.5,0)
\put(219.6,-571.011){\fontsize{12}{1}\usefont{T1}{cmr}{m}{n}\selectfont\color{color_29791}@indirizzo o @nome }
\put(77.8,-591.811){\fontsize{12}{1}\usefont{T1}{cmr}{m}{n}\selectfont\color{color_29791}◦A default comunicazione su UDP porta 514, dove ci sarà un suo syslog che gestisce, }
\put(95.8,-605.611){\fontsize{12}{1}\usefont{T1}{cmr}{m}{n}\selectfont\color{color_29791}l'unica cosa che non potrà fare è fare un altro salto (permesso 1 solo salto)}
\put(41.8,-626.411){\fontsize{12}{1}\usefont{T1}{cmr}{b}{it}\selectfont\color{color_29791}/etc/syslog.conf contiene le regole di configurazione per lo smistamento. Ogni riga è una regola }
\put(41.8,-640.211){\fontsize{12}{1}\usefont{T1}{cmr}{m}{n}\selectfont\color{color_29791}formata da <etichetta di interesse> <destinazione>. Le priority vengono trattate in modo che una }
\put(41.8,-654.011){\fontsize{12}{1}\usefont{T1}{cmr}{m}{n}\selectfont\color{color_29791}regola che ne specifica una faccia match con tutti i messaggi di tale priority e superiori (a meno che }
\put(41.8,-667.811){\fontsize{12}{1}\usefont{T1}{cmr}{m}{n}\selectfont\color{color_29791}questa non sia preceduta da =, in quel caso match solo esatto) e c'è priority speciale none per }
\put(41.8,-681.611){\fontsize{12}{1}\usefont{T1}{cmr}{m}{n}\selectfont\color{color_29791}ignorare i messaggi con la facility specificata prima del punto. Esempi:}
\put(59.8,-702.411){\fontsize{12}{1}\usefont{T1}{cmr}{m}{n}\selectfont\color{color_29791}•kern.* /dev/console invia su device speciale che fa vedere i messaggi sulla console }
\put(77.8,-716.211){\fontsize{12}{1}\usefont{T1}{cmr}{m}{n}\selectfont\color{color_29791}tutti i messaggi di facility kern di qualsiasi priorità}
\put(59.8,-737.011){\fontsize{12}{1}\usefont{T1}{cmr}{m}{n}\selectfont\color{color_29791}•*.info;mail.none; /var/log/messages invia su quel file messaggi con qualunque }
\put(77.8,-750.811){\fontsize{12}{1}\usefont{T1}{cmr}{m}{n}\selectfont\color{color_29791}facility con importanza info o superiore, con esclusione di quelli che hanno facility mail}
\end{picture}
\newpage
\begin{tikzpicture}[overlay]\path(0pt,0pt);\end{tikzpicture}
\begin{picture}(-5,0)(2.5,0)
\put(59.8,-85.01099){\fontsize{12}{1}\usefont{T1}{cmr}{m}{n}\selectfont\color{color_29791}•*.emerg * invia ai terminali di tutti gli utenti messaggi con priority pari o }
\put(77.8,-98.81097){\fontsize{12}{1}\usefont{T1}{cmr}{m}{n}\selectfont\color{color_29791}superiore ad emerg}
\put(59.8,-119.611){\fontsize{12}{1}\usefont{T1}{cmr}{m}{n}\selectfont\color{color_29791}•kern.crit "|/usr/bin/alerter" | significa che lancia processo figlio e tutti messaggi }
\put(77.8,-133.411){\fontsize{12}{1}\usefont{T1}{cmr}{m}{n}\selectfont\color{color_29791}vengono inviati su stdin del processo indicato}
\put(59.8,-154.211){\fontsize{12}{1}\usefont{T1}{cmr}{m}{n}\selectfont\color{color_29791}•*.=warning @loghost invia al server di nome loghost SOLO i messaggi con priority }
\put(77.8,-168.011){\fontsize{12}{1}\usefont{T1}{cmr}{m}{n}\selectfont\color{color_29791}warning}
\put(41.8,-188.811){\fontsize{12}{1}\usefont{T1}{cmr}{b}{it}\selectfont\color{color_29791}rsyslog è un'evoluzione, può ricevere messaggi anche da socket, api di identificazione... Può }
\put(41.8,-202.611){\fontsize{12}{1}\usefont{T1}{cmr}{m}{n}\selectfont\color{color_29791}scrivere non solo su host remoti, file utenti;  ma anche direttamente su qualsiasi database; può }
\put(41.8,-216.411){\fontsize{12}{1}\usefont{T1}{cmr}{m}{n}\selectfont\color{color_29791}lanciare un programma e passargli parametri. È modulare e carica solo le funzioni necessarie. Ad }
\put(41.8,-230.211){\fontsize{12}{1}\usefont{T1}{cmr}{m}{n}\selectfont\color{color_29791}esempio}
\put(59.8,-251.011){\fontsize{12}{1}\usefont{T1}{cmr}{m}{n}\selectfont\color{color_29791}• attivazione della ricezione di messaggi via rete (v8.4 / v8-16): }
\put(77.8,-271.811){\fontsize{12}{1}\usefont{T1}{cmr}{m}{n}\selectfont\color{color_29791}◦\$ModLoad imudp / module(load="imudp")}
\put(77.8,-292.611){\fontsize{12}{1}\usefont{T1}{cmr}{m}{n}\selectfont\color{color_29791}◦\$UDPServerRun 514 / input(type="imudp" port="514")}
\put(59.8,-313.411){\fontsize{12}{1}\usefont{T1}{cmr}{m}{n}\selectfont\color{color_29791}•integrazione del kernel logging}
\put(77.8,-334.211){\fontsize{12}{1}\usefont{T1}{cmr}{m}{n}\selectfont\color{color_29791}◦\$ModLoad imklog / module(load="imklog")}
\put(41.8,-355.011){\fontsize{12}{1}\usefont{T1}{cmr}{m}{n}\selectfont\color{color_29791}Le sue direttive globali si trovano in /etc/rsyslog.conf , quelle specifiche in file separati sotto }
\put(41.8,-368.811){\fontsize{12}{1}\usefont{T1}{cmr}{b}{it}\selectfont\color{color_29791}/etc/rsyslog.d/ (il nome del file generale è quasi uguale a syslog, cambia solo la r. Va a leggere nella }
\put(41.8,-382.611){\fontsize{12}{1}\usefont{T1}{cmr}{m}{n}\selectfont\color{color_29791}directory /etc/rsyslog.d/: se devo automatizzare (config automatica di uno script per syslog) è più }
\put(41.8,-396.411){\fontsize{12}{1}\usefont{T1}{cmr}{m}{n}\selectfont\color{color_29791}robusto mettere un file in una cartella ed eventualmente rimuoverlo, che modificare una riga del file}
\put(41.8,-410.211){\fontsize{12}{1}\usefont{T1}{cmr}{m}{n}\selectfont\color{color_29791}principale. Permette di scartare messaggi ponendo ~ come destinazione}
\end{picture}
\begin{tikzpicture}[overlay]
\path(0pt,0pt);
\draw[color_29791,line width=0.7pt]
(95.8pt, -411.311pt) -- (384.5pt, -411.311pt)
;
\end{tikzpicture}
\begin{picture}(-5,0)(2.5,0)
\put(384.5,-410.211){\fontsize{12}{1}\usefont{T1}{cmr}{m}{n}\selectfont\color{color_29791}. Può fare output complesso,}
\put(41.8,-424.011){\fontsize{12}{1}\usefont{T1}{cmr}{m}{n}\selectfont\color{color_29791}sfruttando template per definire canali; offre anche TCP logging e può passare il messaggio come }
\put(41.8,-437.811){\fontsize{12}{1}\usefont{T1}{cmr}{m}{n}\selectfont\color{color_29791}parametro a un programma. }
\put(41.8,-458.611){\fontsize{12}{1}\usefont{T1}{cmr}{m}{n}\selectfont\color{color_29791}Differenza da cron: ogni utente può riconfigurare la propria tabella di cron; mentre la }
\put(41.8,-472.411){\fontsize{12}{1}\usefont{T1}{cmr}{m}{n}\selectfont\color{color_29791}riconfigurazione del syslog è possibile solo per root: la cartella syslog.d/ non è scrivibile da }
\put(41.8,-486.211){\fontsize{12}{1}\usefont{T1}{cmr}{m}{n}\selectfont\color{color_29791}utenti standard, il comando systemctl restart syslog non è disponibile per utenti standard. Sarebbe }
\put(41.8,-500.011){\fontsize{12}{1}\usefont{T1}{cmr}{m}{n}\selectfont\color{color_29791}possibile dare solo questi particolari permessi usando il file sudoers; ma anziché dare ad un utente }
\put(41.8,-513.811){\fontsize{12}{1}\usefont{T1}{cmr}{m}{n}\selectfont\color{color_29791}questi poteri, potremmo dargli la possibilità di usare un comando che rinomina un file da myconf.off}
\put(41.8,-527.611){\fontsize{12}{1}\usefont{T1}{cmr}{m}{n}\selectfont\color{color_29791}a myconf.conf. Avendo tanti snippet pronti all'uso nella cartella, che non finiscono con .conf e }
\put(41.8,-541.411){\fontsize{12}{1}\usefont{T1}{cmr}{m}{n}\selectfont\color{color_29791}quindi non vengono presi in considerazione.}
\put(41.8,-583.011){\fontsize{12}{1}\usefont{T1}{cmr}{b}{it}\selectfont\color{color_29791}logger è il comando per inviare messaggi di log a rsyslog, quello che si invia al syslog non è }
\put(41.8,-596.811){\fontsize{12}{1}\usefont{T1}{cmr}{m}{n}\selectfont\color{color_29791}esattamente ciò che finisce nel file: viene aggiunta una serie di informazioni utili alla gestione del }
\put(41.8,-610.611){\fontsize{12}{1}\usefont{T1}{cmr}{m}{n}\selectfont\color{color_29791}sistema: }
\put(59.8,-631.411){\fontsize{12}{1}\usefont{T1}{cmr}{m}{n}\selectfont\color{color_29791}•un timestamp, }
\put(59.8,-652.211){\fontsize{12}{1}\usefont{T1}{cmr}{m}{n}\selectfont\color{color_29791}•il nome della macchina che ha prodotto il messaggio, }
\put(59.8,-673.011){\fontsize{12}{1}\usefont{T1}{cmr}{m}{n}\selectfont\color{color_29791}•il nome dell'utente che ha prodotto il messaggio, poi duepunti e il messaggio vero e proprio}
\put(59.8,-693.811){\fontsize{12}{1}\usefont{T1}{cmr}{m}{n}\selectfont\color{color_29791}•formato:Lun 16 15:36:56 hostname utente: testomessaggio}
\put(41.8,-714.611){\fontsize{12}{1}\usefont{T1}{cmr}{b}{it}\selectfont\color{color_29791}logger -p <facility.priority> "testomessaggio" invia a syslog un messaggio con l'etichetta }
\put(41.8,-728.411){\fontsize{12}{1}\usefont{T1}{cmr}{m}{n}\selectfont\color{color_29791}specificata}
\put(41.8,-749.211){\fontsize{12}{1}\usefont{T1}{cmr}{m}{n}\selectfont\color{color_217499}opzione -t permette di scegliere come taggare il messaggio (a default, il tag è il nome utente)}
\end{picture}
\newpage
\begin{tikzpicture}[overlay]\path(0pt,0pt);\end{tikzpicture}
\begin{picture}(-5,0)(2.5,0)
\put(41.8,-85.01099){\fontsize{12}{1}\usefont{T1}{cmr}{m}{n}\selectfont\color{color_217499}opzione -n permette di inviare messaggi ad un server di logging remoto, senza passare da quello }
\put(41.8,-98.81097){\fontsize{12}{1}\usefont{T1}{cmr}{m}{n}\selectfont\color{color_217499}locale}
\put(41.8,-119.611){\fontsize{12}{1}\usefont{T1}{cmr}{m}{n}\selectfont\color{color_29791}I comandi essenziali per il monitoraggio del sistema sono ps, top, uptime per i processi; df, du, free}
\put(41.8,-133.411){\fontsize{12}{1}\usefont{T1}{cmr}{m}{n}\selectfont\color{color_29791}per lo spazio in memoria, fuser per i file. Mettendoli in script, si hanno fotografie in tempo reale di }
\put(41.8,-147.211){\fontsize{12}{1}\usefont{T1}{cmr}{m}{n}\selectfont\color{color_29791}cosa succede nel sistema. La maggior parte di essi sono interfacce verso il filesystem /proc, che }
\put(41.8,-161.011){\fontsize{12}{1}\usefont{T1}{cmr}{m}{n}\selectfont\color{color_29791}appare come FS ma in realtà occupa 0 byte. Quando si prova ad accedere a /proc, in realtà si stanno }
\put(41.8,-174.811){\fontsize{12}{1}\usefont{T1}{cmr}{m}{n}\selectfont\color{color_29791}chiedendo informazioni al sistema. }
\put(41.8,-195.611){\fontsize{12}{1}\usefont{T1}{cmr}{m}{n}\selectfont\color{color_29791}– ps: stato dei processi}
\put(41.8,-216.411){\fontsize{12}{1}\usefont{T1}{cmr}{m}{n}\selectfont\color{color_29791}– uptime: carico del sistema}
\put(41.8,-237.211){\fontsize{12}{1}\usefont{T1}{cmr}{m}{n}\selectfont\color{color_29791}– free: occupazione memoria}
\put(41.8,-258.011){\fontsize{12}{1}\usefont{T1}{cmr}{b}{it}\selectfont\color{color_35081}ps ha moltissime opzioni avendo tre diverse sintassi (UNIX singolo trattino e singole lettere, BSD }
\put(41.8,-271.811){\fontsize{12}{1}\usefont{T1}{cmr}{m}{n}\selectfont\color{color_35081}singole lettere senza trattino, GNU estensioni con parole precedute da doppio trattino). Per gli usi }
\put(41.8,-285.611){\fontsize{12}{1}\usefont{T1}{cmr}{m}{n}\selectfont\color{color_35081}più comuni all'inizio della man page abbiamo }
\put(59.8,-306.411){\fontsize{12}{1}\usefont{T1}{cmr}{m}{n}\selectfont\color{color_35081}•la sezione PROCESS SELECTION BY LIST mostra come ottenere una lista di processi }
\put(77.8,-320.211){\fontsize{12}{1}\usefont{T1}{cmr}{m}{n}\selectfont\color{color_35081}secondo le loro proprietà (es. comando lanciato, pid, utente, ecc.), molto meglio che fare }
\put(77.8,-334.011){\fontsize{12}{1}\usefont{T1}{cmr}{m}{it}\selectfont\color{color_35081}grep dell'intera lista di processi}
\put(59.8,-354.811){\fontsize{12}{1}\usefont{T1}{cmr}{m}{n}\selectfont\color{color_35081}•la sezione OUTPUT FORMAT CONTROL mostra come formattare la lista prodotta (in }
\put(77.8,-368.611){\fontsize{12}{1}\usefont{T1}{cmr}{m}{n}\selectfont\color{color_35081}particolare le opzioni (equivalenti) -o, o, --format seguite da una stringa di specificatori }
\put(77.8,-382.411){\fontsize{12}{1}\usefont{T1}{cmr}{m}{n}\selectfont\color{color_35081}documentata nella sezione STANDARD FORMAT SPECIFIERS permettono un controllo }
\put(77.8,-396.211){\fontsize{12}{1}\usefont{T1}{cmr}{m}{n}\selectfont\color{color_35081}completo sui campi che si vogliono far comparire nella lista)}
\put(59.8,-417.011){\fontsize{12}{1}\usefont{T1}{cmr}{m}{n}\selectfont\color{color_35081}•la sezione PROCESS STATE CODES spiega il significato della colonna STAT e dà }
\put(77.8,-430.811){\fontsize{12}{1}\usefont{T1}{cmr}{m}{n}\selectfont\color{color_35081}un’indicazione fondamentale dello stato del processo}
\put(41.8,-451.611){\fontsize{12}{1}\usefont{T1}{cmr}{b}{it}\selectfont\color{color_29791}ps aux mostra tutti i processi dell'utente, ordinati in base all'utente del processo, e include i processi }
\put(41.8,-465.411){\fontsize{12}{1}\usefont{T1}{cmr}{m}{n}\selectfont\color{color_29791}non agganciati ad un terminale (es. crond). Opzione w mostra la riga di comando completa che ha }
\put(41.8,-479.211){\fontsize{12}{1}\usefont{T1}{cmr}{m}{n}\selectfont\color{color_217499}originato il processo, f mostra i rapporti di discendenza tra i processi, h rimuove header dall'output}
\put(41.8,-500.011){\fontsize{12}{1}\usefont{T1}{cmr}{b}{it}\selectfont\color{color_35081}uptime riporta il tempo trascorso dal boot e il carico del sistema: lo scheduler registra il numero }
\put(41.8,-513.811){\fontsize{12}{1}\usefont{T1}{cmr}{m}{n}\selectfont\color{color_35081}totale di processi in stato R (runnable) o D (uninterruptable sleep). Questo viene mostrato come }
\put(41.8,-527.611){\fontsize{12}{1}\usefont{T1}{cmr}{m}{n}\selectfont\color{color_35081}media su tre periodi diversi. Formato output :}
\put(41.8,-548.411){\fontsize{12}{1}\usefont{T1}{cmr}{m}{it}\selectfont\color{color_35081}21:27:56 up 7:10, 2 users, load average: 0.00, 0.00, 0.00 }
\put(41.8,-569.211){\fontsize{12}{1}\usefont{T1}{cmr}{m}{n}\selectfont\color{color_35081}ORA , tempo trascorso dal boot, numero utenti connessi, carico medio negli ultimi 1/5/15 minuti}
\put(41.8,-590.011){\fontsize{12}{1}\usefont{T1}{cmr}{b}{it}\selectfont\color{color_35081}free mostra informazioni sulla memoria. La maggior parte della memoria usata per cache può essere}
\put(41.8,-603.811){\fontsize{12}{1}\usefont{T1}{cmr}{m}{n}\selectfont\color{color_35081}liberata per usi prioritari, per cui available ≈ free + buff/cache. L’impatto sulle prestazioni della }
\put(41.8,-617.611){\fontsize{12}{1}\usefont{T1}{cmr}{m}{n}\selectfont\color{color_35081}rinuncia alle cache non è nullo. Used swap > 0 significa solo che in qualche momento è servita la }
\put(41.8,-631.411){\fontsize{12}{1}\usefont{T1}{cmr}{m}{n}\selectfont\color{color_35081}swap}
\put(41.8,-652.211){\fontsize{12}{1}\usefont{T1}{cmr}{b}{it}\selectfont\color{color_35081}top riassume ps, uptime, free + uso dettagliato cpu, è interattivo e viene aggiornato regolarmente, }
\put(41.8,-666.011){\fontsize{12}{1}\usefont{T1}{cmr}{m}{n}\selectfont\color{color_35081}permette di interagire coi processi. Utile per stima intuitiva dello stato di salute, esaminando lo stato}
\put(41.8,-679.811){\fontsize{12}{1}\usefont{T1}{cmr}{m}{n}\selectfont\color{color_35081}della CPU: }
\put(59.8,-700.611){\fontsize{12}{1}\usefont{T1}{cmr}{m}{n}\selectfont\color{color_35081}•CPU scarica – molti processi in D → un dispositivo non risponde}
\put(59.8,-721.411){\fontsize{12}{1}\usefont{T1}{cmr}{m}{n}\selectfont\color{color_35081}•CPU usata principalmente in userspace – sistema CPU-bound}
\put(59.8,-742.211){\fontsize{12}{1}\usefont{T1}{cmr}{m}{n}\selectfont\color{color_35081}•CPU usata principalmente in iowait – sistema I/O-bound (possono esserci periferiche lente }
\put(77.8,-756.011){\fontsize{12}{1}\usefont{T1}{cmr}{m}{n}\selectfont\color{color_35081}ma anche sovraccaricate per altri motivi, esempio swap molto usata in sistemi memory-}
\put(77.8,-769.811){\fontsize{12}{1}\usefont{T1}{cmr}{m}{n}\selectfont\color{color_35081}bound)}
\end{picture}
\newpage
\begin{tikzpicture}[overlay]\path(0pt,0pt);\end{tikzpicture}
\begin{picture}(-5,0)(2.5,0)
\put(59.8,-85.01099){\fontsize{12}{1}\usefont{T1}{cmr}{m}{n}\selectfont\color{color_35081}•indagini più approfondite con vmstat e iostat disponendo di una baseline (valori tipici }
\put(77.8,-98.81097){\fontsize{12}{1}\usefont{T1}{cmr}{m}{n}\selectfont\color{color_35081}misurati durante uso ottimale del sistema)}
\put(41.8,-128.611){\fontsize{14.1}{1}\usefont{T1}{cmr}{b}{n}\selectfont\color{color_217499}Hardening e sicurezza di base}
\put(41.8,-148.811){\fontsize{12}{1}\usefont{T1}{cmr}{m}{n}\selectfont\color{color_217499}La sicurezza può essere gestita in modo efficace solo se si vieta ogni comportamento non }
\put(41.8,-162.611){\fontsize{12}{1}\usefont{T1}{cmr}{m}{n}\selectfont\color{color_217499}esplicitamente consentito (default deny) e se ciò che è consentito viene svolto con i minimi }
\put(41.8,-176.411){\fontsize{12}{1}\usefont{T1}{cmr}{m}{n}\selectfont\color{color_217499}permessi necessari (minimum privilege). Sicurezza comprende riservatezza, integrità, autenticità }
\put(41.8,-190.211){\fontsize{12}{1}\usefont{T1}{cmr}{b}{n}\selectfont\color{color_217499}e disponibilità, tutte da individuare e difendere caso per caso a seconda dei sistemi e delle }
\put(41.8,-204.011){\fontsize{12}{1}\usefont{T1}{cmr}{m}{n}\selectfont\color{color_217499}informazioni trattate. Solitamente ci si preoccupa di difendere un sistema dagli attacchi via rete a }
\put(41.8,-217.811){\fontsize{12}{1}\usefont{T1}{cmr}{m}{n}\selectfont\color{color_217499}software come SO e applicazioni, le contromisure sono facili da scavalcare con accesso fisico al }
\put(41.8,-231.611){\fontsize{12}{1}\usefont{T1}{cmr}{m}{n}\selectfont\color{color_217499}sistema: prima di tutto va considerata la sicurezza fisica del server: lo storage potrebbe essere }
\put(41.8,-245.411){\fontsize{12}{1}\usefont{T1}{cmr}{m}{n}\selectfont\color{color_217499}sottratto, potrebbero essere connessi apparati di raccolta dati alle interfacce, potrebbe essere avviato}
\put(41.8,-259.211){\fontsize{12}{1}\usefont{T1}{cmr}{m}{n}\selectfont\color{color_217499}un sistema operativo arbitrario. In ambito cloud queste preoccupazioni cambiano faccia ma }
\put(41.8,-273.011){\fontsize{12}{1}\usefont{T1}{cmr}{m}{n}\selectfont\color{color_217499}sussistono ancora. Se la collocazione della macchina è fuori dal controllo diretto, si può scegliere un}
\put(41.8,-286.811){\fontsize{12}{1}\usefont{T1}{cmr}{m}{n}\selectfont\color{color_217499}case che si possa chiudere e fissare al rack, installare dispositivi di rilevazione delle intrusioni, }
\put(41.8,-300.611){\fontsize{12}{1}\usefont{T1}{cmr}{m}{n}\selectfont\color{color_217499}disabilitare periferiche non utilizzate.}
\put(41.8,-321.411){\fontsize{12}{1}\usefont{T1}{cmr}{m}{n}\selectfont\color{color_217499}Il default deny inizia dall'installazione, scegliendo solo software strettamente necessario, evitando }
\put(41.8,-335.211){\fontsize{12}{1}\usefont{T1}{cmr}{m}{n}\selectfont\color{color_217499}pacchetti ignoti e di dubbia utilità. }
\put(41.8,-365.011){\fontsize{14.1}{1}\usefont{T1}{cmr}{b}{n}\selectfont\color{color_29791}Il processo di avvio}
\put(41.8,-399.011){\fontsize{12}{1}\usefont{T1}{cmr}{m}{n}\selectfont\color{color_29791}Per andare a regime il sistema segue un processo di boot, dove tipicamente }
\put(59.8,-419.811){\fontsize{12}{1}\usefont{T1}{cmr}{m}{n}\selectfont\color{color_29791}•il BIOS individua i dispositivi di possibile caricamento di bootloader per esaminarli in un }
\put(77.8,-433.611){\fontsize{12}{1}\usefont{T1}{cmr}{m}{n}\selectfont\color{color_29791}certo ordine}
\put(59.8,-454.411){\fontsize{12}{1}\usefont{T1}{cmr}{m}{n}\selectfont\color{color_29791}•il bootloader sceglie il sistema operativo (sia BIOS che bootloader possono avere password)}
\put(59.8,-475.211){\fontsize{12}{1}\usefont{T1}{cmr}{m}{n}\selectfont\color{color_29791}•il sistema operativo carica di device driver e lancia il processo init}
\put(59.8,-496.011){\fontsize{12}{1}\usefont{T1}{cmr}{m}{n}\selectfont\color{color_29791}•init avvia i servizi necessari all’inizializzazione del sistema nell’ordine corretto}
\put(41.8,-516.811){\fontsize{12}{1}\usefont{T1}{cmr}{m}{n}\selectfont\color{color_217499}Si può avere password sia sul BIOS che sul bootloader. Se l'avvio del sistema richiede una }
\put(41.8,-530.611){\fontsize{12}{1}\usefont{T1}{cmr}{m}{n}\selectfont\color{color_217499}password può essere difficile il funzionamento non supervisionato (es. mancanza di alimentazione e}
\put(41.8,-544.411){\fontsize{12}{1}\usefont{T1}{cmr}{m}{n}\selectfont\color{color_217499}impossibilità a ripartire per ore), ma è importante proteggersi almeno dai cambi di configurazione, e}
\put(41.8,-558.211){\fontsize{12}{1}\usefont{T1}{cmr}{m}{n}\selectfont\color{color_217499}inoltre non contare su un unico strato di protezione (le password del BIOS sono solitamente }
\put(41.8,-572.011){\fontsize{12}{1}\usefont{T1}{cmr}{m}{n}\selectfont\color{color_217499}resettabili, o si possono indovinare).}
\put(41.8,-592.811){\fontsize{12}{1}\usefont{T1}{cmr}{m}{n}\selectfont\color{color_217499}I bootloader tipici sono LILO (dalle origini) e GRUB (più potente e flessibile, ha una shell che }
\put(41.8,-606.611){\fontsize{12}{1}\usefont{T1}{cmr}{m}{n}\selectfont\color{color_217499}permette di eseguire comandi per modificare al volo la procedura d'avvio, permettendo abusi). }
\put(41.8,-627.411){\fontsize{12}{1}\usefont{T1}{cmr}{m}{n}\selectfont\color{color_217499}Entrambi hanno delle direttive per specificare password globali (richieste al boot) o specifiche per }
\put(41.8,-641.211){\fontsize{12}{1}\usefont{T1}{cmr}{m}{n}\selectfont\color{color_217499}alcuni item (password richiesta solo per alcune immagini).}
\put(41.8,-662.011){\fontsize{12}{1}\usefont{T1}{cmr}{m}{n}\selectfont\color{color_217499}Per garantire l'esecuzione di software autentico, è necessaria una chain of trust: anti-malware }
\put(41.8,-675.811){\fontsize{12}{1}\usefont{T1}{cmr}{m}{n}\selectfont\color{color_217499}verifica applicazioni, a sua volta è verificato da SO, a sua volta verificato dal bootloader. Per }
\put(41.8,-689.611){\fontsize{12}{1}\usefont{T1}{cmr}{m}{n}\selectfont\color{color_217499}verificare il bootloader, il BIOS si avvale di HW speciale non modificabile dal SO quindi immune a}
\put(41.8,-703.411){\fontsize{12}{1}\usefont{T1}{cmr}{m}{n}\selectfont\color{color_217499}infezioni: per garantire la root of trust si segue l'ordine del trusted boot (basato su un modulo TPM }
\put(41.8,-717.211){\fontsize{12}{1}\usefont{T1}{cmr}{m}{n}\selectfont\color{color_217499}Trusted Platform Module, un chip con funzioni crittografiche) o del secure boot mediante UEFI }
\put(41.8,-731.011){\fontsize{12}{1}\usefont{T1}{cmr}{m}{n}\selectfont\color{color_217499}(che può usare TPM per velocizzare i controlli).}
\put(41.8,-751.811){\fontsize{12}{1}\usefont{T1}{cmr}{m}{n}\selectfont\color{color_29791}UEFI è la standardizzazione di EFI, interfaccia tra SO e firmware creata da Intel per avere più }
\put(41.8,-765.611){\fontsize{12}{1}\usefont{T1}{cmr}{m}{n}\selectfont\color{color_29791}flessibilità di BIOS. E’ una sorta di pre-sistema operativo, molto più ampio di BIOS avendo qualche}
\end{picture}
\newpage
\begin{tikzpicture}[overlay]\path(0pt,0pt);\end{tikzpicture}
\begin{picture}(-5,0)(2.5,0)
\put(41.8,-85.01099){\fontsize{12}{1}\usefont{T1}{cmr}{m}{n}\selectfont\color{color_29791}milione di righe di codice, e si occupa di verificare la legittimità di ogni componente software prima}
\put(41.8,-98.81097){\fontsize{12}{1}\usefont{T1}{cmr}{m}{n}\selectfont\color{color_29791}di passare il controllo al bootloader e poi al SO, mediante un database di chiavi, bloccando il boot in}
\put(41.8,-112.611){\fontsize{12}{1}\usefont{T1}{cmr}{m}{n}\selectfont\color{color_29791}caso di difformità. Essendo queste chiavi depositate in hardware, all’inizio era possibile autenticare }
\put(41.8,-126.411){\fontsize{12}{1}\usefont{T1}{cmr}{m}{n}\selectfont\color{color_29791}solo sistemi commerciali, ma la fondazione Linux ha poi ottenuto da Microsoft una chiave che }
\put(41.8,-140.211){\fontsize{12}{1}\usefont{T1}{cmr}{m}{n}\selectfont\color{color_29791}permette di verificare sistemi aperti. }
\put(41.8,-161.011){\fontsize{12}{1}\usefont{T1}{cmr}{m}{n}\selectfont\color{color_217499}Permettere l'accesso alle risorse passa attraverso le fasi di autenticazione (verificare se soggetto ha }
\put(41.8,-174.811){\fontsize{12}{1}\usefont{T1}{cmr}{m}{n}\selectfont\color{color_217499}credenziali valide per identificarlo) e autorizzazione (decidere se soggetto ha autorità per effettuare }
\put(41.8,-188.611){\fontsize{12}{1}\usefont{T1}{cmr}{m}{n}\selectfont\color{color_217499}operazione). L'autenticazione ideale dovrebbe disporre di tre prove: qualcosa che sei (conferma }
\put(41.8,-202.411){\fontsize{12}{1}\usefont{T1}{cmr}{m}{n}\selectfont\color{color_217499}identità confrontando caratteristiche fisiologiche con dati "biometrici" di riferimento), qualcosa che }
\put(41.8,-216.211){\fontsize{12}{1}\usefont{T1}{cmr}{m}{n}\selectfont\color{color_217499}hai (conferma identità mediante possesso di oggetto fisico riconoscibile dalla macchina, come token}
\put(41.8,-230.011){\fontsize{12}{1}\usefont{T1}{cmr}{m}{n}\selectfont\color{color_217499}RFID, schede a banda magnetica), qualcosa che sai (conferma identità dimostrando di conoscere un }
\put(41.8,-243.811){\fontsize{12}{1}\usefont{T1}{cmr}{m}{n}\selectfont\color{color_217499}segreto -password, chiave-).}
\put(41.8,-272.811){\fontsize{17.5}{1}\usefont{T1}{cmr}{b}{n}\selectfont\color{color_29791}Utenti e file}
\put(41.8,-293.711){\fontsize{12}{1}\usefont{T1}{cmr}{m}{n}\selectfont\color{color_29791}Gli utenti sono i soggetti di tutte le operazioni svolte sul sistema, utilizzati per determinare i }
\put(41.8,-307.511){\fontsize{12}{1}\usefont{T1}{cmr}{m}{n}\selectfont\color{color_29791}permessi di accesso a qualsiasi risorsa (oggetto). Anche i gruppi possono avere diritti specifici per }
\put(41.8,-321.311){\fontsize{12}{1}\usefont{T1}{cmr}{m}{n}\selectfont\color{color_29791}delle risorse, anche essi sono soggetti: ogni utente appartiene ad almeno un gruppo (quello con cui }
\put(41.8,-335.111){\fontsize{12}{1}\usefont{T1}{cmr}{m}{n}\selectfont\color{color_29791}si trova ad operare appena fa login) e può appartenere anche a più gruppi, durante una sessione di }
\put(41.8,-348.911){\fontsize{12}{1}\usefont{T1}{cmr}{m}{n}\selectfont\color{color_29791}lavoro può cambiare identità di gruppo tra quelli di cui fa parte. }
\put(41.8,-369.711){\fontsize{12}{1}\usefont{T1}{cmr}{b}{it}\selectfont\color{color_29791}useradd serve per creare nuovi utenti, con granularità fine e grande possibilità di automazione. I }
\put(41.8,-383.511){\fontsize{12}{1}\usefont{T1}{cmr}{m}{n}\selectfont\color{color_29791}valori predefiniti per le caratteristiche dell'utente creato stanno in /etc/login.defs. I parametri }
\put(41.8,-397.311){\fontsize{12}{1}\usefont{T1}{cmr}{m}{n}\selectfont\color{color_29791}principali sono }
\put(59.8,-418.111){\fontsize{12}{1}\usefont{T1}{cmr}{m}{n}\selectfont\color{color_29791}•-m crea la home dell'utente seguendo come template i file dentro /etc/skel}
\put(59.8,-438.911){\fontsize{12}{1}\usefont{T1}{cmr}{m}{n}\selectfont\color{color_29791}•-s assegna la shell all'utente, le possibili shell sono indicate dentro il file /etc/shells, }
\put(77.8,-452.711){\fontsize{12}{1}\usefont{T1}{cmr}{m}{n}\selectfont\color{color_29791}altrimenti prende il default}
\put(59.8,-473.511){\fontsize{12}{1}\usefont{T1}{cmr}{m}{n}\selectfont\color{color_29791}•-U crea un gruppo con lo stesso nome dell'utente}
\put(59.8,-494.311){\fontsize{12}{1}\usefont{T1}{cmr}{m}{n}\selectfont\color{color_29791}•-K con questo parametro è possibile specificare la UMASK=0077}
\put(59.8,-515.111){\fontsize{12}{1}\usefont{T1}{cmr}{m}{n}\selectfont\color{color_29791}•-G  per assegnare l'utente ad un gruppo supplementare (es. sudo) alla creazione}
\put(41.8,-535.911){\fontsize{12}{1}\usefont{T1}{cmr}{m}{n}\selectfont\color{color_29791}Le credenziali degli utenti sono in }
\put(59.8,-556.711){\fontsize{12}{1}\usefont{T1}{cmr}{m}{n}\selectfont\color{color_29791}•/etc/passwd, formatoprandini:x:500:500:Marco Prandini:/fat/home:/bin/bash}
\put(77.8,-577.511){\fontsize{12}{1}\usefont{T1}{cmr}{m}{n}\selectfont\color{color_29791}◦Accessibile a tutti, tutti i dati pubblici dell'utente. Ogni riga è un record, i diversi }
\put(95.8,-591.311){\fontsize{12}{1}\usefont{T1}{cmr}{m}{n}\selectfont\color{color_29791}attributi separati da : }
\put(77.8,-612.111){\fontsize{12}{1}\usefont{T1}{cmr}{m}{n}\selectfont\color{color_29791}◦Primo argomento è username, il secondo parametro è una x che ricorda la }
\put(95.8,-625.911){\fontsize{12}{1}\usefont{T1}{cmr}{m}{n}\selectfont\color{color_29791}retrocompatibilità con il campo con password, importante tenerlo così che il sistema }
\put(95.8,-639.711){\fontsize{12}{1}\usefont{T1}{cmr}{m}{n}\selectfont\color{color_29791}cerchi corrispondenza in /etc/shadow }
\put(77.8,-660.511){\fontsize{12}{1}\usefont{T1}{cmr}{m}{n}\selectfont\color{color_29791}◦Poi userid numerico e groupid numerico (relativo al gruppo principale, quello associato }
\put(95.8,-674.311){\fontsize{12}{1}\usefont{T1}{cmr}{m}{n}\selectfont\color{color_29791}al momento del login). }
\put(77.8,-695.111){\fontsize{12}{1}\usefont{T1}{cmr}{m}{n}\selectfont\color{color_29791}◦Poi campo informazioni, home directory dell'utente, cwd (current working directory) da }
\put(95.8,-708.911){\fontsize{12}{1}\usefont{T1}{cmr}{m}{n}\selectfont\color{color_29791}cui parte all'avvio il sistema}
\put(77.8,-729.711){\fontsize{12}{1}\usefont{T1}{cmr}{m}{n}\selectfont\color{color_29791}◦Infine il programma avviato normalmente al login utente (solitamente è una shell, ma }
\put(95.8,-743.511){\fontsize{12}{1}\usefont{T1}{cmr}{m}{n}\selectfont\color{color_29791}potrebbe anche essere un programma diverso. Così si può limitare utente ad un solo }
\put(95.8,-757.311){\fontsize{12}{1}\usefont{T1}{cmr}{m}{n}\selectfont\color{color_29791}programma, che se scritto bene impedisce di uscire). Altra possibilità è /bin/false, }
\put(95.8,-771.111){\fontsize{12}{1}\usefont{T1}{cmr}{m}{n}\selectfont\color{color_29791}programma che appena lanciato termina con esito falso. Questo è usato per gli utenti }
\end{picture}
\newpage
\begin{tikzpicture}[overlay]\path(0pt,0pt);\end{tikzpicture}
\begin{picture}(-5,0)(2.5,0)
\put(95.8,-85.01099){\fontsize{12}{1}\usefont{T1}{cmr}{m}{n}\selectfont\color{color_29791}creati in automatico all'installazione di pacchetti, con l'unico scopo di far girare un }
\put(95.8,-98.81097){\fontsize{12}{1}\usefont{T1}{cmr}{m}{n}\selectfont\color{color_29791}demone con l'identità di quell'utente. Così facendo non si dà shell interattiva, se }
\put(95.8,-112.611){\fontsize{12}{1}\usefont{T1}{cmr}{m}{n}\selectfont\color{color_29791}qualcuno dovesse riuscire a fare login come questi utenti, avrebbe bin/false e }
\put(95.8,-126.411){\fontsize{12}{1}\usefont{T1}{cmr}{m}{n}\selectfont\color{color_29791}terminerebbe immediatamente}
\put(59.8,-147.211){\fontsize{12}{1}\usefont{T1}{cmr}{m}{n}\selectfont\color{color_29791}•/etc/shadow, formato }
\put(77.8,-161.011){\fontsize{12}{1}\usefont{T1}{cmr}{m}{n}\selectfont\color{color_29791}prandini:\$1\$/PBy29Md\$kjC1F8dvHxKhnvMTWelnX/:12156:0:99999:7:::}
\put(77.8,-181.811){\fontsize{12}{1}\usefont{T1}{cmr}{m}{n}\selectfont\color{color_29791}◦Linee corrispondenti a passwd ma accessibile solo a root. Spostando le password }
\put(95.8,-195.611){\fontsize{12}{1}\usefont{T1}{cmr}{m}{n}\selectfont\color{color_29791}offuscate dal file passwd, si rende più difficile tentare di indovinarle essendo shadow }
\put(95.8,-209.411){\fontsize{12}{1}\usefont{T1}{cmr}{m}{n}\selectfont\color{color_29791}accessibile solo a root.}
\put(77.8,-230.211){\fontsize{12}{1}\usefont{T1}{cmr}{m}{n}\selectfont\color{color_29791}◦Lasciare passwd pubblico è pratico: ad esempio per leggere gli id numerici utente, da }
\put(95.8,-244.011){\fontsize{12}{1}\usefont{T1}{cmr}{m}{n}\selectfont\color{color_29791}usare in diversi contesti per identificare gli utenti. come quando facciamo ls -l ci dice l'id}
\put(95.8,-257.811){\fontsize{12}{1}\usefont{T1}{cmr}{m}{n}\selectfont\color{color_29791}utente, non il nome}
\put(77.8,-278.611){\fontsize{12}{1}\usefont{T1}{cmr}{m}{n}\selectfont\color{color_29791}◦Contiene dati sulla validità temporale della password, esaminabili e modificabili con }
\put(95.8,-292.411){\fontsize{12}{1}\usefont{T1}{cmr}{b}{it}\selectfont\color{color_29791}chage: quando si crea un nuovo utente, i valori di default vanno presi da qualche parte: }
\put(95.8,-306.211){\fontsize{12}{1}\usefont{T1}{cmr}{m}{n}\selectfont\color{color_29791}alcuni sono in /etc/login.defs (come PASS\_MAX\_DAYS numero massimo di giorni }
\put(95.8,-320.011){\fontsize{12}{1}\usefont{T1}{cmr}{m}{n}\selectfont\color{color_29791}prima che vada cambiata la password, e PASS\_MIN\_DAYS numero minimo di giorni }
\put(95.8,-333.811){\fontsize{12}{1}\usefont{T1}{cmr}{m}{n}\selectfont\color{color_29791}prima che si possa fare un nuovo cambio dopo l'ultimo effettuato), altri in }
\put(95.8,-347.611){\fontsize{12}{1}\usefont{T1}{cmr}{m}{n}\selectfont\color{color_29791}/etc/default/useradd (come EXPIRE, scadenza dell'account), utili per configurare il }
\put(95.8,-361.411){\fontsize{12}{1}\usefont{T1}{cmr}{m}{n}\selectfont\color{color_29791}sistema con valori arbitrari.}
\put(77.8,-382.211){\fontsize{12}{1}\usefont{T1}{cmr}{m}{n}\selectfont\color{color_29791}◦È importante non avere password deboli, magari impedendole con PAM. }
\put(95.8,-396.011){\fontsize{12}{1}\usefont{T1}{cmr}{m}{it}\selectfont\color{color_217499}pam\_cracklib.so è la libreria per il controllo della robustezza delle password (all'atto }
\put(95.8,-409.811){\fontsize{12}{1}\usefont{T1}{cmr}{m}{n}\selectfont\color{color_217499}della scelta di queste), e nei file /etc/pam.d/system-auth o /etc/pam.d/common-password, }
\put(95.8,-423.611){\fontsize{12}{1}\usefont{T1}{cmr}{m}{n}\selectfont\color{color_217499}nella riga che inizia con password requisite si possono specificare i parametri minlen }
\put(95.8,-437.411){\fontsize{12}{1}\usefont{T1}{cmr}{m}{n}\selectfont\color{color_217499}(lunghezza minima) e Xcredit (X è tra l u d o , per lower/uppercase, digits, otherchars: }
\put(95.8,-451.211){\fontsize{12}{1}\usefont{T1}{cmr}{m}{n}\selectfont\color{color_217499}lunghezza minima = minlen – i credits per ogni carattere indicato con il suo peso). }
\put(95.8,-465.011){\fontsize{12}{1}\usefont{T1}{cmr}{m}{n}\selectfont\color{color_217499}Permette inoltre di indicare quanti caratteri devono essere diversi dalla precedente }
\put(95.8,-478.811){\fontsize{12}{1}\usefont{T1}{cmr}{m}{n}\selectfont\color{color_217499}(difok) e quante password precedenti vengono memorizzate (remember=); ma anche di }
\put(95.8,-492.611){\fontsize{12}{1}\usefont{T1}{cmr}{m}{n}\selectfont\color{color_217499}avere lockout dopo tentativi sbagliati con modulo pam\_tally: }
\put(95.8,-513.411){\fontsize{12}{1}\usefont{T1}{cmr}{m}{n}\selectfont\color{color_217499}1.auth required pam\_tally.so onerr=fail no\_magic\_root}
\put(95.8,-534.211){\fontsize{12}{1}\usefont{T1}{cmr}{m}{n}\selectfont\color{color_217499}2.account required pam\_tally.so deny=5 no\_magic\_root reset}
\put(95.8,-555.011){\fontsize{12}{1}\usefont{T1}{cmr}{m}{n}\selectfont\color{color_217499}La prima riga abilita conteggio dei tentativi falliti, la seconda blocca account dopo N=5 }
\put(95.8,-568.811){\fontsize{12}{1}\usefont{T1}{cmr}{m}{n}\selectfont\color{color_217499}tentativi. Un login corretto resetta il contatore, comando faillog permette di esaminare }
\put(95.8,-582.611){\fontsize{12}{1}\usefont{T1}{cmr}{m}{n}\selectfont\color{color_217499}stato account e riattivarlo.}
\put(41.8,-603.411){\fontsize{12}{1}\usefont{T1}{cmr}{m}{n}\selectfont\color{color_217499}In Linux è possibile settare limiti di uso per varie risorse per prevenire DOS; i limiti possono essere }
\put(41.8,-617.211){\fontsize{12}{1}\usefont{T1}{cmr}{m}{n}\selectfont\color{color_217499}hard (set da root, non superabili ad utente) o soft (valori di default, modificabili da utente entro il }
\put(41.8,-631.011){\fontsize{12}{1}\usefont{T1}{cmr}{m}{n}\selectfont\color{color_217499}massimo permesso dal SO e dal limite hard). La configurazione può essere locale ad ogni shell }
\put(41.8,-644.811){\fontsize{12}{1}\usefont{T1}{cmr}{m}{n}\selectfont\color{color_217499}(comando ulimit) o applicata ad ogni login con modulo pam\_limit (file /etc/limits.conf, forma }
\put(41.8,-658.611){\fontsize{12}{1}\usefont{T1}{cmr}{m}{it}\selectfont\color{color_217499}<domain> <type> <item> <value> Domain user, @group o *; type hard, soft o – (both), item ha }
\put(41.8,-672.411){\fontsize{12}{1}\usefont{T1}{cmr}{m}{n}\selectfont\color{color_217499}una sua tabella di limiti (1\_hardening\_access\_control slide 31).}
\put(41.8,-693.211){\fontsize{12}{1}\usefont{T1}{cmr}{m}{n}\selectfont\color{color_29791}L'appartenenza ai gruppi viene verificata unendo informazione in /etc/passwd (che contiene gruppo }
\put(41.8,-707.011){\fontsize{12}{1}\usefont{T1}{cmr}{m}{n}\selectfont\color{color_29791}principale di ogni utente) e il contenuto del file /etc/group, leggibile da tutti, una riga per gruppo }
\put(41.8,-720.811){\fontsize{12}{1}\usefont{T1}{cmr}{m}{n}\selectfont\color{color_29791}(formato sudo:x:27:prandini , il campo 4 dovrebbe contenere i membri del gruppo, ma non }
\put(41.8,-734.611){\fontsize{12}{1}\usefont{T1}{cmr}{m}{n}\selectfont\color{color_29791}necessariamente li tiene tutti; bisogna leggere entrambi i file per scoprire tutti i membri del gruppo, }
\put(41.8,-748.411){\fontsize{12}{1}\usefont{T1}{cmr}{m}{n}\selectfont\color{color_29791}esempio prandini:x:500: )}
\put(41.8,-769.211){\fontsize{12}{1}\usefont{T1}{cmr}{b}{it}\selectfont\color{color_29791}id <USER> riporta tutte le informazioni di identità.}
\end{picture}
\newpage
\begin{tikzpicture}[overlay]\path(0pt,0pt);\end{tikzpicture}
\begin{picture}(-5,0)(2.5,0)
\put(41.8,-85.01099){\fontsize{12}{1}\usefont{T1}{cmr}{b}{it}\selectfont\color{color_29791}usermod permette di modificare, coi suoi diversi parametri, tutte le caratteristiche dell'utente (come }
\put(41.8,-98.81097){\fontsize{12}{1}\usefont{T1}{cmr}{m}{n}\selectfont\color{color_29791}useradd ma può essere usato solo da root. Esiste anche una serie di comandi specifici per cambiare }
\put(41.8,-112.611){\fontsize{12}{1}\usefont{T1}{cmr}{m}{n}\selectfont\color{color_29791}singole proprietà, usabili da utenti standard per agire sul proprio account:}
\put(41.8,-133.411){\fontsize{12}{1}\usefont{T1}{cmr}{b}{it}\selectfont\color{color_29791}chsh modifica la shell di login}
\put(41.8,-154.211){\fontsize{12}{1}\usefont{T1}{cmr}{b}{it}\selectfont\color{color_29791}chfn modifica del nome reale}
\put(41.8,-175.011){\fontsize{12}{1}\usefont{T1}{cmr}{b}{it}\selectfont\color{color_29791}passwd modifica della password}
\put(41.8,-195.811){\fontsize{12}{1}\usefont{T1}{cmr}{b}{it}\selectfont\color{color_29791}adduser non è standard in tutte le distro e chiede i dettagli interattivamente, quindi utile se usato in }
\put(41.8,-209.611){\fontsize{12}{1}\usefont{T1}{cmr}{m}{n}\selectfont\color{color_29791}maniera estemporanea, ma non se dobbiamo scriptare creazione utenti}
\put(41.8,-230.411){\fontsize{12}{1}\usefont{T1}{cmr}{b}{it}\selectfont\color{color_29791}groupadd e addgroup per la creazione di gruppi, analogamente (primo scriptabile, secondo }
\put(41.8,-244.211){\fontsize{12}{1}\usefont{T1}{cmr}{m}{n}\selectfont\color{color_29791}interattivo)}
\put(41.8,-265.011){\fontsize{12}{1}\usefont{T1}{cmr}{b}{it}\selectfont\color{color_29791}gpasswd modifica password e lista utenti di un gruppo}
\put(41.8,-285.811){\fontsize{12}{1}\usefont{T1}{cmr}{b}{it}\selectfont\color{color_29791}getent interroga il db utenti o gruppi (magari utenti sono memorizzati in un db centralizzato }
\put(41.8,-299.611){\fontsize{12}{1}\usefont{T1}{cmr}{m}{n}\selectfont\color{color_29791}accessibile via rete, è possibile configurare macchine per autenticare utenti interrogando il db }
\put(41.8,-313.411){\fontsize{12}{1}\usefont{T1}{cmr}{m}{n}\selectfont\color{color_29791}centralizzato)}
\put(41.8,-334.211){\fontsize{12}{1}\usefont{T1}{cmr}{b}{it}\selectfont\color{color_29791}last mostra i login effettuati sul sistema, lastlog mostra la data di ultimo login di ogni utente, faillog}
\put(41.8,-348.011){\fontsize{12}{1}\usefont{T1}{cmr}{m}{n}\selectfont\color{color_29791}mostra i login falliti sul sistema.}
\put(41.8,-368.811){\fontsize{12}{1}\usefont{T1}{cmr}{m}{n}\selectfont\color{color_217499}L'idea base del controllo dell'accesso è decidere se un soggetto può eseguire una certa operazione su}
\put(41.8,-382.611){\fontsize{12}{1}\usefont{T1}{cmr}{m}{n}\selectfont\color{color_217499}un oggetto: il modo più banale di esprimere i permessi sarebbe una matrice completa; ma in un }
\put(41.8,-396.411){\fontsize{12}{1}\usefont{T1}{cmr}{m}{n}\selectfont\color{color_217499}sistema reale si avrebbero migliaia di soggetti e milioni di oggetti, con la maggior parte delle celle }
\put(41.8,-410.211){\fontsize{12}{1}\usefont{T1}{cmr}{m}{n}\selectfont\color{color_217499}al valore di default. Per risparmiare spazio, si può partizionare la matrice in due modi:}
\put(59.8,-431.011){\fontsize{12}{1}\usefont{T1}{cmr}{m}{n}\selectfont\color{color_217499}•Per soggetto: capability lists, una lista associata a ogni soggetto che contiene solo gli oggetti }
\put(77.8,-444.811){\fontsize{12}{1}\usefont{T1}{cmr}{m}{n}\selectfont\color{color_217499}con permessi diversi dal default }
\put(59.8,-465.611){\fontsize{12}{1}\usefont{T1}{cmr}{m}{n}\selectfont\color{color_217499}•Per oggetto: access control lists ACL, una lista associata ad ogni oggetto del sistema, }
\put(77.8,-479.411){\fontsize{12}{1}\usefont{T1}{cmr}{m}{n}\selectfont\color{color_217499}contente i soli soggetti con permessi diversi dal default. Esplicitamente implementata in }
\put(77.8,-493.211){\fontsize{12}{1}\usefont{T1}{cmr}{m}{n}\selectfont\color{color_217499}POSIX e MS Windows, il filesystem UNIX tradizionale ha una ACL con 3 soli soggetti: }
\put(77.8,-507.011){\fontsize{12}{1}\usefont{T1}{cmr}{m}{n}\selectfont\color{color_217499}utente proprietario U, gruppo proprietario G, others O (gruppo implicito di tutti i soggetti }
\put(77.8,-520.811){\fontsize{12}{1}\usefont{T1}{cmr}{m}{n}\selectfont\color{color_217499}non U e non appartenenti a G)}
\put(41.8,-541.611){\fontsize{12}{1}\usefont{T1}{cmr}{m}{n}\selectfont\color{color_217499}Il controllo dell'accesso viene gestito secondo due modelli}
\put(59.8,-562.411){\fontsize{12}{1}\usefont{T1}{cmr}{m}{n}\selectfont\color{color_217499}•DAC (Discretionary Access Control), ogni oggetto un proprietario e questo decide i }
\put(77.8,-576.211){\fontsize{12}{1}\usefont{T1}{cmr}{m}{n}\selectfont\color{color_217499}permessi, le ACL sono esempio tipico (setfacl per impostare, getfacl per visualizzare: ls -l }
\put(77.8,-590.011){\fontsize{12}{1}\usefont{T1}{cmr}{m}{n}\selectfont\color{color_217499}mostra un + se ACL è presente per un file).}
\put(59.8,-610.811){\fontsize{12}{1}\usefont{T1}{cmr}{m}{n}\selectfont\color{color_217499}•MAC (Mandatory Access Control), la proprietà di un oggetto non consente di cambiarne i }
\put(77.8,-624.611){\fontsize{12}{1}\usefont{T1}{cmr}{m}{n}\selectfont\color{color_217499}permessi, c'è una policy centralizzata da un security manager, espressa di solito con qualche}
\put(77.8,-638.411){\fontsize{12}{1}\usefont{T1}{cmr}{m}{n}\selectfont\color{color_217499}capability list}
\put(41.8,-659.211){\fontsize{12}{1}\usefont{T1}{cmr}{m}{n}\selectfont\color{color_29791}Ogni file (regolare, directory, link, socket, block/char speciali) è memorizzato in un i-node, che }
\put(41.8,-673.011){\fontsize{12}{1}\usefont{T1}{cmr}{m}{n}\selectfont\color{color_29791}mantiene anche informazioni sulle autorizzazioni: un (solo) utente proprietario del file e un (solo) }
\put(41.8,-686.811){\fontsize{12}{1}\usefont{T1}{cmr}{m}{n}\selectfont\color{color_29791}gruppo proprietario del file, oltre a 12 bit che rappresentano permessi standard e speciali.}
\put(41.8,-707.611){\fontsize{12}{1}\usefont{T1}{cmr}{m}{n}\selectfont\color{color_29791}I 9 bit meno significativi sono i bit di accesso (User,Group,Other); i 3 più significativi sono speciali,}
\put(41.8,-721.411){\fontsize{12}{1}\usefont{T1}{cmr}{m}{n}\selectfont\color{color_29791}in ordine di importanza: SUID (set user id), SGID (set group id), sticky.}
\put(41.8,-742.211){\fontsize{12}{1}\usefont{T1}{cmr}{m}{n}\selectfont\color{color_29791}Il significato dei bit di accesso cambia tra file e directory, ma è deducibile ricordando che una }
\put(41.8,-756.011){\fontsize{12}{1}\usefont{T1}{cmr}{m}{n}\selectfont\color{color_29791}directory è un file che contiene un DB di coppie (nome,i-node; i cosiddetti hardlink (link di }
\put(41.8,-769.811){\fontsize{12}{1}\usefont{T1}{cmr}{m}{n}\selectfont\color{color_29791}filesystem), nome dato al file - inode che rappresenta quel file nel filesystem)}
\end{picture}
\newpage
\begin{tikzpicture}[overlay]\path(0pt,0pt);\end{tikzpicture}
\begin{picture}(-5,0)(2.5,0)
\put(59.8,-85.01099){\fontsize{12}{1}\usefont{T1}{cmr}{m}{n}\selectfont\color{color_29791}•R read:permette la lettura di un file, per le directory equivale a leggere la prima }
\put(77.8,-98.81097){\fontsize{12}{1}\usefont{T1}{cmr}{m}{n}\selectfont\color{color_29791}colonna (elenco dei file)}
\put(59.8,-119.611){\fontsize{12}{1}\usefont{T1}{cmr}{m}{n}\selectfont\color{color_29791}•W write: scrittura dentro un file, aggiunta/cancellazione}
\end{picture}
\begin{tikzpicture}[overlay]
\path(0pt,0pt);
\draw[color_29791,line width=0.7pt]
(306.3pt, -120.711pt) -- (371.6pt, -120.711pt)
;
\end{tikzpicture}
\begin{picture}(-5,0)(2.5,0)
\put(371.6,-119.611){\fontsize{12}{1}\usefont{T1}{cmr}{m}{n}\selectfont\color{color_29791}/rinomina di file in una }
\put(77.8,-133.411){\fontsize{12}{1}\usefont{T1}{cmr}{m}{n}\selectfont\color{color_29791}directory (permesso W in una directory consente a un utente di cancellare file sul cui }
\put(77.8,-147.211){\fontsize{12}{1}\usefont{T1}{cmr}{m}{n}\selectfont\color{color_29791}contenuto non ha alcun diritto)}
\put(59.8,-168.011){\fontsize{12}{1}\usefont{T1}{cmr}{m}{n}\selectfont\color{color_29791}•X execute: esegue il file come programma, permette lookup tra nome e inode della }
\put(77.8,-181.811){\fontsize{12}{1}\usefont{T1}{cmr}{m}{n}\selectfont\color{color_29791}directory (se ho la R posso vedere i nomi, se ho la X posso vedere quali inode corrispondono}
\put(77.8,-195.611){\fontsize{12}{1}\usefont{T1}{cmr}{m}{n}\selectfont\color{color_29791}a un nome: averli entrambi su una directory permette di vederli e recuperarne l'inode. Per }
\put(77.8,-209.411){\fontsize{12}{1}\usefont{T1}{cmr}{m}{n}\selectfont\color{color_29791}avere l'accesso a un file -che richiede lookup di tutti gli i-node corrispondenti ai nomi delle }
\put(77.8,-223.211){\fontsize{12}{1}\usefont{T1}{cmr}{m}{n}\selectfont\color{color_29791}directory nel path- serve il permesso X per ognuna, mentre R non è necessario.)}
\put(41.8,-244.011){\fontsize{12}{1}\usefont{T1}{cmr}{m}{n}\selectfont\color{color_29791}Assegnazione di ownership}
\put(59.8,-264.811){\fontsize{12}{1}\usefont{T1}{cmr}{m}{n}\selectfont\color{color_29791}•viene fatta alla creazione di un file, l'utente creatore è assegnato come proprietario del file, il}
\put(77.8,-278.611){\fontsize{12}{1}\usefont{T1}{cmr}{m}{n}\selectfont\color{color_29791}gruppo attivo dell'utente creatore è assegnato come gruppo proprietario (default gruppo }
\put(77.8,-292.411){\fontsize{12}{1}\usefont{T1}{cmr}{m}{n}\selectfont\color{color_29791}predefinito da /etc/passwd, gruppo attivo può essere cambiato con un altro tra quelli di }
\put(77.8,-306.211){\fontsize{12}{1}\usefont{T1}{cmr}{m}{n}\selectfont\color{color_29791}appartenenza con comando newgrp, comando locale alla shell in cui viene chiamato). }
\put(59.8,-327.011){\fontsize{12}{1}\usefont{T1}{cmr}{m}{n}\selectfont\color{color_29791}•Per cambiarla dopo la creazione del file, si usano}
\put(77.8,-347.811){\fontsize{12}{1}\usefont{T1}{cmr}{m}{n}\selectfont\color{color_29791}◦chown [new\_owner]:[new\_group] <file> modifica owner e/o group owner del file}
\put(77.8,-368.611){\fontsize{12}{1}\usefont{T1}{cmr}{m}{n}\selectfont\color{color_29791}◦chgrp [new\_group] <file> cambia group owner del file (comunque solo tra quelli di cui }
\put(95.8,-382.411){\fontsize{12}{1}\usefont{T1}{cmr}{m}{n}\selectfont\color{color_29791}l'utente è membro; con opzione -R attiva ricorsione su cartelle)}
\put(41.8,-403.211){\fontsize{12}{1}\usefont{T1}{cmr}{m}{n}\selectfont\color{color_29791}Assegnazione dei permessi }
\put(59.8,-424.011){\fontsize{12}{1}\usefont{T1}{cmr}{m}{n}\selectfont\color{color_29791}•alla creazione di file viene fatta seguendo logica negativa, ovvero con tutti quelli sensati }
\put(77.8,-437.811){\fontsize{12}{1}\usefont{T1}{cmr}{m}{n}\selectfont\color{color_29791}meno quelli della umask. Quelli sensati sono diversi per file (rw-rw-rw- oppure 666) e }
\put(77.8,-451.611){\fontsize{12}{1}\usefont{T1}{cmr}{m}{n}\selectfont\color{color_29791}directory (rwxrwxrwx oppure 777), perché per un file non è mai ragionevole assegnare }
\put(77.8,-465.411){\fontsize{12}{1}\usefont{T1}{cmr}{m}{n}\selectfont\color{color_29791}automaticamente bit X, ma se una directory non ha X, invece, serve a poco. La umask può }
\put(77.8,-479.211){\fontsize{12}{1}\usefont{T1}{cmr}{m}{n}\selectfont\color{color_29791}quindi essere unica e specificare quali permessi non concedere. Poiché in Linux il gruppo di }
\put(77.8,-493.011){\fontsize{12}{1}\usefont{T1}{cmr}{m}{n}\selectfont\color{color_29791}default di un utente è uguale all'utente stesso, una umask sensata è 006 (toglie agli other }
\put(77.8,-506.811){\fontsize{12}{1}\usefont{T1}{cmr}{m}{n}\selectfont\color{color_29791}lettura e scrittura: file per uso personale avranno gruppo uguale all'utente, così facendo si }
\put(77.8,-520.611){\fontsize{12}{1}\usefont{T1}{cmr}{m}{n}\selectfont\color{color_29791}apre strada alla collaborazione non essendo necessario cambiare i permessi se si vuole }
\put(77.8,-534.411){\fontsize{12}{1}\usefont{T1}{cmr}{m}{n}\selectfont\color{color_29791}condividere, basta cambiare gruppo quando lo si crea). Col comando umask si può }
\put(77.8,-548.211){\fontsize{12}{1}\usefont{T1}{cmr}{m}{n}\selectfont\color{color_29791}interrogare e settare interattivamente nella sessione corrente, per rendere persistente la scelta}
\put(77.8,-562.011){\fontsize{12}{1}\usefont{T1}{cmr}{m}{n}\selectfont\color{color_29791}si usano i file di configurazione della shell. }
\put(59.8,-582.811){\fontsize{12}{1}\usefont{T1}{cmr}{m}{n}\selectfont\color{color_29791}•Successivamente alla creazione, si usa chmod per modificare i permessi:}
\put(77.8,-603.611){\fontsize{12}{1}\usefont{T1}{cmr}{m}{n}\selectfont\color{color_29791}◦modo numerico in base ottale, es. 2770 (la cifra meno significativa riguarda le conf per }
\put(95.8,-617.411){\fontsize{12}{1}\usefont{T1}{cmr}{m}{n}\selectfont\color{color_29791}gli other, la seconda meno sign. per il gruppo, la terza per l'utente, la quarta per i bit }
\put(95.8,-631.211){\fontsize{12}{1}\usefont{T1}{cmr}{m}{n}\selectfont\color{color_29791}speciali quindi es.}
\put(95.8,-652.011){\fontsize{12}{1}\usefont{T1}{cmr}{m}{n}\selectfont\color{color_29791}1.chmod 2770 miadirectory sarà SUID}
\end{picture}
\begin{tikzpicture}[overlay]
\path(0pt,0pt);
\draw[color_29791,line width=0.7pt]
(313.4pt, -648.611pt) -- (341.4pt, -648.611pt)
;
\end{tikzpicture}
\begin{picture}(-5,0)(2.5,0)
\put(341.4,-652.011){\fontsize{12}{1}\usefont{T1}{cmr}{m}{n}\selectfont\color{color_29791} SGID STICKY}
\end{picture}
\begin{tikzpicture}[overlay]
\path(0pt,0pt);
\draw[color_29791,line width=0.7pt]
(375.4pt, -648.611pt) -- (418.7pt, -648.611pt)
;
\end{tikzpicture}
\begin{picture}(-5,0)(2.5,0)
\put(418.7,-652.011){\fontsize{12}{1}\usefont{T1}{cmr}{m}{n}\selectfont\color{color_29791} rwx rwx ---}
\put(95.8,-672.811){\fontsize{12}{1}\usefont{T1}{cmr}{m}{n}\selectfont\color{color_29791}2.chmod 4555 miocomandosarà SUID SGID}
\end{picture}
\begin{tikzpicture}[overlay]
\path(0pt,0pt);
\draw[color_29791,line width=0.7pt]
(344.4pt, -669.411pt) -- (372.4pt, -669.411pt)
;
\end{tikzpicture}
\begin{picture}(-5,0)(2.5,0)
\put(372.4,-672.811){\fontsize{12}{1}\usefont{T1}{cmr}{m}{n}\selectfont\color{color_29791} STICKY}
\end{picture}
\begin{tikzpicture}[overlay]
\path(0pt,0pt);
\draw[color_29791,line width=0.7pt]
(375.4pt, -669.411pt) -- (418.7pt, -669.411pt)
;
\end{tikzpicture}
\begin{picture}(-5,0)(2.5,0)
\put(418.8,-672.811){\fontsize{12}{1}\usefont{T1}{cmr}{m}{n}\selectfont\color{color_29791} r-x r-x r-x}
\put(77.8,-693.611){\fontsize{12}{1}\usefont{T1}{cmr}{m}{n}\selectfont\color{color_29791}◦modo simbolico utile per modifiche a singoli bit}
\put(95.8,-714.411){\fontsize{12}{1}\usefont{T1}{cmr}{m}{n}\selectfont\color{color_29791}1.soggetto (a all, u user, g group, o other) dice in quale terzina stiamo operando}
\put(95.8,-735.211){\fontsize{12}{1}\usefont{T1}{cmr}{m}{n}\selectfont\color{color_29791}2.operatore (= setta precisamente, con + aggiungiamo permesso, con – rimuoviamo)}
\put(95.8,-756.011){\fontsize{12}{1}\usefont{T1}{cmr}{m}{n}\selectfont\color{color_29791}3.permesso (r read, w write, x execute, s special)}
\end{picture}
\newpage
\begin{tikzpicture}[overlay]\path(0pt,0pt);\end{tikzpicture}
\begin{picture}(-5,0)(2.5,0)
\put(95.8,-85.01099){\fontsize{12}{1}\usefont{T1}{cmr}{m}{n}\selectfont\color{color_29791}4.esempio chmod 'a=r,g-rx,u+rw' file all esatto ad r, group privato di r e x, }
\put(113.8,-98.81097){\fontsize{12}{1}\usefont{T1}{cmr}{m}{n}\selectfont\color{color_29791}user dotato di r e w}
\put(41.8,-119.611){\fontsize{12}{1}\usefont{T1}{cmr}{m}{n}\selectfont\color{color_29791}Il sistema controlla i permessi secondo uno schema: se utente A vuole fare un'operazione su file di }
\put(41.8,-133.411){\fontsize{12}{1}\usefont{T1}{cmr}{m}{n}\selectfont\color{color_29791}utente U e gruppo G; prima si controlla se A è U (in tal caso finisce il controllo e si applicheranno }
\put(41.8,-147.211){\fontsize{12}{1}\usefont{T1}{cmr}{m}{n}\selectfont\color{color_29791}RWX di U), solo in caso negativo poi si controlla se A appartiene a G (in tal caso si usano RWX di }
\put(41.8,-161.011){\fontsize{12}{1}\usefont{T1}{cmr}{m}{n}\selectfont\color{color_29791}G); altrimenti si usano RWX di O. Quindi se A è U e appartiene a G, i permessi associati a G sono }
\put(41.8,-174.811){\fontsize{12}{1}\usefont{T1}{cmr}{m}{n}\selectfont\color{color_29791}ignorati anche se più favorevoli. }
\put(41.8,-195.611){\fontsize{12}{1}\usefont{T1}{cmr}{m}{n}\selectfont\color{color_29791}Ogni programma avviato da U appartenente a G, parte con quattro valori di identità: }
\put(59.8,-216.411){\fontsize{12}{1}\usefont{T1}{cmr}{m}{n}\selectfont\color{color_29791}•real user id (ruid) U, }
\put(59.8,-237.211){\fontsize{12}{1}\usefont{T1}{cmr}{m}{n}\selectfont\color{color_29791}•real group id (rgid) G, }
\put(59.8,-258.011){\fontsize{12}{1}\usefont{T1}{cmr}{m}{n}\selectfont\color{color_29791}•effective user id (euid), identità assunta dal processo per operare con soggetto diverso da U}
\put(59.8,-278.811){\fontsize{12}{1}\usefont{T1}{cmr}{m}{n}\selectfont\color{color_29791}•effective group id (egid), stessa cosa ma per identità di gruppo assunta dal processo per }
\put(77.8,-292.611){\fontsize{12}{1}\usefont{T1}{cmr}{m}{n}\selectfont\color{color_29791}operare come soggetto diverso da G}
\put(41.8,-313.411){\fontsize{12}{1}\usefont{T1}{cmr}{m}{n}\selectfont\color{color_29791}Bit speciali: }
\put(59.8,-334.211){\fontsize{12}{1}\usefont{T1}{cmr}{m}{n}\selectfont\color{color_29791}•Applicati ai file hanno senso solo se eseguibile, possono fare in modo che euid e/o egid }
\put(77.8,-348.011){\fontsize{12}{1}\usefont{T1}{cmr}{m}{n}\selectfont\color{color_29791}siano diversi dai corrispondenti ruid/rgid. : }
\put(77.8,-368.811){\fontsize{12}{1}\usefont{T1}{cmr}{m}{n}\selectfont\color{color_29791}◦bit 11 SUID 1 fa sì che un programma eseguito da utente qualsiasi, venga lanciato in }
\put(95.8,-382.611){\fontsize{12}{1}\usefont{T1}{cmr}{m}{n}\selectfont\color{color_29791}processo che gira con identità dell'utente proprietario. }
\put(77.8,-403.411){\fontsize{12}{1}\usefont{T1}{cmr}{m}{n}\selectfont\color{color_29791}◦bit 10 SGUID stessa cosa ma per identità del gruppo (processo lanciato con gruppo }
\put(95.8,-417.211){\fontsize{12}{1}\usefont{T1}{cmr}{m}{n}\selectfont\color{color_29791}proprietario del file). }
\put(77.8,-438.011){\fontsize{12}{1}\usefont{T1}{cmr}{m}{n}\selectfont\color{color_29791}◦bit 9 STICKY bit è obsoleto, dice al SO di tenere in cache una copia del programma…}
\put(59.8,-458.811){\fontsize{12}{1}\usefont{T1}{cmr}{m}{n}\selectfont\color{color_29791}•Applicati alle directory: }
\put(77.8,-479.611){\fontsize{12}{1}\usefont{T1}{cmr}{m}{n}\selectfont\color{color_29791}◦bit 11 SUID non è usato}
\put(77.8,-500.411){\fontsize{12}{1}\usefont{T1}{cmr}{m}{n}\selectfont\color{color_29791}◦bit 10 SGID serve a poter cambiare automaticamente membership di un utente quando }
\put(95.8,-514.211){\fontsize{12}{1}\usefont{T1}{cmr}{m}{n}\selectfont\color{color_29791}entra a lavorare in una directory:}
\put(95.8,-535.011){\fontsize{12}{1}\usefont{T1}{cmr}{m}{n}\selectfont\color{color_29791}1.precondizione: SGID settato, l'utente appartiene anche al gruppo proprietario }
\put(113.8,-548.811){\fontsize{12}{1}\usefont{T1}{cmr}{m}{n}\selectfont\color{color_29791}della directory}
\put(95.8,-569.611){\fontsize{12}{1}\usefont{T1}{cmr}{m}{n}\selectfont\color{color_29791}2.effetto: l'utente assume come gruppo attivo quello proprietario della }
\put(113.8,-583.411){\fontsize{12}{1}\usefont{T1}{cmr}{m}{n}\selectfont\color{color_29791}directory, i file creati hanno quel gruppo proprietario}
\put(95.8,-604.211){\fontsize{12}{1}\usefont{T1}{cmr}{m}{n}\selectfont\color{color_29791}3.vantaggi, mantenendo una umask 006: nelle aree collaborative i file sono }
\put(113.8,-618.011){\fontsize{12}{1}\usefont{T1}{cmr}{m}{n}\selectfont\color{color_29791}automaticamente resi leggibili e scrivibili da tutti i membri del gruppo. Nelle aree }
\put(113.8,-631.811){\fontsize{12}{1}\usefont{T1}{cmr}{m}{n}\selectfont\color{color_29791}personali i file sono privati perché sono proprietà del gruppo principale dell'utente, }
\put(113.8,-645.611){\fontsize{12}{1}\usefont{T1}{cmr}{m}{n}\selectfont\color{color_29791}che contiene solo lui}
\put(77.8,-666.411){\fontsize{12}{1}\usefont{T1}{cmr}{m}{n}\selectfont\color{color_29791}◦bit 9 Temp le directory temporanee (quelle world-writable predisposte perché le }
\put(95.8,-680.211){\fontsize{12}{1}\usefont{T1}{cmr}{m}{n}\selectfont\color{color_29791}applicazioni dispongano di luoghi noti dove scrivere), hanno un problema, chiunque può}
\put(95.8,-694.011){\fontsize{12}{1}\usefont{T1}{cmr}{m}{n}\selectfont\color{color_29791}cancellare un file (permesso W su directory dà cancellazione). Temp settato a 1 impone }
\put(95.8,-707.811){\fontsize{12}{1}\usefont{T1}{cmr}{m}{n}\selectfont\color{color_29791}che i file nella directory siano cancellabili solo dai rispettivi proprietari.}
\put(41.8,-728.611){\fontsize{12}{1}\usefont{T1}{cmr}{m}{n}\selectfont\color{color_217499}Attributi: usati principalmente per tuning del filesystem, alcuni hanno rilevanza per la sicurezza: a }
\put(41.8,-742.411){\fontsize{12}{1}\usefont{T1}{cmr}{m}{n}\selectfont\color{color_217499}append only, impedisce il taglio dei logfile; I immutable vieta cancellazione, creazione di link, }
\put(41.8,-756.211){\fontsize{12}{1}\usefont{T1}{cmr}{m}{n}\selectfont\color{color_217499}rinomina e scrittura, utile per file di sistema; s secure deletion sovrascrive con 0 i blocchi dei file }
\put(41.8,-770.011){\fontsize{12}{1}\usefont{T1}{cmr}{m}{n}\selectfont\color{color_217499}cancellati, valida contro strumenti in linea. chattr per modificarli, lsattr per visualizzarli.}
\end{picture}
\newpage
\begin{tikzpicture}[overlay]\path(0pt,0pt);\end{tikzpicture}
\begin{picture}(-5,0)(2.5,0)
\put(41.8,-87.01099){\fontsize{14.1}{1}\usefont{T1}{cmr}{b}{n}\selectfont\color{color_29791}Comandi utili per lavorare su file}
\put(59.8,-107.211){\fontsize{12}{1}\usefont{T1}{cmr}{m}{n}\selectfont\color{color_29791}Elenco e navigazione: }
\put(95.8,-128.011){\fontsize{12}{1}\usefont{T1}{cmr}{m}{n}\selectfont\color{color_29791}oIl filesystem Linux è un unico albero anche con più dischi fisici, la radice è ‘/’ e tutti }
\put(113.8,-142.811){\fontsize{12}{1}\usefont{T1}{cmr}{m}{n}\selectfont\color{color_29791}i path completi iniziano da questa.}
\put(95.8,-163.611){\fontsize{12}{1}\usefont{T1}{cmr}{m}{n}\selectfont\color{color_29791}opwd mostra directory di lavoro (present working directory) e tutti i path relativi si }
\put(113.8,-178.411){\fontsize{12}{1}\usefont{T1}{cmr}{m}{n}\selectfont\color{color_29791}riferiscono alla pwd, i processi continuano ad occupare la pwd dov’era l’utente che li}
\put(113.8,-192.211){\fontsize{12}{1}\usefont{T1}{cmr}{m}{n}\selectfont\color{color_29791}ha lanciati. }
\put(95.8,-213.011){\fontsize{12}{1}\usefont{T1}{cmr}{m}{n}\selectfont\color{color_29791}ocd <destinazione> per cambiare directory (homeutente quando senza argomenti, la }
\put(113.8,-227.811){\fontsize{12}{1}\usefont{T1}{cmr}{m}{n}\selectfont\color{color_29791}directory dove si era prima dell’ultimo cd con opzione ‘-‘).}
\put(95.8,-248.611){\fontsize{12}{1}\usefont{T1}{cmr}{m}{n}\selectfont\color{color_29791}oIn ogni directory D si hanno le subdir ‘.’ che coincide con D e ‘..’ che coincide con la}
\put(113.8,-263.411){\fontsize{12}{1}\usefont{T1}{cmr}{m}{n}\selectfont\color{color_29791}dir superiore}
\put(59.8,-284.211){\fontsize{12}{1}\usefont{T1}{cmr}{m}{n}\selectfont\color{color_29791}Analisi dei metadati:}
\put(95.8,-305.011){\fontsize{12}{1}\usefont{T1}{cmr}{m}{n}\selectfont\color{color_29791}ols per elencare contenuto di directory, opzioni}
\put(131.8,-326.811){\fontsize{12}{1}\usefont{T1}{cmr}{m}{n}\selectfont\color{color_29791}.-l abbina al nome le info associate al file}
\put(131.8,-347.611){\fontsize{12}{1}\usefont{T1}{cmr}{m}{n}\selectfont\color{color_29791}.-a non nasconde i file che iniziano per ‘.’ (per convenzione sono quelli di }
\put(149.8,-361.411){\fontsize{12}{1}\usefont{T1}{cmr}{m}{n}\selectfont\color{color_29791}configurazione)}
\put(131.8,-382.211){\fontsize{12}{1}\usefont{T1}{cmr}{m}{n}\selectfont\color{color_29791}.-A come -a ma nasconde i file ‘.’ e ‘..’}
\put(131.8,-403.011){\fontsize{12}{1}\usefont{T1}{cmr}{m}{n}\selectfont\color{color_29791}.-F postpone carattere ‘*’ a eseguibili e ‘/’ alle directory}
\put(131.8,-423.811){\fontsize{12}{1}\usefont{T1}{cmr}{m}{n}\selectfont\color{color_29791}.-d lista nomi dir senza listarne contenuto al contrario del default, può essere }
\put(149.8,-437.611){\fontsize{12}{1}\usefont{T1}{cmr}{m}{n}\selectfont\color{color_29791}utile quando i nomi sono espansi partendo da wildcard}
\put(131.8,-458.411){\fontsize{12}{1}\usefont{T1}{cmr}{m}{n}\selectfont\color{color_29791}.-R percorre ricorsivamente la gerarchia}
\put(131.8,-479.211){\fontsize{12}{1}\usefont{T1}{cmr}{m}{n}\selectfont\color{color_29791}.-i mostra gli i-number dei file oltre al nome}
\put(131.8,-500.011){\fontsize{12}{1}\usefont{T1}{cmr}{m}{n}\selectfont\color{color_29791}.-r inverte l’ordine elenco}
\put(131.8,-520.811){\fontsize{12}{1}\usefont{T1}{cmr}{m}{n}\selectfont\color{color_29791}.-t lista i file in ordine di data/ora di modifica (dal più recente)}
\put(131.8,-541.611){\fontsize{12}{1}\usefont{T1}{cmr}{m}{n}\selectfont\color{color_29791}.-1 forza output singola colonna}
\put(95.8,-562.411){\fontsize{12}{1}\usefont{T1}{cmr}{m}{n}\selectfont\color{color_29791}oMetadati per ls -l: formato -rw-r--r-- 1 root root 1514 Mar 29 11:07 /etc/passwd}
\put(131.8,-584.211){\fontsize{12}{1}\usefont{T1}{cmr}{m}{n}\selectfont\color{color_29791}.primo bit indica il tipo: - file standard, d directory, l link simbolico, b block }
\put(149.8,-598.011){\fontsize{12}{1}\usefont{T1}{cmr}{m}{n}\selectfont\color{color_29791}special, c character special, p named pipe (FIFO), s socket}
\put(131.8,-618.811){\fontsize{12}{1}\usefont{T1}{cmr}{m}{n}\selectfont\color{color_29791}.il numero dopo i permessi indica i link allo stesso inode, possono aversi più }
\put(149.8,-632.611){\fontsize{12}{1}\usefont{T1}{cmr}{m}{n}\selectfont\color{color_29791}nomi per lo stesso inode}
\put(131.8,-653.411){\fontsize{12}{1}\usefont{T1}{cmr}{m}{n}\selectfont\color{color_29791}.seguono utente e gruppo, dimensione, e data di modifica (se l'ultima è }
\put(149.8,-667.211){\fontsize{12}{1}\usefont{T1}{cmr}{m}{n}\selectfont\color{color_29791}avvenuta oltre un anno fa, verrà mostrato l'anno invece che l'ora)}
\put(131.8,-688.011){\fontsize{12}{1}\usefont{T1}{cmr}{m}{n}\selectfont\color{color_29791}.Link simbolici: anche detti softlink, sono un nome diverso da quello }
\put(149.8,-701.811){\fontsize{12}{1}\usefont{T1}{cmr}{m}{n}\selectfont\color{color_29791}principale dato a un file - a differenza dell'hardlink, memorizzato in una }
\put(149.8,-715.611){\fontsize{12}{1}\usefont{T1}{cmr}{m}{n}\selectfont\color{color_29791}directory che punta a un'inode-. Un softlink è un file di nome X con inode Y; }
\put(149.8,-729.411){\fontsize{12}{1}\usefont{T1}{cmr}{m}{n}\selectfont\color{color_29791}dentro questo file c'è solo una linea di testo: il nome di un altro file Z con }
\put(149.8,-743.211){\fontsize{12}{1}\usefont{T1}{cmr}{m}{n}\selectfont\color{color_29791}inode F. Spesso link simbolici creano problemi per manipolazioni massicce }
\put(149.8,-757.011){\fontsize{12}{1}\usefont{T1}{cmr}{m}{n}\selectfont\color{color_29791}di copia e simili: a volte è utile prendere il link simbolico in quanto file a sé, }
\end{picture}
\newpage
\begin{tikzpicture}[overlay]\path(0pt,0pt);\end{tikzpicture}
\begin{picture}(-5,0)(2.5,0)
\put(149.8,-85.01099){\fontsize{12}{1}\usefont{T1}{cmr}{m}{n}\selectfont\color{color_29791}altre volte si vuole fare l'operazione in modo trasparente (quindi operare non }
\put(149.8,-98.81097){\fontsize{12}{1}\usefont{T1}{cmr}{m}{n}\selectfont\color{color_29791}sul file in sé ma sul puntato). }
\put(167.8,-119.611){\fontsize{12}{1}\usefont{T1}{cmr}{m}{n}\selectfont\color{color_29791}L'hardlink referenzia un inode direttamente: si possono creare }
\put(185.8,-133.411){\fontsize{12}{1}\usefont{T1}{cmr}{m}{n}\selectfont\color{color_29791}hardlink SOLO a inode esistenti (il sistema fa lookup dell'inode per }
\put(185.8,-147.211){\fontsize{12}{1}\usefont{T1}{cmr}{m}{n}\selectfont\color{color_29791}creare l'hardlink). Il softlink invece può puntare a file inesistenti, }
\put(185.8,-161.011){\fontsize{12}{1}\usefont{T1}{cmr}{m}{n}\selectfont\color{color_29791}perché non c'è il controllo che il file indicato abbia effettivamente un }
\put(185.8,-174.811){\fontsize{12}{1}\usefont{T1}{cmr}{m}{n}\selectfont\color{color_29791}inode. Softlink fa riferimento ad un nome, non ad un inode.}
\put(167.8,-195.611){\fontsize{12}{1}\usefont{T1}{cmr}{m}{n}\selectfont\color{color_29791}I softlink sono utili per creare link tra diversi filesystem (hardlink }
\put(185.8,-209.411){\fontsize{12}{1}\usefont{T1}{cmr}{m}{n}\selectfont\color{color_29791}fanno riferimento ad inode, i cui numeri sono locali al filesystem. Ma }
\put(185.8,-223.211){\fontsize{12}{1}\usefont{T1}{cmr}{m}{n}\selectfont\color{color_29791}si possono tranquillamente creare connessione con softlink a nomi }
\put(185.8,-237.011){\fontsize{12}{1}\usefont{T1}{cmr}{m}{n}\selectfont\color{color_29791}logici di altro filesystem)}
\put(95.8,-257.811){\fontsize{12}{1}\usefont{T1}{cmr}{m}{n}\selectfont\color{color_29791}oOgni file ha 3 o 4 timestamp distinti Xtime (}
\put(131.8,-279.611){\fontsize{12}{1}\usefont{T1}{cmr}{m}{n}\selectfont\color{color_29791}.mtime modifica contenuto, }
\put(131.8,-300.411){\fontsize{12}{1}\usefont{T1}{cmr}{m}{n}\selectfont\color{color_29791}.atime accesso al contenuto, }
\put(131.8,-321.211){\fontsize{12}{1}\usefont{T1}{cmr}{m}{n}\selectfont\color{color_29791}.ctime modifica metadati, }
\put(131.8,-342.011){\fontsize{12}{1}\usefont{T1}{cmr}{m}{n}\selectfont\color{color_29791}.wtime creazione del file, se supportato)}
\put(131.8,-362.811){\fontsize{12}{1}\usefont{T1}{cmr}{m}{n}\selectfont\color{color_29791}.gestiti dal FS, possono essere cambiati a mano con touch}
\put(131.8,-383.611){\fontsize{12}{1}\usefont{T1}{cmr}{m}{n}\selectfont\color{color_29791}.per estrarre i metadati, comando stat, formato}
\put(167.8,-404.411){\fontsize{12}{1}\usefont{T1}{cmr}{m}{n}\selectfont\color{color_29791}stat --format='\%U \%a \%z' filepath con U utente proprietario,}
\put(185.8,-418.211){\fontsize{12}{1}\usefont{T1}{cmr}{m}{n}\selectfont\color{color_29791}a permessi, z ctime}
\put(59.8,-439.011){\fontsize{12}{1}\usefont{T1}{cmr}{m}{n}\selectfont\color{color_29791}Creazione e generazione file}
\put(95.8,-459.811){\fontsize{12}{1}\usefont{T1}{cmr}{m}{n}\selectfont\color{color_29791}orm – cancella un file o meglio rimuove il link (verrà cancellato quando link = 0, }
\put(113.8,-474.611){\fontsize{12}{1}\usefont{T1}{cmr}{m}{n}\selectfont\color{color_29791}conteggio = n link su filesystem + n FD aperti)}
\put(95.8,-495.411){\fontsize{12}{1}\usefont{T1}{cmr}{m}{n}\selectfont\color{color_29791}ocp – copia 1+ file in una dir, attenzione con file device e softlink: copia concetto o }
\put(113.8,-510.211){\fontsize{12}{1}\usefont{T1}{cmr}{m}{n}\selectfont\color{color_29791}contenuto?}
\put(95.8,-531.011){\fontsize{12}{1}\usefont{T1}{cmr}{m}{n}\selectfont\color{color_29791}omv – sposta 1+ file in una dir}
\put(95.8,-552.811){\fontsize{12}{1}\usefont{T1}{cmr}{m}{n}\selectfont\color{color_29791}oln – crea un link a file (a default hardlink, solo stesso FS e non verso directory. Con }
\put(113.8,-567.611){\fontsize{12}{1}\usefont{T1}{cmr}{m}{n}\selectfont\color{color_29791}opzione -s symlink, senza limitazioni)}
\put(95.8,-588.411){\fontsize{12}{1}\usefont{T1}{cmr}{m}{n}\selectfont\color{color_29791}omkdir – crea dir}
\put(95.8,-610.211){\fontsize{12}{1}\usefont{T1}{cmr}{m}{n}\selectfont\color{color_29791}ormdir – cancella dir: deve essere vuota, con opzione -r cancella ricorsivamente }
\put(113.8,-625.011){\fontsize{12}{1}\usefont{T1}{cmr}{m}{n}\selectfont\color{color_29791}(occhio alla ricorsione, rm -r /* pialla disco)}
\put(59.8,-645.811){\fontsize{12}{1}\usefont{T1}{cmr}{m}{n}\selectfont\color{color_29791}Ricerca nel filesystem: }
\put(95.8,-666.611){\fontsize{12}{1}\usefont{T1}{cmr}{m}{n}\selectfont\color{color_29791}ofind, ricerca in tempo reale il FS, occhio al carico indotto. Si possono combinare più }
\put(113.8,-681.411){\fontsize{12}{1}\usefont{T1}{cmr}{m}{n}\selectfont\color{color_29791}criteri di ricerca, ad esempio}
\put(131.8,-702.211){\fontsize{12}{1}\usefont{T1}{cmr}{m}{n}\selectfont\color{color_29791}.-name ‘expr’ Nome contiene espressione, es. ‘*.c’ tutti i file che finiscono }
\put(149.8,-716.011){\fontsize{12}{1}\usefont{T1}{cmr}{m}{n}\selectfont\color{color_29791}con .c}
\put(131.8,-736.811){\fontsize{12}{1}\usefont{T1}{cmr}{m}{n}\selectfont\color{color_29791}.Timestamp entro periodo }
\put(131.8,-757.611){\fontsize{12}{1}\usefont{T1}{cmr}{m}{n}\selectfont\color{color_29791}.-size +N-M dimensione entro limiti}
\end{picture}
\newpage
\begin{tikzpicture}[overlay]\path(0pt,0pt);\end{tikzpicture}
\begin{picture}(-5,0)(2.5,0)
\put(131.8,-85.01099){\fontsize{12}{1}\usefont{T1}{cmr}{m}{n}\selectfont\color{color_29791}.Tipo specifico (file, dir, link simbolici)}
\put(131.8,-105.811){\fontsize{12}{1}\usefont{T1}{cmr}{m}{n}\selectfont\color{color_29791}.Di proprietà di utente o gruppo specificato, o “orfani”}
\put(131.8,-126.611){\fontsize{12}{1}\usefont{T1}{cmr}{m}{n}\selectfont\color{color_29791}.Permessi di accesso specificati}
\put(131.8,-147.411){\fontsize{12}{1}\usefont{T1}{cmr}{m}{n}\selectfont\color{color_29791}.Esempio: find /usr/src -name '*.c' -size +100k -print}
\put(167.8,-168.211){\fontsize{12}{1}\usefont{T1}{cmr}{m}{n}\selectfont\color{color_29791}ricerca sotto /usr/src tutti i file che finiscono per .c, hanno dimensione}
\put(185.8,-182.011){\fontsize{12}{1}\usefont{T1}{cmr}{m}{n}\selectfont\color{color_29791}maggiore di 100K, ed elencarli sullo standard output}
\put(131.8,-202.811){\fontsize{12}{1}\usefont{T1}{cmr}{m}{n}\selectfont\color{color_29791}.E’ possibile eseguire un comando su ogni oggetto individuato con l’opzione -}
\put(149.8,-216.611){\fontsize{12}{1}\usefont{T1}{cmr}{m}{it}\selectfont\color{color_29791}exec comando \{\} \;  (le graffe vengono espanse con il nome di ogni file, \; }
\put(149.8,-230.411){\fontsize{12}{1}\usefont{T1}{cmr}{m}{n}\selectfont\color{color_29791}serve a find per individuare fine comando)}
\put(167.8,-251.211){\fontsize{12}{1}\usefont{T1}{cmr}{m}{n}\selectfont\color{color_29791}Es. find / -type f -nouser -mtime -2 -exec grep -l TXT \{\} \;}
\put(167.8,-272.011){\fontsize{12}{1}\usefont{T1}{cmr}{m}{n}\selectfont\color{color_29791}Find trova in / i file regolari, orfani, modificati meno di 2 giorni fa }
\put(185.8,-285.811){\fontsize{12}{1}\usefont{T1}{cmr}{m}{n}\selectfont\color{color_29791}(2*24H), su ognuno di essi viene invocato grep che ricerca TXT nel }
\put(185.8,-299.611){\fontsize{12}{1}\usefont{T1}{cmr}{m}{n}\selectfont\color{color_29791}contenuto }
\put(95.8,-320.411){\fontsize{12}{1}\usefont{T1}{cmr}{m}{n}\selectfont\color{color_29791}oIl comando locate ricerca su un db indicizzato (DB aggiornabile con updatedb), con }
\put(113.8,-335.211){\fontsize{12}{1}\usefont{T1}{cmr}{m}{n}\selectfont\color{color_29791}meno carico sul sistema rispetto a find, ma si può specificare solo pattern nel nome e }
\put(113.8,-349.011){\fontsize{12}{1}\usefont{T1}{cmr}{m}{n}\selectfont\color{color_29791}le risposte possono essere obsolete per modifiche effettuate dopo l’esplorazione. }
\put(95.8,-369.811){\fontsize{12}{1}\usefont{T1}{cmr}{m}{n}\selectfont\color{color_29791}oComando file per identificare il contenuto di un file, fa tre test}
\put(131.8,-391.611){\fontsize{12}{1}\usefont{T1}{cmr}{m}{n}\selectfont\color{color_29791}.1) usa stat per capire se il file è vuoto o speciale}
\put(131.8,-412.411){\fontsize{12}{1}\usefont{T1}{cmr}{m}{n}\selectfont\color{color_29791}.2) usa il DB dei "magic number" (verifica dell'header per vedere il magic }
\put(149.8,-426.211){\fontsize{12}{1}\usefont{T1}{cmr}{m}{n}\selectfont\color{color_29791}number)}
\put(131.8,-447.011){\fontsize{12}{1}\usefont{T1}{cmr}{m}{n}\selectfont\color{color_29791}.3) usa metodi empirici per capire se è un file di testo, in tal caso quale sia la }
\put(149.8,-460.811){\fontsize{12}{1}\usefont{T1}{cmr}{m}{n}\selectfont\color{color_29791}lingua naturale o linguaggio di programmazione}
\put(95.8,-481.611){\fontsize{12}{1}\usefont{T1}{cmr}{m}{n}\selectfont\color{color_29791}oDue formati dei file di testo:}
\put(131.8,-503.411){\fontsize{12}{1}\usefont{T1}{cmr}{m}{n}\selectfont\color{color_29791}.nei sistemi UNIX le linee sono terminate da un solo carattere, line feed o }
\put(149.8,-517.211){\fontsize{12}{1}\usefont{T1}{cmr}{m}{n}\selectfont\color{color_29791}LF o \n o 0x0A}
\put(131.8,-538.011){\fontsize{12}{1}\usefont{T1}{cmr}{m}{n}\selectfont\color{color_29791}.nei sistemi DOS/Windows le linee sono terminate da due caratteri: carriage }
\put(149.8,-551.811){\fontsize{12}{1}\usefont{T1}{cmr}{m}{n}\selectfont\color{color_29791}return line feed o CRLF o \r\n o 0x0D0A}
\put(131.8,-572.611){\fontsize{12}{1}\usefont{T1}{cmr}{m}{n}\selectfont\color{color_29791}.senza opportuna conversione i file di origine DOS hanno caratteri extra  fine }
\put(149.8,-586.411){\fontsize{12}{1}\usefont{T1}{cmr}{m}{n}\selectfont\color{color_29791}linea (visualizzati da editor di testo come \^M, possono causare errori negli }
\put(149.8,-600.211){\fontsize{12}{1}\usefont{T1}{cmr}{m}{n}\selectfont\color{color_29791}script e nei file di configurazione). Dopo aver messo in piedi un grande }
\put(149.8,-614.011){\fontsize{12}{1}\usefont{T1}{cmr}{m}{it}\selectfont\color{color_29791}sistema che funzioni per lavorare nativamente su Linux... Un peccato che la }
\put(149.8,-627.811){\fontsize{12}{1}\usefont{T1}{cmr}{m}{it}\selectfont\color{color_29791}gente metta roba scritta su blocco note, piena di caratteri che non }
\put(149.8,-641.611){\fontsize{12}{1}\usefont{T1}{cmr}{m}{it}\selectfont\color{color_29791}permettono di eseguire gli script}
\put(131.8,-662.411){\fontsize{12}{1}\usefont{T1}{cmr}{m}{n}\selectfont\color{color_29791}.conversione: Ubuntu pacchetto tofrodos , comandi todos / fromdos (su altre }
\put(149.8,-676.211){\fontsize{12}{1}\usefont{T1}{cmr}{m}{n}\selectfont\color{color_29791}distro es. unix2dos, dos2unix)}
\put(59.8,-697.011){\fontsize{12}{1}\usefont{T1}{cmr}{m}{n}\selectfont\color{color_29791}Trasferimento dati da/per device (locali): i comandi più ovvi non sono pratici per trasferire }
\put(77.8,-710.811){\fontsize{12}{1}\usefont{T1}{cmr}{m}{n}\selectfont\color{color_29791}dati con file speciali, cat e ridirezioni sono utilizzabili in modo “tutto o niente”, cp non }
\put(77.8,-724.611){\fontsize{12}{1}\usefont{T1}{cmr}{m}{n}\selectfont\color{color_29791}funziona.}
\put(95.8,-745.411){\fontsize{12}{1}\usefont{T1}{cmr}{m}{n}\selectfont\color{color_29791}odd può leggere/scrivere byte da qualsiasi file, bypassa filesystem scrivendo }
\put(113.8,-760.211){\fontsize{12}{1}\usefont{T1}{cmr}{m}{n}\selectfont\color{color_29791}direttamente su memoria fisica, formatodd if=<NOME> of=<NOME>}
\end{picture}
\newpage
\begin{tikzpicture}[overlay]\path(0pt,0pt);\end{tikzpicture}
\begin{picture}(-5,0)(2.5,0)
\put(131.8,-85.01099){\fontsize{12}{1}\usefont{T1}{cmr}{m}{n}\selectfont\color{color_29791}.se NOME= -  si intende STDIN/STDOUT.}
\put(131.8,-105.811){\fontsize{12}{1}\usefont{T1}{cmr}{m}{n}\selectfont\color{color_29791}.skip=<N> da che punto leggere}
\put(131.8,-126.611){\fontsize{12}{1}\usefont{T1}{cmr}{m}{n}\selectfont\color{color_29791}.seek=<N> in che punto scrivere}
\put(131.8,-147.411){\fontsize{12}{1}\usefont{T1}{cmr}{m}{n}\selectfont\color{color_29791}.count<N> quanti dati trasferire}
\put(131.8,-168.211){\fontsize{12}{1}\usefont{T1}{cmr}{m}{n}\selectfont\color{color_29791}.bs=<N> dimensione blocco}
\put(59.8,-189.011){\fontsize{12}{1}\usefont{T1}{cmr}{m}{n}\selectfont\color{color_29791}Archiviazione e compressione:}
\put(95.8,-209.811){\fontsize{12}{1}\usefont{T1}{cmr}{m}{n}\selectfont\color{color_29791}oPer archiviazione si intende prendere una gerarchia di directory e serializzarle }
\put(113.8,-224.611){\fontsize{12}{1}\usefont{T1}{cmr}{m}{n}\selectfont\color{color_29791}(renderle un'unica stringa di byte, che tenga traccia sia della struttura gerarchica dei }
\put(113.8,-238.411){\fontsize{12}{1}\usefont{T1}{cmr}{m}{n}\selectfont\color{color_29791}file che del contenuto). Per gestire più file agevolmente senza perdere i metadati di }
\put(113.8,-252.211){\fontsize{12}{1}\usefont{T1}{cmr}{m}{n}\selectfont\color{color_29791}ognuno si usa tar, con solo uno dei comandi:}
\put(131.8,-273.011){\fontsize{12}{1}\usefont{T1}{cmr}{m}{n}\selectfont\color{color_29791}.-A concatena più tar}
\put(131.8,-293.811){\fontsize{12}{1}\usefont{T1}{cmr}{m}{n}\selectfont\color{color_29791}.-c crea nuovo tar}
\put(131.8,-314.611){\fontsize{12}{1}\usefont{T1}{cmr}{m}{n}\selectfont\color{color_29791}.-d trova differenze tra archivio tar e FS}
\put(131.8,-335.411){\fontsize{12}{1}\usefont{T1}{cmr}{m}{n}\selectfont\color{color_29791}.-r aggiunge file a un tar}
\put(131.8,-356.211){\fontsize{12}{1}\usefont{T1}{cmr}{m}{n}\selectfont\color{color_29791}.-t elenca contenuto di un tar}
\put(131.8,-377.011){\fontsize{12}{1}\usefont{T1}{cmr}{m}{n}\selectfont\color{color_29791}.-u aggiorna file in un tar}
\put(131.8,-397.811){\fontsize{12}{1}\usefont{T1}{cmr}{m}{n}\selectfont\color{color_29791}.-x estrae file da un tar}
\put(131.8,-418.611){\fontsize{12}{1}\usefont{T1}{cmr}{m}{n}\selectfont\color{color_29791}.--delete cancella file da un tar}
\put(131.8,-439.411){\fontsize{12}{1}\usefont{T1}{cmr}{m}{n}\selectfont\color{color_29791}.A default assume che l’archivio sia su /dev/tape, quindi si usa sempre }
\put(149.8,-453.211){\fontsize{12}{1}\usefont{T1}{cmr}{m}{n}\selectfont\color{color_29791}opzione -f <FILENAME>. Con FILENAME= - si può indicare o STDIN da }
\put(149.8,-467.011){\fontsize{12}{1}\usefont{T1}{cmr}{m}{n}\selectfont\color{color_29791}cui leggere archivio con opzioni d,t,x ; o STDOUT su cui scrivere con }
\put(149.8,-480.811){\fontsize{12}{1}\usefont{T1}{cmr}{m}{n}\selectfont\color{color_29791}opzione c}
\put(131.8,-501.611){\fontsize{12}{1}\usefont{T1}{cmr}{m}{n}\selectfont\color{color_29791}.Altre opzioni:}
\put(167.8,-522.411){\fontsize{12}{1}\usefont{T1}{cmr}{m}{n}\selectfont\color{color_29791}-p conserva tutte le informazioni di protezione, ma utenti standard }
\put(185.8,-536.211){\fontsize{12}{1}\usefont{T1}{cmr}{m}{n}\selectfont\color{color_29791}sono forzati a dare la loro ownership quando estraggono  (funziona }
\put(185.8,-550.011){\fontsize{12}{1}\usefont{T1}{cmr}{m}{n}\selectfont\color{color_29791}appieno solo con root)}
\put(167.8,-570.811){\fontsize{12}{1}\usefont{T1}{cmr}{m}{n}\selectfont\color{color_29791}-v verbose stampa i dettagli}
\put(167.8,-591.611){\fontsize{12}{1}\usefont{T1}{cmr}{m}{n}\selectfont\color{color_29791}-T <ELENCO> prende i file da archiviare da ELENCO invece che da }
\put(185.8,-605.411){\fontsize{12}{1}\usefont{T1}{cmr}{m}{n}\selectfont\color{color_29791}parametri}
\put(167.8,-626.211){\fontsize{12}{1}\usefont{T1}{cmr}{m}{n}\selectfont\color{color_29791}-C <DIR> svolge tutto come dopo cd DIR}
\put(167.8,-647.011){\fontsize{12}{1}\usefont{T1}{cmr}{m}{n}\selectfont\color{color_29791}--files-from=nomelistaindica un file da cui leggere (uno per }
\put(185.8,-660.811){\fontsize{12}{1}\usefont{T1}{cmr}{m}{n}\selectfont\color{color_29791}linea) la lista dei file da archiviare}
\put(131.8,-681.611){\fontsize{12}{1}\usefont{T1}{cmr}{m}{n}\selectfont\color{color_29791}.Esempi: non è necessario specificare ‘-‘ fintanto che non è necessario usare }
\put(149.8,-695.411){\fontsize{12}{1}\usefont{T1}{cmr}{m}{n}\selectfont\color{color_29791}più di un’opzione che richiede parametri.}
\put(167.8,-716.211){\fontsize{12}{1}\usefont{T1}{cmr}{m}{n}\selectfont\color{color_29791}Creazione – tar cvpf users.tar /home/* : la barra iniziale verrà }
\put(185.8,-730.011){\fontsize{12}{1}\usefont{T1}{cmr}{m}{n}\selectfont\color{color_29791}rimossa in modo da rendere relativi i path}
\end{picture}
\newpage
\begin{tikzpicture}[overlay]\path(0pt,0pt);\end{tikzpicture}
\begin{picture}(-5,0)(2.5,0)
\put(167.8,-85.01099){\fontsize{12}{1}\usefont{T1}{cmr}{m}{n}\selectfont\color{color_29791}Estrazione – tar -C /newdisk -xvpf users.tar: poiché i path }
\put(185.8,-98.81097){\fontsize{12}{1}\usefont{T1}{cmr}{m}{n}\selectfont\color{color_29791}nell’archivio sono relativi, la directory home viene ricreata dentro }
\put(185.8,-112.611){\fontsize{12}{1}\usefont{T1}{cmr}{m}{n}\selectfont\color{color_29791}/newdisk e tutta la gerarchia sottostante ricostruita}
\put(167.8,-133.411){\fontsize{12}{1}\usefont{T1}{cmr}{m}{n}\selectfont\color{color_29791}In pipeline – tar cvpf   -   /home/* | tar -C /newdisk -xvpf  -}
\put(95.8,-154.211){\fontsize{12}{1}\usefont{T1}{cmr}{m}{n}\selectfont\color{color_29791}oCompressione: tar non comprime, i formati di compressione più comuni in Linux }
\put(113.8,-169.011){\fontsize{12}{1}\usefont{T1}{cmr}{m}{n}\selectfont\color{color_29791}sono}
\put(131.8,-189.811){\fontsize{12}{1}\usefont{T1}{cmr}{m}{n}\selectfont\color{color_29791}..gz con comando gzip (più comune)}
\put(131.8,-210.611){\fontsize{12}{1}\usefont{T1}{cmr}{m}{n}\selectfont\color{color_29791}..bz2 con comando bzip2}
\put(131.8,-231.411){\fontsize{12}{1}\usefont{T1}{cmr}{m}{n}\selectfont\color{color_29791}..xz con comando xz}
\put(131.8,-252.211){\fontsize{12}{1}\usefont{T1}{cmr}{m}{n}\selectfont\color{color_29791}.I comandi base prendono un argomento file e lo comprimono aggiungendo }
\put(149.8,-266.011){\fontsize{12}{1}\usefont{T1}{cmr}{m}{n}\selectfont\color{color_29791}estensione, }
\put(167.8,-286.811){\fontsize{12}{1}\usefont{T1}{cmr}{m}{n}\selectfont\color{color_29791}con opzione -d si decomprime ricreando il file e rimuovendo }
\put(185.8,-300.611){\fontsize{12}{1}\usefont{T1}{cmr}{m}{n}\selectfont\color{color_29791}l’estensione}
\put(167.8,-321.411){\fontsize{12}{1}\usefont{T1}{cmr}{m}{n}\selectfont\color{color_29791}con opzione -c si riversa su STOUT invece che su file (filtro, si può }
\put(185.8,-335.211){\fontsize{12}{1}\usefont{T1}{cmr}{m}{n}\selectfont\color{color_29791}redirigere l’output. Es. tar cf - * | xz -c > archive.tar.xz)}
\put(131.8,-356.011){\fontsize{12}{1}\usefont{T1}{cmr}{m}{n}\selectfont\color{color_29791}.tar ha opzioni per invocare decompressione}
\put(167.8,-376.811){\fontsize{12}{1}\usefont{T1}{cmr}{m}{n}\selectfont\color{color_29791}-z per gzip (estensione .tar.gz o .tgz), }
\put(167.8,-397.611){\fontsize{12}{1}\usefont{T1}{cmr}{m}{n}\selectfont\color{color_29791}-j per bzip2 (.tar.bz2 o .tbz2) }
\put(167.8,-418.411){\fontsize{12}{1}\usefont{T1}{cmr}{m}{n}\selectfont\color{color_29791}-J per xz (.tar.xz o .txz)}
\put(167.8,-439.211){\fontsize{12}{1}\usefont{T1}{cmr}{m}{n}\selectfont\color{color_29791}Es precedente tar cJf archive.tar.xz *  }
\put(59.8,-460.011){\fontsize{12}{1}\usefont{T1}{cmr}{m}{n}\selectfont\color{color_29791}Copia massiva di file (anche remota):}
\put(95.8,-480.811){\fontsize{12}{1}\usefont{T1}{cmr}{m}{n}\selectfont\color{color_29791}oIl trasferimento di gerarchie di file e cartelle, contenenti file non standard non è }
\put(113.8,-495.611){\fontsize{12}{1}\usefont{T1}{cmr}{m}{n}\selectfont\color{color_29791}gestito correttamente da tutte le versioni di cp -a e scp -R (cosa fa cp quando }
\put(113.8,-509.411){\fontsize{12}{1}\usefont{T1}{cmr}{m}{n}\selectfont\color{color_29791}incontra un link? cosa fa quando incontra file device?). tar archivia correttamente i }
\put(113.8,-523.211){\fontsize{12}{1}\usefont{T1}{cmr}{m}{n}\selectfont\color{color_29791}metadati, ma richiede di archiviare → trasferire con scp su altro host → estrarlo nella}
\put(113.8,-537.011){\fontsize{12}{1}\usefont{T1}{cmr}{m}{n}\selectfont\color{color_29791}cartella destinazione }
\put(95.8,-557.811){\fontsize{12}{1}\usefont{T1}{cmr}{m}{n}\selectfont\color{color_29791}orsync permette di effettuare più controlli, come la possibilità di non trasferire file già}
\put(113.8,-572.611){\fontsize{12}{1}\usefont{T1}{cmr}{m}{n}\selectfont\color{color_29791}presenti a destinazione, o di configurare un comportamento per file speciali}
\put(131.8,-593.411){\fontsize{12}{1}\usefont{T1}{cmr}{m}{n}\selectfont\color{color_29791}.sintassi base rsync [OPZIONI] SORGENTE DESTINAZIONE sorgente }
\put(149.8,-607.211){\fontsize{12}{1}\usefont{T1}{cmr}{m}{n}\selectfont\color{color_29791}elenco di file e cartelle, destinazione cartella. Offre copia via rete }
\put(167.8,-628.011){\fontsize{12}{1}\usefont{T1}{cmr}{m}{n}\selectfont\color{color_29791}con protocollo nativo (necessario demone rsyncd) }
\put(203.8,-648.811){\fontsize{12}{1}\usefont{T1}{cmr}{m}{n}\selectfont\color{color_29791}orsync [USER@]HOST::SRCDIR DESTINAZIONE}
\put(203.8,-670.611){\fontsize{12}{1}\usefont{T1}{cmr}{m}{n}\selectfont\color{color_29791}orsync SORGENTE [USER@]HOST::DESTDIR}
\put(167.8,-692.411){\fontsize{12}{1}\usefont{T1}{cmr}{m}{n}\selectfont\color{color_29791}con SSH (non richiede demone rsyncd), stessa sintassi di sopra ma }
\put(185.8,-706.211){\fontsize{12}{1}\usefont{T1}{cmr}{m}{n}\selectfont\color{color_29791}con un solo : e non 2 }
\put(131.8,-727.011){\fontsize{12}{1}\usefont{T1}{cmr}{m}{n}\selectfont\color{color_29791}.opzioni:}
\put(131.8,-747.811){\fontsize{12}{1}\usefont{T1}{cmr}{m}{n}\selectfont\color{color_29791}.Come copiare: }
\put(167.8,-768.611){\fontsize{12}{1}\usefont{T1}{cmr}{m}{n}\selectfont\color{color_29791}-l / -L copia i link come link/come file puntato}
\end{picture}
\newpage
\begin{tikzpicture}[overlay]\path(0pt,0pt);\end{tikzpicture}
\begin{picture}(-5,0)(2.5,0)
\put(167.8,-85.01099){\fontsize{12}{1}\usefont{T1}{cmr}{m}{n}\selectfont\color{color_29791}-p / -o / -gpreserva i permessi, il proprietario, il gruppo}
\put(167.8,-105.811){\fontsize{12}{1}\usefont{T1}{cmr}{m}{n}\selectfont\color{color_29791}-t / -a / -N preserva i timestamp di modifica/accesso/creazione}
\put(167.8,-126.611){\fontsize{12}{1}\usefont{T1}{cmr}{m}{n}\selectfont\color{color_29791}-D preserva i file speciali}
\put(131.8,-147.411){\fontsize{12}{1}\usefont{T1}{cmr}{m}{n}\selectfont\color{color_29791}.Cosa copiare:}
\put(167.8,-168.211){\fontsize{12}{1}\usefont{T1}{cmr}{m}{n}\selectfont\color{color_29791}-u salta i file che sono più nuovi a destinazione o che a }
\put(185.8,-182.011){\fontsize{12}{1}\usefont{T1}{cmr}{m}{n}\selectfont\color{color_29791}parità di età hanno stessa dimensione}
\put(167.8,-202.811){\fontsize{12}{1}\usefont{T1}{cmr}{m}{n}\selectfont\color{color_29791}-c salta i file che a destinazione hanno lo stesso checksum}
\put(41.8,-223.611){\fontsize{12}{1}\usefont{T1}{cmr}{m}{n}\selectfont\color{color_29791}Il backup è la copia dei dati dal sistema live ad un supporto offline, è impegnativo da organizzare ed}
\put(41.8,-237.411){\fontsize{12}{1}\usefont{T1}{cmr}{m}{n}\selectfont\color{color_29791}eseguire ma è l'assicurazione contro qualsiasi causa di distruzione dei dati del sistema principale. }
\put(41.8,-251.211){\fontsize{12}{1}\usefont{T1}{cmr}{m}{n}\selectfont\color{color_29791}Regola d'oro: 3-2-1-1 (3 copie dei dati, 2 media diversi, 1 copia offside (cloud) e 1 copia offline). }
\put(41.8,-265.011){\fontsize{12}{1}\usefont{T1}{cmr}{m}{n}\selectfont\color{color_29791}Va pianificato, considerando tra le altre cose, cosa copiare (compromesso tra praticità di ripristino e }
\put(41.8,-278.811){\fontsize{12}{1}\usefont{T1}{cmr}{m}{n}\selectfont\color{color_29791}tempi/spazi necessari), tipicamente si fa full backup del sistema appena messo in piedi, prima di }
\put(41.8,-292.611){\fontsize{12}{1}\usefont{T1}{cmr}{m}{n}\selectfont\color{color_29791}metterlo in produzione, poi incrementali. }
\put(59.8,-313.411){\fontsize{12}{1}\usefont{T1}{cmr}{m}{n}\selectfont\color{color_29791}•Full backup è la copia completa di ogni singolo file nel/nei filesystem oggetto del backup, }
\put(77.8,-327.211){\fontsize{12}{1}\usefont{T1}{cmr}{m}{n}\selectfont\color{color_29791}ingombrante quindi difficile farlo frequentemente}
\put(59.8,-348.011){\fontsize{12}{1}\usefont{T1}{cmr}{m}{n}\selectfont\color{color_29791}•Incremental backup è la copia dei soli file cambiati da una data di riferimento, tipicamente }
\put(77.8,-361.811){\fontsize{12}{1}\usefont{T1}{cmr}{m}{n}\selectfont\color{color_29791}quella di esecuzione dell'ultimo full backup. Adatto all'esecuzione frequente a tendere verso }
\put(77.8,-375.611){\fontsize{12}{1}\usefont{T1}{cmr}{m}{n}\selectfont\color{color_29791}ogni modifica di file (offre point-in-time restore, ma attenzione al carico dell'operazione di }
\put(77.8,-389.411){\fontsize{12}{1}\usefont{T1}{cmr}{m}{n}\selectfont\color{color_29791}indicizzazione file). Per il ripristino servono sia il full che l'incremental. Si può fare a più }
\put(77.8,-403.211){\fontsize{12}{1}\usefont{T1}{cmr}{m}{n}\selectfont\color{color_29791}livelli:Full → Incremental/level0/volume1 rispetto al full}
\put(185.8,-424.011){\fontsize{12}{1}\usefont{T1}{cmr}{m}{n}\selectfont\color{color_29791}Incremental/level1/volume1 rispetto a incremental/0/1}
\put(185.8,-444.811){\fontsize{12}{1}\usefont{T1}{cmr}{m}{n}\selectfont\color{color_29791}Incremental/level1/volume2 rispetto a incremental/0/1}
\put(148.2,-465.611){\fontsize{12}{1}\usefont{T1}{cmr}{m}{n}\selectfont\color{color_29791}→ Incremental/level0/volume2 rispetto al full}
\put(41.8,-486.411){\fontsize{12}{1}\usefont{T1}{cmr}{m}{n}\selectfont\color{color_29791}Cautele da tenere:}
\put(59.8,-507.211){\fontsize{12}{1}\usefont{T1}{cmr}{m}{n}\selectfont\color{color_29791}•correttezza della copia: si dovrebbe tenere il FS a riposo durante il backup, ma è raro nella }
\put(77.8,-521.011){\fontsize{12}{1}\usefont{T1}{cmr}{m}{n}\selectfont\color{color_29791}pratica quindi va fatta attenzione ai dettagli relativi a lettu di file aperti o di strutture }
\put(77.8,-534.811){\fontsize{12}{1}\usefont{T1}{cmr}{m}{n}\selectfont\color{color_29791}complesse come DB}
\put(59.8,-555.611){\fontsize{12}{1}\usefont{T1}{cmr}{m}{n}\selectfont\color{color_29791}•protezione dei dati: un backup contiene tutti i file del sistema, ma non c'è il sistema }
\put(77.8,-569.411){\fontsize{12}{1}\usefont{T1}{cmr}{m}{n}\selectfont\color{color_29791}operativo a mediare l'accesso quindi in caso di riservatezza dei dati va difeso diversamente }
\put(77.8,-583.211){\fontsize{12}{1}\usefont{T1}{cmr}{m}{n}\selectfont\color{color_29791}(in modo fisico o cifrando i dati)}
\put(59.8,-604.011){\fontsize{12}{1}\usefont{T1}{cmr}{m}{n}\selectfont\color{color_29791}•affidabilità dei supporti: con periodicità dipendente dalla criticità dei sistemi, ci si deve }
\put(77.8,-617.811){\fontsize{12}{1}\usefont{T1}{cmr}{m}{n}\selectfont\color{color_29791}assicurare che i dati siano scritti correttamente e siano leggibili per tutta la durata prevista }
\put(77.8,-631.611){\fontsize{12}{1}\usefont{T1}{cmr}{m}{n}\selectfont\color{color_29791}della copia: protezione da fattori tecnologici (graffi, smagnetizzazione, obsolescenza hw/sw:}
\put(77.8,-645.411){\fontsize{12}{1}\usefont{T1}{cmr}{m}{n}\selectfont\color{color_29791}c'è ancora sistema in grado di leggere i dati?) e da fattori ambientali (polvere, umidità, }
\put(77.8,-659.211){\fontsize{12}{1}\usefont{T1}{cmr}{m}{n}\selectfont\color{color_29791}temperatura).}
\put(41.8,-680.011){\fontsize{12}{1}\usefont{T1}{cmr}{m}{n}\selectfont\color{color_29791}Vengono tipicamente effettuati su HD per sistemi di fascia medio/bassa, i nastri storicamente usati }
\put(41.8,-693.811){\fontsize{12}{1}\usefont{T1}{cmr}{m}{n}\selectfont\color{color_29791}anche in questa fascia continuano a prevalere soprattutto in quella alta; avendo le soluzioni moderne}
\put(41.8,-707.611){\fontsize{12}{1}\usefont{T1}{cmr}{m}{n}\selectfont\color{color_29791}e performanti un alto costo d'ingresso compensato da un basso costo marginale (per GB).}
\end{picture}
\newpage
\begin{tikzpicture}[overlay]\path(0pt,0pt);\end{tikzpicture}
\begin{picture}(-5,0)(2.5,0)
\put(41.8,-92.211){\fontsize{19.6}{1}\usefont{T1}{cmr}{b}{n}\selectfont\color{color_29791}Riga di comando}
\put(41.8,-128.811){\fontsize{17.5}{1}\usefont{T1}{cmr}{b}{n}\selectfont\color{color_29791}Prontuario riga di comando}
\put(59.8,-149.711){\fontsize{12}{1}\usefont{T1}{cmr}{m}{n}\selectfont\color{color_29791}whoami – indica il proprio username}
\put(59.8,-170.511){\fontsize{12}{1}\usefont{T1}{cmr}{m}{n}\selectfont\color{color_29791}id – dà informazioni su identità e gruppo di appartenenza}
\put(59.8,-191.311){\fontsize{12}{1}\usefont{T1}{cmr}{m}{n}\selectfont\color{color_29791}who – chi è collegato alla macchina}
\put(59.8,-212.111){\fontsize{12}{1}\usefont{T1}{cmr}{m}{n}\selectfont\color{color_29791}shutdown [-h|-r] now – solo root, spegne la macchina (-h come comando halt) o la riavvia (-}
\put(77.8,-225.911){\fontsize{12}{1}\usefont{T1}{cmr}{m}{n}\selectfont\color{color_29791}r come reboot), now indica quando}
\put(41.8,-246.711){\fontsize{12}{1}\usefont{T1}{cmr}{m}{n}\selectfont\color{color_29791}I caratteri digitati dall’utente dopo il prompt e terminati da fine linea costituiscono la command line }
\put(41.8,-260.511){\fontsize{12}{1}\usefont{T1}{cmr}{m}{n}\selectfont\color{color_29791}(che termina con \$ per utente, \# per root).}
\put(41.8,-281.311){\fontsize{12}{1}\usefont{T1}{cmr}{m}{n}\selectfont\color{color_29791}I comandi possono essere:}
\put(59.8,-302.111){\fontsize{12}{1}\usefont{T1}{cmr}{m}{n}\selectfont\color{color_29791}•Keyword (es. for per lanciare ciclo for)}
\put(59.8,-322.911){\fontsize{12}{1}\usefont{T1}{cmr}{m}{n}\selectfont\color{color_29791}•Builtin (comandi interpretati direttamente dalla shell; ad esempio cd è builtin, non avrebbe }
\put(77.8,-336.711){\fontsize{12}{1}\usefont{T1}{cmr}{m}{n}\selectfont\color{color_29791}senso lanciare un comando esterno e che poi qualcuno debba dire a me shell dove mi trovo)}
\put(59.8,-357.511){\fontsize{12}{1}\usefont{T1}{cmr}{m}{n}\selectfont\color{color_29791}•Comandi esterni (programma da file, lanciato dalla shell)}
\put(59.8,-378.311){\fontsize{12}{1}\usefont{T1}{cmr}{m}{n}\selectfont\color{color_29791}•Alias (stringa sostituita a un’altra)}
\put(59.8,-399.111){\fontsize{12}{1}\usefont{T1}{cmr}{m}{n}\selectfont\color{color_29791}•Funzioni (come in un linguaggio di programmazione, scrivo codice, do un nome alla }
\put(77.8,-412.911){\fontsize{12}{1}\usefont{T1}{cmr}{m}{n}\selectfont\color{color_29791}funzione e poi lo invoco)}
\put(41.8,-433.711){\fontsize{12}{1}\usefont{T1}{cmr}{b}{it}\selectfont\color{color_29791}type permette di distinguere che tipo di comando eseguo, con opzione -a tutti i modi in cui la shell }
\put(41.8,-447.511){\fontsize{12}{1}\usefont{T1}{cmr}{m}{n}\selectfont\color{color_29791}può trovare il comando. Per alterare ordine di default: }
\put(59.8,-468.311){\fontsize{12}{1}\usefont{T1}{cmr}{m}{n}\selectfont\color{color_29791}•\ davanti al comando previene solo l’espansione degli alias}
\put(59.8,-489.111){\fontsize{12}{1}\usefont{T1}{cmr}{m}{n}\selectfont\color{color_29791}•Keyword builtin previene l’espansione degli alias e l’uso di funzioni e invoca l’esecuzione }
\put(77.8,-502.911){\fontsize{12}{1}\usefont{T1}{cmr}{m}{n}\selectfont\color{color_29791}del builtin specificato}
\put(59.8,-523.711){\fontsize{12}{1}\usefont{T1}{cmr}{m}{n}\selectfont\color{color_29791}•Keyword command utilizza un comando esterno anche se esiste una funzione con lo stesso }
\put(77.8,-537.511){\fontsize{12}{1}\usefont{T1}{cmr}{m}{n}\selectfont\color{color_29791}nome}
\put(59.8,-558.311){\fontsize{12}{1}\usefont{T1}{cmr}{m}{n}\selectfont\color{color_29791}•comando unalias cancella un alias definito in precedenza }
\put(41.8,-579.111){\fontsize{12}{1}\usefont{T1}{cmr}{m}{n}\selectfont\color{color_29791}Gli alias permettono di dare un nome ad una command line (es. alias miols=’ls -l’), queste }
\put(41.8,-592.911){\fontsize{12}{1}\usefont{T1}{cmr}{m}{n}\selectfont\color{color_29791}associazioni vengono perse al termine della sessione (stessa cosa per unalias). Gli alias definiti }
\put(41.8,-606.711){\fontsize{12}{1}\usefont{T1}{cmr}{m}{n}\selectfont\color{color_29791}hanno la priorità rispetto ai builtin e i comandi omonimi}
\put(41.8,-627.511){\fontsize{12}{1}\usefont{T1}{cmr}{m}{n}\selectfont\color{color_29791}Per lanciare un eseguibile lo si può individuare col percorso completo}
\put(59.8,-648.311){\fontsize{12}{1}\usefont{T1}{cmr}{m}{n}\selectfont\color{color_29791}•assoluto usr/local/bin/top}
\put(59.8,-669.111){\fontsize{12}{1}\usefont{T1}{cmr}{m}{n}\selectfont\color{color_29791}•relativo ./mycommand}
\put(41.8,-689.911){\fontsize{12}{1}\usefont{T1}{cmr}{m}{n}\selectfont\color{color_29791}La shell usa var. d’ambiente PATH per eseguire la ricerca dei comandi nel file system. La sua }
\put(41.8,-703.711){\fontsize{12}{1}\usefont{T1}{cmr}{m}{n}\selectfont\color{color_29791}struttura è quella di un elenco di directory separate da :   -->  PATH=/bin:/usr/bin:/sbin }
\put(41.8,-724.511){\fontsize{12}{1}\usefont{T1}{cmr}{m}{n}\selectfont\color{color_29791}Se ci sono eseguibili omonimi in directory diverse il sistema usa la prima istanza che trova nel path:}
\put(41.8,-745.311){\fontsize{12}{1}\usefont{T1}{cmr}{b}{it}\selectfont\color{color_29791}which ci dice quale versione si sta usando}
\end{picture}
\newpage
\begin{tikzpicture}[overlay]\path(0pt,0pt);\end{tikzpicture}
\begin{picture}(-5,0)(2.5,0)
\put(41.8,-85.01099){\fontsize{12}{1}\usefont{T1}{cmr}{m}{n}\selectfont\color{color_29791}Per ottenere configurazione persistente (es. per gli alias, ma anche per valori variabili ambiente, }
\put(41.8,-98.81097){\fontsize{12}{1}\usefont{T1}{cmr}{m}{n}\selectfont\color{color_29791}anche PATH) all’avvio della shell si usano i file di configurazione bash (vedi INVOCATION -verso }
\put(41.8,-112.611){\fontsize{12}{1}\usefont{T1}{cmr}{m}{n}\selectfont\color{color_29791}l’inizio- e FILES -verso il fondo- di man bash): per configurazioni globali applicate a tutti gli utenti}
\put(41.8,-126.411){\fontsize{12}{1}\usefont{T1}{cmr}{m}{n}\selectfont\color{color_29791}i file /etc/profile e /etc/bash.bashrc; per config. personale che scavalca i settaggi globali, i file}
\put(41.8,-140.211){\fontsize{12}{1}\usefont{T1}{cmr}{m}{n}\selectfont\color{color_29791}.bash\_profile .bash\_login .profile .bashrc, si trovano nella home di quell’utente}
\put(41.8,-161.011){\fontsize{12}{1}\usefont{T1}{cmr}{m}{n}\selectfont\color{color_29791}Per controllare la documentazione sui comandi si usano:}
\put(59.8,-181.811){\fontsize{12}{1}\usefont{T1}{cmr}{m}{n}\selectfont\color{color_29791}man pages: si invoca con man <nomepagina>, ogni applicazione installa pagine di }
\put(77.8,-195.611){\fontsize{12}{1}\usefont{T1}{cmr}{m}{n}\selectfont\color{color_29791}manuale, raggruppate in sezioni }
\put(131.8,-216.411){\fontsize{12}{1}\usefont{T1}{cmr}{m}{n}\selectfont\color{color_29791}1.User commands}
\put(131.8,-237.211){\fontsize{12}{1}\usefont{T1}{cmr}{m}{n}\selectfont\color{color_29791}2.System calls}
\put(131.8,-258.011){\fontsize{12}{1}\usefont{T1}{cmr}{m}{n}\selectfont\color{color_29791}3.Funzioni di libreria}
\put(131.8,-278.811){\fontsize{12}{1}\usefont{T1}{cmr}{m}{n}\selectfont\color{color_29791}4.File speciali (/dev/*)}
\put(131.8,-299.611){\fontsize{12}{1}\usefont{T1}{cmr}{m}{n}\selectfont\color{color_29791}5.Formati dei file, protocolli, relative strutture in C}
\put(131.8,-320.411){\fontsize{12}{1}\usefont{T1}{cmr}{m}{n}\selectfont\color{color_29791}6.Giochi}
\put(131.8,-341.211){\fontsize{12}{1}\usefont{T1}{cmr}{m}{n}\selectfont\color{color_29791}7.Varie: macro, header, filesystem}
\put(131.8,-362.011){\fontsize{12}{1}\usefont{T1}{cmr}{m}{n}\selectfont\color{color_29791}8.Comandi di amministrazione solo per root}
\put(41.8,-382.811){\fontsize{12}{1}\usefont{T1}{cmr}{m}{n}\selectfont\color{color_29791}Opzioni utili: }
\put(131.8,-403.611){\fontsize{12}{1}\usefont{T1}{cmr}{m}{n}\selectfont\color{color_29791}.man -a <comando> : cerca in tutte le sezioni}
\put(131.8,-424.411){\fontsize{12}{1}\usefont{T1}{cmr}{m}{n}\selectfont\color{color_29791}.man <sez> <comando> : cerca in una sezione}
\put(131.8,-445.211){\fontsize{12}{1}\usefont{T1}{cmr}{m}{n}\selectfont\color{color_29791}.man -k <keyword>:  }
\end{picture}
\begin{tikzpicture}[overlay]
\path(0pt,0pt);
\draw[color_29791,line width=0.7pt]
(149.7pt, -446.311pt) -- (246.1pt, -446.311pt)
;
\end{tikzpicture}
\begin{picture}(-5,0)(2.5,0)
\put(246.2,-445.211){\fontsize{12}{1}\usefont{T1}{cmr}{m}{n}\selectfont\color{color_29791} cerca tutte le pagine attinenti alla parola chiave  }
\end{picture}
\begin{tikzpicture}[overlay]
\path(0pt,0pt);
\draw[color_29791,line width=0.7pt]
(246.1pt, -446.311pt) -- (475.3pt, -446.311pt)
;
\end{tikzpicture}
\begin{picture}(-5,0)(2.5,0)
\put(59.8,-466.011){\fontsize{12}{1}\usefont{T1}{cmr}{m}{n}\selectfont\color{color_29791}i builtin bash non hanno man page, si possono avere info sommarie con help <builtin> o in }
\put(77.8,-479.811){\fontsize{12}{1}\usefont{T1}{cmr}{m}{n}\selectfont\color{color_29791}particolare nella man page bash(1)}
\put(59.8,-500.611){\fontsize{12}{1}\usefont{T1}{cmr}{m}{n}\selectfont\color{color_29791}info files: si leggono con il comando info, a metà strada tra man page e ipertesto}
\put(59.8,-521.411){\fontsize{12}{1}\usefont{T1}{cmr}{m}{n}\selectfont\color{color_29791}HOWTO: si trovano in /usr/[share]/doc/HOWTO, utili per la risoluzione di problemi pratici.}
\put(77.8,-535.211){\fontsize{12}{1}\usefont{T1}{cmr}{m}{n}\selectfont\color{color_29791}Nello stesso path sotto /translations molti sono presenti anche in italiano}
\put(59.8,-556.011){\fontsize{12}{1}\usefont{T1}{cmr}{m}{n}\selectfont\color{color_29791}On-line: molte risorse, in particolare tldp.org The Linux Documentation Project}
\put(41.8,-576.811){\fontsize{12}{1}\usefont{T1}{cmr}{m}{n}\selectfont\color{color_29791}Ogni comando accede ai caratteri che lo seguono nella command line quando viene invocato, i }
\put(41.8,-590.611){\fontsize{12}{1}\usefont{T1}{cmr}{m}{n}\selectfont\color{color_29791}gruppi di caratteri separati da spazi costituiscono gli argomenti. La shell carica ARGV in memoria }
\put(41.8,-604.411){\fontsize{12}{1}\usefont{T1}{cmr}{m}{n}\selectfont\color{color_29791}prima di generare  il processo. Un argomento che inizia con ‘-‘ è un’opzione, non da elaborare di }
\put(41.8,-618.211){\fontsize{12}{1}\usefont{T1}{cmr}{m}{n}\selectfont\color{color_29791}per sé ma usato per specificare varianti al comportamento del comando. Più opzioni si possono }
\put(41.8,-632.011){\fontsize{12}{1}\usefont{T1}{cmr}{m}{n}\selectfont\color{color_29791}raggruppare in una sola stringa. }
\put(41.8,-652.811){\fontsize{12}{1}\usefont{T1}{cmr}{m}{n}\selectfont\color{color_29791}I comandi possono essere autocompletati con TAB, che con due pressioni mostra i possibili }
\put(41.8,-666.611){\fontsize{12}{1}\usefont{T1}{cmr}{m}{n}\selectfont\color{color_29791}completamenti quando c’è ambiguità. history mostra l’elenco di tutti i comandi eseguiti in un }
\put(41.8,-680.411){\fontsize{12}{1}\usefont{T1}{cmr}{m}{n}\selectfont\color{color_29791}terminale, si scorre con freccia-su che mostra editabili i precedenti, e può essere ricercata con }
\put(41.8,-694.211){\fontsize{12}{1}\usefont{T1}{cmr}{m}{n}\selectfont\color{color_29791}CTRL-r (premuto ancora per andare a istanze meno recenti). Una volta trovato il match, si può }
\put(41.8,-708.011){\fontsize{12}{1}\usefont{T1}{cmr}{m}{n}\selectfont\color{color_29791}lanciare con invio o renderlo editabile con freccia-dx/sx. script permette di catturare in un file un }
\put(41.8,-721.811){\fontsize{12}{1}\usefont{T1}{cmr}{m}{n}\selectfont\color{color_29791}sessione di terminale, sia comandi che risultati; si termina con exit o CTRL-d. Con opzione -t }
\put(41.8,-735.611){\fontsize{12}{1}\usefont{T1}{cmr}{m}{n}\selectfont\color{color_29791}include timestamps, il file prodotto può essere utilizzato come documentazione o come argomento }
\put(41.8,-749.411){\fontsize{12}{1}\usefont{T1}{cmr}{m}{n}\selectfont\color{color_29791}per scriptreplay che visualizza senza rieseguire i comandi, ma funziona solo se il file include i }
\put(41.8,-763.211){\fontsize{12}{1}\usefont{T1}{cmr}{m}{n}\selectfont\color{color_29791}timestamp}
\end{picture}
\newpage
\begin{tikzpicture}[overlay]\path(0pt,0pt);\end{tikzpicture}
\begin{picture}(-5,0)(2.5,0)
\put(41.8,-87.01099){\fontsize{14.1}{1}\usefont{T1}{cmr}{b}{n}\selectfont\color{color_29791}Intro a VIM}
\put(41.8,-107.211){\fontsize{12}{1}\usefont{T1}{cmr}{m}{n}\selectfont\color{color_29791}VIM è un’editor a tutto schermo, versione più amichevole dello storico VI. Ha un’interfaccia }
\put(41.8,-121.011){\fontsize{12}{1}\usefont{T1}{cmr}{m}{n}\selectfont\color{color_29791}modale, il programma può trovarsi in uno dei seguenti stati, e si cambia tra uno e l’altro digitando }
\put(41.8,-134.811){\fontsize{12}{1}\usefont{T1}{cmr}{m}{n}\selectfont\color{color_29791}alcuni caratteri:}
\put(59.8,-155.611){\fontsize{12}{1}\usefont{T1}{cmr}{m}{n}\selectfont\color{color_29791}COMMAND}
\put(95.8,-176.411){\fontsize{12}{1}\usefont{T1}{cmr}{m}{n}\selectfont\color{color_29791}oIl cursore è posizionato sul testo}
\put(95.8,-198.211){\fontsize{12}{1}\usefont{T1}{cmr}{m}{n}\selectfont\color{color_29791}oLa tastiera è utilizzabile solo per richiedere l’esecuzione di comandi, e non per }
\put(113.8,-213.011){\fontsize{12}{1}\usefont{T1}{cmr}{m}{n}\selectfont\color{color_29791}introdurre testo}
\put(95.8,-233.811){\fontsize{12}{1}\usefont{T1}{cmr}{m}{n}\selectfont\color{color_29791}oI caratteri digitati non vengono visualizzati}
\put(95.8,-255.611){\fontsize{12}{1}\usefont{T1}{cmr}{m}{n}\selectfont\color{color_29791}oSi passa a INPUT con oOiIaACR}
\put(95.8,-277.411){\fontsize{12}{1}\usefont{T1}{cmr}{m}{n}\selectfont\color{color_29791}oSi passa a DIRECTIVE con :/?}
\put(95.8,-299.211){\fontsize{12}{1}\usefont{T1}{cmr}{m}{n}\selectfont\color{color_29791}oSi può richiedere lo spostamento del cursore con i comandi di movimento}
\put(131.8,-321.011){\fontsize{12}{1}\usefont{T1}{cmr}{m}{n}\selectfont\color{color_29791}.h 1 a sinistra (come backspace)}
\put(131.8,-341.811){\fontsize{12}{1}\usefont{T1}{cmr}{m}{n}\selectfont\color{color_29791}.l 1 a destra (come space)}
\put(131.8,-362.611){\fontsize{12}{1}\usefont{T1}{cmr}{m}{n}\selectfont\color{color_29791}.k 1 linea sopra (stessa colonna)}
\put(131.8,-383.411){\fontsize{12}{1}\usefont{T1}{cmr}{m}{n}\selectfont\color{color_29791}.j 1 linea sotto (stessa colonna)}
\put(131.8,-404.211){\fontsize{12}{1}\usefont{T1}{cmr}{m}{n}\selectfont\color{color_29791}.Questi 4 Sostituibili con le frecce}
\put(131.8,-425.011){\fontsize{12}{1}\usefont{T1}{cmr}{m}{n}\selectfont\color{color_29791}.\^ inizio riga}
\put(131.8,-445.811){\fontsize{12}{1}\usefont{T1}{cmr}{m}{n}\selectfont\color{color_29791}.\$ fine riga}
\put(131.8,-466.611){\fontsize{12}{1}\usefont{T1}{cmr}{m}{n}\selectfont\color{color_29791}.gg prima linea del testo}
\put(131.8,-487.411){\fontsize{12}{1}\usefont{T1}{cmr}{m}{n}\selectfont\color{color_29791}.G ultima linea del testo}
\put(131.8,-508.211){\fontsize{12}{1}\usefont{T1}{cmr}{m}{n}\selectfont\color{color_29791}.\#G sulla linea numero \#}
\put(95.8,-529.011){\fontsize{12}{1}\usefont{T1}{cmr}{m}{n}\selectfont\color{color_29791}oE’ anche possibile apportare modifiche (con comandi, in sostanza, di modifica)}
\put(131.8,-550.811){\fontsize{12}{1}\usefont{T1}{cmr}{m}{n}\selectfont\color{color_29791}.x cancella il carattere su cui si trova il cursore}
\put(131.8,-571.611){\fontsize{12}{1}\usefont{T1}{cmr}{m}{n}\selectfont\color{color_29791}.dd cancella la linea su cui si trova il cursore}
\put(131.8,-592.411){\fontsize{12}{1}\usefont{T1}{cmr}{m}{n}\selectfont\color{color_29791}.rX rimpiazza il carattere sotto il cursore con X (modifica singolo carattere, }
\put(149.8,-606.211){\fontsize{12}{1}\usefont{T1}{cmr}{m}{n}\selectfont\color{color_29791}non si passa ad INPUT)}
\put(131.8,-627.011){\fontsize{12}{1}\usefont{T1}{cmr}{m}{n}\selectfont\color{color_29791}.}
\put(131.8,-647.811){\fontsize{12}{1}\usefont{T1}{cmr}{m}{n}\selectfont\color{color_29791}.I successivi sono comandi di modifica, causano l’ingresso in INPUT}
\put(131.8,-668.611){\fontsize{12}{1}\usefont{T1}{cmr}{m}{n}\selectfont\color{color_29791}.i inserimento nella posizione del cursore}
\put(131.8,-689.411){\fontsize{12}{1}\usefont{T1}{cmr}{m}{n}\selectfont\color{color_29791}.I inserimento a inizio riga}
\put(131.8,-710.211){\fontsize{12}{1}\usefont{T1}{cmr}{m}{n}\selectfont\color{color_29791}.a append nella posizione del cursore}
\put(131.8,-731.011){\fontsize{12}{1}\usefont{T1}{cmr}{m}{n}\selectfont\color{color_29791}.A append a inizio riga}
\put(131.8,-751.811){\fontsize{12}{1}\usefont{T1}{cmr}{m}{n}\selectfont\color{color_29791}.R replace (sovrascrittura)}
\end{picture}
\newpage
\begin{tikzpicture}[overlay]\path(0pt,0pt);\end{tikzpicture}
\begin{picture}(-5,0)(2.5,0)
\put(131.8,-85.01099){\fontsize{12}{1}\usefont{T1}{cmr}{m}{n}\selectfont\color{color_29791}.cw change word, elimina la parola che inizia sotto il cursore ed edita}
\put(131.8,-105.811){\fontsize{12}{1}\usefont{T1}{cmr}{m}{n}\selectfont\color{color_29791}.C change, elimina fino a fine riga}
\put(131.8,-126.611){\fontsize{12}{1}\usefont{T1}{cmr}{m}{n}\selectfont\color{color_29791}.o linea vuota sotto al cursore}
\put(131.8,-147.411){\fontsize{12}{1}\usefont{T1}{cmr}{m}{n}\selectfont\color{color_29791}.O linea vuota sopra al cursore}
\put(95.8,-168.211){\fontsize{12}{1}\usefont{T1}{cmr}{m}{n}\selectfont\color{color_29791}oRicerca e spostamenti possono essere usati come terminatore per alcuni comandi di }
\put(113.8,-183.011){\fontsize{12}{1}\usefont{T1}{cmr}{m}{n}\selectfont\color{color_29791}modifica}
\put(131.8,-203.811){\fontsize{12}{1}\usefont{T1}{cmr}{m}{n}\selectfont\color{color_29791}.d\$ cancella dalla posizione corrente a fine riga}
\put(131.8,-224.611){\fontsize{12}{1}\usefont{T1}{cmr}{m}{n}\selectfont\color{color_29791}.dG cancella dalla posizione corrente a fine file}
\put(131.8,-245.411){\fontsize{12}{1}\usefont{T1}{cmr}{m}{n}\selectfont\color{color_29791}.c/ciao<RET> cancella dalla posizione corrente alla prima occorrenza della }
\put(149.8,-259.211){\fontsize{12}{1}\usefont{T1}{cmr}{m}{n}\selectfont\color{color_29791}stringa ciao e si porta in insert mode}
\put(95.8,-280.011){\fontsize{12}{1}\usefont{T1}{cmr}{m}{n}\selectfont\color{color_29791}oCon il punto . si ripete l’ultimo comando impartito}
\put(95.8,-301.811){\fontsize{12}{1}\usefont{T1}{cmr}{m}{n}\selectfont\color{color_29791}oPrecedendo un comando con un numero N, il comando verrà eseguito N volte }
\put(113.8,-316.611){\fontsize{12}{1}\usefont{T1}{cmr}{m}{n}\selectfont\color{color_29791}consecutivamente (Es. 10x cancella 10 caratteri)}
\put(95.8,-337.411){\fontsize{12}{1}\usefont{T1}{cmr}{m}{n}\selectfont\color{color_29791}ou annulla l’ultima azione eseguita (undo)}
\put(95.8,-359.211){\fontsize{12}{1}\usefont{T1}{cmr}{m}{n}\selectfont\color{color_29791}oCopia e incolla:}
\put(131.8,-381.011){\fontsize{12}{1}\usefont{T1}{cmr}{m}{n}\selectfont\color{color_29791}.La copia viene eseguita con buffer interni a vi, è possibile specificarne altri}
\put(131.8,-401.811){\fontsize{12}{1}\usefont{T1}{cmr}{m}{n}\selectfont\color{color_29791}.yy copia la linea corrente nel buffer}
\put(131.8,-422.611){\fontsize{12}{1}\usefont{T1}{cmr}{m}{n}\selectfont\color{color_29791}.“ayy copia la linea corrente nel buffer “a”}
\put(131.8,-443.411){\fontsize{12}{1}\usefont{T1}{cmr}{m}{n}\selectfont\color{color_29791}.d esegue il cut}
\put(131.8,-464.211){\fontsize{12}{1}\usefont{T1}{cmr}{m}{n}\selectfont\color{color_29791}.p incolla dopo la linea corrente}
\put(131.8,-485.011){\fontsize{12}{1}\usefont{T1}{cmr}{m}{n}\selectfont\color{color_29791}.P incolla prima della linea corrente}
\put(131.8,-505.811){\fontsize{12}{1}\usefont{T1}{cmr}{m}{n}\selectfont\color{color_29791}.Per copiare blocchi di linee si può}
\put(167.8,-526.611){\fontsize{12}{1}\usefont{T1}{cmr}{m}{n}\selectfont\color{color_29791}Usare marcatori: ma marca la posizione con il simbolo “a”. y’a copia }
\put(185.8,-540.411){\fontsize{12}{1}\usefont{T1}{cmr}{m}{n}\selectfont\color{color_29791}nel buffer tutto il testo dalla posizione marcata “a” in precedenza fino }
\put(185.8,-554.211){\fontsize{12}{1}\usefont{T1}{cmr}{m}{n}\selectfont\color{color_29791}alla posizione corrente}
\put(167.8,-575.011){\fontsize{12}{1}\usefont{T1}{cmr}{m}{n}\selectfont\color{color_29791}Usando le ripetizioni: ci si posiziona sulla prima linea del blocco, per }
\put(185.8,-588.811){\fontsize{12}{1}\usefont{T1}{cmr}{m}{n}\selectfont\color{color_29791}copiare 10 righe si  digita 10yy}
\put(167.8,-609.611){\fontsize{12}{1}\usefont{T1}{cmr}{m}{n}\selectfont\color{color_29791}Usando la ricerca: ci si posiziona sulla prima linea del blocco, per }
\put(185.8,-623.411){\fontsize{12}{1}\usefont{T1}{cmr}{m}{n}\selectfont\color{color_29791}copiare fino alla parola ‘basta’ (esclusa) si digita y/basta<RET>}
\put(59.8,-644.211){\fontsize{12}{1}\usefont{T1}{cmr}{m}{n}\selectfont\color{color_29791}INPUT}
\put(95.8,-665.011){\fontsize{12}{1}\usefont{T1}{cmr}{m}{n}\selectfont\color{color_29791}oTutti i caratteri digitati vengono visualizzati ed inseriti nel testo}
\put(95.8,-686.811){\fontsize{12}{1}\usefont{T1}{cmr}{m}{n}\selectfont\color{color_29791}oSi passa a COMMAND con <ESC>}
\put(59.8,-708.611){\fontsize{12}{1}\usefont{T1}{cmr}{m}{n}\selectfont\color{color_29791}DIRECTIVE}
\put(95.8,-729.411){\fontsize{12}{1}\usefont{T1}{cmr}{m}{n}\selectfont\color{color_29791}oCi si trova posizionati con il cursore nella linea direttive (l’ultima linea del video) e }
\put(113.8,-744.211){\fontsize{12}{1}\usefont{T1}{cmr}{m}{n}\selectfont\color{color_29791}si possono richiedere tutti i comandi per il controllo del file}
\put(95.8,-765.011){\fontsize{12}{1}\usefont{T1}{cmr}{m}{n}\selectfont\color{color_29791}oSi passa a COMMAND con <RET>}
\end{picture}
\newpage
\begin{tikzpicture}[overlay]\path(0pt,0pt);\end{tikzpicture}
\begin{picture}(-5,0)(2.5,0)
\put(95.8,-85.01099){\fontsize{12}{1}\usefont{T1}{cmr}{m}{n}\selectfont\color{color_29791}oI comandi per caricare/salvare/uscire sono DIRECTIVE}
\put(131.8,-106.811){\fontsize{12}{1}\usefont{T1}{cmr}{m}{n}\selectfont\color{color_29791}.:r <file> inserisce il contenuto di file al punto del cursore}
\put(131.8,-127.611){\fontsize{12}{1}\usefont{T1}{cmr}{m}{n}\selectfont\color{color_29791}.:w scrive il file corrente}
\put(131.8,-148.411){\fontsize{12}{1}\usefont{T1}{cmr}{m}{n}\selectfont\color{color_29791}.:q esce (:q! per uscire senza salvare)}
\put(131.8,-169.211){\fontsize{12}{1}\usefont{T1}{cmr}{m}{n}\selectfont\color{color_29791}.ZZ scrive ed esce}
\put(95.8,-190.011){\fontsize{12}{1}\usefont{T1}{cmr}{m}{n}\selectfont\color{color_29791}oDigitando la barra / si entra in DIRECTIVE per cercare stringhe (es. /ciao<RET>)}
\put(131.8,-211.811){\fontsize{12}{1}\usefont{T1}{cmr}{m}{n}\selectfont\color{color_29791}.Con n si passa alla successiva occorrenza, con N alla precedente. Se si entra }
\put(149.8,-225.611){\fontsize{12}{1}\usefont{T1}{cmr}{m}{n}\selectfont\color{color_29791}con ? invece che con /, il verso di ricerca è verso l’inizio (e i significati di n e}
\put(149.8,-239.411){\fontsize{12}{1}\usefont{T1}{cmr}{b}{n}\selectfont\color{color_29791}N si adeguano)}
\put(131.8,-260.211){\fontsize{12}{1}\usefont{T1}{cmr}{m}{n}\selectfont\color{color_29791}./<RET> e ?<RET> ripetono ultima ricerca}
\put(131.8,-281.011){\fontsize{12}{1}\usefont{T1}{cmr}{m}{n}\selectfont\color{color_29791}.:s/trova/sostituisci/ cerca “trova” e lo sostituisce con “sostituisci”}
\put(131.8,-301.811){\fontsize{12}{1}\usefont{T1}{cmr}{m}{n}\selectfont\color{color_29791}.:\%s/trova/sostituisci/cgi cerca “trova” in ogni linea del file e lo sostituisce }
\put(149.8,-315.611){\fontsize{12}{1}\usefont{T1}{cmr}{m}{n}\selectfont\color{color_29791}con “sostituisci”}
\put(167.8,-336.411){\fontsize{12}{1}\usefont{T1}{cmr}{m}{n}\selectfont\color{color_29791}Dopo aver chiesto conferma (c)}
\put(167.8,-357.211){\fontsize{12}{1}\usefont{T1}{cmr}{m}{n}\selectfont\color{color_29791}Anche più volte nella stessa linea (g)}
\put(167.8,-378.011){\fontsize{12}{1}\usefont{T1}{cmr}{m}{n}\selectfont\color{color_29791}Case insensitive (i)}
\put(167.8,-398.811){\fontsize{12}{1}\usefont{T1}{cmr}{m}{n}\selectfont\color{color_29791}\% è una scorciatoia per 1,\$ , in realtà il comando può essere invocato }
\put(185.8,-412.611){\fontsize{12}{1}\usefont{T1}{cmr}{m}{n}\selectfont\color{color_29791}come :I,Fs/trova/sostituisci/ per applicarlo tra le linee I ed F}
\put(41.8,-433.411){\fontsize{12}{1}\usefont{T1}{cmr}{m}{n}\selectfont\color{color_29791}Digitando q<lettera minuscola> inizia la registrazione della macro identificata dalla lettera }
\put(41.8,-447.211){\fontsize{12}{1}\usefont{T1}{cmr}{m}{n}\selectfont\color{color_29791}indicata. Tutte le azioni compiute saranno registrate, finché non si preme nuovamente q per }
\put(41.8,-461.011){\fontsize{12}{1}\usefont{T1}{cmr}{m}{n}\selectfont\color{color_29791}terminare la registrazione. }
\put(41.8,-481.811){\fontsize{12}{1}\usefont{T1}{cmr}{m}{n}\selectfont\color{color_29791}Digitando @<lettera minuscola> viene invocata la macro. (Ovviamente ripetibile con }
\put(41.8,-495.611){\fontsize{12}{1}\usefont{T1}{cmr}{b}{it}\selectfont\color{color_29791}N@<lettera>) }
\put(41.8,-525.411){\fontsize{14.1}{1}\usefont{T1}{cmr}{b}{n}\selectfont\color{color_29791}Composizione comandi e filtri}
\put(41.8,-545.611){\fontsize{12}{1}\usefont{T1}{cmr}{m}{n}\selectfont\color{color_29791}Tutti i programmi *nix che lavorano su stream di testo (filtri) hanno 3 stream standard:}
\put(59.8,-566.411){\fontsize{12}{1}\usefont{T1}{cmr}{m}{n}\selectfont\color{color_29791}Fd 0 STDIN}
\put(59.8,-587.211){\fontsize{12}{1}\usefont{T1}{cmr}{m}{n}\selectfont\color{color_29791}Fd 1 STDOUT}
\put(59.8,-608.011){\fontsize{12}{1}\usefont{T1}{cmr}{m}{n}\selectfont\color{color_29791}Fd 2 STDERR}
\put(41.8,-628.811){\fontsize{12}{1}\usefont{T1}{cmr}{m}{n}\selectfont\color{color_29791}Bash può disconnettere gli stream predefiniti (chiudendoli nel figlio dopo la fork) e far trovare gli }
\put(41.8,-642.611){\fontsize{12}{1}\usefont{T1}{cmr}{m}{n}\selectfont\color{color_29791}stessi fd aperti su un file diverso (aprendolo prima della exec).}
\put(59.8,-663.411){\fontsize{12}{1}\usefont{T1}{cmr}{m}{n}\selectfont\color{color_29791}Ridirezione stdout:ls > miofile:  scrive lo stdout di ls su miofile, troncandolo. se miofile non }
\put(77.8,-677.211){\fontsize{12}{1}\usefont{T1}{cmr}{m}{n}\selectfont\color{color_29791}esiste viene creato. >> scrive in append}
\put(59.8,-698.011){\fontsize{12}{1}\usefont{T1}{cmr}{m}{n}\selectfont\color{color_29791}Ridirezione stderr: 2> e 2>>}
\put(59.8,-718.811){\fontsize{12}{1}\usefont{T1}{cmr}{m}{n}\selectfont\color{color_29791}Confluenza: ls > miofile 2>\&1: ridirige stderr dentro stdout, e poi stdout su file. L'ordine è }
\put(77.8,-732.611){\fontsize{12}{1}\usefont{T1}{cmr}{m}{n}\selectfont\color{color_29791}importante}
\put(59.8,-753.411){\fontsize{12}{1}\usefont{T1}{cmr}{m}{n}\selectfont\color{color_29791}Ridirezione stdin: sort < miofile riversa il contenuto di miofile su stdin di sort}
\end{picture}
\newpage
\begin{tikzpicture}[overlay]\path(0pt,0pt);\end{tikzpicture}
\begin{picture}(-5,0)(2.5,0)
\put(59.8,-85.01099){\fontsize{12}{1}\usefont{T1}{cmr}{m}{n}\selectfont\color{color_29791}Piping: ls | sort : bash  fa sì che ciò che ls produce su stdout venga riportato su stdin di sort}
\put(41.8,-121.011){\fontsize{17.5}{1}\usefont{T1}{cmr}{b}{n}\selectfont\color{color_29791}Filtri}
\put(41.8,-141.911){\fontsize{12}{1}\usefont{T1}{cmr}{m}{n}\selectfont\color{color_29791}La ridirezione è usata da comandi pensati per elaborare stream testo da stdin e produrre risultati in }
\put(41.8,-155.711){\fontsize{12}{1}\usefont{T1}{cmr}{m}{n}\selectfont\color{color_29791}stdout, i filtri. I più importanti sono: }
\put(62.1,-176.511){\fontsize{12}{1}\usefont{T1}{cmr}{m}{n}\selectfont\color{color_29791}cat: copia stdin su stdout. tac produce le righe in ordine inverso}
\put(62.1,-197.311){\fontsize{12}{1}\usefont{T1}{cmr}{m}{n}\selectfont\color{color_29791}less: non è un vero filtro essendo l’output destinato al terminale ma è utile per l’uso }
\put(80.1,-211.111){\fontsize{12}{1}\usefont{T1}{cmr}{m}{n}\selectfont\color{color_29791}interattivo. Posto al termine di una pipeline intercetta l’output e lo mostra riempiendo il }
\put(80.1,-224.911){\fontsize{12}{1}\usefont{T1}{cmr}{m}{n}\selectfont\color{color_29791}terminale, opzioni}
\put(98.1,-245.711){\fontsize{12}{1}\usefont{T1}{cmr}{m}{n}\selectfont\color{color_29791}o-h comandi disponibili}
\put(98.1,-267.511){\fontsize{12}{1}\usefont{T1}{cmr}{m}{n}\selectfont\color{color_29791}ofrecce – scorrimento}
\put(98.1,-289.311){\fontsize{12}{1}\usefont{T1}{cmr}{m}{n}\selectfont\color{color_29791}o<N>g si porta alla riga num <g> default:1}
\put(98.1,-311.111){\fontsize{12}{1}\usefont{T1}{cmr}{m}{n}\selectfont\color{color_29791}oG si porta al termine del file}
\put(98.1,-332.911){\fontsize{12}{1}\usefont{T1}{cmr}{m}{n}\selectfont\color{color_29791}o/<pattern> cerca la riga successiva al cursore contenente <pattern>}
\put(98.1,-354.711){\fontsize{12}{1}\usefont{T1}{cmr}{m}{n}\selectfont\color{color_29791}o?<pattern> cerca la riga successiva al cursore contenente <pattern>}
\put(98.1,-376.511){\fontsize{12}{1}\usefont{T1}{cmr}{m}{n}\selectfont\color{color_29791}on ripete la ricerca fatta in precedenza}
\put(98.1,-398.311){\fontsize{12}{1}\usefont{T1}{cmr}{m}{n}\selectfont\color{color_29791}oN ripete la ricerca fatta in precedenza, ma nel verso opposto}
\put(98.1,-420.111){\fontsize{12}{1}\usefont{T1}{cmr}{m}{n}\selectfont\color{color_29791}oq esce da less}
\put(62.1,-441.911){\fontsize{12}{1}\usefont{T1}{cmr}{m}{n}\selectfont\color{color_29791}rev: filtro che inverte l’ordine dei caratteri di ogni linea in stdin, su stdout. Usato di solito }
\put(80.1,-455.711){\fontsize{12}{1}\usefont{T1}{cmr}{m}{n}\selectfont\color{color_29791}con cut per estrarre campi la cui posizione è nota dal fine linea }
\put(62.1,-476.511){\fontsize{12}{1}\usefont{T1}{cmr}{m}{n}\selectfont\color{color_29791}head e tail: estraggono la parte iniziale/finale di un file. A default prendono 10 righe, con le }
\put(80.1,-490.311){\fontsize{12}{1}\usefont{T1}{cmr}{m}{n}\selectfont\color{color_29791}opzioni}
\put(98.1,-511.111){\fontsize{12}{1}\usefont{T1}{cmr}{m}{n}\selectfont\color{color_29791}o-c NUM i primi/ultimi NUM caratteri, con -/+NUM tutto il file eccetto/a partire da }
\put(116.1,-525.911){\fontsize{12}{1}\usefont{T1}{cmr}{m}{n}\selectfont\color{color_29791}NUM caratteri}
\put(98.1,-546.711){\fontsize{12}{1}\usefont{T1}{cmr}{m}{n}\selectfont\color{color_29791}o-n NUM prime/ultime NUM righe, con -/+ NUM tutto il file eccetto/a partire da }
\put(116.1,-561.511){\fontsize{12}{1}\usefont{T1}{cmr}{m}{n}\selectfont\color{color_29791}NUM righe}
\put(98.1,-582.311){\fontsize{12}{1}\usefont{T1}{cmr}{m}{n}\selectfont\color{color_29791}o-f per tail, mantiene il file aperto e mostra in tempo reale le ultime righe. Utile per }
\put(116.1,-597.111){\fontsize{12}{1}\usefont{T1}{cmr}{m}{n}\selectfont\color{color_29791}vedere output di processo che scrive su file. Per far ciò ignora EOF, quindi va usato }
\put(116.1,-610.911){\fontsize{12}{1}\usefont{T1}{cmr}{m}{n}\selectfont\color{color_29791}–pid=PID per terminare quando termina processo PID. –retry fa riprovare tail finché}
\put(116.1,-624.711){\fontsize{12}{1}\usefont{T1}{cmr}{m}{n}\selectfont\color{color_29791}non riesce ad aprire file quando produttore è in ritardo. -F == -f –retry}
\put(62.1,-645.511){\fontsize{12}{1}\usefont{T1}{cmr}{m}{n}\selectfont\color{color_29791}cut: taglia parti di righe. Opzioni}
\put(98.1,-666.311){\fontsize{12}{1}\usefont{T1}{cmr}{m}{n}\selectfont\color{color_29791}o-c ELENCO\_POSIZIONI\_CARATTERI intervallo per caratteri scelti (es. 1-8 per }
\put(116.1,-681.111){\fontsize{12}{1}\usefont{T1}{cmr}{m}{n}\selectfont\color{color_29791}ogni riga)}
\put(98.1,-701.911){\fontsize{12}{1}\usefont{T1}{cmr}{m}{n}\selectfont\color{color_29791}oPer i file a record (un record per riga), opzioni -d e -f per uno o più campi di ogni }
\put(116.1,-716.711){\fontsize{12}{1}\usefont{T1}{cmr}{m}{n}\selectfont\color{color_29791}record}
\put(134.1,-737.511){\fontsize{12}{1}\usefont{T1}{cmr}{m}{n}\selectfont\color{color_29791}.cut -d<CARATTERE\_DELIMITATORE> -f<ELENCO\_CAMPI>}
\put(134.1,-758.311){\fontsize{12}{1}\usefont{T1}{cmr}{m}{n}\selectfont\color{color_29791}.-s evita che vengano poste in output le righe senza delimitatore}
\end{picture}
\newpage
\begin{tikzpicture}[overlay]\path(0pt,0pt);\end{tikzpicture}
\begin{picture}(-5,0)(2.5,0)
\put(134.1,-85.01099){\fontsize{12}{1}\usefont{T1}{cmr}{m}{n}\selectfont\color{color_29791}.Es. cat /etc/passwd | cut -f1 -d: -s Estrae campo username da  file passwd}
\put(134.1,-105.811){\fontsize{12}{1}\usefont{T1}{cmr}{m}{n}\selectfont\color{color_29791}.Es2. cat /etc/passwd | cut -f5 -d: -s | cut -f2 -d’ ‘ | cut -c1 Se nel campo note }
\put(152.1,-119.611){\fontsize{12}{1}\usefont{T1}{cmr}{m}{n}\selectfont\color{color_29791}(campo 5) metto ‘Nome Cognome’ (separati da spazio), estrae dal cognome }
\put(152.1,-133.411){\fontsize{12}{1}\usefont{T1}{cmr}{m}{n}\selectfont\color{color_29791}degli utenti (campo 2) l’iniziale (carattere 1)}
\put(62.1,-154.211){\fontsize{12}{1}\usefont{T1}{cmr}{m}{n}\selectfont\color{color_29791}sort: ordina le linee di uno stream. L’ordine dei caratteri è stabilito dal locale scelto, con }
\put(80.1,-168.011){\fontsize{12}{1}\usefont{T1}{cmr}{m}{n}\selectfont\color{color_29791}opzione LC\_ALL=C è il valore dei byte che li codificano.}
\put(98.1,-188.811){\fontsize{12}{1}\usefont{T1}{cmr}{m}{n}\selectfont\color{color_29791}o-u elimina entry multiple (equivale a sort | uniq)}
\put(98.1,-210.611){\fontsize{12}{1}\usefont{T1}{cmr}{m}{n}\selectfont\color{color_29791}o-r reverse (ordinamento decrescente)}
\put(98.1,-232.411){\fontsize{12}{1}\usefont{T1}{cmr}{m}{n}\selectfont\color{color_29791}o-R random (permutazione casuale righe)}
\put(98.1,-254.211){\fontsize{12}{1}\usefont{T1}{cmr}{m}{n}\selectfont\color{color_29791}o-m merge di file già ordinato}
\put(98.1,-276.011){\fontsize{12}{1}\usefont{T1}{cmr}{m}{n}\selectfont\color{color_29791}o-c controlla se il file è già ordinato}
\put(41.8,-297.811){\fontsize{12}{1}\usefont{T1}{cmr}{m}{n}\selectfont\color{color_29791}Criteri aggiuntivi all’ordinamento di default}
\put(98.1,-318.611){\fontsize{12}{1}\usefont{T1}{cmr}{m}{n}\selectfont\color{color_29791}o-b ignora spazi a inizio riga}
\put(98.1,-340.411){\fontsize{12}{1}\usefont{T1}{cmr}{m}{n}\selectfont\color{color_29791}o-d considera solo i caratteri alfanumerici e gli spazi}
\put(98.1,-362.211){\fontsize{12}{1}\usefont{T1}{cmr}{m}{n}\selectfont\color{color_29791}o-f ignora maiusc/minusc}
\put(98.1,-384.011){\fontsize{12}{1}\usefont{T1}{cmr}{m}{n}\selectfont\color{color_29791}o-n interpreta stringhe di numeri per valore numerico}
\put(98.1,-405.811){\fontsize{12}{1}\usefont{T1}{cmr}{m}{n}\selectfont\color{color_29791}o-h interpreta numeri leggibili come 2K, 1G…}
\put(41.8,-427.611){\fontsize{12}{1}\usefont{T1}{cmr}{m}{n}\selectfont\color{color_29791}Può inoltre cercare le chiavi di ordinamento in posizioni specifiche della riga, senza considerarla }
\put(41.8,-441.411){\fontsize{12}{1}\usefont{T1}{cmr}{m}{n}\selectfont\color{color_29791}per intero. }
\put(98.1,-462.211){\fontsize{12}{1}\usefont{T1}{cmr}{m}{n}\selectfont\color{color_29791}o-t<SEP> imposta SEP come separatore, a default spazi}
\put(98.1,-484.011){\fontsize{12}{1}\usefont{T1}{cmr}{m}{n}\selectfont\color{color_29791}o-k<KEY> chiave di ordinamento, se usato più volte, ordina per la prima chiave, poi }
\put(116.1,-498.811){\fontsize{12}{1}\usefont{T1}{cmr}{m}{n}\selectfont\color{color_29791}la seconda…}
\put(98.1,-519.611){\fontsize{12}{1}\usefont{T1}{cmr}{m}{n}\selectfont\color{color_29791}oKEY è nella forma semplificata F[.C] [,F[.C]] [OPTS] dove}
\put(134.1,-541.411){\fontsize{12}{1}\usefont{T1}{cmr}{m}{n}\selectfont\color{color_29791}.F = numero di campo}
\put(134.1,-562.211){\fontsize{12}{1}\usefont{T1}{cmr}{m}{n}\selectfont\color{color_29791}.C = posizione in caratteri nel campo}
\put(134.1,-583.011){\fontsize{12}{1}\usefont{T1}{cmr}{m}{n}\selectfont\color{color_29791}.OPTS = una delle opzioni di ordinamento (n, f, d...)}
\put(41.8,-603.811){\fontsize{12}{1}\usefont{T1}{cmr}{m}{n}\selectfont\color{color_29791}Es. sort -t. -k 1,1n -k 2,2n -k 3,3n -k 4,4n Ordina un elenco di IP addess (byte1.b2.b3.b4)}
\put(62.1,-624.611){\fontsize{12}{1}\usefont{T1}{cmr}{m}{n}\selectfont\color{color_29791}uniq: elimina i duplicati consecutivi. Con -c indica il numero di righe compattate, con -d }
\put(80.1,-638.411){\fontsize{12}{1}\usefont{T1}{cmr}{m}{n}\selectfont\color{color_29791}mostra solo entry non singole}
\put(62.1,-659.211){\fontsize{12}{1}\usefont{T1}{cmr}{m}{n}\selectfont\color{color_29791}wc: con -c conta i caratteri, con -l le linee e con -w le parole (stringhe separate da spazi)}
\put(62.1,-680.011){\fontsize{12}{1}\usefont{T1}{cmr}{m}{n}\selectfont\color{color_29791}grep: esamina le righe in ingresso e produce in uscita quelle che matchano un pattern }
\put(80.1,-693.811){\fontsize{12}{1}\usefont{T1}{cmr}{m}{n}\selectfont\color{color_29791}passato come argomento (espressione regolare, nel caso più semplice una sottostringa).}
\put(41.8,-735.411){\fontsize{12}{1}\usefont{T1}{cmr}{m}{n}\selectfont\color{color_29791}Più usata è la variante egrep che permette espressioni regolari “moderne”. Corrisponde a grep -E}
\put(41.8,-756.211){\fontsize{12}{1}\usefont{T1}{cmr}{m}{n}\selectfont\color{color_29791}Segue il principio di “greediness” per selezionare i match:}
\end{picture}
\newpage
\begin{tikzpicture}[overlay]\path(0pt,0pt);\end{tikzpicture}
\begin{picture}(-5,0)(2.5,0)
\put(98.1,-85.01099){\fontsize{12}{1}\usefont{T1}{cmr}{m}{n}\selectfont\color{color_29791}oNel caso una RE corrisponda a più di una sottostringa in una stringa, la RE matcha }
\put(116.1,-99.81097){\fontsize{12}{1}\usefont{T1}{cmr}{m}{n}\selectfont\color{color_29791}quella che inizia per prima.}
\put(98.1,-120.611){\fontsize{12}{1}\usefont{T1}{cmr}{m}{n}\selectfont\color{color_29791}oA partire da quel punto, se la RE matcha più di una sottostringa, si seleziona la più }
\put(116.1,-135.411){\fontsize{12}{1}\usefont{T1}{cmr}{m}{n}\selectfont\color{color_29791}lunga}
\put(41.8,-156.211){\fontsize{12}{1}\usefont{T1}{cmr}{m}{n}\selectfont\color{color_29791}Nelle RE multilivello, le sottoespressioni selezionano sempre le sottostringhe più lunghe, tranne }
\put(41.8,-170.011){\fontsize{12}{1}\usefont{T1}{cmr}{m}{n}\selectfont\color{color_29791}quando l’intera corrispondenza è la più lunga possibile. La priorità viene data a sottoespressioni che}
\put(41.8,-183.811){\fontsize{12}{1}\usefont{T1}{cmr}{m}{n}\selectfont\color{color_29791}iniziano prima nella RE su quelle che iniziano dopo. }
\put(41.8,-204.611){\fontsize{12}{1}\usefont{T1}{cmr}{m}{n}\selectfont\color{color_29791}Opzioni principali di grep:}
\put(98.1,-225.411){\fontsize{12}{1}\usefont{T1}{cmr}{m}{n}\selectfont\color{color_29791}oControllo matching:}
\put(134.1,-247.211){\fontsize{12}{1}\usefont{T1}{cmr}{m}{n}\selectfont\color{color_29791}.-E usa extended RE (come egrep senza parametri)}
\put(134.1,-268.011){\fontsize{12}{1}\usefont{T1}{cmr}{m}{n}\selectfont\color{color_29791}.-F disattiva le RE e usa il parametro come stringa letterale}
\put(134.1,-288.811){\fontsize{12}{1}\usefont{T1}{cmr}{m}{n}\selectfont\color{color_29791}.-w/-x fa match solo con RE “whole word” o “whole line”}
\put(134.1,-309.611){\fontsize{12}{1}\usefont{T1}{cmr}{m}{n}\selectfont\color{color_29791}.-i rende l’espressione case insensitive}
\put(98.1,-330.411){\fontsize{12}{1}\usefont{T1}{cmr}{m}{n}\selectfont\color{color_29791}oControllo input}
\put(134.1,-352.211){\fontsize{12}{1}\usefont{T1}{cmr}{m}{n}\selectfont\color{color_29791}.-r cerca ricorsivamente nei file di una cartella}
\put(134.1,-373.011){\fontsize{12}{1}\usefont{T1}{cmr}{m}{n}\selectfont\color{color_29791}.-f FILE prende le RE da un FILE invece che come parametro}
\put(98.1,-393.811){\fontsize{12}{1}\usefont{T1}{cmr}{m}{n}\selectfont\color{color_29791}oControllo output}
\put(134.1,-415.611){\fontsize{12}{1}\usefont{T1}{cmr}{m}{n}\selectfont\color{color_29791}.-o restituisce solo le sottostringhe che corrispondono alla RE invece }
\put(152.1,-429.411){\fontsize{12}{1}\usefont{T1}{cmr}{m}{n}\selectfont\color{color_29791}della riga che le contiene, separatamente una per riga di output}
\put(134.1,-450.211){\fontsize{12}{1}\usefont{T1}{cmr}{m}{n}\selectfont\color{color_29791}.-v restituisce le linee che non contengono l’espressione}
\put(134.1,-471.011){\fontsize{12}{1}\usefont{T1}{cmr}{m}{n}\selectfont\color{color_29791}.-l utile passando a grep più file su cui cercare: restituisce solo i nomi }
\put(152.1,-484.811){\fontsize{12}{1}\usefont{T1}{cmr}{m}{n}\selectfont\color{color_29791}dei file in cui l’espressione è stata trovata}
\put(134.1,-505.611){\fontsize{12}{1}\usefont{T1}{cmr}{m}{n}\selectfont\color{color_29791}.-n restituisce anche il numero della riga contenente l’espressione}
\put(134.1,-526.411){\fontsize{12}{1}\usefont{T1}{cmr}{m}{n}\selectfont\color{color_29791}.-c restituisce solo il conteggio delle righe che contengono la RE}
\put(134.1,-547.211){\fontsize{12}{1}\usefont{T1}{cmr}{m}{n}\selectfont\color{color_29791}.-q quiet, non scrive niente su stdout e esce con 0 appena trova match }
\put(152.1,-561.011){\fontsize{12}{1}\usefont{T1}{cmr}{m}{n}\selectfont\color{color_29791}(anche se rileva errori)}
\put(134.1,-581.811){\fontsize{12}{1}\usefont{T1}{cmr}{m}{n}\selectfont\color{color_29791}.--line-buffered disattiva il buffering}
\put(41.8,-611.611){\fontsize{14.1}{1}\usefont{T1}{cmr}{b}{n}\selectfont\color{color_29791}Espressioni regolari moderne (o estese): regexp o RE, documentazione }
\put(41.8,-627.811){\fontsize{14.1}{1}\usefont{T1}{cmr}{m}{n}\selectfont\color{color_29791}in man page regex(7)}
\put(59.8,-648.011){\fontsize{12}{1}\usefont{T1}{cmr}{m}{n}\selectfont\color{color_29791}RE = uno o più rami non vuoti separati da |}
\put(59.8,-668.811){\fontsize{12}{1}\usefont{T1}{cmr}{m}{n}\selectfont\color{color_29791}Ramo = uno o più pezzi concatenati}
\put(59.8,-689.611){\fontsize{12}{1}\usefont{T1}{cmr}{m}{n}\selectfont\color{color_29791}Pezzo = atomo eventualmente con moltiplicatore}
\put(59.8,-710.411){\fontsize{12}{1}\usefont{T1}{cmr}{m}{n}\selectfont\color{color_29791}Atomo = uno di }
\put(95.8,-731.211){\fontsize{12}{1}\usefont{T1}{cmr}{m}{n}\selectfont\color{color_29791}o( ) contiene RE}
\put(95.8,-753.011){\fontsize{12}{1}\usefont{T1}{cmr}{m}{n}\selectfont\color{color_29791}o[ ] contiene charset}
\end{picture}
\newpage
\begin{tikzpicture}[overlay]\path(0pt,0pt);\end{tikzpicture}
\begin{picture}(-5,0)(2.5,0)
\put(95.8,-85.01099){\fontsize{12}{1}\usefont{T1}{cmr}{m}{n}\selectfont\color{color_29791}o\^ o \$ o .}
\put(95.8,-106.811){\fontsize{12}{1}\usefont{T1}{cmr}{m}{n}\selectfont\color{color_29791}oBackslash sequence}
\put(95.8,-128.611){\fontsize{12}{1}\usefont{T1}{cmr}{m}{n}\selectfont\color{color_29791}oSingolo carattere}
\put(41.8,-150.411){\fontsize{12}{1}\usefont{T1}{cmr}{m}{n}\selectfont\color{color_29791}Atomi speciali:}
\put(113.2,-171.211){\fontsize{12}{1}\usefont{T1}{cmr}{m}{n}\selectfont\color{color_29791}o.  qualsiasi carattere}
\put(113.2,-193.011){\fontsize{12}{1}\usefont{T1}{cmr}{m}{n}\selectfont\color{color_29791}o\^  inizio linea}
\put(113.2,-214.811){\fontsize{12}{1}\usefont{T1}{cmr}{m}{n}\selectfont\color{color_29791}o\$  fine linea}
\put(113.2,-236.611){\fontsize{12}{1}\usefont{T1}{cmr}{m}{n}\selectfont\color{color_29791}oBackslash sequence:}
\put(113.2,-258.411){\fontsize{12}{1}\usefont{T1}{cmr}{m}{n}\selectfont\color{color_29791}o\< - \>  stringa vuota a inizio/fine parola}
\put(113.2,-280.211){\fontsize{12}{1}\usefont{T1}{cmr}{m}{n}\selectfont\color{color_29791}o\b  stringa vuota a confine di parola}
\put(113.2,-302.011){\fontsize{12}{1}\usefont{T1}{cmr}{m}{n}\selectfont\color{color_29791}o\B stringa vuota a condizione che non sia confine di parola}
\put(113.2,-323.811){\fontsize{12}{1}\usefont{T1}{cmr}{m}{n}\selectfont\color{color_29791}o \w qualsiasi lettera o numero }
\put(113.2,-345.611){\fontsize{12}{1}\usefont{T1}{cmr}{m}{n}\selectfont\color{color_29791}o\W qualsiasi carattere non in \w}
\put(41.8,-367.411){\fontsize{12}{1}\usefont{T1}{cmr}{m}{n}\selectfont\color{color_29791}Moltiplicatori:}
\put(77.8,-388.211){\fontsize{12}{1}\usefont{T1}{cmr}{m}{n}\selectfont\color{color_29791}o\{n,m\} indica da n a m occorrenze dell’atomo che lo precede}
\put(77.8,-410.011){\fontsize{12}{1}\usefont{T1}{cmr}{m}{n}\selectfont\color{color_29791}o? indica zero o una occorrenze dell’atomo che lo precede}
\put(77.8,-431.811){\fontsize{12}{1}\usefont{T1}{cmr}{m}{n}\selectfont\color{color_29791}o* zero o più occorrenze atomo precedente}
\put(113.8,-453.611){\fontsize{12}{1}\usefont{T1}{cmr}{m}{n}\selectfont\color{color_29791}o+ una o più occorrenze atomo precedente}
\put(41.8,-475.411){\fontsize{12}{1}\usefont{T1}{cmr}{m}{n}\selectfont\color{color_29791}Esempi charset:}
\put(59.8,-496.211){\fontsize{12}{1}\usefont{T1}{cmr}{m}{n}\selectfont\color{color_29791}•[abc] UN qualsiasi carattere tra a,b o c}
\put(59.8,-517.011){\fontsize{12}{1}\usefont{T1}{cmr}{m}{n}\selectfont\color{color_29791}•[a-z] UN qualsiasi carattere tra a e z compresi}
\put(59.8,-537.811){\fontsize{12}{1}\usefont{T1}{cmr}{m}{n}\selectfont\color{color_29791}•[\^dc] UN qualsiasi carattere che non sia né d né c}
\put(41.8,-558.611){\fontsize{12}{1}\usefont{T1}{cmr}{m}{n}\selectfont\color{color_29791}Charset basati su character class [:NOME\_CLASSE:] dove NOME\_CLASSE appartiene }
\put(41.8,-572.411){\fontsize{12}{1}\usefont{T1}{cmr}{m}{n}\selectfont\color{color_29791}all’insieme definito in wctype(3) tipicamente: }
\put(77.3,-593.211){\fontsize{12}{1}\usefont{T1}{cmr}{b}{it}\selectfont\color{color_29791}alnumdigitpunct alpha graph space}
\put(77.3,-614.011){\fontsize{12}{1}\usefont{T1}{cmr}{b}{it}\selectfont\color{color_29791}blank lower upper cntrl print xdigit o eventualmente nel locale attivo}
\put(41.8,-634.811){\fontsize{12}{1}\usefont{T1}{cmr}{m}{n}\selectfont\color{color_29791}Esempi RE}
\put(41.8,-655.611){\fontsize{12}{1}\usefont{T1}{cmr}{m}{n}\selectfont\color{color_29791}Essendo molti caratteri speciali delle RE anche caratteri speciali shell, è necessario proteggerli }
\put(41.8,-669.411){\fontsize{12}{1}\usefont{T1}{cmr}{m}{n}\selectfont\color{color_29791}dall'espansione quando possibile, ponendo l’intera RE tra apici ‘’}
\put(59.8,-692.111){\fontsize{12}{1}\usefont{T1}{cmr}{m}{n}\selectfont\color{color_29791}•egrep ‘\^Nel.*vita\.\$’ miofileha come output tutte le righe di miofile che }
\put(77.8,-706.311){\fontsize{12}{1}\usefont{T1}{cmr}{m}{n}\selectfont\color{color_29791}iniziano per Nel e finiscono per vita. (notare il punto quotato con \)}
\put(59.8,-729.011){\fontsize{14}{1}\usefont{T1}{cmr}{m}{n}\selectfont\color{color_29791}•egrep ‘.es[\^es]\{3,5\}e’ miofile ha come output tutte le righe che contengono in }
\put(77.8,-743.211){\fontsize{12}{1}\usefont{T1}{cmr}{m}{n}\selectfont\color{color_29791}qualsiasi posizione la sequenza: }
\put(77.8,-764.011){\fontsize{12}{1}\usefont{T1}{cmr}{m}{n}\selectfont\color{color_29791}◦1 carattere qualsiasi (il punto)}
\end{picture}
\newpage
\begin{tikzpicture}[overlay]\path(0pt,0pt);\end{tikzpicture}
\begin{picture}(-5,0)(2.5,0)
\put(77.8,-85.01099){\fontsize{12}{1}\usefont{T1}{cmr}{m}{n}\selectfont\color{color_29791}◦es }
\put(77.8,-105.811){\fontsize{12}{1}\usefont{T1}{cmr}{m}{n}\selectfont\color{color_29791}◦una sequenza di 3-5 caratteri potenzialmente diversi uno dall’altro, a patto che ognuno }
\put(95.8,-119.611){\fontsize{12}{1}\usefont{T1}{cmr}{m}{n}\selectfont\color{color_29791}sia diverso da e ed s}
\put(77.8,-140.411){\fontsize{12}{1}\usefont{T1}{cmr}{m}{n}\selectfont\color{color_29791}◦e}
\put(41.8,-161.211){\fontsize{12}{1}\usefont{T1}{cmr}{m}{n}\selectfont\color{color_29791}Per altri esempi https://www.cyberciti.biz/faq/grep-regular-expressions/}
\end{picture}
\begin{tikzpicture}[overlay]
\path(0pt,0pt);
\draw[color_29919,line width=0.7pt]
(120.1pt, -162.311pt) -- (385.6pt, -162.311pt)
;
\end{tikzpicture}
\begin{picture}(-5,0)(2.5,0)
\put(41.8,-182.011){\fontsize{12}{1}\usefont{T1}{cmr}{b}{it}\selectfont\color{color_29791}tee legge stdin e lo scrive sia su stdout che sui file indicati.}
\put(59.8,-202.811){\fontsize{14}{1}\usefont{T1}{cmr}{m}{n}\selectfont\color{color_29791}•-a apre il file in append}
\put(59.8,-224.011){\fontsize{14}{1}\usefont{T1}{cmr}{m}{n}\selectfont\color{color_29791}•utile per tenere file intermedio comando1 | tee FILE | comando2 }
\put(59.8,-245.311){\fontsize{14}{1}\usefont{T1}{cmr}{m}{n}\selectfont\color{color_29791}•con process substitution permette anche varianti più complesse in cui ad ogni stadio pipeline}
\put(77.8,-259.511){\fontsize{12}{1}\usefont{T1}{cmr}{m}{n}\selectfont\color{color_29791}si elabora: ls | tee >(grep foo | wc > foo.count) | tee >(grep bar | wc > bar.count) | grep }
\put(77.8,-273.311){\fontsize{12}{1}\usefont{T1}{cmr}{m}{it}\selectfont\color{color_29791}baz | wc > baz.count}
\put(41.8,-294.111){\fontsize{12}{1}\usefont{T1}{cmr}{b}{it}\selectfont\color{color_217499}diff mostra le differenze tra due file}
\put(41.8,-314.911){\fontsize{12}{1}\usefont{T1}{cmr}{b}{it}\selectfont\color{color_217499}paste unisce righe di posizione omologa in più file }
\put(41.8,-335.711){\fontsize{12}{1}\usefont{T1}{cmr}{b}{it}\selectfont\color{color_217499}join funziona in modo simile a paste, ma seleziona le righe non in base alla posizione: unisce quelle}
\put(41.8,-349.511){\fontsize{12}{1}\usefont{T1}{cmr}{m}{n}\selectfont\color{color_217499}che iniziano con la stessa "chiave" (necessita di due file ordinati in modo identico sulla chiave }
\put(41.8,-363.311){\fontsize{12}{1}\usefont{T1}{cmr}{m}{n}\selectfont\color{color_217499}selezionata)}
\put(41.8,-384.111){\fontsize{12}{1}\usefont{T1}{cmr}{b}{it}\selectfont\color{color_35081}sed e awk non sono semplici filtri avendo un vero e proprio linguaggio di programmazione, }
\put(41.8,-397.911){\fontsize{12}{1}\usefont{T1}{cmr}{m}{n}\selectfont\color{color_35081}vediamo solo alcuni esempi pratici.}
\put(41.8,-418.711){\fontsize{12}{1}\usefont{T1}{cmr}{b}{it}\selectfont\color{color_35081}sed Stream EDitor, segue formato base sed -e 'comando' o sed -f 'script', non useremo script }
\put(41.8,-432.511){\fontsize{12}{1}\usefont{T1}{cmr}{m}{n}\selectfont\color{color_35081}generici ma il solo comando di sostituzione:}
\put(41.8,-453.311){\fontsize{12}{1}\usefont{T1}{cmr}{b}{it}\selectfont\color{color_35081}sed 's/VECCHIO\_PATTERN/NUOVO\_VALORE/[modificatori]' }
\put(41.8,-474.111){\fontsize{12}{1}\usefont{T1}{cmr}{m}{n}\selectfont\color{color_35081}sostituisce in ogni riga il NUOVO\_VALORE alla parte di testo coincidente con }
\put(41.8,-487.911){\fontsize{12}{1}\usefont{T1}{cmr}{m}{n}\selectfont\color{color_35081}VECCHIO\_PATTERN. Con sed -E i pattern sono circa quelli di egrep, es. (inserisce la stringa }
\put(41.8,-501.711){\fontsize{12}{1}\usefont{T1}{cmr}{m}{n}\selectfont\color{color_35081}“Linea:” all'inizio di ogni riga di passwd: cat /etc/passwd | sed 's/\^/Linea:/' ). I modificatori del }
\put(41.8,-515.511){\fontsize{12}{1}\usefont{T1}{cmr}{m}{n}\selectfont\color{color_35081}comando di sostituzione sono (si mettono dopo l'ultimo slash es. sed 's/\^/Linea:/i' }
\put(59.8,-536.311){\fontsize{12}{1}\usefont{T1}{cmr}{m}{n}\selectfont\color{color_35081}•i case insensitive}
\put(59.8,-557.111){\fontsize{12}{1}\usefont{T1}{cmr}{m}{n}\selectfont\color{color_35081}•g global (sostituisce tutte le occorrenze sulla riga)}
\put(59.8,-577.911){\fontsize{12}{1}\usefont{T1}{cmr}{m}{n}\selectfont\color{color_35081}•NUM sostituisce solo l’occorrenza NUM-esima}
\put(41.8,-598.711){\fontsize{12}{1}\usefont{T1}{cmr}{m}{n}\selectfont\color{color_35081}Opzioni su riga di comando:}
\put(59.8,-619.511){\fontsize{12}{1}\usefont{T1}{cmr}{m}{n}\selectfont\color{color_35081}•-i[SUFFIX] edita il file dato [backup con estensione SUFFIX se fornita]}
\put(59.8,-640.311){\fontsize{12}{1}\usefont{T1}{cmr}{m}{n}\selectfont\color{color_35081}•-u unbuffered}
\put(41.8,-661.111){\fontsize{12}{1}\usefont{T1}{cmr}{b}{it}\selectfont\color{color_35081}awk è un interprete per AWK (Turing-completo) che useremo solo come evoluzione di cut perché }
\put(41.8,-674.911){\fontsize{12}{1}\usefont{T1}{cmr}{m}{n}\selectfont\color{color_35081}permette di considerare qualsiasi sequenza di caratteri come unico delimitatore. Nell’uso più }
\put(41.8,-688.711){\fontsize{12}{1}\usefont{T1}{cmr}{m}{n}\selectfont\color{color_35081}comune permette di superare uno dei più evidenti limiti di cut in presenza di più delimitatori }
\put(41.8,-702.511){\fontsize{12}{1}\usefont{T1}{cmr}{m}{n}\selectfont\color{color_35081}consecutivi: ad esempio, cut -f2 -d' ' se ci sono due spazi dopo il primo campo considera il }
\put(41.8,-716.311){\fontsize{12}{1}\usefont{T1}{cmr}{m}{n}\selectfont\color{color_35081}2° spazio come 2° campo. }
\put(41.8,-737.111){\fontsize{12}{1}\usefont{T1}{cmr}{m}{n}\selectfont\color{color_35081}Esempi:}
\put(59.8,-757.911){\fontsize{12}{1}\usefont{T1}{cmr}{m}{n}\selectfont\color{color_35081}•stampa il secondo campo del file, purché sia separato dal primo da un numero qualunque di }
\put(77.8,-771.711){\fontsize{12}{1}\usefont{T1}{cmr}{m}{n}\selectfont\color{color_35081}blanks cat personale | awk '\{print \$2\}'}
\end{picture}
\newpage
\begin{tikzpicture}[overlay]\path(0pt,0pt);\end{tikzpicture}
\begin{picture}(-5,0)(2.5,0)
\put(59.8,-85.01099){\fontsize{12}{1}\usefont{T1}{cmr}{m}{n}\selectfont\color{color_35081}•in un file che riporta il risultato di un’operaz. come [stringhe…] stat=esito [stringhe …] }
\put(77.8,-98.81097){\fontsize{12}{1}\usefont{T1}{cmr}{m}{n}\selectfont\color{color_35081}estrae tutti gli esiti: cat log | awk –F 'stat=' '\{print \$2\}' | awk '\{print \$1\}' dove -F }
\put(77.8,-112.611){\fontsize{12}{1}\usefont{T1}{cmr}{m}{it}\selectfont\color{color_35081}‘stat=’ definisce il separatore dei campi}
\put(59.8,-133.411){\fontsize{12}{1}\usefont{T1}{cmr}{m}{n}\selectfont\color{color_35081}•a differenza di cut non ha il concetto di “-f 5-” (dal quinto campo in poi), ma: cat file | }
\put(77.8,-147.211){\fontsize{12}{1}\usefont{T1}{cmr}{b}{it}\selectfont\color{color_35081}awk '\{print substr(\$0, index(\$0,\$5)) \}'}
\put(41.8,-168.011){\fontsize{12}{1}\usefont{T1}{cmr}{b}{it}\selectfont\color{color_35081}tr è utile per sostituire più rapidamente singoli caratteri, senza regex. Esempi:}
\put(59.8,-188.811){\fontsize{12}{1}\usefont{T1}{cmr}{m}{n}\selectfont\color{color_35081}•tr 'A-Z' 'a-z' trasforma maiuscole in minuscole}
\put(59.8,-209.611){\fontsize{12}{1}\usefont{T1}{cmr}{m}{n}\selectfont\color{color_35081}•tr ';:.!?' ',' sostituisce qualsiasi occorrenza dei caratteri nel primo set con ,}
\put(59.8,-230.411){\fontsize{12}{1}\usefont{T1}{cmr}{m}{n}\selectfont\color{color_35081}•in generale, se il secondo set è più limitato del primo set, il suo ultimo carattere viene }
\put(77.8,-244.211){\fontsize{12}{1}\usefont{T1}{cmr}{m}{n}\selectfont\color{color_35081}ripetuto quanto basta a generare la corrispondenza 1:1 }
\put(77.8,-265.011){\fontsize{12}{1}\usefont{T1}{cmr}{m}{n}\selectfont\color{color_35081}◦tr ';:.!?' ',-' In questo caso quindi ; → ,: → -. → -! → -? → -}
\put(59.8,-285.811){\fontsize{12}{1}\usefont{T1}{cmr}{m}{n}\selectfont\color{color_29791}•tr -d '\r' elimina ogni occorrenza del carriage return}
\put(41.8,-306.611){\fontsize{12}{1}\usefont{T1}{cmr}{b}{it}\selectfont\color{color_217499}xargs <comando> si aspetta su stdin un elenco di stringhe, da dare come argomento al comando }
\put(41.8,-320.411){\fontsize{12}{1}\usefont{T1}{cmr}{m}{n}\selectfont\color{color_217499}indicato. Es. find /usr/src -name '*.c' -size +100k -print | xargs catlancia cat con tutti i file }
\put(41.8,-334.211){\fontsize{12}{1}\usefont{T1}{cmr}{m}{n}\selectfont\color{color_217499}risultanti che find mette su stdin }
\put(41.8,-370.211){\fontsize{17.5}{1}\usefont{T1}{cmr}{b}{n}\selectfont\color{color_29791}Shell e gestione processi}
\put(41.8,-391.111){\fontsize{12}{1}\usefont{T1}{cmr}{m}{n}\selectfont\color{color_29791}La shell, in particolare l’incarnazione bash che studiamo, nasce per automatizzare task, evitando di }
\put(41.8,-404.911){\fontsize{12}{1}\usefont{T1}{cmr}{m}{n}\selectfont\color{color_29791}richiedere l’inserimento di comandi manuale. Si pone quindi non come ambiente per codice general}
\put(41.8,-418.711){\fontsize{12}{1}\usefont{T1}{cmr}{m}{n}\selectfont\color{color_29791}purpose: lo scopo fondamentale è avviare processi di sistema, predisporre flussi di comunicazione }
\put(41.8,-432.511){\fontsize{12}{1}\usefont{T1}{cmr}{m}{n}\selectfont\color{color_29791}tra questi, controllare come terminano e che diagnostica producono. Gli aspetti principali da }
\put(41.8,-446.311){\fontsize{12}{1}\usefont{T1}{cmr}{m}{n}\selectfont\color{color_29791}considerare rispetto a un linguaggio come C o Java sono due: il primo è che gli elementi di base }
\put(41.8,-460.111){\fontsize{12}{1}\usefont{T1}{cmr}{m}{n}\selectfont\color{color_29791}gestiti sono file e processi, (quindi è fondamentale pensare, quando si scrive o analizza una riga di }
\put(41.8,-473.911){\fontsize{12}{1}\usefont{T1}{cmr}{m}{n}\selectfont\color{color_29791}comando, a quali processi verranno eseguiti e quali file sono coinvolti); il secondo è che bash è un }
\put(41.8,-487.711){\fontsize{12}{1}\usefont{T1}{cmr}{m}{n}\selectfont\color{color_29791}linguaggio interpretato e non compilato, quindi la riga di comando passa attraverso un processo }
\put(41.8,-501.511){\fontsize{12}{1}\usefont{T1}{cmr}{m}{n}\selectfont\color{color_29791}detto espansione, che vedremo più avanti, prima di essere eseguita.}
\put(41.8,-522.311){\fontsize{12}{1}\usefont{T1}{cmr}{m}{n}\selectfont\color{color_29791}In ambiente UNIX i processi sono processi pesanti, e possono essere creati usando}
\put(59.8,-543.111){\fontsize{12}{1}\usefont{T1}{cmr}{m}{n}\selectfont\color{color_29791}•fork, che crea una copia del processo corrente duplicando tutte le risorse (inclusi file) e }
\put(77.8,-556.911){\fontsize{12}{1}\usefont{T1}{cmr}{m}{n}\selectfont\color{color_29791}condivide il codice}
\put(59.8,-577.711){\fontsize{12}{1}\usefont{T1}{cmr}{m}{n}\selectfont\color{color_29791}•exec, che sostituisce il codice del processo con quello caricato da un programma }
\put(77.8,-591.511){\fontsize{12}{1}\usefont{T1}{cmr}{m}{n}\selectfont\color{color_29791}modificando la text table}
\put(41.8,-612.311){\fontsize{12}{1}\usefont{T1}{cmr}{m}{n}\selectfont\color{color_29791}Quando si lancia un programma quindi la prima cosa che accade è che viene duplicato il processo }
\put(41.8,-626.111){\fontsize{12}{1}\usefont{T1}{cmr}{m}{n}\selectfont\color{color_29791}bash con tutte le sue risorse. Questo segue direttamente dal processo d’avvio del terminale: una }
\put(41.8,-639.911){\fontsize{12}{1}\usefont{T1}{cmr}{m}{n}\selectfont\color{color_29791}volta che il kernel ha inizializzato i dispositivi HW e li ha esposti come device driver (/dev/tty* per }
\put(41.8,-653.711){\fontsize{12}{1}\usefont{T1}{cmr}{m}{n}\selectfont\color{color_29791}i terminali virtuali collegati alla console, /dev/pts/* per pseudoterminali collegati a finestre }
\put(41.8,-667.511){\fontsize{12}{1}\usefont{T1}{cmr}{m}{n}\selectfont\color{color_29791}grafiche), lancia init e smette di occuparsi dell’avvio, lasciando a init il compito di lanciare tutti gli }
\put(41.8,-681.311){\fontsize{12}{1}\usefont{T1}{cmr}{m}{n}\selectfont\color{color_29791}altri processi. Il primo che parte è getty che fa una open su tty0 in lettura e una in scrittura (il }
\put(41.8,-695.111){\fontsize{12}{1}\usefont{T1}{cmr}{m}{n}\selectfont\color{color_29791}processo ha quindi ora un fd0 da cui può leggere da tastiera e fd1 e 2 con cui può scrivere sullo }
\put(41.8,-708.911){\fontsize{12}{1}\usefont{T1}{cmr}{m}{n}\selectfont\color{color_29791}schermo). Getty dopo la config del terminale fa exec e si trasforma in login, che attende inserimento}
\put(41.8,-722.711){\fontsize{12}{1}\usefont{T1}{cmr}{m}{n}\selectfont\color{color_29791}user:pass avendo ereditato i fd da login. Una volta inseriti us:pw, il processo fa fork/exec e lancia }
\put(41.8,-736.511){\fontsize{12}{1}\usefont{T1}{cmr}{m}{n}\selectfont\color{color_29791}bash, che a sua volta eredita i fd. Da questo momento in poi ogni comando aperto da bash segue la }
\put(41.8,-750.311){\fontsize{12}{1}\usefont{T1}{cmr}{m}{n}\selectfont\color{color_29791}stessa filosofia, ereditando i fd dopo fork(nuova bash figlia)/exec(cambio codice).}
\end{picture}
\newpage
\begin{tikzpicture}[overlay]\path(0pt,0pt);\end{tikzpicture}
\begin{picture}(-5,0)(2.5,0)
\put(41.8,-85.01099){\fontsize{12}{1}\usefont{T1}{cmr}{m}{n}\selectfont\color{color_29791}Per convenzione tutti i comandi *nix che operano su stream di testo (filtri) sono progettati per }
\put(41.8,-98.81097){\fontsize{12}{1}\usefont{T1}{cmr}{m}{n}\selectfont\color{color_29791}disporre di tre stream con cui comunicare con il resto del sistema: stdin su fd0, stdout su fd1, stderr }
\put(41.8,-112.611){\fontsize{12}{1}\usefont{T1}{cmr}{m}{n}\selectfont\color{color_29791}su fd2. }
\put(41.8,-133.411){\fontsize{12}{1}\usefont{T1}{cmr}{m}{it}\selectfont\color{color_29791}A=40 mycommand | othercommand > outfile}
\put(41.8,-154.211){\fontsize{12}{1}\usefont{T1}{cmr}{m}{n}\selectfont\color{color_29791}L’interpretazione della linea di comando passa attraverso fasi ordinate:}
\put(59.8,-175.011){\fontsize{12}{1}\usefont{T1}{cmr}{m}{n}\selectfont\color{color_29791}•Espansione degli elementi della riga (descritta in seguito, la shell interpreta alcuni caratteri }
\put(77.8,-188.811){\fontsize{12}{1}\usefont{T1}{cmr}{m}{n}\selectfont\color{color_29791}speciali, usati per gestire come eseguire) \{mycommand, othercommand\}}
\put(59.8,-209.611){\fontsize{12}{1}\usefont{T1}{cmr}{m}{n}\selectfont\color{color_29791}•Preparazione ridirezioni \{| , > outfile\}}
\put(59.8,-230.411){\fontsize{12}{1}\usefont{T1}{cmr}{m}{n}\selectfont\color{color_29791}•Assegnamento di variabili \{A=40\}}
\put(59.8,-251.211){\fontsize{12}{1}\usefont{T1}{cmr}{m}{n}\selectfont\color{color_29791}•Esecuzione comandi}
\put(41.8,-281.011){\fontsize{14.1}{1}\usefont{T1}{cmr}{b}{n}\selectfont\color{color_29791}Ridirezione}
\put(41.8,-301.211){\fontsize{12}{1}\usefont{T1}{cmr}{m}{n}\selectfont\color{color_29791}La ridirezione per un processo figlio di shell consiste nel disconnettere gli stream predefiniti dal }
\put(41.8,-315.011){\fontsize{12}{1}\usefont{T1}{cmr}{m}{n}\selectfont\color{color_29791}terminale (chiudendoli dopo la fork, nel figlio) e far trovare gli stessi fd aperti su un file diverso }
\put(41.8,-328.811){\fontsize{12}{1}\usefont{T1}{cmr}{m}{n}\selectfont\color{color_29791}(aprendoli prima della exec). Questo è possibile perché quando lancia un comando, la shell per }
\put(41.8,-342.611){\fontsize{12}{1}\usefont{T1}{cmr}{m}{n}\selectfont\color{color_29791}prima cosa duplica sé stessa e quindi il figlio avrà gli stessi fd, facendo close dopo la fork e open di }
\put(41.8,-356.411){\fontsize{12}{1}\usefont{T1}{cmr}{m}{n}\selectfont\color{color_29791}un file prima della exec il SO porrà quel file nel primo fd disponibile. Una volta fatta la exec il }
\put(41.8,-370.211){\fontsize{12}{1}\usefont{T1}{cmr}{m}{n}\selectfont\color{color_29791}processo avrà gli fd già settati e aperti come da modello, e per lui non farà differenza dove puntino.}
\put(59.8,-391.011){\fontsize{12}{1}\usefont{T1}{cmr}{m}{n}\selectfont\color{color_29791}•L’stdout si ridirige con > myfile per aprire myfile da file pointer 0 (quindi troncandolo o }
\put(71.2,-404.811){\fontsize{12}{1}\usefont{T1}{cmr}{m}{n}\selectfont\color{color_29791}creandolo), con >> per aprirlo da file pointer in fondo al file; stderr si ridirige con 2> o 2>>; }
\put(71.2,-418.611){\fontsize{12}{1}\usefont{T1}{cmr}{m}{n}\selectfont\color{color_29791}stdin con command < myfile porta il contenuto di myfile su stdin di command. }
\put(59.8,-439.411){\fontsize{12}{1}\usefont{T1}{cmr}{m}{n}\selectfont\color{color_29791}•Particolare è la confluenza: ls > myfile 2>\&1 prima ridirige stderr di ls su stdout (\&1 indica }
\put(71.2,-453.211){\fontsize{12}{1}\usefont{T1}{cmr}{m}{n}\selectfont\color{color_29791}proprio fd1, non un file di nome 1); poi stdout su myfile. Il risultato è che entrambi gli stream }
\put(71.2,-467.011){\fontsize{12}{1}\usefont{T1}{cmr}{m}{n}\selectfont\color{color_29791}finiranno su myfile, ma l’ordine di ridirezione è importante, hanno priorità quelle fatte più a }
\put(71.2,-480.811){\fontsize{12}{1}\usefont{T1}{cmr}{m}{n}\selectfont\color{color_29791}destra.}
\put(59.8,-501.611){\fontsize{12}{1}\usefont{T1}{cmr}{m}{n}\selectfont\color{color_29791}• Inoltre si può evitare l’uso di file temporanei per inviare dati su stdin di un comando, usando }
\put(71.2,-515.411){\fontsize{12}{1}\usefont{T1}{cmr}{m}{it}\selectfont\color{color_29791}command <<MARCATORE (scrivendo poi più linee, e chiudendo l’input con una linea }
\put(71.2,-529.211){\fontsize{12}{1}\usefont{T1}{cmr}{m}{n}\selectfont\color{color_29791}MARCATORE) o command <<< “testo” per un sola linea. }
\put(59.8,-550.011){\fontsize{12}{1}\usefont{T1}{cmr}{m}{n}\selectfont\color{color_29791}•Per ridirigere stream permanentemente si usa exec [ridir], che ridirige gli stessi stream della }
\put(71.2,-563.811){\fontsize{12}{1}\usefont{T1}{cmr}{m}{n}\selectfont\color{color_29791}shell quindi tutti i comandi successivi avranno quegli fd (2>/dev/null, che toglie stderr da }
\put(71.2,-577.611){\fontsize{12}{1}\usefont{T1}{cmr}{m}{n}\selectfont\color{color_29791}terminale. interattivamente fa sparire anche prompt e echo ma è utile negli script). }
\put(59.8,-598.411){\fontsize{12}{1}\usefont{T1}{cmr}{m}{n}\selectfont\color{color_29791}•Con exec si possono anche creare nuovi fd: exec 3< fin 4> fout 5<> frw permette letture con }
\put(71.2,-612.211){\fontsize{12}{1}\usefont{T1}{cmr}{m}{n}\selectfont\color{color_29791}<\&3 su fin, scritture con >\&4 su fout e frw sia in lettura che in scrittura; per chiudere un fd si }
\put(71.2,-626.011){\fontsize{12}{1}\usefont{T1}{cmr}{m}{n}\selectfont\color{color_29791}usa exec fd>\&-.}
\put(41.8,-655.811){\fontsize{14.1}{1}\usefont{T1}{cmr}{b}{n}\selectfont\color{color_29791}Pipe e subshell}
\put(41.8,-676.011){\fontsize{12}{1}\usefont{T1}{cmr}{m}{n}\selectfont\color{color_29791}La bash pipe è un array di due fd [0,1], con elemento 1 scrivibile e tutti ciò che viene scritto su 1 }
\put(41.8,-689.811){\fontsize{12}{1}\usefont{T1}{cmr}{m}{n}\selectfont\color{color_29791}può essere letto da 0 (0 leggibile 1 scrivibile come standard). Quando c’è una pipeline (Es. ls | sort) }
\put(59.8,-710.611){\fontsize{12}{1}\usefont{T1}{cmr}{m}{n}\selectfont\color{color_29791}•bash fa due fork (due bash figlie, ognuna con 3 fd in lettura e scrittura, standard + i due }
\put(77.8,-724.411){\fontsize{12}{1}\usefont{T1}{cmr}{m}{n}\selectfont\color{color_29791}pipe)}
\end{picture}
\newpage
\begin{tikzpicture}[overlay]\path(0pt,0pt);\end{tikzpicture}
\begin{picture}(-5,0)(2.5,0)
\put(59.8,-85.01099){\fontsize{12}{1}\usefont{T1}{cmr}{m}{n}\selectfont\color{color_29791}•la prima shell figlia chiama dup2(fd[1],1] per dirottare il suo stdout su fd[1] (quello della }
\put(77.8,-98.81097){\fontsize{12}{1}\usefont{T1}{cmr}{m}{n}\selectfont\color{color_29791}pipe) e viceversa la seconda chiama dup2(fd[0],0) per dirottare sul suo stdin quello della }
\put(77.8,-112.611){\fontsize{12}{1}\usefont{T1}{cmr}{m}{n}\selectfont\color{color_29791}pipe}
\put(59.8,-133.411){\fontsize{12}{1}\usefont{T1}{cmr}{m}{n}\selectfont\color{color_29791}•infine la prima chiude fd[0] e la seconda fd[1] per evitare inconsistenze sulla pipe, }
\put(59.8,-154.211){\fontsize{12}{1}\usefont{T1}{cmr}{m}{n}\selectfont\color{color_29791}•poi ognuna fa exec del suo comando facendoli partire con gli fd già settati in modo da avere }
\put(77.8,-168.011){\fontsize{12}{1}\usefont{T1}{cmr}{m}{n}\selectfont\color{color_29791}trasparenza senza sapere della pipe. }
\put(41.8,-188.811){\fontsize{12}{1}\usefont{T1}{cmr}{m}{n}\selectfont\color{color_29791}La pipe è quindi un mezzo di comunicazione tra processi fornito dal SO, il buffering offerto da }
\put(41.8,-202.611){\fontsize{12}{1}\usefont{T1}{cmr}{m}{n}\selectfont\color{color_29791}questo permette di passare dati dal primo al secondo processo senza usare file temporanei. È il SO a}
\put(41.8,-216.411){\fontsize{12}{1}\usefont{T1}{cmr}{m}{n}\selectfont\color{color_29791}occuparsi della sincronizzazione: se il secondo comando è lento a svuotare il buffer e questo si }
\put(41.8,-230.211){\fontsize{12}{1}\usefont{T1}{cmr}{m}{n}\selectfont\color{color_29791}riempie, quando il primo chiama syscall write viene reso non schedulabile (sospeso) dal SO. }
\put(41.8,-244.011){\fontsize{12}{1}\usefont{T1}{cmr}{m}{n}\selectfont\color{color_29791}Appena il secondo fa una syscall read e il buffer si libera, il SO va a richiamare il primo. }
\put(41.8,-264.811){\fontsize{12}{1}\usefont{T1}{cmr}{m}{n}\selectfont\color{color_29791}Caso particolare è quando il secondo elemento di una pipeline è un builtin o una funzione, in quanto}
\put(41.8,-278.611){\fontsize{12}{1}\usefont{T1}{cmr}{m}{n}\selectfont\color{color_29791}non va caricato codice binario ma vanno interpretati da una shell: il secondo figlio non chiamerà }
\put(41.8,-292.411){\fontsize{12}{1}\usefont{T1}{cmr}{m}{it}\selectfont\color{color_29791}exec e quindi resterà bash. Una pipeline con builtin crea quindi una subshell implicita, che è ancora }
\put(41.8,-306.211){\fontsize{12}{1}\usefont{T1}{cmr}{m}{n}\selectfont\color{color_29791}un processo pesante (coincide il text segment avendo accesso alle stesse istruzioni, ma ha suo }
\put(41.8,-320.011){\fontsize{12}{1}\usefont{T1}{cmr}{m}{n}\selectfont\color{color_29791}program counter), quindi ha i propri dati: variabili ereditate dal padre e poi modificate nel figlio non}
\put(41.8,-333.811){\fontsize{12}{1}\usefont{T1}{cmr}{m}{n}\selectfont\color{color_29791}riflettono le modifiche sulle variabili nel padre (processi pesanti, copia per valore). }
\put(41.8,-354.611){\fontsize{12}{1}\usefont{T1}{cmr}{m}{n}\selectfont\color{color_29791}Si può forzare la creazione di subshell per eseguire sequenze di comandi nello stesso processo bash }
\put(41.8,-368.411){\fontsize{12}{1}\usefont{T1}{cmr}{m}{n}\selectfont\color{color_29791}usando (comando1 ; comando 2; …) dove ; equivale ad un accapo (eseguiti in ordine). Ciò è utile }
\put(41.8,-382.211){\fontsize{12}{1}\usefont{T1}{cmr}{m}{n}\selectfont\color{color_29791}ad esempio quando nei processi figli modifico variabili d'ambiente: le modifiche si vedrebbero sul }
\put(41.8,-396.011){\fontsize{12}{1}\usefont{T1}{cmr}{m}{n}\selectfont\color{color_29791}comportamento della shell padre, quindi preferisco lanciare shell figlie. Tutto ciò che viene dato su }
\put(41.8,-409.811){\fontsize{12}{1}\usefont{T1}{cmr}{m}{n}\selectfont\color{color_29791}stdin della subshell è disponibile su stdin dei comandi e ciò che questi producono su stdout/err è }
\put(41.8,-423.611){\fontsize{12}{1}\usefont{T1}{cmr}{m}{n}\selectfont\color{color_29791}prodotto dagli stream corrispondenti della subshell (Es. producer | ( step1 ; step2 ; step3 ) }
\put(41.8,-437.411){\fontsize{12}{1}\usefont{T1}{cmr}{m}{it}\selectfont\color{color_29791}2>/dev/null | consumer dove producer legge stdin da tastiera non avendo <, scrive stdout su stdin }
\put(41.8,-451.211){\fontsize{12}{1}\usefont{T1}{cmr}{m}{n}\selectfont\color{color_29791}della subshell che esegue i 3 comandi in ordine; ognuno dei 3 ha facoltà di scrivere sui propri std, e }
\put(41.8,-465.011){\fontsize{12}{1}\usefont{T1}{cmr}{m}{n}\selectfont\color{color_29791}tutte le scritture su stderr della subshell finiscono su /dev/null. Infine l’stdout di subshell finisce su }
\put(41.8,-478.811){\fontsize{12}{1}\usefont{T1}{cmr}{m}{n}\selectfont\color{color_29791}stdin di consumer).}
\put(41.8,-508.611){\fontsize{14.1}{1}\usefont{T1}{cmr}{b}{n}\selectfont\color{color_29791}Interazione coi processi e segnali}
\put(41.8,-528.811){\fontsize{12}{1}\usefont{T1}{cmr}{m}{n}\selectfont\color{color_29791}Ogni comando lanciato da shell diventa un processo, identificato da un Process ID o PID globale (in}
\put(41.8,-542.611){\fontsize{12}{1}\usefont{T1}{cmr}{m}{n}\selectfont\color{color_29791}alcuni casi anche da un JobID, localmente alla shell che l’ha lanciato; anche se usando ps su altra }
\put(41.8,-556.411){\fontsize{12}{1}\usefont{T1}{cmr}{m}{n}\selectfont\color{color_29791}shell è visibile con lo stesso PID). Un processo svolge le sue azioni a nome dell’utente che l’ha }
\put(41.8,-570.211){\fontsize{12}{1}\usefont{T1}{cmr}{m}{n}\selectfont\color{color_29791}lanciato (processi root possono assumere altre identità ma perdono il potere di tornare indietro) e }
\put(41.8,-584.011){\fontsize{12}{1}\usefont{T1}{cmr}{m}{n}\selectfont\color{color_29791}processi anche non lanciati da una stessa pipeline possono comunicare tra loro, mediante strumenti }
\put(41.8,-597.811){\fontsize{12}{1}\usefont{T1}{cmr}{m}{n}\selectfont\color{color_29791}da preparare come socket, o in modo limitato ma semplice usando segnali.}
\put(41.8,-618.611){\fontsize{12}{1}\usefont{T1}{cmr}{m}{n}\selectfont\color{color_29791}I segnali sono eventi asincroni notificati dal kernel a un processo, generati dal kernel stesso (Es. }
\put(41.8,-632.411){\fontsize{12}{1}\usefont{T1}{cmr}{m}{n}\selectfont\color{color_29791}SIGPIPE quando termina un processo in lettura su una pipe, così che l’altro possa smettere di }
\put(41.8,-646.211){\fontsize{12}{1}\usefont{T1}{cmr}{m}{n}\selectfont\color{color_29791}scrivere) o da un altro processo; il cui contenuto informativo è limitato ad un numero. Non vengono}
\put(41.8,-660.011){\fontsize{12}{1}\usefont{T1}{cmr}{m}{n}\selectfont\color{color_29791}ricevuti istantaneamente, il controllo avviene ogni volta che il processo rientra in user space (Es. }
\put(41.8,-673.811){\fontsize{12}{1}\usefont{T1}{cmr}{m}{n}\selectfont\color{color_29791}dopo una syscall o quando schedulato da CPU): se tra un controllo e il successivo sono stati ricevuti}
\put(41.8,-687.611){\fontsize{12}{1}\usefont{T1}{cmr}{m}{n}\selectfont\color{color_29791}più segnali diversi, vengono posti in uno stato “pending” senza notificare ricezioni multiple (viene }
\put(41.8,-701.411){\fontsize{12}{1}\usefont{T1}{cmr}{m}{n}\selectfont\color{color_29791}settato il flag pending nel process descriptor nella process table, identificatore dei segnali ricevuti }
\put(41.8,-715.211){\fontsize{12}{1}\usefont{T1}{cmr}{m}{n}\selectfont\color{color_29791}ma non gestiti) e vengono poi gestiti in modo non deterministico (come sviluppatore di script non si}
\put(41.8,-729.011){\fontsize{12}{1}\usefont{T1}{cmr}{m}{n}\selectfont\color{color_29791}può assumere che vengano ricevuti nello stesso ordine in cui sono stati inviati). La gestione a livello}
\put(41.8,-742.811){\fontsize{12}{1}\usefont{T1}{cmr}{m}{n}\selectfont\color{color_29791}di SO avviene mediante l’esecuzione automatica di handler, dirottando il flusso di esecuzione di un }
\put(41.8,-756.611){\fontsize{12}{1}\usefont{T1}{cmr}{m}{n}\selectfont\color{color_29791}dato processo, in seguito alla rilevazione di segnale pending diretto a tale processo. L’esecuzione di }
\end{picture}
\newpage
\begin{tikzpicture}[overlay]\path(0pt,0pt);\end{tikzpicture}
\begin{picture}(-5,0)(2.5,0)
\put(41.8,-85.01099){\fontsize{12}{1}\usefont{T1}{cmr}{m}{n}\selectfont\color{color_29791}un handler blocca segnali dello stesso tipo, ricezioni durante l’handler non causano esecuzioni }
\put(41.8,-98.81097){\fontsize{12}{1}\usefont{T1}{cmr}{m}{n}\selectfont\color{color_29791}annidate ma settano il flag.}
\put(41.8,-119.611){\fontsize{12}{1}\usefont{T1}{cmr}{m}{n}\selectfont\color{color_29791}Il comportamento di un processo alla ricezione di un segnale può essere terminare, ignorarlo, }
\put(41.8,-133.411){\fontsize{12}{1}\usefont{T1}{cmr}{m}{n}\selectfont\color{color_29791}sospendersi (stato stop) o riprendersi da stop (cont), le disposition predefinite sono in signal(7). La }
\put(41.8,-147.211){\fontsize{12}{1}\usefont{T1}{cmr}{m}{n}\selectfont\color{color_29791}disposition per un processo può essere modificata: opzione di default; ignorare il segnale; eseguire }
\put(41.8,-161.011){\fontsize{12}{1}\usefont{T1}{cmr}{m}{n}\selectfont\color{color_29791}un handler. Fanno eccezione KILL e STOP, che non possono essere bloccati, ignorati, o intercettati }
\put(41.8,-174.811){\fontsize{12}{1}\usefont{T1}{cmr}{m}{n}\selectfont\color{color_29791}da un handler. }
\put(41.8,-195.611){\fontsize{12}{1}\usefont{T1}{cmr}{m}{n}\selectfont\color{color_29791}Gli handler possono essere definiti in bash usando il builtin trap (trap [-lp] [[codice\_da\_eseguire] }
\put(41.8,-209.411){\fontsize{12}{1}\usefont{T1}{cmr}{m}{it}\selectfont\color{color_29791}segnale …]), bash riconosce oltre ai segnali standard anche pseudosegnali come }
\put(59.8,-230.211){\fontsize{12}{1}\usefont{T1}{cmr}{m}{n}\selectfont\color{color_29791}•DEBUG (lanciato dalla shell prima di eseguire ogni comando. Es. utile per lanciare uno }
\put(77.8,-244.011){\fontsize{12}{1}\usefont{T1}{cmr}{m}{n}\selectfont\color{color_29791}script ogni volta che un comando esegue), }
\put(59.8,-264.811){\fontsize{12}{1}\usefont{T1}{cmr}{m}{n}\selectfont\color{color_29791}•RETURN (lanciato da shell dopo ritorno da chiamata a funzione o dopo inclusione di file }
\put(77.8,-278.611){\fontsize{12}{1}\usefont{T1}{cmr}{m}{n}\selectfont\color{color_29791}con source), }
\put(59.8,-299.411){\fontsize{12}{1}\usefont{T1}{cmr}{m}{n}\selectfont\color{color_29791}•ERR (lanciato dopo ogni comando che fallisce), }
\put(59.8,-320.211){\fontsize{12}{1}\usefont{T1}{cmr}{m}{n}\selectfont\color{color_29791}•EXIT (shell in uscita, da fine script, exit, o segnale di terminazione tranne KILL, con il }
\put(77.8,-334.011){\fontsize{12}{1}\usefont{T1}{cmr}{m}{n}\selectfont\color{color_29791}quale l’handler verrebbe eliminato prima che possa essere eseguito). }
\put(41.8,-354.811){\fontsize{12}{1}\usefont{T1}{cmr}{m}{n}\selectfont\color{color_29791}Nota che i signal handler non vengono ereditati dai figli (mentre con export si possono passare loro }
\put(41.8,-368.611){\fontsize{12}{1}\usefont{T1}{cmr}{m}{n}\selectfont\color{color_29791}variabili per riferimento, è invece necessario usare di nuovo trap nella subshell), l’esecuzione di un }
\put(41.8,-382.411){\fontsize{12}{1}\usefont{T1}{cmr}{m}{n}\selectfont\color{color_29791}handler non blocca segnali dello stesso tipo e quando bash esegue un comando, il processo non è }
\put(41.8,-396.211){\fontsize{12}{1}\usefont{T1}{cmr}{m}{n}\selectfont\color{color_29791}schedulato fino alla terminazione del child, quindi non vengono controllati i segnali finché il child }
\put(41.8,-410.011){\fontsize{12}{1}\usefont{T1}{cmr}{m}{n}\selectfont\color{color_29791}non termina. }
\put(41.8,-430.811){\fontsize{12}{1}\usefont{T1}{cmr}{m}{n}\selectfont\color{color_29791}Per inviare un segnale a un processo si può usare kill [options] <pid> […], dove pid <0 }
\end{picture}
\begin{tikzpicture}[overlay]
\path(0pt,0pt);
\draw[color_29791,line width=0.7pt]
(433.3pt, -431.911pt) -- (467.4pt, -431.911pt)
;
\end{tikzpicture}
\begin{picture}(-5,0)(2.5,0)
\put(41.8,-444.611){\fontsize{12}{1}\usefont{T1}{cmr}{m}{n}\selectfont\color{color_29791}identificano l’intero process group}
\end{picture}
\begin{tikzpicture}[overlay]
\path(0pt,0pt);
\draw[color_29791,line width=0.7pt]
(41.8pt, -445.711pt) -- (207.4pt, -445.711pt)
;
\end{tikzpicture}
\begin{picture}(-5,0)(2.5,0)
\put(207.4,-444.611){\fontsize{12}{1}\usefont{T1}{cmr}{m}{n}\selectfont\color{color_29791}; inoltre getty trasforma alcune combo di tasti su terminale in }
\put(41.8,-458.411){\fontsize{12}{1}\usefont{T1}{cmr}{m}{n}\selectfont\color{color_29791}segnali inviati al processo che lo occupa: }
\put(59.8,-479.211){\fontsize{12}{1}\usefont{T1}{cmr}{m}{n}\selectfont\color{color_29791}•CTRL + Z per SIGSTP, }
\put(59.8,-500.011){\fontsize{12}{1}\usefont{T1}{cmr}{m}{n}\selectfont\color{color_29791}•CTRL + C per SIGINT, }
\put(59.8,-520.811){\fontsize{12}{1}\usefont{T1}{cmr}{m}{n}\selectfont\color{color_29791}•CTRL + \ per SIGQUIT. }
\put(59.8,-541.611){\fontsize{12}{1}\usefont{T1}{cmr}{m}{n}\selectfont\color{color_29791}•Inoltre CTRL+D produce carattere EOF che non è un segnale, CTRL + S stoppa scrolling }
\put(71.2,-555.411){\fontsize{12}{1}\usefont{T1}{cmr}{m}{n}\selectfont\color{color_29791}congelando il terminale e CTRL + Q riavvia scrolling.}
\put(41.8,-576.211){\fontsize{12}{1}\usefont{T1}{cmr}{m}{n}\selectfont\color{color_217499}Esempio di chiusura pulita gerarchia processi in 9\_job\_control slide 12}
\put(41.8,-597.011){\fontsize{12}{1}\usefont{T1}{cmr}{m}{n}\selectfont\color{color_29791}Per esempi e approfondimenti sulla propagazione di segnali a child process:}
\put(41.8,-610.811){\fontsize{12}{1}\usefont{T1}{cmr}{m}{n}\selectfont\color{color_29919}https://linuxconfig.org/how-to-propagate-a-signal-to-child-processes-from-a-bash-script}
\end{picture}
\begin{tikzpicture}[overlay]
\path(0pt,0pt);
\draw[color_29919,line width=0.7pt]
(41.8pt, -611.911pt) -- (464.5pt, -611.911pt)
;
\end{tikzpicture}
\begin{picture}(-5,0)(2.5,0)
\put(41.8,-638.411){\fontsize{12}{1}\usefont{T1}{cmr}{m}{n}\selectfont\color{color_29791}Il comando sleep pone un timer per far dormire il processo per il numero di secondi indicato nel }
\put(41.8,-652.211){\fontsize{12}{1}\usefont{T1}{cmr}{m}{n}\selectfont\color{color_29791}parametro. E’ un comando esterno quindi valgono le solite regole: essendo eseguito in un processo }
\put(41.8,-666.011){\fontsize{12}{1}\usefont{T1}{cmr}{m}{n}\selectfont\color{color_29791}figlio, inviare un segnale alla shell che lo genera non lo tocca (la shell padre non è schedulabile }
\put(41.8,-679.811){\fontsize{12}{1}\usefont{T1}{cmr}{m}{n}\selectfont\color{color_29791}mentre esegue il figlio); inoltre anche sleep stesso è immune ai segnali (ricevuti ma non vengono }
\put(41.8,-693.611){\fontsize{12}{1}\usefont{T1}{cmr}{m}{n}\selectfont\color{color_29791}processati) dato che non rientra in user space fino alla sua terminazione, poiché usa syscall per }
\put(41.8,-707.411){\fontsize{12}{1}\usefont{T1}{cmr}{m}{n}\selectfont\color{color_29791}avere l’attesa, non fa busy wait.}
\put(41.8,-728.211){\fontsize{12}{1}\usefont{T1}{cmr}{m}{n}\selectfont\color{color_29791}E’ possibile usare una sola shell per eseguire in contemporanea più comandi che non hanno }
\put(41.8,-742.011){\fontsize{12}{1}\usefont{T1}{cmr}{m}{n}\selectfont\color{color_29791}necessità di accedere al terminale, lanciandoli in background postponendo \& alla command line. La }
\put(41.8,-755.811){\fontsize{12}{1}\usefont{T1}{cmr}{m}{n}\selectfont\color{color_29791}shell risponde con [job\_id], che identifica il job localmente a quella shell: per usarlo al posto di un }
\end{picture}
\begin{tikzpicture}[overlay]
\path(0pt,0pt);
\draw[color_29791,line width=0.7pt]
(399.4pt, -756.911pt) -- (518pt, -756.911pt)
;
\end{tikzpicture}
\newpage
\begin{tikzpicture}[overlay]\path(0pt,0pt);\end{tikzpicture}
\begin{picture}(-5,0)(2.5,0)
\put(41.8,-85.01099){\fontsize{12}{1}\usefont{T1}{cmr}{m}{n}\selectfont\color{color_29791}pid si usa   }
\end{picture}
\begin{tikzpicture}[overlay]
\path(0pt,0pt);
\draw[color_29791,line width=0.7pt]
(41.7pt, -86.11096pt) -- (90.10001pt, -86.11096pt)
;
\end{tikzpicture}
\begin{picture}(-5,0)(2.5,0)
\put(90.2,-85.01099){\fontsize{12}{1}\usefont{T1}{cmr}{m}{it}\selectfont\color{color_29791}\%job\_id  }
\end{picture}
\begin{tikzpicture}[overlay]
\path(0pt,0pt);
\draw[color_29791,line width=0.7pt]
(90.1pt, -86.11096pt) -- (130.7pt, -86.11096pt)
;
\end{tikzpicture}
\begin{picture}(-5,0)(2.5,0)
\put(130.8,-85.01099){\fontsize{12}{1}\usefont{T1}{cmr}{m}{n}\selectfont\color{color_29791}, mentre il PID del processo viene salvato nella variabile \$!. In \$! viene messo il }
\put(41.8,-98.81097){\fontsize{12}{1}\usefont{T1}{cmr}{m}{n}\selectfont\color{color_29791}pid dell'ultimo processo mandato in background, utile salvarselo se serve in seguito. }
\put(41.8,-119.611){\fontsize{12}{1}\usefont{T1}{cmr}{m}{n}\selectfont\color{color_29791}Il processo in background non riceve più lo stdin, in quanto appena si lancia comando\& lo stdin }
\put(41.8,-133.411){\fontsize{12}{1}\usefont{T1}{cmr}{m}{n}\selectfont\color{color_29791}ritorna alla shell che lo ha lanciato e si può inserire un altro comando; l’output resta però agganciato}
\put(41.8,-147.211){\fontsize{12}{1}\usefont{T1}{cmr}{m}{n}\selectfont\color{color_29791}quindi il terminale potrebbe essere “sporcato” se il comando ha uscite, se non ci interessa l’output }
\put(41.8,-161.011){\fontsize{12}{1}\usefont{T1}{cmr}{m}{n}\selectfont\color{color_29791}sarà opportuno ridirigerlo su /dev/null al lancio. }
\put(41.8,-181.811){\fontsize{12}{1}\usefont{T1}{cmr}{m}{n}\selectfont\color{color_29791}Se si lancia un comando senza \& e si vuole rimediare, si può dare un segnale di STOP con }
\put(41.8,-195.611){\fontsize{12}{1}\usefont{T1}{cmr}{m}{n}\selectfont\color{color_29791}CTRL+Z, ricevendo comunque un job\_id; usandolo con bg \%job\_id si invia SIGCONT che riavvia }
\put(41.8,-209.411){\fontsize{12}{1}\usefont{T1}{cmr}{m}{n}\selectfont\color{color_29791}il processo mettendolo in background.}
\put(41.8,-230.211){\fontsize{12}{1}\usefont{T1}{cmr}{m}{n}\selectfont\color{color_29791}Il builtin wait permette di bloccare l’esecuzione fino al completamento dei job in background. Di }
\put(41.8,-244.011){\fontsize{12}{1}\usefont{T1}{cmr}{m}{n}\selectfont\color{color_29791}default attende il completamento di tutti i job. Utile quando si deve lanciare massa di processi in }
\put(41.8,-257.811){\fontsize{12}{1}\usefont{T1}{cmr}{m}{n}\selectfont\color{color_29791}background ma dobbiamo aspettare che tutti abbiano concluso prima di tornare.(sarà utile }
\put(41.8,-271.611){\fontsize{12}{1}\usefont{T1}{cmr}{m}{n}\selectfont\color{color_29791}per monitoraggio, avendo molte macchine che devono rispondere anche in modo lento. Se lanciamo}
\put(41.8,-285.411){\fontsize{12}{1}\usefont{T1}{cmr}{m}{n}\selectfont\color{color_29791}in background in parallelo la funzione facendole monitorare 10 macchine.  Faremo uno script dove }
\put(41.8,-299.211){\fontsize{12}{1}\usefont{T1}{cmr}{m}{n}\selectfont\color{color_29791}lanciamo le verifiche, e per attendere tutti i risultati metteremo un wait ).Esempio }
\put(41.8,-313.011){\fontsize{12}{1}\usefont{T1}{cmr}{m}{n}\selectfont\color{color_29791}frequente: dovendo monitorare 40 macchine, abbiamo una funzione che interroga una macchina per }
\put(41.8,-326.811){\fontsize{12}{1}\usefont{T1}{cmr}{m}{n}\selectfont\color{color_29791}sapere es. se ha abbastanza memoria, sappiamo che questo comando può metterci 3 secondi perché }
\put(41.8,-340.611){\fontsize{12}{1}\usefont{T1}{cmr}{m}{n}\selectfont\color{color_29791}lato macchina interrogata è un quesito pesante. Non si fa un ciclo che interroga una macchina alla }
\put(41.8,-354.411){\fontsize{12}{1}\usefont{T1}{cmr}{m}{n}\selectfont\color{color_29791}volta (impiego max 120 sec), si fanno 40 processi in background (monitora macchina \$i), ognuno }
\put(41.8,-368.211){\fontsize{12}{1}\usefont{T1}{cmr}{m}{n}\selectfont\color{color_29791}dei quali è leggero e ci mette 3 secondi ad interrogare la macchina di riferimento. Dopo si usa wait }
\put(41.8,-382.011){\fontsize{12}{1}\usefont{T1}{cmr}{m}{n}\selectfont\color{color_29791}per attendere che siano terminati tutti i processi in background; l’output di ognuno dei processi }
\put(41.8,-395.811){\fontsize{12}{1}\usefont{T1}{cmr}{m}{n}\selectfont\color{color_29791}finirebbe comunque su stdout, ma sarebbe più opportuno salvarlo. Usando un solo file ogni }
\put(41.8,-409.611){\fontsize{12}{1}\usefont{T1}{cmr}{m}{n}\selectfont\color{color_29791}processo avrebbe la sua idea di fd, quindi anche scrivendo in append il file sarebbe sovrascritto da }
\put(41.8,-423.411){\fontsize{12}{1}\usefont{T1}{cmr}{m}{n}\selectfont\color{color_29791}ogni processo essendo il file una struttura non concorrente; quindi si crea un file per ognuna delle }
\put(41.8,-437.211){\fontsize{12}{1}\usefont{T1}{cmr}{m}{n}\selectfont\color{color_29791}interazioni, ridirigendo output per ogni processo su file \$i. }
\put(41.8,-458.011){\fontsize{12}{1}\usefont{T1}{cmr}{b}{it}\selectfont\color{color_29791}watch prende come argomento altri comandi, li lancia periodicamente e aggiorna il suo output.}
\put(41.8,-478.811){\fontsize{12}{1}\usefont{T1}{cmr}{m}{n}\selectfont\color{color_29791}Se è necessario riportare in foreground un processo ricollegandolo così al terminale, si usa fg }
\put(41.8,-492.611){\fontsize{12}{1}\usefont{T1}{cmr}{b}{it}\selectfont\color{color_29791}\%job\_id. Il comando jobs mostra l’elenco dei job, cioè di tutti i processi avviati dalla shell corrente }
\put(41.8,-506.411){\fontsize{12}{1}\usefont{T1}{cmr}{m}{n}\selectfont\color{color_29791}(l'elenco dei jobs è locale, quelli avviati in un'altra shell non vengono mostrati: usando ps invece }
\put(41.8,-520.211){\fontsize{12}{1}\usefont{T1}{cmr}{m}{n}\selectfont\color{color_29791}avremo lo stesso risultato anche in'un altra shell, passandogli il pid del job), indicandone lo stato }
\put(41.8,-534.011){\fontsize{12}{1}\usefont{T1}{cmr}{m}{n}\selectfont\color{color_29791}(attivo o stoppato). jobs -l mostra il pid dei job, nel caso non ce lo fossimo segnati.}
\put(41.8,-554.811){\fontsize{12}{1}\usefont{T1}{cmr}{m}{n}\selectfont\color{color_29791}Per i processi in background si possono usare i comandi nohup <command> per evitare che sia }
\put(41.8,-568.611){\fontsize{12}{1}\usefont{T1}{cmr}{m}{n}\selectfont\color{color_29791}inviato SIGHUP al comando quando la shell chiude (ne causerebbe la terminazione, inoltre l'output }
\put(41.8,-582.411){\fontsize{12}{1}\usefont{T1}{cmr}{m}{n}\selectfont\color{color_217499}viene rediretto -a default su nohup.out-) e nice <command> lancia command con niceness diversa }
\put(41.8,-596.211){\fontsize{12}{1}\usefont{T1}{cmr}{m}{n}\selectfont\color{color_29791}da zero, modificandone la priorità. Sono comandi esterni, anche combinabili. Invece il builtin }
\put(41.8,-610.011){\fontsize{12}{1}\usefont{T1}{cmr}{b}{it}\selectfont\color{color_29791}disown che agisce su PID/job\_id lanciati in precedenza, rimuove un job dalla job table di shell (a }
\put(41.8,-623.811){\fontsize{12}{1}\usefont{T1}{cmr}{m}{n}\selectfont\color{color_29791}default l’ultimo, con opzione -h offre anche immunità all’hangup).}
\put(41.8,-644.611){\fontsize{12}{1}\usefont{T1}{cmr}{m}{n}\selectfont\color{color_29791}Per gestire i processi è utile il comando ps (process status), che ha molte opzioni. Utile avere una }
\put(41.8,-660.311){\fontsize{12}{1}\usefont{T1}{cmr}{m}{n}\selectfont\color{color_29791}cheat sheet. https://www.golinuxcloud.com/ps-command-in-linux/}
\end{picture}
\begin{tikzpicture}[overlay]
\path(0pt,0pt);
\draw[color_29919,line width=0.8pt]
(100.8pt, -661.511pt) -- (418.1pt, -661.511pt)
;
\end{tikzpicture}
\begin{picture}(-5,0)(2.5,0)
\put(41.8,-681.511){\fontsize{12}{1}\usefont{T1}{cmr}{b}{it}\selectfont\color{color_29791}top mostra i processi più attivi e altre statistiche.}
\put(41.8,-717.511){\fontsize{17.5}{1}\usefont{T1}{cmr}{b}{n}\selectfont\color{color_29791}Shell scripting}
\put(41.8,-738.411){\fontsize{12}{1}\usefont{T1}{cmr}{m}{n}\selectfont\color{color_29791}Il linguaggio di bash è interpretato, non compilato: il significato di molti caratteri è sintattico e non }
\put(41.8,-752.211){\fontsize{12}{1}\usefont{T1}{cmr}{m}{n}\selectfont\color{color_29791}letterale: la riga di comando effettivamente eseguita risulta da un procedimento, detto espansione, }
\put(41.8,-766.011){\fontsize{12}{1}\usefont{T1}{cmr}{m}{n}\selectfont\color{color_29791}che individua sottostringhe contrassegnate da caratteri speciali, sostituendole col risultato di una }
\end{picture}
\newpage
\begin{tikzpicture}[overlay]\path(0pt,0pt);\end{tikzpicture}
\begin{picture}(-5,0)(2.5,0)
\put(41.8,-85.01099){\fontsize{12}{1}\usefont{T1}{cmr}{m}{n}\selectfont\color{color_29791}corrispondente elaborazione. L’espansione consiste in 12 passi in sequenza, alcuni possono essere }
\put(41.8,-98.81097){\fontsize{12}{1}\usefont{T1}{cmr}{m}{n}\selectfont\color{color_29791}saltati usando il quoting, ovvero proteggendo i meta-caratteri da non interpretare per mezzo di altri }
\put(41.8,-112.611){\fontsize{12}{1}\usefont{T1}{cmr}{m}{n}\selectfont\color{color_29791}caratteri speciali (‘ “ e \).}
\put(59.8,-133.411){\fontsize{12}{1}\usefont{T1}{cmr}{m}{n}\selectfont\color{color_29791}(1)Tokenizzazione: La riga viene divisa in token usando come separatori i metacaratteri }
\put(77.8,-147.211){\fontsize{12}{1}\usefont{T1}{cmr}{m}{n}\selectfont\color{color_29791}\{SPACE TAB NEWLINE ; ( ) < > | \&\}. I token possono essere stringhe, parole chiave, }
\put(77.8,-161.011){\fontsize{12}{1}\usefont{T1}{cmr}{m}{n}\selectfont\color{color_29791}caratteri di ridirezione, carattere “:” (è il null command, comando che non fa niente). I }
\put(77.8,-174.811){\fontsize{12}{1}\usefont{T1}{cmr}{m}{n}\selectfont\color{color_29791}caratteri speciali possono far comparire degli elementi che anche apparendo come normali }
\put(77.8,-188.611){\fontsize{12}{1}\usefont{T1}{cmr}{m}{n}\selectfont\color{color_29791}valori, potrebbero avere un significato sintattico speciale per la shell, che potrebbe venir }
\put(77.8,-202.411){\fontsize{12}{1}\usefont{T1}{cmr}{m}{n}\selectfont\color{color_29791}interpretato nei passi successivi. Parti di riga quotate tra singoli apici ‘ saltano a 12) }
\put(77.8,-216.211){\fontsize{12}{1}\usefont{T1}{cmr}{b}{n}\selectfont\color{color_29791}command lookup, parti tra doppi apici “ saltano a 6) parameter expansion}
\put(59.8,-237.011){\fontsize{12}{1}\usefont{T1}{cmr}{m}{n}\selectfont\color{color_29791}(2)1st token = alias?: La shell cerca il primo token nella lista degli alias. Se lo trova, lo }
\put(77.8,-250.811){\fontsize{12}{1}\usefont{T1}{cmr}{m}{n}\selectfont\color{color_29791}espande e riparte col processing dal punto 1. Sono consentiti alias ricorsivi, bash non }
\put(77.8,-264.611){\fontsize{12}{1}\usefont{T1}{cmr}{m}{n}\selectfont\color{color_29791}espande due volte uno stesso alias. Saltato con “}
\put(59.8,-285.411){\fontsize{12}{1}\usefont{T1}{cmr}{m}{n}\selectfont\color{color_29791}(3)1st token = keyword?: La shell controlla se il primo token è una parola chiave che inizia un}
\put(77.8,-299.211){\fontsize{12}{1}\usefont{T1}{cmr}{m}{n}\selectfont\color{color_29791}comando composto, es. if, while, function, \{, (. Se c’è bisogno crea una subshell per il }
\put(77.8,-313.011){\fontsize{12}{1}\usefont{T1}{cmr}{m}{n}\selectfont\color{color_29791}comando composto e va a leggere il primo token. Saltato con “}
\put(59.8,-333.811){\fontsize{12}{1}\usefont{T1}{cmr}{m}{n}\selectfont\color{color_29791}(4)Brace expansion: elementi tra \{\} vengono espansi in una lista. Estensiva }
\put(77.8,-347.611){\fontsize{12}{1}\usefont{T1}{cmr}{m}{it}\selectfont\color{color_29791}\{a,pippo,mamma\} o sequenza \{min..max[..incr.]\}, ci sono anche molti altri tipi. Es. utile per }
\put(77.8,-361.411){\fontsize{12}{1}\usefont{T1}{cmr}{m}{n}\selectfont\color{color_29791}creare un ciclo, espandendo un lungo elenco partendo dall’indicazione dei soli estremi. }
\put(77.8,-375.211){\fontsize{12}{1}\usefont{T1}{cmr}{m}{n}\selectfont\color{color_29791}Saltato con “}
\put(59.8,-396.011){\fontsize{12}{1}\usefont{T1}{cmr}{m}{n}\selectfont\color{color_29791}(5)Tilde expansion: se c’è un token nella forma ~username, viene sostituito con la home }
\put(77.8,-409.811){\fontsize{12}{1}\usefont{T1}{cmr}{m}{n}\selectfont\color{color_29791}directory dell’utente username (se username è vuoto, si usa l’utente corrente). Quindi è la }
\put(77.8,-423.611){\fontsize{12}{1}\usefont{T1}{cmr}{m}{n}\selectfont\color{color_29791}shell a sostituire ~, non i comandi che usano percorsi. Saltato con “}
\put(59.8,-444.411){\fontsize{12}{1}\usefont{T1}{cmr}{m}{n}\selectfont\color{color_29791}(6)Parameter expansion: carattere \$ può marcare l’inizio di diverse espansioni (6, 7, 8), la più}
\put(77.8,-458.211){\fontsize{12}{1}\usefont{T1}{cmr}{m}{n}\selectfont\color{color_29791}semplice PE è la sostituzione della stringa \$NAME con il valore contenuto nella variabile }
\put(77.8,-472.011){\fontsize{12}{1}\usefont{T1}{cmr}{m}{n}\selectfont\color{color_29791}NAME. Eseguito anche con “}
\put(59.8,-492.811){\fontsize{12}{1}\usefont{T1}{cmr}{m}{n}\selectfont\color{color_29791}(7)Command substitution: il token \$(command) causa: creazione subshell, esecuzione }
\put(77.8,-506.611){\fontsize{12}{1}\usefont{T1}{cmr}{m}{n}\selectfont\color{color_29791}comando, stdout di command viene posto sulla riga di comando al posto del token originale, }
\put(77.8,-520.411){\fontsize{12}{1}\usefont{T1}{cmr}{m}{n}\selectfont\color{color_29791}a parte eventuali righe vuote alla fine. NON USARE COMMAND SUBSTITUTION }
\put(77.8,-534.211){\fontsize{12}{1}\usefont{T1}{cmr}{m}{n}\selectfont\color{color_29791}quando comandi hanno grandi output perché viene messo tutto in memoria, MAI porre }
\put(77.8,-548.011){\fontsize{12}{1}\usefont{T1}{cmr}{m}{n}\selectfont\color{color_29791}comandi che non terminano. Equivale a usare backtick, \$() è però più leggibile e inoltre è }
\put(77.8,-561.811){\fontsize{12}{1}\usefont{T1}{cmr}{m}{n}\selectfont\color{color_29791}annidabile \$(c1(\$c2)) ma è difficile da usare. Eseguito anche con “}
\put(59.8,-582.611){\fontsize{12}{1}\usefont{T1}{cmr}{m}{n}\selectfont\color{color_29791}(8)Arithmetic expansion: il token ((expr)) causa la valutazione di expr, un’espressione }
\put(77.8,-596.411){\fontsize{12}{1}\usefont{T1}{cmr}{m}{n}\selectfont\color{color_29791}aritmetica. Se preceduto da \$, il risultato viene posto sulla riga di comando, altrimenti }
\put(77.8,-610.211){\fontsize{12}{1}\usefont{T1}{cmr}{m}{n}\selectfont\color{color_29791}l’unico effetto è eventualmente sulle variabili usate. Eseguito anche con “}
\put(59.8,-631.011){\fontsize{12}{1}\usefont{T1}{cmr}{m}{n}\selectfont\color{color_29791}(9)Process substitution: con il token <(comando) o >(comando), viene eseguito comando in }
\put(77.8,-644.811){\fontsize{12}{1}\usefont{T1}{cmr}{m}{n}\selectfont\color{color_29791}modo concorrente e asincrono rispetto al resto della riga. Serve quando abbiamo comandi }
\put(77.8,-658.611){\fontsize{12}{1}\usefont{T1}{cmr}{m}{n}\selectfont\color{color_29791}che esigono di leggere da stdin o scrivere su stdout. Eseguito anche con “}
\put(77.8,-679.411){\fontsize{12}{1}\usefont{T1}{cmr}{m}{n}\selectfont\color{color_29791}◦Quando abbiamo un cmd\_producer\_su\_stdout e un cmd\_consumer\_da\_file (che si }
\put(95.8,-693.211){\fontsize{12}{1}\usefont{T1}{cmr}{m}{n}\selectfont\color{color_29791}aspetta un file come parametro), non possiamo metterli in pipeline: con process }
\put(95.8,-707.011){\fontsize{12}{1}\usefont{T1}{cmr}{m}{n}\selectfont\color{color_29791}substituton cmd\_consumer <(cmd\_producer\_su\_stdout) il processo tra parentesi viene }
\put(95.8,-720.811){\fontsize{12}{1}\usefont{T1}{cmr}{m}{n}\selectfont\color{color_29791}lanciato concorrentemente e la shell genera un nome di file (una named pipe) da fornire }
\put(95.8,-734.611){\fontsize{12}{1}\usefont{T1}{cmr}{m}{n}\selectfont\color{color_29791}al primo (simmetrico cmd\_producer\_su\_file >(cmd\_consumer\_da\_stdin) )}
\end{picture}
\newpage
\begin{tikzpicture}[overlay]\path(0pt,0pt);\end{tikzpicture}
\begin{picture}(-5,0)(2.5,0)
\put(59.8,-85.01099){\fontsize{12}{1}\usefont{T1}{cmr}{m}{n}\selectfont\color{color_29791}(10)Word splitting: i risultati dei passi 6..9 sono esaminati e separati in word }
\put(77.8,-98.81097){\fontsize{12}{1}\usefont{T1}{cmr}{m}{n}\selectfont\color{color_29791}indipendenti. Separatore = qualsiasi carattere presente nella variabile IFS, a default }
\put(77.8,-112.611){\fontsize{12}{1}\usefont{T1}{cmr}{m}{n}\selectfont\color{color_29791}IFS=<space><tab><newline>. Saltato con “}
\put(59.8,-133.411){\fontsize{12}{1}\usefont{T1}{cmr}{m}{n}\selectfont\color{color_29791}(11)Pathname expansion: ogni word viene esaminata e se contiene uno tra *, ?, [ viene }
\put(77.8,-147.211){\fontsize{12}{1}\usefont{T1}{cmr}{m}{n}\selectfont\color{color_29791}considerata un pattern e sostituita con tutti i file che concordano (quindi * non è un carattere }
\put(77.8,-161.011){\fontsize{12}{1}\usefont{T1}{cmr}{m}{n}\selectfont\color{color_29791}speciale per i comandi, è la shell che sostituisce con tutti i file presenti e crea una command }
\put(77.8,-174.811){\fontsize{12}{1}\usefont{T1}{cmr}{m}{n}\selectfont\color{color_29791}line dove tutti i file sono argomento del comando). Saltato con “}
\put(59.8,-195.611){\fontsize{12}{1}\usefont{T1}{cmr}{m}{n}\selectfont\color{color_29791}(12)Quote removal e command lookup: vengono rimosse tutte le occorrenze di }
\put(77.8,-209.411){\fontsize{12}{1}\usefont{T1}{cmr}{m}{n}\selectfont\color{color_29791}caratteri di quoting “usate” effettivamente (non protette da altri quoting, non generate dai }
\put(77.8,-223.211){\fontsize{12}{1}\usefont{T1}{cmr}{m}{n}\selectfont\color{color_29791}passi 6..9), vengono impostati gli stream in caso di ridirezione, viene cercato il comando in }
\put(77.8,-237.011){\fontsize{12}{1}\usefont{T1}{cmr}{m}{n}\selectfont\color{color_29791}ordine tra: funzioni, builtin, eseguibili in \$PATH}
\put(41.8,-266.811){\fontsize{14.1}{1}\usefont{T1}{cmr}{b}{n}\selectfont\color{color_29791}Quoting}
\put(41.8,-287.011){\fontsize{12}{1}\usefont{T1}{cmr}{m}{n}\selectfont\color{color_29791}Il quoting consente di proteggere parametri che contengono i simboli [ ] ! * ? \$ \{ \} ( ) “ ‘ \` \ | > < ; }
\put(41.8,-300.811){\fontsize{12}{1}\usefont{T1}{cmr}{m}{n}\selectfont\color{color_29791}che se non venissero protetti verrebbero espansi, impedendo l’uso del loro valore letterale}
\put(59.8,-321.611){\fontsize{12}{1}\usefont{T1}{cmr}{m}{n}\selectfont\color{color_29791}•\ backslash: protegge solo il carattere successivo, che non verrà interpretato come simbolo }
\put(77.8,-335.411){\fontsize{12}{1}\usefont{T1}{cmr}{m}{n}\selectfont\color{color_29791}speciale}
\put(59.8,-356.211){\fontsize{12}{1}\usefont{T1}{cmr}{m}{n}\selectfont\color{color_29791}•‘ apice: ogni carattere di una stringa racchiusa tra una coppia di apici viene protetto }
\put(77.8,-370.011){\fontsize{12}{1}\usefont{T1}{cmr}{m}{n}\selectfont\color{color_29791}dall’espansione e trattato letteralmente, senza eccezioni}
\put(77.8,-390.811){\fontsize{12}{1}\usefont{T1}{cmr}{m}{n}\selectfont\color{color_29791}◦1 → 12}
\put(59.8,-411.611){\fontsize{12}{1}\usefont{T1}{cmr}{m}{n}\selectfont\color{color_29791}•“ doppio apice: ogni carattere di una stringa racchiusa tra una coppia di virgolette viene }
\put(77.8,-425.411){\fontsize{12}{1}\usefont{T1}{cmr}{m}{n}\selectfont\color{color_29791}protetto dall’espansione, con l’eccezione di \$, \` backtick, \ backslash, e altri casi particolari}
\put(77.8,-446.211){\fontsize{12}{1}\usefont{T1}{cmr}{m}{n}\selectfont\color{color_29791}◦1 → (6 7 8 9) → 12}
\put(77.8,-467.011){\fontsize{12}{1}\usefont{T1}{cmr}{m}{n}\selectfont\color{color_29791}◦vengono eseguiti solo i passi 1) Tokenizzazione, 6) Parameter Exp, 7) Command Sub, 8) }
\put(95.8,-480.811){\fontsize{12}{1}\usefont{T1}{cmr}{m}{n}\selectfont\color{color_29791}Arithm. Sub, 9) Process Sub, 12) Quote Removal e Command Lookup}
\put(77.8,-501.611){\fontsize{12}{1}\usefont{T1}{cmr}{m}{n}\selectfont\color{color_29791}◦vengono saltati 2) Check token alias, 3) Check token keyword, 4) Brace Exp, 5) Tilde }
\put(95.8,-515.411){\fontsize{12}{1}\usefont{T1}{cmr}{m}{n}\selectfont\color{color_29791}Exp e poi 10) Word Split, 11) Path Exp}
\put(41.8,-536.211){\fontsize{12}{1}\usefont{T1}{cmr}{m}{n}\selectfont\color{color_29791}I simboli stessi di quoting vanno protetti dall’espansione se vanno usati per il loro valore letterale. È}
\put(41.8,-550.011){\fontsize{12}{1}\usefont{T1}{cmr}{m}{n}\selectfont\color{color_29791}possibile separare frammenti protetti in modo diverso, verranno semplicemente concatenati dopo }
\put(41.8,-563.811){\fontsize{12}{1}\usefont{T1}{cmr}{m}{n}\selectfont\color{color_29791}l’espansione e la quote removal:}
\put(59.8,-584.611){\fontsize{12}{1}\usefont{T1}{cmr}{m}{n}\selectfont\color{color_29791}•es. “protetto da virgolette”\*’o da apici’ sarà espanso come SINGOLO token (la }
\put(77.8,-598.411){\fontsize{12}{1}\usefont{T1}{cmr}{m}{n}\selectfont\color{color_29791}separazione in pezzi diversi viene fatta dal word splitting, che avviene prima della quoting }
\put(77.8,-612.211){\fontsize{12}{1}\usefont{T1}{cmr}{m}{n}\selectfont\color{color_29791}removal. Quindi gli spazi del contenuto tra " " non vengono considerati, come quelli tra ' '. }
\put(77.8,-626.011){\fontsize{12}{1}\usefont{T1}{cmr}{m}{n}\selectfont\color{color_29791}Tutto il pezzo, spazi inclusi, sono considerati come singolo token) di valore}
\put(77.8,-646.811){\fontsize{12}{1}\usefont{T1}{cmr}{b}{it}\selectfont\color{color_29791}protetto da virgolette*o da apici}
\put(41.8,-667.611){\fontsize{12}{1}\usefont{T1}{cmr}{m}{n}\selectfont\color{color_29791}Il comando echo stampa a video i caratteri che seguono, è utile per visualizzare il valore di una }
\put(41.8,-681.411){\fontsize{12}{1}\usefont{T1}{cmr}{m}{n}\selectfont\color{color_29791}variabile (echo \$PATH visualizza il contenuto della variabile PATH) o di una pathname expansion }
\put(41.8,-695.211){\fontsize{12}{1}\usefont{T1}{cmr}{m}{n}\selectfont\color{color_29791}(echo * visualizza tutti i nomi di file nella directory corrente) sfruttando l’espansione di bash (echo }
\put(41.8,-709.011){\fontsize{12}{1}\usefont{T1}{cmr}{m}{it}\selectfont\color{color_29791}non sa cosa sia una variabile o un pathname).}
\end{picture}
\newpage
\begin{tikzpicture}[overlay]\path(0pt,0pt);\end{tikzpicture}
\begin{picture}(-5,0)(2.5,0)
\put(41.8,-87.01099){\fontsize{14.1}{1}\usefont{T1}{cmr}{b}{n}\selectfont\color{color_29791}Pathname expansion}
\put(41.8,-107.211){\fontsize{12}{1}\usefont{T1}{cmr}{m}{n}\selectfont\color{color_29791}La pathname expansion è molto utile per gestire file e directory, ed essendo uno degli ultimi passi }
\put(41.8,-121.011){\fontsize{12}{1}\usefont{T1}{cmr}{m}{n}\selectfont\color{color_29791}svolti da bash opera su stringhe che potrebbero essere generate ai passi precedenti. Viene saltata con}
\put(41.8,-134.811){\fontsize{12}{1}\usefont{T1}{cmr}{m}{n}\selectfont\color{color_29791}quoting a doppi apici “”. I pattern tipici sono }
\put(59.8,-155.611){\fontsize{12}{1}\usefont{T1}{cmr}{m}{n}\selectfont\color{color_29791}•* rappresenta una qualunque stringa di zero o più caratteri}
\put(59.8,-176.411){\fontsize{12}{1}\usefont{T1}{cmr}{m}{n}\selectfont\color{color_29791}•? rappresenta un qualunque carattere singolo}
\put(59.8,-197.211){\fontsize{12}{1}\usefont{T1}{cmr}{m}{n}\selectfont\color{color_29791}•[SET] rappresenta un carattere appartenente a SET:}
\put(77.8,-218.011){\fontsize{12}{1}\usefont{T1}{cmr}{m}{n}\selectfont\color{color_29791}◦[afhOV] elenco}
\put(77.8,-238.811){\fontsize{12}{1}\usefont{T1}{cmr}{m}{n}\selectfont\color{color_29791}◦[a-k] intervallo, il cui ordine dipende dal locale (più intervalli uniti con virgola es. [a-}
\put(95.8,-252.611){\fontsize{12}{1}\usefont{T1}{cmr}{m}{n}\selectfont\color{color_29791}d,0-5])}
\put(77.8,-273.411){\fontsize{12}{1}\usefont{T1}{cmr}{m}{n}\selectfont\color{color_29791}◦[!a] [\^A-Z] per negare il contenuto del SET (es. tranne a, tranne da A a Z)}
\put(77.8,-294.211){\fontsize{12}{1}\usefont{T1}{cmr}{m}{n}\selectfont\color{color_29791}◦[[:alnum:]] classe di caratteri come per egrep}
\put(77.8,-315.011){\fontsize{12}{1}\usefont{T1}{cmr}{m}{n}\selectfont\color{color_29791}◦per includere i caratteri – o ], metterli come primi carattere}
\put(41.8,-335.811){\fontsize{12}{1}\usefont{T1}{cmr}{m}{n}\selectfont\color{color_29791}Bash cerca se può sostituire la stringa che contiene uno di questi pattern con qualcosa che }
\put(41.8,-349.611){\fontsize{12}{1}\usefont{T1}{cmr}{m}{n}\selectfont\color{color_29791}corrisponde, dal path in cui si trova, e sostituisce con file che corrispondono, in ordine alfabetico. }
\put(41.8,-363.411){\fontsize{12}{1}\usefont{T1}{cmr}{m}{n}\selectfont\color{color_29791}Se non esistono file che corrispondono, il pattern resta inalterato sulla linea di comando. Vedi sez. }
\put(41.8,-377.211){\fontsize{12}{1}\usefont{T1}{cmr}{m}{it}\selectfont\color{color_29791}pathname expansion di man bash(1).}
\put(41.8,-398.011){\fontsize{12}{1}\usefont{T1}{cmr}{m}{n}\selectfont\color{color_29791}Esempi pathname expansion:}
\put(59.8,-418.811){\fontsize{12}{1}\usefont{T1}{cmr}{m}{n}\selectfont\color{color_29791}•echo *tutti i file del direttorio corrente}
\put(59.8,-439.611){\fontsize{12}{1}\usefont{T1}{cmr}{m}{n}\selectfont\color{color_29791}•echo [a-p,1-7]*[cfd]? File con nomi che iniziano per caratt. compreso tra a e p o tra 1}
\put(77.8,-453.411){\fontsize{12}{1}\usefont{T1}{cmr}{m}{n}\selectfont\color{color_29791}e 7, se il penultimo caratt. è c, f oppure d}
\put(59.8,-474.211){\fontsize{12}{1}\usefont{T1}{cmr}{m}{n}\selectfont\color{color_29791}•echo \* fa echo del carattere *, non visto come wildcard da bash ma con }
\put(77.8,-488.011){\fontsize{12}{1}\usefont{T1}{cmr}{m}{n}\selectfont\color{color_29791}valore letterale (quoting con backslash)}
\put(59.8,-508.811){\fontsize{12}{1}\usefont{T1}{cmr}{m}{n}\selectfont\color{color_29791}•echo *[!\*\?]* elenca tutti i file del direttorio corrente che hanno almeno un caratt. }
\put(77.8,-522.611){\fontsize{12}{1}\usefont{T1}{cmr}{m}{n}\selectfont\color{color_29791}Diverso dalle wildcard * e ?}
\put(59.8,-543.411){\fontsize{12}{1}\usefont{T1}{cmr}{m}{n}\selectfont\color{color_29791}•echo /*/*/* elenca tutti i file dei direttori di secondo livello a partire dalla root }
\put(77.8,-557.211){\fontsize{12}{1}\usefont{T1}{cmr}{m}{n}\selectfont\color{color_29791}(tutti i file che stanno in una directory seguita da qualsiasi carattere che sta in una directory }
\put(77.8,-571.011){\fontsize{12}{1}\usefont{T1}{cmr}{m}{n}\selectfont\color{color_29791}seguita da qualsiasi carattere che sta in una qualsiasi directory seguita da qualsiasi carattere)}
\put(41.8,-591.811){\fontsize{12}{1}\usefont{T1}{cmr}{m}{n}\selectfont\color{color_29791}È possibile abilitare comportamenti più complessi (Moltiplicatori, inversione del matching) }
\put(41.8,-605.611){\fontsize{12}{1}\usefont{T1}{cmr}{m}{n}\selectfont\color{color_29791}attivando l’opzione extglob con shopt.}
\put(41.8,-635.411){\fontsize{14.1}{1}\usefont{T1}{cmr}{b}{n}\selectfont\color{color_29791}Brace expansion}
\put(41.8,-655.611){\fontsize{12}{1}\usefont{T1}{cmr}{m}{n}\selectfont\color{color_29791}Un meccanismo di espansione per generare sequenze di stringhe secondo un pattern, con la stessa }
\put(41.8,-669.411){\fontsize{12}{1}\usefont{T1}{cmr}{m}{n}\selectfont\color{color_29791}sintassi della pathname expansion, ma le stringhe sono generate indipendentemente dal fatto che }
\put(41.8,-683.211){\fontsize{12}{1}\usefont{T1}{cmr}{m}{n}\selectfont\color{color_29791}esistano o meno file che rispettano il pattern (al contrario della path exp). Avviene presto }
\put(41.8,-697.011){\fontsize{12}{1}\usefont{T1}{cmr}{m}{n}\selectfont\color{color_29791}nell’espansione della riga, non si possono usare variabili che contengono valori che la shell espande}
\put(41.8,-710.811){\fontsize{12}{1}\usefont{T1}{cmr}{m}{n}\selectfont\color{color_29791}dopo. Viene saltata con quoting a doppi apici “”}
\put(41.8,-731.611){\fontsize{12}{1}\usefont{T1}{cmr}{m}{n}\selectfont\color{color_29791}Sintassi [PRE]\{LISTA\}[POST] oppure [PRE]\{SEQUENZA\}[POST]}
\put(41.8,-752.411){\fontsize{12}{1}\usefont{T1}{cmr}{m}{n}\selectfont\color{color_29791}Esempio lista:a\{d,c,b\}e espanso dalla shell in ade ace abe}
\end{picture}
\newpage
\begin{tikzpicture}[overlay]\path(0pt,0pt);\end{tikzpicture}
\begin{picture}(-5,0)(2.5,0)
\put(41.8,-85.01099){\fontsize{12}{1}\usefont{T1}{cmr}{m}{n}\selectfont\color{color_29791}Esempi sequenza: file\{9..13..2\}.c espanso in file9.c file11.c file13.c}
\put(148.2,-105.811){\fontsize{12}{1}\usefont{T1}{cmr}{m}{n}\selectfont\color{color_29791}doc\{009..11\} espanso in doc009 doc010 doc011 (shell offre zero-}
\put(254.5,-119.611){\fontsize{12}{1}\usefont{T1}{cmr}{m}{n}\selectfont\color{color_29791}padding)}
\put(148.2,-140.411){\fontsize{12}{1}\usefont{T1}{cmr}{m}{n}\selectfont\color{color_29791}\{a..j..3\} a d g j (solo per singoli carattere alfabetico)}
\put(148.2,-161.211){\fontsize{12}{1}\usefont{T1}{cmr}{m}{n}\selectfont\color{color_29791}\{a..c\}\{1,3\} a1 a3 b1 b3 c1 c3 (tutte le combo, “prod. cartesiano”)}
\put(148.2,-182.011){\fontsize{12}{1}\usefont{T1}{cmr}{m}{n}\selectfont\color{color_29791}p\{1\{a,b\},2,3\{b,d\}\} p1a p1b p2 p3b p3d (nesting)}
\put(41.8,-211.811){\fontsize{14.1}{1}\usefont{T1}{cmr}{b}{n}\selectfont\color{color_29791}Variabili}
\put(41.8,-232.011){\fontsize{12}{1}\usefont{T1}{cmr}{m}{n}\selectfont\color{color_29791}Le variabili sono offerte da shell per memorizzare stringhe di testo sotto dato nome. Si creano o }
\put(41.8,-245.811){\fontsize{12}{1}\usefont{T1}{cmr}{m}{n}\selectfont\color{color_29791}modificano semplicemente con pippo=valore (senza spazi, altrimenti tokenizzazione impedisce }
\put(41.8,-259.611){\fontsize{12}{1}\usefont{T1}{cmr}{m}{n}\selectfont\color{color_29791}assegnamento). La param. Exp sostituisce \$NOME con il valore della variabile NOME (se NOME è}
\put(41.8,-273.411){\fontsize{12}{1}\usefont{T1}{cmr}{m}{n}\selectfont\color{color_29791}composto o ambiguo, si protegge con \{\} quindi \$\{NOME\}).}
\put(41.8,-294.211){\fontsize{12}{1}\usefont{T1}{cmr}{m}{n}\selectfont\color{color_29791}Esempi di interazione con quoting:}
\put(59.8,-315.011){\fontsize{12}{1}\usefont{T1}{cmr}{m}{n}\selectfont\color{color_29791}•ls *\** lista tutti i nomi dei file che contengono il carattere * in qualunque posizione}
\put(59.8,-335.811){\fontsize{12}{1}\usefont{T1}{cmr}{m}{n}\selectfont\color{color_29791}•echo “\$A” stampa esattamente il contenuto della variabile A (inibizione dei passi }
\put(77.8,-349.611){\fontsize{12}{1}\usefont{T1}{cmr}{m}{n}\selectfont\color{color_29791}successivi alla param.exp)}
\put(59.8,-370.411){\fontsize{12}{1}\usefont{T1}{cmr}{m}{n}\selectfont\color{color_29791}•echo ‘\$A’ stampa esattamente \$A (inibisce quasi tutti i passi, inclusa param. Exp)}
\put(59.8,-391.211){\fontsize{12}{1}\usefont{T1}{cmr}{m}{n}\selectfont\color{color_29791}•echo “’\$A’” Chi vince? La prima cosa che incontra la shell è l'apice doppio, e va a }
\put(148.7,-405.011){\fontsize{12}{1}\usefont{T1}{cmr}{m}{n}\selectfont\color{color_29791}cercarne la coppia. Gli apici effettivamente usati per proteggere delle }
\put(148.7,-418.811){\fontsize{12}{1}\usefont{T1}{cmr}{m}{n}\selectfont\color{color_29791}sequenze vengono rimossi in 12) Quote rem, quindi l’apice singolo mantiene }
\put(148.7,-432.611){\fontsize{12}{1}\usefont{T1}{cmr}{m}{n}\selectfont\color{color_29791}il suo significato letterale. Utile per usare due protezioni diverse quando si }
\put(148.7,-446.411){\fontsize{12}{1}\usefont{T1}{cmr}{m}{n}\selectfont\color{color_29791}deve passare attraverso due espansioni, in particolare sarà utile per proteggere}
\put(148.7,-460.211){\fontsize{12}{1}\usefont{T1}{cmr}{m}{n}\selectfont\color{color_29791}cose da passare alla shell remota delle VM.}
\put(59.8,-481.011){\fontsize{12}{1}\usefont{T1}{cmr}{m}{n}\selectfont\color{color_29791}•echo \$(ls) ; echo “\$(ls)” La differenza è che nel primo viene fatto 10)Word split, nel }
\put(219.6,-494.811){\fontsize{12}{1}\usefont{T1}{cmr}{m}{n}\selectfont\color{color_29791}secondo no quindi tutti gli accapo restano tali e non cambiano }
\put(219.6,-508.611){\fontsize{12}{1}\usefont{T1}{cmr}{m}{n}\selectfont\color{color_29791}in spazi}
\put(41.8,-529.411){\fontsize{12}{1}\usefont{T1}{cmr}{m}{n}\selectfont\color{color_29791}Le variabili d’ambiente sono usate per mantenere dati riguardo il sistema o le preferenze utente, }
\put(41.8,-543.211){\fontsize{12}{1}\usefont{T1}{cmr}{m}{n}\selectfont\color{color_29791}utili a tutti i comandi. Per evitare di passarle a ogni comando, la shell dispone dell’esportazione }
\put(41.8,-557.011){\fontsize{12}{1}\usefont{T1}{cmr}{m}{n}\selectfont\color{color_29791}(export pippo) con cui variabili standard diventano d’ambiente (ereditabili dai processi figli), }
\put(41.8,-570.811){\fontsize{12}{1}\usefont{T1}{cmr}{m}{n}\selectfont\color{color_29791}altrimenti resterebbero confinate nella shell stessa. L’ambiente (environment) corrente si visualizza }
\put(41.8,-584.611){\fontsize{12}{1}\usefont{T1}{cmr}{m}{n}\selectfont\color{color_29791}con builtin set. unset si può usare per eliminare una variabile}
\put(41.8,-605.411){\fontsize{12}{1}\usefont{T1}{cmr}{m}{n}\selectfont\color{color_29791}Un assegnamento prima di un comando modifica solo per tale esecuzione l’ambiente (VAR=x }
\put(41.8,-619.211){\fontsize{12}{1}\usefont{T1}{cmr}{m}{it}\selectfont\color{color_29791}command), comando env permette un controllo più preciso.}
\put(41.8,-640.011){\fontsize{12}{1}\usefont{T1}{cmr}{m}{n}\selectfont\color{color_29791}Variabili notevoli: (man bash → Shell variables)}
\put(59.8,-660.811){\fontsize{12}{1}\usefont{T1}{cmr}{m}{n}\selectfont\color{color_29791}•Settate da bash}
\put(77.8,-681.611){\fontsize{12}{1}\usefont{T1}{cmr}{m}{n}\selectfont\color{color_29791}◦\$ pid della shell capostipite (per vederne il valore, \$\$)}
\put(77.8,-702.411){\fontsize{12}{1}\usefont{T1}{cmr}{m}{n}\selectfont\color{color_29791}◦HOSTNAME nome dell’host}
\put(77.8,-723.211){\fontsize{12}{1}\usefont{T1}{cmr}{m}{n}\selectfont\color{color_29791}◦RANDOM num. Casuale tra 0 e 32767 (as es. utile per assegnare carico di }
\put(202.2,-737.011){\fontsize{12}{1}\usefont{T1}{cmr}{m}{n}\selectfont\color{color_29791}lavoro tra vari computer)}
\put(59.8,-757.811){\fontsize{12}{1}\usefont{T1}{cmr}{m}{n}\selectfont\color{color_29791}•Usate da bashfor F in "\$(cat /etc/passwd | cut -f5 -d:)" ; do echo \$F | tr 'a-z' 'A-Z'; done}
\end{picture}
\newpage
\begin{tikzpicture}[overlay]\path(0pt,0pt);\end{tikzpicture}
\begin{picture}(-5,0)(2.5,0)
\put(77.8,-85.01099){\fontsize{12}{1}\usefont{T1}{cmr}{m}{n}\selectfont\color{color_29791}◦PS0..PS4 prompt in diversi contesti}
\put(77.8,-105.811){\fontsize{12}{1}\usefont{T1}{cmr}{m}{n}\selectfont\color{color_29791}◦HOME home dir dell’utente}
\put(41.8,-126.611){\fontsize{12}{1}\usefont{T1}{cmr}{m}{n}\selectfont\color{color_29791}Le variabili posizionali sono usate da ogni script per accedere agli argomenti sulla riga di comando }
\put(41.8,-140.411){\fontsize{12}{1}\usefont{T1}{cmr}{m}{n}\selectfont\color{color_29791}(come argv) usando le variabili che hanno per nome un numero. \$0 è il nome del comando, se il }
\put(41.8,-154.211){\fontsize{12}{1}\usefont{T1}{cmr}{m}{n}\selectfont\color{color_29791}numero è >9 va usato \$\{NUM\}. }
\put(41.8,-175.011){\fontsize{12}{1}\usefont{T1}{cmr}{m}{n}\selectfont\color{color_29791}Se non si sa quanti parametri sono stati passati (es. ciclo senza sapere il totale) si può usare \$* }
\put(41.8,-188.811){\fontsize{12}{1}\usefont{T1}{cmr}{m}{n}\selectfont\color{color_29791}oppure \$@, con differenza nel quoting: }
\put(59.8,-209.611){\fontsize{12}{1}\usefont{T1}{cmr}{m}{n}\selectfont\color{color_29791}•"\$*" viene espanso in "\$1 \$2 \$3" }
\put(59.8,-230.411){\fontsize{12}{1}\usefont{T1}{cmr}{m}{n}\selectfont\color{color_29791}•"\$@" viene espanso in "\$1" "\$2" quindi mantiene la protezione per ognuno dei contenuti}
\put(41.8,-272.011){\fontsize{12}{1}\usefont{T1}{cmr}{m}{n}\selectfont\color{color_29791}Il builtin read legge stringhe da stdin e le assegna a variabili. L’input viene tokenizzato usando IFS }
\put(41.8,-285.811){\fontsize{12}{1}\usefont{T1}{cmr}{m}{n}\selectfont\color{color_29791}(di default qualsiasi spaziatore), se ci sono più token che variabili, quelli in eccesso finiscono tutti }
\put(41.8,-299.611){\fontsize{12}{1}\usefont{T1}{cmr}{m}{n}\selectfont\color{color_29791}nell’ultima variabile specificata, come unica stringa separatori inclusi.  }
\put(41.8,-320.411){\fontsize{12}{1}\usefont{T1}{cmr}{m}{n}\selectfont\color{color_29791}(Facendo IFS=: read A B C... cambio IFS in :  , ma solo per read, spazio non è più tokenizzatore }
\put(41.8,-334.211){\fontsize{12}{1}\usefont{T1}{cmr}{m}{n}\selectfont\color{color_29791}per read. Se cambio IFS nella shell generale si scombussola tutto)}
\put(41.8,-355.011){\fontsize{12}{1}\usefont{T1}{cmr}{m}{n}\selectfont\color{color_29791}Opzioni utili -p PROMPT stampa PROMPT prima di leggere input}
\put(148.2,-375.811){\fontsize{12}{1}\usefont{T1}{cmr}{m}{n}\selectfont\color{color_29791}-u FD legge da FD invece che da stdin}
\put(148.2,-396.611){\fontsize{12}{1}\usefont{T1}{cmr}{m}{n}\selectfont\color{color_29791}-a ARRAY assegna i token a elementi di ARRAY}
\put(41.8,-417.411){\fontsize{12}{1}\usefont{T1}{cmr}{m}{n}\selectfont\color{color_29791}Nella realizzazione degli script è utile tenere a mente i sottoprocessi generati in pipe: }
\put(59.8,-438.211){\fontsize{12}{1}\usefont{T1}{cmr}{m}{n}\selectfont\color{color_29791}•manipolare variabili nei processi figli senza perdere i risultati prima di poterli usare → si }
\put(77.8,-452.011){\fontsize{12}{1}\usefont{T1}{cmr}{m}{n}\selectfont\color{color_29791}può usare subshell con echo ciao | ( read A ; echo \$A )}
\put(59.8,-472.811){\fontsize{12}{1}\usefont{T1}{cmr}{m}{n}\selectfont\color{color_29791}•necessità di usare read per acquisire dati interattivamente dall’utente in un processo figlio }
\put(77.8,-486.611){\fontsize{12}{1}\usefont{T1}{cmr}{m}{n}\selectfont\color{color_29791}che ha stdin alimentato da una pipe anziché da un terminale → creare file descriptor per il }
\put(77.8,-500.411){\fontsize{12}{1}\usefont{T1}{cmr}{m}{n}\selectfont\color{color_29791}terminale con exec 3<\$(tty) ; exec ciao | ( read -u 3 A ; echo \$A ) (tty restituisce file }
\put(77.8,-514.211){\fontsize{12}{1}\usefont{T1}{cmr}{m}{n}\selectfont\color{color_29791}speciale che descrive il terminale connesso a stdin)}
\put(41.8,-535.011){\fontsize{12}{1}\usefont{T1}{cmr}{b}{it}\selectfont\color{color_29791}shift sposta gli argomenti della riga di comando (es. \$1 diventa il contenuto di \$2, viene buttato via }
\put(41.8,-548.811){\fontsize{12}{1}\usefont{T1}{cmr}{m}{n}\selectfont\color{color_29791}il contenuto di \$1), \$0 rimane il nome dello scripting.}
\put(41.8,-569.611){\fontsize{12}{1}\usefont{T1}{cmr}{b}{it}\selectfont\color{color_217499}select è builtin che può essere utile per automatizzare la selezionare di alternative , esempio}
\put(41.8,-587.611){\fontsize{9}{1}\usefont{T1}{cmr}{m}{n}\selectfont\color{color_217499}directorylist="Finished \$(for i in /*;do [ -d "\$i" ] \&\& echo \$i; done)"}
\put(41.8,-604.911){\fontsize{9}{1}\usefont{T1}{cmr}{m}{n}\selectfont\color{color_217499}PS3='Directory to process? ' \# Set a useful select prompt}
\put(41.8,-622.311){\fontsize{9}{1}\usefont{T1}{cmr}{m}{n}\selectfont\color{color_217499}until [ "\$directory" == "Finished" ]; do}
\put(77.3,-639.611){\fontsize{9}{1}\usefont{T1}{cmr}{m}{n}\selectfont\color{color_217499}printf "\%b" "\a\n\nSelect a directory to process:\n" >\&2}
\put(77.3,-657.011){\fontsize{9}{1}\usefont{T1}{cmr}{m}{n}\selectfont\color{color_217499}select directory in \$directorylist; do}
\put(112.7,-674.311){\fontsize{9}{1}\usefont{T1}{cmr}{m}{n}\selectfont\color{color_217499}\# User types a number which is stored in \$REPLY, but select}
\put(112.7,-691.711){\fontsize{9}{1}\usefont{T1}{cmr}{m}{n}\selectfont\color{color_217499}\# returns the value of the entry}
\put(112.7,-709.011){\fontsize{9}{1}\usefont{T1}{cmr}{m}{n}\selectfont\color{color_217499}if [ "\$directory" = "Finished" ]; then}
\put(148.2,-726.411){\fontsize{9}{1}\usefont{T1}{cmr}{m}{n}\selectfont\color{color_217499}echo "Finished processing directories."}
\put(148.2,-743.711){\fontsize{9}{1}\usefont{T1}{cmr}{m}{n}\selectfont\color{color_217499}break}
\put(112.7,-761.111){\fontsize{9}{1}\usefont{T1}{cmr}{m}{n}\selectfont\color{color_217499}elif [ -n "\$directory" ]; then}
\end{picture}
\newpage
\begin{tikzpicture}[overlay]\path(0pt,0pt);\end{tikzpicture}
\begin{picture}(-5,0)(2.5,0)
\put(148.2,-82.211){\fontsize{9}{1}\usefont{T1}{cmr}{m}{n}\selectfont\color{color_217499}echo "You chose number \$REPLY, processing \$directory..."}
\put(148.2,-99.51099){\fontsize{9}{1}\usefont{T1}{cmr}{m}{n}\selectfont\color{color_217499}\# Do something here}
\put(148.2,-116.911){\fontsize{9}{1}\usefont{T1}{cmr}{m}{n}\selectfont\color{color_217499}break}
\put(112.7,-134.211){\fontsize{9}{1}\usefont{T1}{cmr}{m}{n}\selectfont\color{color_217499}else}
\put(148.2,-151.611){\fontsize{9}{1}\usefont{T1}{cmr}{m}{n}\selectfont\color{color_217499}echo "Invalid selection!"}
\put(112.7,-168.911){\fontsize{9}{1}\usefont{T1}{cmr}{m}{n}\selectfont\color{color_217499}fi \# end of handle user's selection}
\put(77.3,-186.311){\fontsize{9}{1}\usefont{T1}{cmr}{m}{n}\selectfont\color{color_217499}done \# end of select a directory}
\put(41.8,-203.611){\fontsize{9}{1}\usefont{T1}{cmr}{m}{n}\selectfont\color{color_217499}done \# end of while not finished}
\put(41.8,-223.811){\fontsize{12}{1}\usefont{T1}{cmr}{b}{it}\selectfont\color{color_217499}getopts è builtin utile per realizzare script che supportino sintassi con opzioni precedute da trattino. }
\put(41.8,-237.611){\fontsize{12}{1}\usefont{T1}{cmr}{m}{n}\selectfont\color{color_217499}Sintassi è getopts optstring NAME [arg] dove}
\put(59.8,-258.411){\fontsize{12}{1}\usefont{T1}{cmr}{m}{n}\selectfont\color{color_29791}•optstringdefinisce i caratteri da riconoscere come opzioni, se un carattere è seguito da :}
\put(77.8,-272.211){\fontsize{12}{1}\usefont{T1}{cmr}{m}{n}\selectfont\color{color_217499}significa che si attende un parametro per quell'opzione}
\put(59.8,-293.011){\fontsize{12}{1}\usefont{T1}{cmr}{m}{n}\selectfont\color{color_29791}•NAMEè il nome di variabile in cui collocare il parametro correntemente analizzato}
\put(41.8,-313.811){\fontsize{12}{1}\usefont{T1}{cmr}{m}{n}\selectfont\color{color_217499}Ad ogni invocazione getopts esamina una variabile posizionale, assegnando il suo indice ad }
\put(41.8,-327.611){\fontsize{12}{1}\usefont{T1}{cmr}{m}{n}\selectfont\color{color_217499}OPTIND ed il suo contenuto a NAME. Se un'opzione richiede argomento nella successiva variabile }
\put(41.8,-341.411){\fontsize{12}{1}\usefont{T1}{cmr}{m}{n}\selectfont\color{color_217499}posizionale, questo viene letto (incrementando OPTIND) ed assegnato ad OPTARG. Esempio:}
\put(41.8,-359.411){\fontsize{9}{1}\usefont{T1}{cmr}{m}{n}\selectfont\color{color_217499}\#!/usr/bin/env bash}
\put(41.8,-376.711){\fontsize{9}{1}\usefont{T1}{cmr}{m}{n}\selectfont\color{color_217499}aflag=}
\put(41.8,-394.111){\fontsize{9}{1}\usefont{T1}{cmr}{m}{n}\selectfont\color{color_217499}bflag=}
\put(41.8,-411.411){\fontsize{9}{1}\usefont{T1}{cmr}{m}{n}\selectfont\color{color_217499}while getopts 'ab:' OPTION ; do}
\put(77.3,-428.811){\fontsize{9}{1}\usefont{T1}{cmr}{m}{n}\selectfont\color{color_217499}case \$OPTION in}
\put(112.7,-446.111){\fontsize{9}{1}\usefont{T1}{cmr}{m}{n}\selectfont\color{color_217499}a)aflag=1 ;;}
\put(112.7,-463.511){\fontsize{9}{1}\usefont{T1}{cmr}{m}{n}\selectfont\color{color_217499}b)bflag=1}
\put(148.2,-480.811){\fontsize{9}{1}\usefont{T1}{cmr}{m}{n}\selectfont\color{color_217499}bval="\$OPTARG" ;;}
\put(112.7,-498.211){\fontsize{9}{1}\usefont{T1}{cmr}{m}{n}\selectfont\color{color_217499}?)printf "Usage: \%s: [-a] [-b value] args\n" \$(basename \$0) >\&2}
\put(148.2,-515.511){\fontsize{9}{1}\usefont{T1}{cmr}{m}{n}\selectfont\color{color_217499}exit 2 ;;}
\put(77.3,-532.911){\fontsize{9}{1}\usefont{T1}{cmr}{m}{n}\selectfont\color{color_217499}esac}
\put(41.8,-550.211){\fontsize{9}{1}\usefont{T1}{cmr}{m}{n}\selectfont\color{color_217499}done}
\put(41.8,-567.611){\fontsize{9}{1}\usefont{T1}{cmr}{m}{n}\selectfont\color{color_217499}shift \$((\$OPTIND – 1))\# getopts cycles over \$*, doesn't shift it}
\put(41.8,-584.911){\fontsize{9}{1}\usefont{T1}{cmr}{m}{n}\selectfont\color{color_217499}if [ "\$aflag" ] ; then}
\put(77.3,-602.311){\fontsize{9}{1}\usefont{T1}{cmr}{m}{n}\selectfont\color{color_217499}printf "Option -a specified\n"}
\put(41.8,-619.611){\fontsize{9}{1}\usefont{T1}{cmr}{m}{n}\selectfont\color{color_217499}fi}
\put(41.8,-637.011){\fontsize{9}{1}\usefont{T1}{cmr}{m}{n}\selectfont\color{color_217499}if [ "\$bflag" ] ; then}
\put(77.3,-654.311){\fontsize{9}{1}\usefont{T1}{cmr}{m}{n}\selectfont\color{color_217499}printf 'Option -b "\%s" specified\n' "\$bval"}
\put(41.8,-671.711){\fontsize{9}{1}\usefont{T1}{cmr}{m}{n}\selectfont\color{color_217499}fi}
\put(41.8,-689.011){\fontsize{9}{1}\usefont{T1}{cmr}{m}{n}\selectfont\color{color_217499}printf "Remaining arguments are: \%s\n" "\$*"}
\put(41.8,-718.211){\fontsize{14.1}{1}\usefont{T1}{cmr}{b}{n}\selectfont\color{color_217499}Array}
\put(41.8,-738.411){\fontsize{12}{1}\usefont{T1}{cmr}{m}{n}\selectfont\color{color_217499}Bash supporta array monodimensionali, si dichiarano con declare -a MYVECTOR}
\put(59.8,-759.211){\fontsize{12}{1}\usefont{T1}{cmr}{m}{n}\selectfont\color{color_29791}•si assegnano più valori mettendoli tra parentesi}
\end{picture}
\newpage
\begin{tikzpicture}[overlay]\path(0pt,0pt);\end{tikzpicture}
\begin{picture}(-5,0)(2.5,0)
\put(77.8,-85.01099){\fontsize{12}{1}\usefont{T1}{cmr}{m}{n}\selectfont\color{color_29791}◦MYVECTOR=(un elenco di elementi)}
\put(95.8,-105.811){\fontsize{12}{1}\usefont{T1}{cmr}{m}{n}\selectfont\color{color_217499}echo \$\{MYVECTOR[2]\}\#output→ di}
\put(59.8,-126.611){\fontsize{12}{1}\usefont{T1}{cmr}{m}{n}\selectfont\color{color_29791}•si parsa come read manipolando IFS}
\put(77.8,-147.411){\fontsize{12}{1}\usefont{T1}{cmr}{m}{n}\selectfont\color{color_29791}◦STRING="fai.il.parsing.come.read"}
\put(95.8,-168.211){\fontsize{12}{1}\usefont{T1}{cmr}{m}{n}\selectfont\color{color_217499}IFS='.' MYVECTOR=(\$STRINGA) \#→ valori sempre tra parentesi}
\put(95.8,-189.011){\fontsize{12}{1}\usefont{T1}{cmr}{m}{n}\selectfont\color{color_217499}echo \$\{MYVECTOR[2]\}\#output → parsing}
\put(41.8,-209.811){\fontsize{12}{1}\usefont{T1}{cmr}{b}{n}\selectfont\color{color_29791}Accenno agli array}
\end{picture}
\begin{tikzpicture}[overlay]
\path(0pt,0pt);
\draw[color_29791,line width=0.7pt]
(41.8pt, -206.4109pt) -- (139.1pt, -206.4109pt)
;
\end{tikzpicture}
\begin{picture}(-5,0)(2.5,0)
\put(139.1,-209.811){\fontsize{12}{1}\usefont{T1}{cmr}{m}{n}\selectfont\color{color_29791}:}
\end{picture}
\begin{tikzpicture}[overlay]
\path(0pt,0pt);
\draw[color_29791,line width=0.7pt]
(139.1pt, -206.4109pt) -- (142.4pt, -206.4109pt)
;
\end{tikzpicture}
\begin{picture}(-5,0)(2.5,0)
\put(142.5,-209.811){\fontsize{12}{1}\usefont{T1}{cmr}{m}{n}\selectfont\color{color_29791} gli array in bash usano indici non necessariamente consecutivi. Usarne uno }
\put(41.8,-223.611){\fontsize{12}{1}\usefont{T1}{cmr}{m}{n}\selectfont\color{color_29791}può essere utile per gestire multipli processi, indicizzandoli con il loro PID. }
\put(77.3,-244.411){\fontsize{12}{1}\usefont{T1}{cmr}{b}{it}\selectfont\color{color_29791}echo \$\{!PID[@]\} mostra tutti gli indici validi per l'array di nome PID}
\put(77.3,-265.211){\fontsize{12}{1}\usefont{T1}{cmr}{b}{it}\selectfont\color{color_29791}echo \$\{PID[@]\}mostra tutti i valori contenuti nell'array di nome PID}
\put(41.8,-286.011){\fontsize{12}{1}\usefont{T1}{cmr}{m}{n}\selectfont\color{color_29791}Quindi ! chiede alla shell gli indici dell'array, non i valori; viceversa senza ! mostra i valori }
\put(41.8,-299.811){\fontsize{12}{1}\usefont{T1}{cmr}{m}{n}\selectfont\color{color_29791}contenuti. La protezione dell'espansione della chiocciola è coerente alle variabili posizionali:}
\put(41.8,-320.611){\fontsize{12}{1}\usefont{T1}{cmr}{m}{it}\selectfont\color{color_29791}for i in "\$\{A[@]\}" ; do echo \$i ; done espande ogni valore dell'array A con quoting tra }
\put(41.8,-334.411){\fontsize{12}{1}\usefont{T1}{cmr}{m}{n}\selectfont\color{color_29791}doppi apici per ognuno di essi}
\put(41.8,-355.211){\fontsize{12}{1}\usefont{T1}{cmr}{m}{n}\selectfont\color{color_29791}Es. "1" "2" "3"}
\put(41.8,-376.011){\fontsize{12}{1}\usefont{T1}{cmr}{m}{it}\selectfont\color{color_29791}for i in "\$\{A[*]\}" ; do echo \$i ; done espande l'intera sequenza dei valori dell'array A }
\put(41.8,-389.811){\fontsize{12}{1}\usefont{T1}{cmr}{m}{n}\selectfont\color{color_29791}con quoting tra doppi apici}
\put(41.8,-410.611){\fontsize{12}{1}\usefont{T1}{cmr}{m}{n}\selectfont\color{color_29791}Es. "1 2 3"}
\put(41.8,-431.411){\fontsize{12}{1}\usefont{T1}{cmr}{m}{n}\selectfont\color{color_217499}È possibile ottenere l'espansione di variabili al nome a cui sono assegnate per avere riferimento }
\put(41.8,-445.211){\fontsize{12}{1}\usefont{T1}{cmr}{m}{n}\selectfont\color{color_217499}indiretto ai valori corrispondenti a quel nome: (indirezione)}
\put(77.8,-466.011){\fontsize{12}{1}\usefont{T1}{cmr}{m}{n}\selectfont\color{color_29791}•CHIAVE=PIPPO}
\put(95.8,-486.811){\fontsize{12}{1}\usefont{T1}{cmr}{m}{n}\selectfont\color{color_217499}PIPPO=VALORE}
\put(95.8,-507.611){\fontsize{12}{1}\usefont{T1}{cmr}{m}{n}\selectfont\color{color_217499}echo \$\{!CHIAVE\}\#output→ VALORE}
\put(41.8,-528.411){\fontsize{12}{1}\usefont{T1}{cmr}{m}{n}\selectfont\color{color_217499}In Bash 4+ si hanno veri array associativi (indice può essere una stringa, non solo un numero)}
\put(77.8,-549.211){\fontsize{12}{1}\usefont{T1}{cmr}{m}{n}\selectfont\color{color_29791}•declare -A ASAR\#uso di -A per indicare array associativo, non -a}
\put(95.8,-570.011){\fontsize{12}{1}\usefont{T1}{cmr}{m}{n}\selectfont\color{color_217499}ASAR[chiaveuno]=valoreuno}
\put(95.8,-590.811){\fontsize{12}{1}\usefont{T1}{cmr}{m}{n}\selectfont\color{color_217499}echo \$\{ASAR[chiaveuno]\}\#output → valoreuno}
\put(95.8,-611.611){\fontsize{12}{1}\usefont{T1}{cmr}{m}{n}\selectfont\color{color_217499}KEY=chiaveuno}
\put(95.8,-632.411){\fontsize{12}{1}\usefont{T1}{cmr}{m}{n}\selectfont\color{color_217499}echo \$\{ASAR[KEY]\}\#output → valoreuno}
\put(41.8,-662.211){\fontsize{14.1}{1}\usefont{T1}{cmr}{b}{n}\selectfont\color{color_29791}Arithmetic Expansion}
\put(41.8,-682.411){\fontsize{12}{1}\usefont{T1}{cmr}{m}{n}\selectfont\color{color_29791}Tutto viene trattato come stringa in bash, salvo contesti particolari in cui può svolgere operazioni }
\put(41.8,-696.211){\fontsize{12}{1}\usefont{T1}{cmr}{m}{n}\selectfont\color{color_29791}aritmetiche. La shell ha il limite di poter lavorare solo con numeri interi e non c’è modo di }
\put(41.8,-710.011){\fontsize{12}{1}\usefont{T1}{cmr}{m}{n}\selectfont\color{color_29791}accorgersi del rollover (se si ha overflow sul signed int dell’architettura); bc è calcolatore a riga di }
\put(41.8,-723.811){\fontsize{12}{1}\usefont{T1}{cmr}{m}{n}\selectfont\color{color_29791}comando che può essere usato per avere risultati a precisione arbitraria in script.}
\put(41.8,-744.611){\fontsize{12}{1}\usefont{T1}{cmr}{m}{n}\selectfont\color{color_29791}Si ha valutazione aritmetica in due casi}
\put(59.8,-765.411){\fontsize{12}{1}\usefont{T1}{cmr}{m}{n}\selectfont\color{color_29791}•declare -i N per dichiarare variabili come intere:}
\end{picture}
\newpage
\begin{tikzpicture}[overlay]\path(0pt,0pt);\end{tikzpicture}
\begin{picture}(-5,0)(2.5,0)
\put(77.8,-85.01099){\fontsize{12}{1}\usefont{T1}{cmr}{m}{n}\selectfont\color{color_29791}◦La shell non ha concetto di tipi, le variabili sono stringhe. SOLO quando si dichiara una }
\put(95.8,-98.81097){\fontsize{12}{1}\usefont{T1}{cmr}{m}{n}\selectfont\color{color_29791}var come intera e viene assegnata una stringa che può essere interpretata come intero, }
\put(95.8,-112.611){\fontsize{12}{1}\usefont{T1}{cmr}{m}{n}\selectfont\color{color_29791}questa viene così assegnata. }
\put(77.8,-133.411){\fontsize{12}{1}\usefont{T1}{cmr}{m}{n}\selectfont\color{color_29791}◦Con -p print mostra tipo e valore del simbolo (output – nessun tipo)}
\put(59.8,-154.211){\fontsize{12}{1}\usefont{T1}{cmr}{m}{n}\selectfont\color{color_29791}•Builtin let, o equivalente comando composto (( ))}
\put(77.8,-175.011){\fontsize{12}{1}\usefont{T1}{cmr}{m}{n}\selectfont\color{color_29791}◦let N++ l’espressione viene valutata, ha effetti sulle variabili, ritorna true se }
\put(95.8,-188.811){\fontsize{12}{1}\usefont{T1}{cmr}{m}{n}\selectfont\color{color_29791}risultato non è nullo, produce errori su stder ma nulla su stdout}
\put(77.8,-209.611){\fontsize{12}{1}\usefont{T1}{cmr}{m}{n}\selectfont\color{color_29791}◦(()) si comporta come “” per proteggere elementi dell’expr, riconosce variabili anche se }
\put(95.8,-223.411){\fontsize{12}{1}\usefont{T1}{cmr}{m}{n}\selectfont\color{color_29791}non sono dichiarate intere e senza bisogno di prefisso \$ per espansione (ammessa, ma se }
\put(95.8,-237.211){\fontsize{12}{1}\usefont{T1}{cmr}{m}{n}\selectfont\color{color_29791}aggiunta non vale il seguente) e interpretandole come zero se non definite}
\put(77.8,-258.011){\fontsize{12}{1}\usefont{T1}{cmr}{m}{n}\selectfont\color{color_29791}◦Variabili mai definite non danno errore in contesto aritmetico, vengono valutate come }
\put(95.8,-271.811){\fontsize{12}{1}\usefont{T1}{cmr}{m}{n}\selectfont\color{color_29791}fossero 0. Gli operatori sono quelli del C}
\put(95.8,-292.611){\fontsize{12}{1}\usefont{T1}{cmr}{m}{n}\selectfont\color{color_29791}▪id++ ++id post/pre incremento (decr. Con –) la variabile viene espansa }
\put(113.8,-306.411){\fontsize{12}{1}\usefont{T1}{cmr}{m}{n}\selectfont\color{color_29791}prima/dopo l’incremento}
\put(95.8,-327.211){\fontsize{12}{1}\usefont{T1}{cmr}{m}{n}\selectfont\color{color_29791}▪+ - * / somma sottraz. prodotto divis.}
\put(95.8,-348.011){\fontsize{12}{1}\usefont{T1}{cmr}{m}{n}\selectfont\color{color_29791}▪** \% elevamento a potenza, modulo}
\put(95.8,-368.811){\fontsize{12}{1}\usefont{T1}{cmr}{m}{n}\selectfont\color{color_29791}▪! ~ \& | \^ NOT NOT AND OR XOR bit a bit}
\put(95.8,-389.611){\fontsize{12}{1}\usefont{T1}{cmr}{m}{n}\selectfont\color{color_29791}▪<< >> shift binario a sx/dx}
\put(95.8,-410.411){\fontsize{12}{1}\usefont{T1}{cmr}{m}{n}\selectfont\color{color_29791}▪= *= /= += -= <<= >>= \&= \^= |= assegnamenti}
\put(95.8,-431.211){\fontsize{12}{1}\usefont{T1}{cmr}{m}{n}\selectfont\color{color_29791}▪<= >= < > == != confronto }
\put(95.8,-452.011){\fontsize{12}{1}\usefont{T1}{cmr}{m}{n}\selectfont\color{color_29791}▪\&\& || AND OR logico}
\put(95.8,-472.811){\fontsize{12}{1}\usefont{T1}{cmr}{m}{n}\selectfont\color{color_29791}▪expr?expr2:expr3restituisce risultato di expr2 o expr3 se expr è }
\put(291.1,-486.611){\fontsize{12}{1}\usefont{T1}{cmr}{m}{n}\selectfont\color{color_29791}true/false}
\put(77.8,-507.411){\fontsize{12}{1}\usefont{T1}{cmr}{m}{n}\selectfont\color{color_29791}◦I numeri si possono esprimere tra base 2 e base 64 con prefisso B\#num→ base B}
\put(95.8,-528.211){\fontsize{12}{1}\usefont{T1}{cmr}{m}{n}\selectfont\color{color_29791}▪default 10, prefisso 0 ottale, prefisso 0x esadecimale}
\put(95.8,-549.011){\fontsize{12}{1}\usefont{T1}{cmr}{m}{n}\selectfont\color{color_29791}▪cifre utilizzabili 0..9a..z..Z@\_}
\end{picture}
\begin{tikzpicture}[overlay]
\path(0pt,0pt);
\draw[color_29919,line width=0.7pt]
(233.8pt, -550.111pt) -- (258.2pt, -550.111pt)
;
\end{tikzpicture}
\begin{picture}(-5,0)(2.5,0)
\put(258.2,-549.011){\fontsize{12}{1}\usefont{T1}{cmr}{m}{n}\selectfont\color{color_29791} (servono 64 cifre per base64)}
\put(41.8,-578.811){\fontsize{14.1}{1}\usefont{T1}{cmr}{b}{n}\selectfont\color{color_29791}Controllo di flusso}
\put(41.8,-608.011){\fontsize{14.1}{1}\usefont{T1}{cmr}{b}{n}\selectfont\color{color_29791}Sequenze}
\put(41.8,-628.211){\fontsize{12}{1}\usefont{T1}{cmr}{m}{n}\selectfont\color{color_29791}Con ( ) Si possono raggruppare comandi per trattarli come uno solo aprendo una subshell, per }
\put(41.8,-642.011){\fontsize{12}{1}\usefont{T1}{cmr}{m}{n}\selectfont\color{color_29791}creare una sequenza senza aprire una subshell (mantenere esecuzione nello spazio di memoria della }
\put(41.8,-655.811){\fontsize{12}{1}\usefont{T1}{cmr}{m}{n}\selectfont\color{color_29791}shell principale), si usa \{ \}. Ogni comando della sequenza deve terminare con ; o newline}
\put(41.8,-676.611){\fontsize{12}{1}\usefont{T1}{cmr}{m}{n}\selectfont\color{color_29791}In entrambi i casi l’exit code della sequenza è quello dell’ultimo comando eseguito}
\put(41.8,-697.411){\fontsize{12}{1}\usefont{T1}{cmr}{b}{it}\selectfont\color{color_29791}\$'\x0a' produce ASCII newline (vedi man bash → \$’string’)}
\put(41.8,-727.211){\fontsize{14.1}{1}\usefont{T1}{cmr}{b}{n}\selectfont\color{color_29791}Funzioni}
\put(41.8,-747.411){\fontsize{12}{1}\usefont{T1}{cmr}{m}{n}\selectfont\color{color_29791}Le funzioni sono sequenze con un nome function NOME() \{ SEQUENZA ; \} (il punto e virgola }
\put(41.8,-761.211){\fontsize{12}{1}\usefont{T1}{cmr}{m}{n}\selectfont\color{color_29791}serve per ricordare l’accapo, se non si mettesse il ; l’ultima graffa dovrebbe andare accapo). }
\end{picture}
\newpage
\begin{tikzpicture}[overlay]\path(0pt,0pt);\end{tikzpicture}
\begin{picture}(-5,0)(2.5,0)
\put(41.8,-85.01099){\fontsize{12}{1}\usefont{T1}{cmr}{m}{n}\selectfont\color{color_29791}Possono ricevere parametri, utilizzabili all’interno allo stesso modo dei parametri posizionali degli }
\put(41.8,-98.81097){\fontsize{12}{1}\usefont{T1}{cmr}{m}{n}\selectfont\color{color_29791}script \$1,\$2… ma appunto per questo i valori passati allo script non possono essere visti }
\put(41.8,-112.611){\fontsize{12}{1}\usefont{T1}{cmr}{m}{n}\selectfont\color{color_29791}internamente alla funzione, solo \$0 resta impostato al nome dello script (si possono passare per }
\put(41.8,-126.411){\fontsize{12}{1}\usefont{T1}{cmr}{m}{n}\selectfont\color{color_29791}valore all’interno: se si passa \$1 del main come arg funzione, copia il valore \$1 del main). Il nome }
\put(41.8,-140.211){\fontsize{12}{1}\usefont{T1}{cmr}{m}{n}\selectfont\color{color_29791}della funzione viene posto in \$FUNCNAME.}
\put(41.8,-161.011){\fontsize{12}{1}\usefont{T1}{cmr}{m}{n}\selectfont\color{color_29791}Se invocate semplicemente, sono eseguite nel contesto del chiamante. Al contrario di C e assembly, }
\put(41.8,-174.811){\fontsize{12}{1}\usefont{T1}{cmr}{m}{n}\selectfont\color{color_29791}non c'è creazione di contesto diverso: non viene creato altro processo, è lo stesso che devia e va ad }
\put(41.8,-188.611){\fontsize{12}{1}\usefont{T1}{cmr}{m}{n}\selectfont\color{color_29791}eseguire prima di riprendere da dov'è arrivato. }
\put(41.8,-209.411){\fontsize{12}{1}\usefont{T1}{cmr}{m}{n}\selectfont\color{color_29791}Le variabili sono “globali”: una variabile A modificata in una funzione cambia valore nello spazio }
\put(41.8,-223.211){\fontsize{12}{1}\usefont{T1}{cmr}{m}{n}\selectfont\color{color_29791}di memoria generale, si possono dichiarare variabili locali con local. Per le invocazioni in pipeline }
\put(41.8,-237.011){\fontsize{12}{1}\usefont{T1}{cmr}{m}{n}\selectfont\color{color_29791}la funzione verrà eseguita dalla shell figlia creata in automatico: scrivendo cat miofile | }
\put(41.8,-250.811){\fontsize{12}{1}\usefont{T1}{cmr}{m}{it}\selectfont\color{color_29791}function | sort si ha creazione automatica di 3 shell; in questo caso non è più vero che lo spazio di }
\put(41.8,-264.611){\fontsize{12}{1}\usefont{T1}{cmr}{m}{n}\selectfont\color{color_29791}memoria è lo stesso del chiamante! La seconda fork non fa exec , è una subshell che interpreta }
\put(41.8,-278.411){\fontsize{12}{1}\usefont{T1}{cmr}{m}{n}\selectfont\color{color_29791}direttamente la funzione (shell diversa dalla shell che prepara la pipeline). }
\put(41.8,-308.211){\fontsize{14.1}{1}\usefont{T1}{cmr}{b}{n}\selectfont\color{color_29791}Valutazione delle condizioni}
\put(41.8,-328.411){\fontsize{12}{1}\usefont{T1}{cmr}{m}{n}\selectfont\color{color_29791}I comandi di controllo di flusso non fanno valutazione logica delle espressioni, decidono il percorso}
\put(41.8,-342.211){\fontsize{12}{1}\usefont{T1}{cmr}{m}{n}\selectfont\color{color_29791}in base all’exit code di un processo (0==true, altri valori==false). L’exit code di un processo si trova}
\put(41.8,-356.011){\fontsize{12}{1}\usefont{T1}{cmr}{m}{n}\selectfont\color{color_29791}nella variabile speciale \$? settata dalla shell al termine del processo. E’ comodo ma abbiamo }
\put(41.8,-369.811){\fontsize{12}{1}\usefont{T1}{cmr}{m}{n}\selectfont\color{color_29791}interpreti di espressioni logiche per quando serve, sia come builtin che come comandi esterni:}
\put(59.8,-390.611){\fontsize{12}{1}\usefont{T1}{cmr}{m}{n}\selectfont\color{color_29791}•Builtin test, [ ], [[ ]]}
\put(59.8,-411.411){\fontsize{12}{1}\usefont{T1}{cmr}{m}{n}\selectfont\color{color_29791}•Comandi test, [ ]}
\put(41.8,-432.211){\fontsize{12}{1}\usefont{T1}{cmr}{m}{n}\selectfont\color{color_29791}Essendo i builtin eseguiti prima di cercare i comandi equivalenti faremo riferimento a quelli, ma è }
\put(41.8,-446.011){\fontsize{12}{1}\usefont{T1}{cmr}{m}{n}\selectfont\color{color_29791}utile ricordare che se serve portabilità degli script, i comandi esterni sono standard su molti sistemi }
\put(41.8,-459.811){\fontsize{12}{1}\usefont{T1}{cmr}{m}{n}\selectfont\color{color_29791}indipendentemente dalla shell usata, tipicamente hanno sintassi identiche.}
\put(41.8,-480.611){\fontsize{12}{1}\usefont{T1}{cmr}{b}{n}\selectfont\color{color_29791}test e [ sono lo stesso builtin (solo, [ richiede ] come ultimo parametro) e seguiti da un’espressione }
\put(41.8,-494.411){\fontsize{12}{1}\usefont{T1}{cmr}{m}{n}\selectfont\color{color_29791}la valutano e ritornano 0 o 1 in base al risultato (0 TRUE, 1 FALSE}
\end{picture}
\begin{tikzpicture}[overlay]
\path(0pt,0pt);
\draw[color_29791,line width=0.7pt]
(274.8pt, -495.511pt) -- (365.7pt, -495.511pt)
;
\end{tikzpicture}
\begin{picture}(-5,0)(2.5,0)
\put(365.7,-494.411){\fontsize{12}{1}\usefont{T1}{cmr}{m}{n}\selectfont\color{color_29791})}
\put(59.8,-515.211){\fontsize{12}{1}\usefont{T1}{cmr}{m}{n}\selectfont\color{color_29791}•Test unari, un solo parametro: test OP ARG}
\put(77.8,-536.011){\fontsize{12}{1}\usefont{T1}{cmr}{m}{n}\selectfont\color{color_29791}◦su stringhe}
\put(95.8,-556.811){\fontsize{12}{1}\usefont{T1}{cmr}{m}{n}\selectfont\color{color_29791}▪-z true se stringa è vuota}
\put(95.8,-577.611){\fontsize{12}{1}\usefont{T1}{cmr}{m}{n}\selectfont\color{color_29791}▪-n true se string è non vuota}
\put(77.8,-598.411){\fontsize{12}{1}\usefont{T1}{cmr}{m}{n}\selectfont\color{color_29791}◦su file}
\put(95.8,-619.211){\fontsize{12}{1}\usefont{T1}{cmr}{m}{n}\selectfont\color{color_29791}▪-e true se file esiste}
\put(95.8,-640.011){\fontsize{12}{1}\usefont{T1}{cmr}{m}{n}\selectfont\color{color_29791}▪-f true se è file regolare}
\put(95.8,-660.811){\fontsize{12}{1}\usefont{T1}{cmr}{m}{n}\selectfont\color{color_29791}▪-d true se è directory}
\put(95.8,-681.611){\fontsize{12}{1}\usefont{T1}{cmr}{m}{n}\selectfont\color{color_29791}▪-s true se file non è vuoto}
\put(77.8,-702.411){\fontsize{12}{1}\usefont{T1}{cmr}{m}{n}\selectfont\color{color_29791}◦altri 20 su help test (è builtin, quindi o help o man bash e ricerca)}
\put(59.8,-723.211){\fontsize{12}{1}\usefont{T1}{cmr}{m}{n}\selectfont\color{color_29791}•Test binari, due parametri: test ARG1 OP ARG2}
\put(77.8,-744.011){\fontsize{12}{1}\usefont{T1}{cmr}{m}{n}\selectfont\color{color_29791}◦confronto lessicale (alfabetico) tra stringhe}
\end{picture}
\newpage
\begin{tikzpicture}[overlay]\path(0pt,0pt);\end{tikzpicture}
\begin{picture}(-5,0)(2.5,0)
\put(95.8,-85.01099){\fontsize{12}{1}\usefont{T1}{cmr}{m}{n}\selectfont\color{color_29791}▪=, !=, <, > per qualche motivo < e > non sono per verifiche numeriche tra }
\put(113.8,-98.81097){\fontsize{12}{1}\usefont{T1}{cmr}{m}{n}\selectfont\color{color_29791}stringhe ma per verifica lessicale (alfabetico)}
\put(77.8,-119.611){\fontsize{12}{1}\usefont{T1}{cmr}{m}{n}\selectfont\color{color_29791}◦confronto numerico tra stringhe}
\put(95.8,-140.411){\fontsize{12}{1}\usefont{T1}{cmr}{m}{n}\selectfont\color{color_29791}▪-eq, -ne equal, not equal}
\put(95.8,-161.211){\fontsize{12}{1}\usefont{T1}{cmr}{m}{n}\selectfont\color{color_29791}▪-lt, -le less than, less or equal}
\put(95.8,-182.011){\fontsize{12}{1}\usefont{T1}{cmr}{m}{n}\selectfont\color{color_29791}▪-gt, -ge greater than, greater or equal}
\put(77.8,-202.811){\fontsize{12}{1}\usefont{T1}{cmr}{m}{n}\selectfont\color{color_29791}◦confronto tra file}
\put(95.8,-223.611){\fontsize{12}{1}\usefont{T1}{cmr}{m}{n}\selectfont\color{color_29791}▪-nt newer than}
\put(95.8,-244.411){\fontsize{12}{1}\usefont{T1}{cmr}{m}{n}\selectfont\color{color_29791}▪-ot older than}
\put(41.8,-265.211){\fontsize{12}{1}\usefont{T1}{cmr}{b}{n}\selectfont\color{color_29791}[[ ]] è builtin presente nelle versioni più recenti di bash, in più di test / [ offre}
\put(59.8,-286.011){\fontsize{12}{1}\usefont{T1}{cmr}{m}{n}\selectfont\color{color_29791}•op binari == e != matchano param. Sinistro con pattern espresso dal parametro di destra con }
\put(77.8,-299.811){\fontsize{12}{1}\usefont{T1}{cmr}{m}{n}\selectfont\color{color_29791}stessa sintassi di pathname expansion ([[]] equivale a quoting, i metacaratteri vengono }
\put(77.8,-313.611){\fontsize{12}{1}\usefont{T1}{cmr}{m}{n}\selectfont\color{color_29791}protetti)}
\put(77.8,-334.411){\fontsize{12}{1}\usefont{T1}{cmr}{m}{n}\selectfont\color{color_29791}◦Es. [[ “ciao” == c?a[l-z] ]] → true}
\put(59.8,-355.211){\fontsize{12}{1}\usefont{T1}{cmr}{m}{n}\selectfont\color{color_29791}•op binario =~ matcha param. Sinistro con una regular expression specificata dal param, }
\put(77.8,-369.011){\fontsize{12}{1}\usefont{T1}{cmr}{m}{n}\selectfont\color{color_29791}destro}
\put(77.8,-389.811){\fontsize{12}{1}\usefont{T1}{cmr}{m}{n}\selectfont\color{color_29791}◦Es. [[ “ciao” =~ \^c.\{2\}o\$ ]] → true}
\put(77.3,-410.611){\fontsize{12}{1}\usefont{T1}{cmr}{m}{n}\selectfont\color{color_29791}Simboli speciali uguali hanno significati diversi nei due casi (sintassi pathname exp != }
\put(77.3,-424.411){\fontsize{12}{1}\usefont{T1}{cmr}{m}{n}\selectfont\color{color_29791}sintassi regex)}
\put(41.8,-445.211){\fontsize{12}{1}\usefont{T1}{cmr}{m}{n}\selectfont\color{color_29791}E’ possibile combinare controlli elementari:}
\put(59.8,-466.011){\fontsize{12}{1}\usefont{T1}{cmr}{m}{n}\selectfont\color{color_29791}•con test / [ ]}
\put(77.8,-486.811){\fontsize{12}{1}\usefont{T1}{cmr}{m}{n}\selectfont\color{color_29791}◦operatori AND, OR, NOT sono rispettivamente -a, -o, !}
\put(77.8,-507.611){\fontsize{12}{1}\usefont{T1}{cmr}{m}{n}\selectfont\color{color_29791}◦si possono fare raggruppamenti con ( ), vanno protette con backslash , si usano \( \)}
\put(59.8,-528.411){\fontsize{12}{1}\usefont{T1}{cmr}{m}{n}\selectfont\color{color_29791}•con [[ ]]}
\put(77.8,-549.211){\fontsize{12}{1}\usefont{T1}{cmr}{m}{n}\selectfont\color{color_29791}◦operatori AND, OR, NOT sono rispettivamente \&\&, ||, !}
\put(77.8,-570.011){\fontsize{12}{1}\usefont{T1}{cmr}{m}{n}\selectfont\color{color_29791}◦si possono fare raggruppamenti con (), protette già dalle [[ ]], si usano così come sono}
\put(41.8,-590.811){\fontsize{12}{1}\usefont{T1}{cmr}{m}{n}\selectfont\color{color_29791}Gli operatori \&\&, ||, ! si possono usare sulla riga di comando per combinare logicamente gli exit }
\put(41.8,-604.611){\fontsize{12}{1}\usefont{T1}{cmr}{m}{n}\selectfont\color{color_29791}code di qualsiasi processo. Sia in questo caso che quando usati dentro [[ ]], valutazione shortcut: se }
\put(41.8,-618.411){\fontsize{12}{1}\usefont{T1}{cmr}{m}{n}\selectfont\color{color_29791}il primo risultato è sufficiente a sapere il risultato, il secondo non viene valutato. Nel caso dei }
\put(41.8,-632.211){\fontsize{12}{1}\usefont{T1}{cmr}{m}{n}\selectfont\color{color_29791}processi è come un if, quindi con AND se il primo comando torna false il secondo non viene }
\put(41.8,-646.011){\fontsize{12}{1}\usefont{T1}{cmr}{m}{n}\selectfont\color{color_29791}eseguito, viceversa con OR solo se il primo torna false il secondo viene eseguito. Utile per avere }
\put(41.8,-659.811){\fontsize{12}{1}\usefont{T1}{cmr}{m}{n}\selectfont\color{color_29791}check:}
\put(41.8,-680.611){\fontsize{12}{1}\usefont{T1}{cmr}{b}{it}\selectfont\color{color_29791}cd "\$MYDIR" || echo ("Errore") ; exit(1) il ramo destro viene eseguito solo se fallisce il }
\put(290,-694.411){\fontsize{12}{1}\usefont{T1}{cmr}{m}{n}\selectfont\color{color_29791}sinistro, con messaggio e bloccante (uscita)}
\put(41.8,-715.211){\fontsize{12}{1}\usefont{T1}{cmr}{b}{n}\selectfont\color{color_29791}if è utile tutte le volte che voglio esecuzione condizionata: la keyword if vuole un COMANDO }
\put(41.8,-729.011){\fontsize{12}{1}\usefont{T1}{cmr}{m}{n}\selectfont\color{color_29791}come argomento (esterno eseguito da shell figlia, builtin, funzione… ecc) e solo se il comando }
\put(41.8,-742.811){\fontsize{12}{1}\usefont{T1}{cmr}{m}{n}\selectfont\color{color_29791}ritorna 0 viene considerato come TRUE, qualsiasi altro ritorno viene visto come FALSE. elif per }
\put(41.8,-756.611){\fontsize{12}{1}\usefont{T1}{cmr}{m}{n}\selectfont\color{color_29791}seconde verifiche, else per caso alternativo globale (questi due vanno racchiusi tra [ ] ), fi chiude la }
\put(41.8,-770.411){\fontsize{12}{1}\usefont{T1}{cmr}{m}{n}\selectfont\color{color_29791}sequenza. In una pipeline (o anche funzione senza nome con molti comandi con ; ; ;) l’exit code è }
\end{picture}
\newpage
\begin{tikzpicture}[overlay]\path(0pt,0pt);\end{tikzpicture}
\begin{picture}(-5,0)(2.5,0)
\put(41.8,-85.01099){\fontsize{12}{1}\usefont{T1}{cmr}{m}{n}\selectfont\color{color_29791}quello dell’ultimo comando eseguito. I COMANDI possono essere composti (sequenze, subshell, }
\put(41.8,-98.81097){\fontsize{12}{1}\usefont{T1}{cmr}{m}{n}\selectfont\color{color_29791}combinazioni logiche...)}
\end{picture}
\begin{tikzpicture}[overlay]
\path(0pt,0pt);
\filldraw[color_282751][even odd rule]
(41.7pt, -122.161pt) -- (49.05pt, -122.161pt)
 -- (49.05pt, -122.161pt)
 -- (49.05pt, -108.411pt)
 -- (49.05pt, -108.411pt)
 -- (41.7pt, -108.411pt) -- cycle
;
\end{tikzpicture}
\begin{picture}(-5,0)(2.5,0)
\put(41.8,-119.611){\fontsize{12}{1}\usefont{T1}{cmr}{m}{n}\selectfont\color{color_29791}if COMANDO1}
\end{picture}
\begin{tikzpicture}[overlay]
\path(0pt,0pt);
\filldraw[color_282751][even odd rule]
(41.7pt, -142.961pt) -- (62.35pt, -142.961pt)
 -- (62.35pt, -142.961pt)
 -- (62.35pt, -129.211pt)
 -- (62.35pt, -129.211pt)
 -- (41.7pt, -129.211pt) -- cycle
;
\end{tikzpicture}
\begin{picture}(-5,0)(2.5,0)
\put(41.8,-140.411){\fontsize{12}{1}\usefont{T1}{cmr}{m}{n}\selectfont\color{color_29791}then }
\put(77.3,-161.211){\fontsize{12}{1}\usefont{T1}{cmr}{m}{n}\selectfont\color{color_29791}comandi eseguiti se COMANDO1 ritorna true}
\put(41.8,-182.011){\fontsize{12}{1}\usefont{T1}{cmr}{m}{n}\selectfont\color{color_29791}[ }
\end{picture}
\begin{tikzpicture}[overlay]
\path(0pt,0pt);
\filldraw[color_282751][even odd rule]
(48.7pt, -184.561pt) -- (64.65pt, -184.561pt)
 -- (64.65pt, -184.561pt)
 -- (64.65pt, -170.811pt)
 -- (64.65pt, -170.811pt)
 -- (48.7pt, -170.811pt) -- cycle
;
\end{tikzpicture}
\begin{picture}(-5,0)(2.5,0)
\put(48.8,-182.011){\fontsize{12}{1}\usefont{T1}{cmr}{m}{n}\selectfont\color{color_29791}elif COMANDO 2}
\end{picture}
\begin{tikzpicture}[overlay]
\path(0pt,0pt);
\filldraw[color_282751][even odd rule]
(41.7pt, -205.361pt) -- (62.35pt, -205.361pt)
 -- (62.35pt, -205.361pt)
 -- (62.35pt, -191.611pt)
 -- (62.35pt, -191.611pt)
 -- (41.7pt, -191.611pt) -- cycle
;
\end{tikzpicture}
\begin{picture}(-5,0)(2.5,0)
\put(41.8,-202.811){\fontsize{12}{1}\usefont{T1}{cmr}{m}{n}\selectfont\color{color_29791}then }
\put(77.3,-223.611){\fontsize{12}{1}\usefont{T1}{cmr}{m}{n}\selectfont\color{color_29791}comandi eseguiti se COMANDO2 ritorna true ]}
\put(41.8,-244.411){\fontsize{12}{1}\usefont{T1}{cmr}{m}{n}\selectfont\color{color_29791}[ }
\end{picture}
\begin{tikzpicture}[overlay]
\path(0pt,0pt);
\filldraw[color_282751][even odd rule]
(48.7pt, -246.961pt) -- (67.35pt, -246.961pt)
 -- (67.35pt, -246.961pt)
 -- (67.35pt, -233.211pt)
 -- (67.35pt, -233.211pt)
 -- (48.7pt, -233.211pt) -- cycle
;
\end{tikzpicture}
\begin{picture}(-5,0)(2.5,0)
\put(48.8,-244.411){\fontsize{12}{1}\usefont{T1}{cmr}{m}{n}\selectfont\color{color_29791}else }
\put(77.3,-265.211){\fontsize{12}{1}\usefont{T1}{cmr}{m}{n}\selectfont\color{color_29791}comandi eseguiti se nessun ritorno true ]}
\end{picture}
\begin{tikzpicture}[overlay]
\path(0pt,0pt);
\filldraw[color_282751][even odd rule]
(41.7pt, -288.561pt) -- (49.05pt, -288.561pt)
 -- (49.05pt, -288.561pt)
 -- (49.05pt, -274.811pt)
 -- (49.05pt, -274.811pt)
 -- (41.7pt, -274.811pt) -- cycle
;
\end{tikzpicture}
\begin{picture}(-5,0)(2.5,0)
\put(41.8,-286.011){\fontsize{12}{1}\usefont{T1}{cmr}{m}{n}\selectfont\color{color_29791}fi}
\put(41.8,-306.811){\fontsize{12}{1}\usefont{T1}{cmr}{b}{n}\selectfont\color{color_29791}case permette di esprimere condizioni multiple in modo più leggibile che in una catena di elif. Si }
\put(41.8,-320.611){\fontsize{12}{1}\usefont{T1}{cmr}{m}{n}\selectfont\color{color_29791}usa la stessa sintassi della path exp per i casi, (ma non viene davvero eseguita espansione del nome }
\put(41.8,-334.411){\fontsize{12}{1}\usefont{T1}{cmr}{m}{n}\selectfont\color{color_29791}del case, ovviamente). Se nella variabile c’è valore corrispondente alla sintassi, si entra nel caso. }
\put(41.8,-348.211){\fontsize{12}{1}\usefont{T1}{cmr}{m}{n}\selectfont\color{color_29791}Vedi esempio}
\end{picture}
\begin{tikzpicture}[overlay]
\path(0pt,0pt);
\filldraw[color_282751][even odd rule]
(41.7pt, -371.561pt) -- (62.35pt, -371.561pt)
 -- (62.35pt, -371.561pt)
 -- (62.35pt, -357.811pt)
 -- (62.35pt, -357.811pt)
 -- (41.7pt, -357.811pt) -- cycle
;
\end{tikzpicture}
\begin{picture}(-5,0)(2.5,0)
\put(41.8,-369.011){\fontsize{12}{1}\usefont{T1}{cmr}{m}{n}\selectfont\color{color_29791}case “\$variabile” }
\end{picture}
\begin{tikzpicture}[overlay]
\path(0pt,0pt);
\filldraw[color_282751][even odd rule]
(127pt, -371.561pt) -- (136.35pt, -371.561pt)
 -- (136.35pt, -371.561pt)
 -- (136.35pt, -357.811pt)
 -- (136.35pt, -357.811pt)
 -- (127pt, -357.811pt) -- cycle
;
\end{tikzpicture}
\begin{picture}(-5,0)(2.5,0)
\put(127.1,-369.011){\fontsize{12}{1}\usefont{T1}{cmr}{m}{n}\selectfont\color{color_29791}in }
\put(41.8,-389.811){\fontsize{12}{1}\usefont{T1}{cmr}{m}{n}\selectfont\color{color_29791}nome1}
\end{picture}
\begin{tikzpicture}[overlay]
\path(0pt,0pt);
\filldraw[color_282751][even odd rule]
(74.4pt, -392.361pt) -- (78.35pt, -392.361pt)
 -- (78.35pt, -392.361pt)
 -- (78.35pt, -378.611pt)
 -- (78.35pt, -378.611pt)
 -- (74.4pt, -378.611pt) -- cycle
;
\end{tikzpicture}
\begin{picture}(-5,0)(2.5,0)
\put(74.5,-389.811){\fontsize{12}{1}\usefont{T1}{cmr}{m}{n}\selectfont\color{color_29791}) echo vale nome1 }
\end{picture}
\begin{tikzpicture}[overlay]
\path(0pt,0pt);
\filldraw[color_282751][even odd rule]
(165.7pt, -392.361pt) -- (172.25pt, -392.361pt)
 -- (172.25pt, -392.361pt)
 -- (172.25pt, -378.611pt)
 -- (172.25pt, -378.611pt)
 -- (165.7pt, -378.611pt) -- cycle
;
\end{tikzpicture}
\begin{picture}(-5,0)(2.5,0)
\put(165.8,-389.811){\fontsize{12}{1}\usefont{T1}{cmr}{m}{n}\selectfont\color{color_29791};;}
\put(41.8,-410.611){\fontsize{12}{1}\usefont{T1}{cmr}{m}{n}\selectfont\color{color_29791}nome?}
\end{picture}
\begin{tikzpicture}[overlay]
\path(0pt,0pt);
\filldraw[color_282751][even odd rule]
(73.7pt, -413.161pt) -- (77.64999pt, -413.161pt)
 -- (77.64999pt, -413.161pt)
 -- (77.64999pt, -399.411pt)
 -- (77.64999pt, -399.411pt)
 -- (73.7pt, -399.411pt) -- cycle
;
\end{tikzpicture}
\begin{picture}(-5,0)(2.5,0)
\put(73.8,-410.611){\fontsize{12}{1}\usefont{T1}{cmr}{m}{n}\selectfont\color{color_29791}) echo vale nome2, nomea, nomez }
\end{picture}
\begin{tikzpicture}[overlay]
\path(0pt,0pt);
\filldraw[color_282751][even odd rule]
(241pt, -413.161pt) -- (247.55pt, -413.161pt)
 -- (247.55pt, -413.161pt)
 -- (247.55pt, -399.411pt)
 -- (247.55pt, -399.411pt)
 -- (241pt, -399.411pt) -- cycle
;
\end{tikzpicture}
\begin{picture}(-5,0)(2.5,0)
\put(241.1,-410.611){\fontsize{12}{1}\usefont{T1}{cmr}{m}{n}\selectfont\color{color_29791};;}
\put(41.8,-431.411){\fontsize{12}{1}\usefont{T1}{cmr}{m}{n}\selectfont\color{color_29791}nome*}
\end{picture}
\begin{tikzpicture}[overlay]
\path(0pt,0pt);
\filldraw[color_282751][even odd rule]
(74.4pt, -433.961pt) -- (78.35pt, -433.961pt)
 -- (78.35pt, -433.961pt)
 -- (78.35pt, -420.211pt)
 -- (78.35pt, -420.211pt)
 -- (74.4pt, -420.211pt) -- cycle
;
\end{tikzpicture}
\begin{picture}(-5,0)(2.5,0)
\put(74.5,-431.411){\fontsize{12}{1}\usefont{T1}{cmr}{m}{n}\selectfont\color{color_29791}) echo vale nome11, nome, nomepippo }
\end{picture}
\begin{tikzpicture}[overlay]
\path(0pt,0pt);
\filldraw[color_282751][even odd rule]
(263.9pt, -433.961pt) -- (270.45pt, -433.961pt)
 -- (270.45pt, -433.961pt)
 -- (270.45pt, -420.211pt)
 -- (270.45pt, -420.211pt)
 -- (263.9pt, -420.211pt) -- cycle
;
\end{tikzpicture}
\begin{picture}(-5,0)(2.5,0)
\put(264,-431.411){\fontsize{12}{1}\usefont{T1}{cmr}{m}{n}\selectfont\color{color_29791};;}
\put(41.8,-452.211){\fontsize{12}{1}\usefont{T1}{cmr}{m}{n}\selectfont\color{color_29791}[1-9]nome}
\end{picture}
\begin{tikzpicture}[overlay]
\path(0pt,0pt);
\filldraw[color_282751][even odd rule]
(92.4pt, -454.761pt) -- (96.35pt, -454.761pt)
 -- (96.35pt, -454.761pt)
 -- (96.35pt, -441.011pt)
 -- (96.35pt, -441.011pt)
 -- (92.4pt, -441.011pt) -- cycle
;
\end{tikzpicture}
\begin{picture}(-5,0)(2.5,0)
\put(92.5,-452.211){\fontsize{12}{1}\usefont{T1}{cmr}{m}{n}\selectfont\color{color_29791}) echo vale 1nome, 2nome, …, 9nome }
\end{picture}
\begin{tikzpicture}[overlay]
\path(0pt,0pt);
\filldraw[color_282751][even odd rule]
(279pt, -454.761pt) -- (285.55pt, -454.761pt)
 -- (285.55pt, -454.761pt)
 -- (285.55pt, -441.011pt)
 -- (285.55pt, -441.011pt)
 -- (279pt, -441.011pt) -- cycle
;
\end{tikzpicture}
\begin{picture}(-5,0)(2.5,0)
\put(279.1,-452.211){\fontsize{12}{1}\usefont{T1}{cmr}{m}{n}\selectfont\color{color_29791};;}
\put(41.8,-473.011){\fontsize{12}{1}\usefont{T1}{cmr}{m}{n}\selectfont\color{color_29791}*}
\end{picture}
\begin{tikzpicture}[overlay]
\path(0pt,0pt);
\filldraw[color_282751][even odd rule]
(47.7pt, -475.561pt) -- (51.65pt, -475.561pt)
 -- (51.65pt, -475.561pt)
 -- (51.65pt, -461.811pt)
 -- (51.65pt, -461.811pt)
 -- (47.7pt, -461.811pt) -- cycle
;
\end{tikzpicture}
\begin{picture}(-5,0)(2.5,0)
\put(47.8,-473.011){\fontsize{12}{1}\usefont{T1}{cmr}{m}{n}\selectfont\color{color_29791}) echo non cade in nessuna delle precedenti }
\end{picture}
\begin{tikzpicture}[overlay]
\path(0pt,0pt);
\filldraw[color_282751][even odd rule]
(259pt, -475.561pt) -- (265.55pt, -475.561pt)
 -- (265.55pt, -475.561pt)
 -- (265.55pt, -461.811pt)
 -- (265.55pt, -461.811pt)
 -- (259pt, -461.811pt) -- cycle
;
\end{tikzpicture}
\begin{picture}(-5,0)(2.5,0)
\put(259.1,-473.011){\fontsize{12}{1}\usefont{T1}{cmr}{m}{n}\selectfont\color{color_29791};;}
\end{picture}
\begin{tikzpicture}[overlay]
\path(0pt,0pt);
\filldraw[color_282751][even odd rule]
(41.7pt, -496.361pt) -- (62.35pt, -496.361pt)
 -- (62.35pt, -496.361pt)
 -- (62.35pt, -482.611pt)
 -- (62.35pt, -482.611pt)
 -- (41.7pt, -482.611pt) -- cycle
;
\end{tikzpicture}
\begin{picture}(-5,0)(2.5,0)
\put(41.8,-493.811){\fontsize{12}{1}\usefont{T1}{cmr}{m}{n}\selectfont\color{color_29791}esac}
\put(41.8,-514.611){\fontsize{12}{1}\usefont{T1}{cmr}{b}{n}\selectfont\color{color_29791}for itera su una lista di elementi, sintassi for NAME  [in WORDS … ] ; do COMMANDS; done}
\put(41.8,-535.411){\fontsize{12}{1}\usefont{T1}{cmr}{m}{n}\selectfont\color{color_29791}Casi d’uso:}
\put(59.8,-556.211){\fontsize{12}{1}\usefont{T1}{cmr}{m}{n}\selectfont\color{color_29791}•WORDS = pattern di pathname expansion}
\put(77.8,-577.011){\fontsize{12}{1}\usefont{T1}{cmr}{m}{n}\selectfont\color{color_29791}◦itera direttamente sui nomi di file prodotti dall’espansione }
\put(95.8,-597.811){\fontsize{12}{1}\usefont{T1}{cmr}{b}{it}\selectfont\color{color_29791}for F in /tmp/*.bak ; do rm -f “\$F” ; done }
\put(95.8,-618.611){\fontsize{12}{1}\usefont{T1}{cmr}{m}{n}\selectfont\color{color_29791}valuta condizione for: prima viene espansa la sequenza, al posto di /tmp/*.bak appaiono }
\put(95.8,-632.411){\fontsize{12}{1}\usefont{T1}{cmr}{m}{n}\selectfont\color{color_29791}tutti i file corrispondenti all’espressione. Solo dopo la shell sa quante iterazioni dovrà }
\put(95.8,-646.211){\fontsize{12}{1}\usefont{T1}{cmr}{m}{n}\selectfont\color{color_29791}fare. A questo punto viene eseguito il codice tra do e done aggiornando il valore di F per}
\put(95.8,-660.011){\fontsize{12}{1}\usefont{T1}{cmr}{m}{n}\selectfont\color{color_29791}ogni iterazione.}
\put(95.8,-680.811){\fontsize{12}{1}\usefont{T1}{cmr}{m}{n}\selectfont\color{color_29791}La shell evita espansione caratteri speciali se sono nei nomi di file, ma per ogni }
\put(95.8,-694.611){\fontsize{12}{1}\usefont{T1}{cmr}{m}{n}\selectfont\color{color_29791}iterazione poi ci sarà una nuova espansione! Es. se assegno filename “Marco Prandini” }
\put(95.8,-708.411){\fontsize{12}{1}\usefont{T1}{cmr}{m}{n}\selectfont\color{color_29791}(con spazio) a F, viene preso come un solo elemento ma nell’iterazone, se non metto }
\put(95.8,-722.211){\fontsize{12}{1}\usefont{T1}{cmr}{m}{n}\selectfont\color{color_29791}quoting intorno a \$F, cambia il significato: viene espanso in rm -f Marco Prandini e il }
\put(95.8,-736.011){\fontsize{12}{1}\usefont{T1}{cmr}{m}{n}\selectfont\color{color_29791}comando vede due parametri file separati, provando a eliminarli ma non esistono. }
\put(95.8,-749.811){\fontsize{12}{1}\usefont{T1}{cmr}{m}{n}\selectfont\color{color_29791}Occhio quindi ad usare “\$@” che protegge ognuno dei parametri con lo stesso quoting}
\put(59.8,-770.611){\fontsize{12}{1}\usefont{T1}{cmr}{m}{n}\selectfont\color{color_29791}•WORDS = parametri della command line}
\end{picture}
\newpage
\begin{tikzpicture}[overlay]\path(0pt,0pt);\end{tikzpicture}
\begin{picture}(-5,0)(2.5,0)
\put(77.8,-85.01099){\fontsize{12}{1}\usefont{T1}{cmr}{m}{n}\selectfont\color{color_29791}◦for PAR in “\$@” ; do echo “\$PAR” ; done}
\put(59.8,-105.811){\fontsize{12}{1}\usefont{T1}{cmr}{m}{n}\selectfont\color{color_29791}•WORDS = command substitution}
\put(77.8,-126.611){\fontsize{12}{1}\usefont{T1}{cmr}{m}{n}\selectfont\color{color_29791}◦for USER in \$(cat /etc/passwd | cut -f1 -d:)}
\put(95.8,-147.411){\fontsize{12}{1}\usefont{T1}{cmr}{m}{n}\selectfont\color{color_29791}vedi pattern comune per processing righe}
\put(59.8,-168.211){\fontsize{12}{1}\usefont{T1}{cmr}{m}{n}\selectfont\color{color_29791}•WORDS = brace expansion}
\put(77.8,-189.011){\fontsize{12}{1}\usefont{T1}{cmr}{m}{n}\selectfont\color{color_29791}◦for ITEM in item\_\{a..z\}}
\put(59.8,-209.811){\fontsize{12}{1}\usefont{T1}{cmr}{m}{n}\selectfont\color{color_29791}•Si possono impiegare sequenze numeriche usando NON MOSTRATO}
\put(77.8,-230.611){\fontsize{12}{1}\usefont{T1}{cmr}{m}{n}\selectfont\color{color_29791}◦for BACKWARDSTENTHS in \$(seq 1 -0.1 0)      dove (start incremento end)}
\put(59.8,-251.411){\fontsize{12}{1}\usefont{T1}{cmr}{m}{n}\selectfont\color{color_29791}•Nelle versioni recenti di bash sintassi C-like NON MOSTRATO}
\put(77.8,-272.211){\fontsize{12}{1}\usefont{T1}{cmr}{m}{n}\selectfont\color{color_29791}◦for (( i=0, j=0 ; i+j < 10 ; i++, j+=2 )) }
\put(77.8,-293.011){\fontsize{12}{1}\usefont{T1}{cmr}{m}{n}\selectfont\color{color_29791}◦Espressioni di inizializzazione, test di terminazione (espressione logica uguale a true o }
\put(95.8,-306.811){\fontsize{12}{1}\usefont{T1}{cmr}{m}{n}\selectfont\color{color_29791}aritmetica che dà 0. Se c’è più di un test vengono eseguiti tutti ma solo l’ultimo }
\put(95.8,-320.611){\fontsize{12}{1}\usefont{T1}{cmr}{m}{n}\selectfont\color{color_29791}determina se il ciclo prosegue), espressioni eseguite ad ogni iterazione}
\put(41.8,-341.411){\fontsize{12}{1}\usefont{T1}{cmr}{b}{it}\selectfont\color{color_29791}while permette cicli indefiniti basandosi su un comando da eseguire per verificare se l’exit code è }
\put(41.8,-355.211){\fontsize{12}{1}\usefont{T1}{cmr}{m}{n}\selectfont\color{color_29791}TRUE (solo 0, tutto il resto FALSE), (until itera quando si ha FALSE) sintassi}
\put(77.3,-376.011){\fontsize{12}{1}\usefont{T1}{cmr}{b}{it}\selectfont\color{color_29791}while COMANDO (oppure until COMANDO) ; do LISTA ;COMANDI; ITERATI; done}
\put(41.8,-396.811){\fontsize{12}{1}\usefont{T1}{cmr}{m}{n}\selectfont\color{color_29791}Naturalmente COMANDO può essere composto (sequenze, subshell, combinazioni logiche…)}
\put(41.8,-417.611){\fontsize{12}{1}\usefont{T1}{cmr}{m}{n}\selectfont\color{color_29791}Pattern comune usato per fare processing di dati:}
\put(59.8,-438.411){\fontsize{12}{1}\usefont{T1}{cmr}{m}{n}\selectfont\color{color_29791}•cat /etc/passwd | while read riga ; do echo \$(( counter++ )) \$riga ; done}
\put(77.8,-459.211){\fontsize{12}{1}\usefont{T1}{cmr}{m}{n}\selectfont\color{color_29791}◦cat legge tutto e mette su stdout, pipeline porta stdout su stdin del processo dopo, figlio }
\put(95.8,-473.011){\fontsize{12}{1}\usefont{T1}{cmr}{m}{n}\selectfont\color{color_29791}(shell figlia che fa il while) read mangia una riga di input e restituisce true; while ha }
\put(95.8,-486.811){\fontsize{12}{1}\usefont{T1}{cmr}{m}{n}\selectfont\color{color_29791}ricevuto true e fa iterazione: echo di valore di counter incrementato dopo averlo espanso }
\put(95.8,-500.611){\fontsize{12}{1}\usefont{T1}{cmr}{m}{n}\selectfont\color{color_29791}(espansione non da errore perché parte da 0 essendo counter non definito) seguito dal }
\put(95.8,-514.411){\fontsize{12}{1}\usefont{T1}{cmr}{m}{n}\selectfont\color{color_29791}valore di riga. }
\put(77.8,-535.211){\fontsize{12}{1}\usefont{T1}{cmr}{m}{n}\selectfont\color{color_217499}◦Tenere a mente importanza di aprire subshell esplicitamente per il while in pipe, così da }
\put(95.8,-549.011){\fontsize{12}{1}\usefont{T1}{cmr}{m}{n}\selectfont\color{color_217499}garantire corretto spazio di memoria per eventuali handler e variabili}
\put(77.8,-569.811){\fontsize{12}{1}\usefont{T1}{cmr}{m}{n}\selectfont\color{color_217499}◦Se necessario input interattivo dentro al ciclo, il secondo read va alimentato da terminale}
\put(95.8,-583.611){\fontsize{12}{1}\usefont{T1}{cmr}{m}{n}\selectfont\color{color_217499}es. T=\$(ps h \$\$ | awk '\{ print \$2 \}') ; cat /etc/passwd | while read U ; do \{ read VAR }
\put(95.8,-597.411){\fontsize{12}{1}\usefont{T1}{cmr}{m}{it}\selectfont\color{color_217499}< /dev/\$T ; echo \$VAR \} ; done oppure esempio messo nella desc. del builtin read}
\put(59.8,-618.211){\fontsize{12}{1}\usefont{T1}{cmr}{m}{n}\selectfont\color{color_29791}•Versione con for: for riga in \$(cat /etc/passwd) ; do echo \$((counter++)) \$riga ; done}
\put(77.8,-639.011){\fontsize{12}{1}\usefont{T1}{cmr}{m}{n}\selectfont\color{color_29791}◦Il for è vantaggioso perché non c'è pipeline, quindi tutto quello che avviene nelle }
\put(95.8,-652.811){\fontsize{12}{1}\usefont{T1}{cmr}{m}{n}\selectfont\color{color_29791}iterazioni è nella stessa shell che lo lancia: si vede che se si rilancia il comando due }
\put(95.8,-666.611){\fontsize{12}{1}\usefont{T1}{cmr}{m}{n}\selectfont\color{color_29791}volte, parte con il counter dallo stesso valore, la variabile è dello stesso ambiente.}
\put(77.8,-687.411){\fontsize{12}{1}\usefont{T1}{cmr}{m}{n}\selectfont\color{color_29791}◦Se l’elemento informativo è la riga evito di usare un for, con while possiamo sfruttare }
\put(95.8,-701.211){\fontsize{12}{1}\usefont{T1}{cmr}{m}{n}\selectfont\color{color_29791}read che ha il concetto di riga. Fintanto che nelle righe ci sono solo spazi non c’è }
\put(95.8,-715.011){\fontsize{12}{1}\usefont{T1}{cmr}{m}{n}\selectfont\color{color_29791}problema ad usare for, ma se ci sono spazi e accapo invece è un problema perché }
\put(95.8,-728.811){\fontsize{12}{1}\usefont{T1}{cmr}{m}{n}\selectfont\color{color_29791}perdiamo la differenza tra i due separatori. }
\put(41.8,-749.611){\fontsize{12}{1}\usefont{T1}{cmr}{b}{it}\selectfont\color{color_29791}break [N] esce da un ciclo for, while o until (se specificato, esce da N cicli annidati)}
\end{picture}
\newpage
\begin{tikzpicture}[overlay]\path(0pt,0pt);\end{tikzpicture}
\begin{picture}(-5,0)(2.5,0)
\put(41.8,-85.01099){\fontsize{12}{1}\usefont{T1}{cmr}{b}{it}\selectfont\color{color_29791}continue [N] salta alla successiva iterazione di un ciclo (se specificato, riparte risalendo di N cicli }
\put(41.8,-98.81097){\fontsize{12}{1}\usefont{T1}{cmr}{m}{n}\selectfont\color{color_29791}annidati)}
\put(41.8,-128.611){\fontsize{14.1}{1}\usefont{T1}{cmr}{b}{n}\selectfont\color{color_29791}Accorgimenti utili}
\put(41.8,-148.811){\fontsize{12}{1}\usefont{T1}{cmr}{b}{it}\selectfont\color{color_29791}source : utile per condividere parametri tra script correlati tra loro e per creare librerie di funzioni }
\put(41.8,-162.611){\fontsize{12}{1}\usefont{T1}{cmr}{m}{n}\selectfont\color{color_29791}importabili. Può essere usato per eseguire uno script nel contesto di un altro (inclusa riga di }
\put(41.8,-176.411){\fontsize{12}{1}\usefont{T1}{cmr}{m}{n}\selectfont\color{color_29791}comando interattiva). Es. supponiamo che lo script common.sh contenga assegnamenti di valori a }
\put(41.8,-190.211){\fontsize{12}{1}\usefont{T1}{cmr}{m}{n}\selectfont\color{color_29791}variabili, definizioni di funzioni e alias. Dopo l’esecuzione di source common.sh le variabili, }
\put(41.8,-204.011){\fontsize{12}{1}\usefont{T1}{cmr}{m}{n}\selectfont\color{color_29791}funzioni e alias saranno definite anche nello script chiamante.}
\put(41.8,-224.811){\fontsize{12}{1}\usefont{T1}{cmr}{b}{it}\selectfont\color{color_29791}su è utile da root (da altri utenti chiederebbe password, non pratico da scriptare) per eseguire }
\put(41.8,-238.611){\fontsize{12}{1}\usefont{T1}{cmr}{m}{n}\selectfont\color{color_29791}comandi con altre identità, ma non è exec: non si mettono comandi a seguire, apre una shell e finché}
\put(41.8,-252.411){\fontsize{12}{1}\usefont{T1}{cmr}{m}{n}\selectfont\color{color_29791}non si esce da quella non prosegue nel contesto chiamante. Sintassi su -c “COMANDO” - }
\put(41.8,-266.211){\fontsize{12}{1}\usefont{T1}{cmr}{b}{it}\selectfont\color{color_29791}UTENTE. }
\put(41.8,-287.011){\fontsize{12}{1}\usefont{T1}{cmr}{b}{it}\selectfont\color{color_29791}sudo COMANDO è utile ad usare comandi come root, ma richiede la configurazione di sudoers }
\put(41.8,-300.811){\fontsize{12}{1}\usefont{T1}{cmr}{m}{n}\selectfont\color{color_29791}(possibilmente con NOPASSWD) }
\put(59.8,-321.611){\fontsize{12}{1}\usefont{T1}{cmr}{m}{n}\selectfont\color{color_29791}•Sintassi per /etc/sudoers:https://toroid.org/sudoers-syntax}
\end{picture}
\begin{tikzpicture}[overlay]
\path(0pt,0pt);
\draw[color_29919,line width=0.7pt]
(255.1pt, -322.711pt) -- (411.2pt, -322.711pt)
;
\end{tikzpicture}
\begin{picture}(-5,0)(2.5,0)
\put(77.8,-342.411){\fontsize{12}{1}\usefont{T1}{cmr}{m}{n}\selectfont\color{color_29791}◦formato delle direttive: User Host = (Runas) Tag: Command}
\put(95.8,-363.211){\fontsize{12}{1}\usefont{T1}{cmr}{m}{n}\selectfont\color{color_29791}1.“User puo eseguire Command coi privilegi di Runas su Host"}
\put(113.8,-384.011){\fontsize{12}{1}\usefont{T1}{cmr}{m}{n}\selectfont\color{color_29791}1.\%sudo ALL=(ALL:ALL) ALLriga base tipica, per gruppo sudo}
\put(113.8,-404.811){\fontsize{12}{1}\usefont{T1}{cmr}{m}{n}\selectfont\color{color_29791}2.usermod -aG sudo las Aggiunge "las" al gruppo sudo, efficace solo}
\put(131.8,-418.611){\fontsize{12}{1}\usefont{T1}{cmr}{m}{n}\selectfont\color{color_29791}dal successivo login}
\put(95.8,-439.411){\fontsize{12}{1}\usefont{T1}{cmr}{m}{n}\selectfont\color{color_29791}2.Runas si può specificare come utente[:gruppo] ; se manca, è implicito root (e nessun }
\put(113.8,-453.211){\fontsize{12}{1}\usefont{T1}{cmr}{m}{n}\selectfont\color{color_29791}altro)}
\put(41.8,-474.011){\fontsize{12}{1}\usefont{T1}{cmr}{b}{it}\selectfont\color{color_29791}date ha molte opzioni:}
\put(59.8,-494.811){\fontsize{12}{1}\usefont{T1}{cmr}{m}{n}\selectfont\color{color_29791}•con -s può impostare orologio di sistema}
\put(59.8,-515.611){\fontsize{12}{1}\usefont{T1}{cmr}{m}{n}\selectfont\color{color_29791}•può restituire l’orologio in formati diversi}
\put(77.8,-536.411){\fontsize{12}{1}\usefont{T1}{cmr}{m}{n}\selectfont\color{color_29791}◦date +FORMAT permette di selezionare cosa e come visualizzare}
\put(77.8,-557.211){\fontsize{12}{1}\usefont{T1}{cmr}{m}{n}\selectfont\color{color_29791}◦FORMAT è una stringa in cui vengono interpretate sequenze speciali contrassegnate dal }
\put(95.8,-571.011){\fontsize{12}{1}\usefont{T1}{cmr}{m}{n}\selectfont\color{color_29791}carattere \% (vedi man date) es. date +”\%Y\%m\%d \%H:\%M:\%S” AnnoMeseGiorni }
\put(95.8,-584.811){\fontsize{12}{1}\usefont{T1}{cmr}{m}{n}\selectfont\color{color_29791}Ora:Minuto:Secondo; date +“\%A \%e \%B” Giornosettimana Giorno Mese (locale)}
\put(95.8,-605.611){\fontsize{12}{1}\usefont{T1}{cmr}{b}{it}\selectfont\color{color_29791}\%s numero secondi da epoca Unix (1/1/70), (date +s\%) ; con \%s\%n appende i }
\put(95.8,-619.411){\fontsize{12}{1}\usefont{T1}{cmr}{m}{n}\selectfont\color{color_29791}nanosecondi}
\put(77.8,-640.211){\fontsize{12}{1}\usefont{T1}{cmr}{m}{n}\selectfont\color{color_29791}◦con -d si può fornire un timestamp da usare al posto del tempo corrente (es. convertire }
\put(95.8,-654.011){\fontsize{12}{1}\usefont{T1}{cmr}{m}{n}\selectfont\color{color_29791}timestamp tra diversi formati: N=\$(date -d ‘2020-05-15 10:01’ +\%s) dà ora di evento }
\put(95.8,-667.811){\fontsize{12}{1}\usefont{T1}{cmr}{m}{n}\selectfont\color{color_29791}interessante convertito in epoch, ((N+=1800)) modifica tempo, date -d “@\$N” stampa in}
\put(95.8,-681.611){\fontsize{12}{1}\usefont{T1}{cmr}{m}{n}\selectfont\color{color_29791}modo leggibile: @ operatore per cambiare da epoch a timestamp}
\put(41.8,-702.411){\fontsize{12}{1}\usefont{T1}{cmr}{b}{it}\selectfont\color{color_29791}mktemp è utile per creare file temporanei, restituisce il nome del file creato (default formato }
\put(41.8,-716.211){\fontsize{12}{1}\usefont{T1}{cmr}{m}{n}\selectfont\color{color_29791}/tmp/tmp.XXXXXXXXXX. Opzioni}
\put(59.8,-737.011){\fontsize{12}{1}\usefont{T1}{cmr}{m}{n}\selectfont\color{color_29791}•-d crea una directory}
\put(59.8,-757.811){\fontsize{12}{1}\usefont{T1}{cmr}{m}{n}\selectfont\color{color_29791}•-p DIR crea all’interno di DIR invece che in /tmp}
\end{picture}
\newpage
\begin{tikzpicture}[overlay]\path(0pt,0pt);\end{tikzpicture}
\begin{picture}(-5,0)(2.5,0)
\put(41.8,-85.01099){\fontsize{12}{1}\usefont{T1}{cmr}{b}{it}\selectfont\color{color_29791}basename rimuove path ed eventualmente suffisso da un nome di file. }
\put(59.8,-105.811){\fontsize{12}{1}\usefont{T1}{cmr}{m}{n}\selectfont\color{color_29791}•Es. basename s .h include/stdio.h ritorna stdio}
\put(41.8,-126.611){\fontsize{12}{1}\usefont{T1}{cmr}{b}{it}\selectfont\color{color_29791}dirname rimuove l’ultimo componente del percorso da un nome di file (se non contiene /, }
\put(41.8,-140.411){\fontsize{12}{1}\usefont{T1}{cmr}{m}{n}\selectfont\color{color_29791}restituisce “.” directory corrente)}
\put(59.8,-161.211){\fontsize{12}{1}\usefont{T1}{cmr}{m}{n}\selectfont\color{color_29791}•Es. dirname /usr/binritorna /usr ; dirname data.txt ritorna .}
\put(41.8,-182.011){\fontsize{12}{1}\usefont{T1}{cmr}{b}{it}\selectfont\color{color_35081}eval permette di processare un file come se fosse uno script, sottoponendolo ai 12 passi di }
\put(41.8,-195.811){\fontsize{12}{1}\usefont{T1}{cmr}{m}{n}\selectfont\color{color_35081}valutazione. Questo permette ad uno script di generare altri script ed eseguirli correttamente. Utile }
\put(41.8,-209.611){\fontsize{12}{1}\usefont{T1}{cmr}{m}{n}\selectfont\color{color_217499}per rivalutare separatori: es. listpage="ls | more"; \$listpage param. Exp avviene dopo la }
\put(41.8,-223.411){\fontsize{12}{1}\usefont{T1}{cmr}{m}{n}\selectfont\color{color_217499}tokenization, quindi i caratteri "|" e "more" vengono intepretati come argomenti di ls. Con }
\put(77.3,-237.211){\fontsize{12}{1}\usefont{T1}{cmr}{b}{it}\selectfont\color{color_217499}eval \$listpage si ha invece una valutazione corretta.}
\put(41.8,-277.211){\fontsize{19.6}{1}\usefont{T1}{cmr}{b}{n}\selectfont\color{color_29791}Funzionamento in rete}
\put(41.8,-313.811){\fontsize{17.5}{1}\usefont{T1}{cmr}{b}{n}\selectfont\color{color_29791}Configurazione di rete}
\put(41.8,-343.711){\fontsize{14.1}{1}\usefont{T1}{cmr}{b}{n}\selectfont\color{color_29791}Richiami di reti, reti locali}
\put(41.8,-363.911){\fontsize{12}{1}\usefont{T1}{cmr}{m}{n}\selectfont\color{color_29791}Internet è una grande "rete di reti", la cui componente base (isola) è la singola rete IP; che contiene }
\put(41.8,-377.711){\fontsize{12}{1}\usefont{T1}{cmr}{m}{n}\selectfont\color{color_29791}calcolatori che fanno da nodi terminali detti host. Gli indirizzi IP si dividono in globali (validi per }
\put(41.8,-391.511){\fontsize{12}{1}\usefont{T1}{cmr}{m}{n}\selectfont\color{color_29791}tutta la rete, devono essere univoci e quindi vanno assegnati da una procedura di gestione globale }
\put(41.8,-405.311){\fontsize{12}{1}\usefont{T1}{cmr}{m}{n}\selectfont\color{color_29791}che garantisca l'univocità) e locali, validi limitatamente ad una certa sottoparte della rete, che }
\put(41.8,-419.111){\fontsize{12}{1}\usefont{T1}{cmr}{m}{n}\selectfont\color{color_29791}possono essere non globalmente univoci. (vedremo un ibrido tra queste due visioni quando vedremo che tra i vai }
\put(41.8,-430.111){\fontsize{9}{1}\usefont{T1}{cmr}{m}{n}\selectfont\color{color_29791}metodi di assegnazione degli indirizzi IP ci sono metodi che partono da univocità di indirizzo locale (livello 2, MAC di scheda di }
\put(41.8,-440.411){\fontsize{9}{1}\usefont{T1}{cmr}{m}{n}\selectfont\color{color_29791}rete) per derivare indirizzo di liv 3,IP che abbia univocità locale (senza considerare univocità globale) così si può ignorare network }
\put(41.8,-450.811){\fontsize{9}{1}\usefont{T1}{cmr}{m}{n}\selectfont\color{color_29791}fisico, consentendo a questi di scoprire con automatismi come collegarsi con esterno (si sfrutta univocità indirizzi locali))}
\put(41.8,-470.911){\fontsize{12}{1}\usefont{T1}{cmr}{m}{n}\selectfont\color{color_29791}Viene detta rete fisica la rete (tipicamente LAN) a cui un host è effettivamente connesso; mentre è }
\put(41.8,-484.711){\fontsize{12}{1}\usefont{T1}{cmr}{m}{n}\selectfont\color{color_29791}detta rete logica la network IP o (subnet) a cui un Host appartiene logicamente: sulla stessa LAN }
\put(41.8,-498.511){\fontsize{12}{1}\usefont{T1}{cmr}{m}{n}\selectfont\color{color_29791}(fisica) possono esserci più subnet. A livello logico queste penseranno di non poter comunicare tra }
\put(41.8,-512.311){\fontsize{12}{1}\usefont{T1}{cmr}{m}{n}\selectfont\color{color_29791}loro pur essendo sulla stessa struttura fisica. Tutti gli host appartenenti alla medesima network IP }
\put(41.8,-526.111){\fontsize{12}{1}\usefont{T1}{cmr}{m}{n}\selectfont\color{color_29791}sono in grado di parlare tra loro grazie alla tecnologia con cui essa viene implementata. Per }
\put(41.8,-539.911){\fontsize{12}{1}\usefont{T1}{cmr}{m}{n}\selectfont\color{color_29791}raggiungere hosts che si trovano sulla stessa LAN non è necessario sapere indirizzo di ognuno di }
\put(41.8,-553.711){\fontsize{12}{1}\usefont{T1}{cmr}{m}{n}\selectfont\color{color_29791}loro, basta sapere come raggiungere la subnet.}
\put(41.8,-574.511){\fontsize{12}{1}\usefont{T1}{cmr}{m}{n}\selectfont\color{color_29791}Per il sistemista diventa necessario saper configurare un computer diversamente dal semplice host }
\put(41.8,-588.311){\fontsize{12}{1}\usefont{T1}{cmr}{m}{n}\selectfont\color{color_29791}nodo terminale in una subnet: esempio, un router è semplicemente un computer con 2 interfacce }
\put(41.8,-602.111){\fontsize{12}{1}\usefont{T1}{cmr}{m}{n}\selectfont\color{color_29791}(in/out) specializzato nello smistamento pacchetti.}
\put(41.8,-622.911){\fontsize{12}{1}\usefont{T1}{cmr}{m}{n}\selectfont\color{color_29791}Gli indirizzi IPv4 sono formati da 32bit divisi in 4 byte, ogni indirizzo fa parte di una subnet che }
\put(41.8,-636.711){\fontsize{12}{1}\usefont{T1}{cmr}{m}{n}\selectfont\color{color_29791}inizia con l'indirizzo di network; l'estensione della subnet era implicita in origine (mediante classi di}
\put(41.8,-650.511){\fontsize{12}{1}\usefont{T1}{cmr}{m}{n}\selectfont\color{color_29791}indirizzi, in cui i byte erano divisi tra net id e host id (qui indicata come *): Classe A da 0.*** a }
\put(41.8,-664.311){\fontsize{12}{1}\usefont{T1}{cmr}{m}{n}\selectfont\color{color_29791}127.*** -128 reti, fino a 16M hosts-; Classe B da 128.*** a 192.255.** -16K reti, fino a 65K }
\put(41.8,-678.111){\fontsize{12}{1}\usefont{T1}{cmr}{m}{n}\selectfont\color{color_29791}hosts-, Classe C da 192.0.0.* a 223.255.255.* -2M reti, fino a 254 hosts-; indirizzi da 224.*** a }
\put(41.8,-691.911){\fontsize{12}{1}\usefont{T1}{cmr}{m}{n}\selectfont\color{color_29791}239.*** sono per il multicast, da 240.*** a 255.*** per usi futuri), oggi l'estensione è specificata da}
\put(41.8,-705.711){\fontsize{12}{1}\usefont{T1}{cmr}{m}{n}\selectfont\color{color_29791}una netmask. }
\put(41.8,-726.511){\fontsize{12}{1}\usefont{T1}{cmr}{m}{n}\selectfont\color{color_29791}Avere poche classi con dimensioni fisse porta a spreco di indirizzi (es. chiedere una subnet di classe }
\put(41.8,-740.311){\fontsize{12}{1}\usefont{T1}{cmr}{m}{n}\selectfont\color{color_29791}A ricevendo 16 milioni di indirizzi anche se ne servono solo 100.000…), quindi con CIDR }
\put(41.8,-754.111){\fontsize{12}{1}\usefont{T1}{cmr}{m}{n}\selectfont\color{color_29791}(Classless Internet-Domain Routing) la divisione tra net-id e host-id è in un punto arbitrario (es. con}
\put(41.8,-767.911){\fontsize{12}{1}\usefont{T1}{cmr}{m}{n}\selectfont\color{color_29791}2\^6=64 indirizzi, 6 bit di host id: 144.156.166.151 10010000.10011100.10100110.10  010111 unico}
\end{picture}
\newpage
\begin{tikzpicture}[overlay]\path(0pt,0pt);\end{tikzpicture}
\begin{picture}(-5,0)(2.5,0)
\put(41.8,-85.01099){\fontsize{12}{1}\usefont{T1}{cmr}{m}{n}\selectfont\color{color_29791}vincolo è che la dimensione della rete sia potenza di due), mentre nel caso delle classi bastava }
\put(41.8,-98.81097){\fontsize{12}{1}\usefont{T1}{cmr}{m}{n}\selectfont\color{color_29791}guardare il primo numero per identificare la rete locale, ora non è più sufficiente: è necessario }
\put(41.8,-112.611){\fontsize{12}{1}\usefont{T1}{cmr}{m}{n}\selectfont\color{color_29791}conoscere la netmask. Questa è un valore a 32 bit composto da tanti 1 quanti sono i bit della subnet }
\put(41.8,-126.411){\fontsize{12}{1}\usefont{T1}{cmr}{m}{n}\selectfont\color{color_29791}e tanti 0 quanti sono i bit che specificano l'host (nell'esempio precedente, 26 1 e 6 0 }
\put(41.8,-140.211){\fontsize{12}{1}\usefont{T1}{cmr}{m}{n}\selectfont\color{color_29791}11111111.11111111.11111111.11000000 = 255.255.255.192). Si può fare AND tra netmask ed }
\put(41.8,-154.011){\fontsize{12}{1}\usefont{T1}{cmr}{m}{n}\selectfont\color{color_29791}indirizzo, per confrontare indirizzi di destinazione e verificare se è nella stessa subnet di indirizzo di}
\put(41.8,-167.811){\fontsize{12}{1}\usefont{T1}{cmr}{m}{n}\selectfont\color{color_29791}partenza. In ogni subnet, due indirizzi hanno significati speciali e non possono essere assegnati ad }
\put(41.8,-181.611){\fontsize{12}{1}\usefont{T1}{cmr}{m}{n}\selectfont\color{color_29791}un host: quello con host-id con tutti 0 identifica la subnet (network address), con tutti 1 è indirizzo }
\put(41.8,-195.411){\fontsize{12}{1}\usefont{T1}{cmr}{m}{n}\selectfont\color{color_29791}broadcast della subnet (se supportato, fa giungere pacchetto a tutti gli host della rete).}
\put(41.8,-216.211){\fontsize{12}{1}\usefont{T1}{cmr}{m}{n}\selectfont\color{color_29791}In ogni LAN ogni dispositivo ha un indirizzo MAC (mediante cui si ha inoltro fisico del traffico tra }
\put(41.8,-230.011){\fontsize{12}{1}\usefont{T1}{cmr}{m}{n}\selectfont\color{color_29791}schede di rete), ma ha anche un indirizzo IP della rete. La traduzione di uno nell'altro avviene }
\put(41.8,-243.811){\fontsize{12}{1}\usefont{T1}{cmr}{m}{n}\selectfont\color{color_29791}mediante ARP (Address Resolution Protocol). ARPsfrutta broadcast della rete fisica per chiedere }
\put(41.8,-257.611){\fontsize{12}{1}\usefont{T1}{cmr}{m}{n}\selectfont\color{color_29791}chi possiede un  certo indirizzo IP: se un host sa che appartiene ad una subnet, e vuole parlare con }
\put(41.8,-271.411){\fontsize{12}{1}\usefont{T1}{cmr}{m}{n}\selectfont\color{color_29791}un host nella stessa subnet, SA ANCHE CHE è sulla stessa LAN, quindi gli basta chiedere in }
\put(41.8,-285.211){\fontsize{12}{1}\usefont{T1}{cmr}{m}{n}\selectfont\color{color_29791}broadcast (tutti faranno caching opportunistico della richiesta, che contiene IP-MAC a cui }
\put(41.8,-299.011){\fontsize{12}{1}\usefont{T1}{cmr}{m}{n}\selectfont\color{color_29791}rispondere) alla LAN chi possiede un certo IP per ottenere da lui indirizzo fisico a cui inviare }
\put(41.8,-312.811){\fontsize{12}{1}\usefont{T1}{cmr}{m}{n}\selectfont\color{color_29791}pacchetti (risposta in unicast con coppia IP-MAC). }
\put(41.8,-333.611){\fontsize{12}{1}\usefont{T1}{cmr}{m}{n}\selectfont\color{color_29791}Nel livello fisico concreto (sul ferro), l'interfaccia di rete è un vero dispositivo, che ha un MAC }
\put(41.8,-347.411){\fontsize{12}{1}\usefont{T1}{cmr}{m}{n}\selectfont\color{color_29791}tipicamente cablato (MAC modificabili sono eccezione più che regola). Far parte di una LAN }
\put(41.8,-361.211){\fontsize{12}{1}\usefont{T1}{cmr}{m}{n}\selectfont\color{color_29791}cablata è determinato da connessione fisica di un cavo, oppure da gesto attivo di connettersi a }
\put(41.8,-375.011){\fontsize{12}{1}\usefont{T1}{cmr}{m}{n}\selectfont\color{color_29791}wireless LAN, l'indirizzo IP va configurato. In una VM, l'interfaccia di rete è un artefatto gestito }
\put(41.8,-388.811){\fontsize{12}{1}\usefont{T1}{cmr}{m}{n}\selectfont\color{color_29791}dall'hypervisor, mentre il SO guest la percepisce come un dispositivo e fornisce strumenti per }
\put(41.8,-402.611){\fontsize{12}{1}\usefont{T1}{cmr}{m}{n}\selectfont\color{color_29791}configurarlo. L'hypervisor può impostare MAC (chiaramente è configurabile da chi gestisce }
\put(41.8,-416.411){\fontsize{12}{1}\usefont{T1}{cmr}{m}{n}\selectfont\color{color_29791}hypervisor, non essendo cablato in una scheda fsica), definire a quale subnet è connessa la VM }
\put(41.8,-430.211){\fontsize{12}{1}\usefont{T1}{cmr}{m}{n}\selectfont\color{color_29791}(instradamento è definito dall'hypervisor, il guest consegna a lui i pacchetti), può a volte gestire }
\put(41.8,-444.011){\fontsize{12}{1}\usefont{T1}{cmr}{m}{n}\selectfont\color{color_29791}indirizzamento logico in modi particolari: può dire "Se esiste protocollo standard per configurare }
\put(41.8,-457.811){\fontsize{12}{1}\usefont{T1}{cmr}{m}{n}\selectfont\color{color_29791}rete (esiste, DHCP), allora non chiedo al sysadm di farlo ma sfruttando il protocollo lascio che }
\put(41.8,-471.611){\fontsize{12}{1}\usefont{T1}{cmr}{m}{n}\selectfont\color{color_29791}venga configurato in automazione".}
\put(41.8,-492.411){\fontsize{12}{1}\usefont{T1}{cmr}{m}{n}\selectfont\color{color_29791}In particolare in VirtualBox è permesso attivare molteplici interfacce per ogni VM, ognuna delle }
\end{picture}
\begin{tikzpicture}[overlay]
\path(0pt,0pt);
\draw[color_29791,line width=0.7pt]
(41.8pt, -489.011pt) -- (507.4pt, -489.011pt)
;
\end{tikzpicture}
\begin{picture}(-5,0)(2.5,0)
\put(41.8,-506.211){\fontsize{12}{1}\usefont{T1}{cmr}{m}{n}\selectfont\color{color_29791}quali può connettersi in diversi modi, i principali:}
\end{picture}
\begin{tikzpicture}[overlay]
\path(0pt,0pt);
\draw[color_29791,line width=0.7pt]
(41.8pt, -502.811pt) -- (279.7pt, -502.811pt)
;
\end{tikzpicture}
\begin{picture}(-5,0)(2.5,0)
\put(279.7,-506.211){\fontsize{12}{1}\usefont{T1}{cmr}{m}{n}\selectfont\color{color_29791}   }
\end{picture}
\begin{tikzpicture}[overlay]
\path(0pt,0pt);
\draw[color_29791,line width=0.7pt]
(279.7pt, -502.811pt) -- (289.9pt, -502.811pt)
;
\end{tikzpicture}
\begin{picture}(-5,0)(2.5,0)
\put(290,-506.211){\fontsize{12}{1}\usefont{T1}{cmr}{m}{n}\selectfont\color{color_29791}(comando VBoxManage utile per gestire varie }
\end{picture}
\begin{tikzpicture}[overlay]
\path(0pt,0pt);
\draw[color_29791,line width=0.7pt]
(290pt, -502.811pt) -- (515.7pt, -502.811pt)
;
\end{tikzpicture}
\begin{picture}(-5,0)(2.5,0)
\put(41.8,-520.011){\fontsize{12}{1}\usefont{T1}{cmr}{m}{n}\selectfont\color{color_29791}VM da script)}
\end{picture}
\begin{tikzpicture}[overlay]
\path(0pt,0pt);
\draw[color_29791,line width=0.7pt]
(41.8pt, -516.611pt) -- (109.1pt, -516.611pt)
;
\end{tikzpicture}
\begin{picture}(-5,0)(2.5,0)
\put(59.8,-540.811){\fontsize{12}{1}\usefont{T1}{cmr}{m}{n}\selectfont\color{color_29791}•}
\end{picture}
\begin{tikzpicture}[overlay]
\path(0pt,0pt);
\draw[color_29791,line width=0.7pt]
(59.8pt, -537.411pt) -- (64.10001pt, -537.411pt)
;
\end{tikzpicture}
\begin{picture}(-5,0)(2.5,0)
\put(64,-540.811){\fontsize{12}{1}\usefont{T1}{cmr}{m}{n}\selectfont\color{color_29791}  }
\end{picture}
\begin{tikzpicture}[overlay]
\path(0pt,0pt);
\draw[color_29791,line width=0.7pt]
(64pt, -537.411pt) -- (77.7pt, -537.411pt)
;
\end{tikzpicture}
\begin{picture}(-5,0)(2.5,0)
\put(77.8,-540.811){\fontsize{12}{1}\usefont{T1}{cmr}{b}{n}\selectfont\color{color_29791}NAT}
\end{picture}
\begin{tikzpicture}[overlay]
\path(0pt,0pt);
\draw[color_29791,line width=0.7pt]
(77.8pt, -537.411pt) -- (102.2pt, -537.411pt)
;
\end{tikzpicture}
\begin{picture}(-5,0)(2.5,0)
\put(77.8,-561.611){\fontsize{12}{1}\usefont{T1}{cmr}{m}{n}\selectfont\color{color_29791}◦}
\end{picture}
\begin{tikzpicture}[overlay]
\path(0pt,0pt);
\draw[color_29791,line width=0.7pt]
(77.8pt, -558.211pt) -- (87.3pt, -558.211pt)
;
\end{tikzpicture}
\begin{picture}(-5,0)(2.5,0)
\put(87.3,-561.611){\fontsize{12}{1}\usefont{T1}{cmr}{m}{n}\selectfont\color{color_29791} }
\end{picture}
\begin{tikzpicture}[overlay]
\path(0pt,0pt);
\draw[color_29791,line width=0.7pt]
(87.3pt, -558.211pt) -- (93.3pt, -558.211pt)
;
\end{tikzpicture}
\begin{picture}(-5,0)(2.5,0)
\put(95.8,-561.611){\fontsize{12}{1}\usefont{T1}{cmr}{m}{n}\selectfont\color{color_29791}A default c'è una sola interfaccia in NAT: permette al guest di chiedere direttamente }
\end{picture}
\begin{tikzpicture}[overlay]
\path(0pt,0pt);
\draw[color_29791,line width=0.7pt]
(95.8pt, -558.211pt) -- (500.5pt, -558.211pt)
;
\end{tikzpicture}
\begin{picture}(-5,0)(2.5,0)
\put(95.8,-575.411){\fontsize{12}{1}\usefont{T1}{cmr}{m}{n}\selectfont\color{color_29791}all'hypervisor un indirizzo per uscire sulla rete. Vedi dopo -->}
\end{picture}
\begin{tikzpicture}[overlay]
\path(0pt,0pt);
\draw[color_29791,line width=0.7pt]
(95.8pt, -572.011pt) -- (391.1pt, -572.011pt)
;
\end{tikzpicture}
\begin{picture}(-5,0)(2.5,0)
\put(59.8,-596.211){\fontsize{12}{1}\usefont{T1}{cmr}{m}{n}\selectfont\color{color_29791}•}
\end{picture}
\begin{tikzpicture}[overlay]
\path(0pt,0pt);
\draw[color_29791,line width=0.7pt]
(59.8pt, -592.811pt) -- (64.10001pt, -592.811pt)
;
\end{tikzpicture}
\begin{picture}(-5,0)(2.5,0)
\put(64,-596.211){\fontsize{12}{1}\usefont{T1}{cmr}{m}{n}\selectfont\color{color_29791}  }
\end{picture}
\begin{tikzpicture}[overlay]
\path(0pt,0pt);
\draw[color_29791,line width=0.7pt]
(64pt, -592.811pt) -- (77.7pt, -592.811pt)
;
\end{tikzpicture}
\begin{picture}(-5,0)(2.5,0)
\put(77.8,-596.211){\fontsize{12}{1}\usefont{T1}{cmr}{b}{n}\selectfont\color{color_29791}Bridged}
\end{picture}
\begin{tikzpicture}[overlay]
\path(0pt,0pt);
\draw[color_29791,line width=0.7pt]
(77.8pt, -592.811pt) -- (119.1pt, -592.811pt)
;
\end{tikzpicture}
\begin{picture}(-5,0)(2.5,0)
\put(77.8,-617.011){\fontsize{12}{1}\usefont{T1}{cmr}{m}{n}\selectfont\color{color_29791}◦}
\end{picture}
\begin{tikzpicture}[overlay]
\path(0pt,0pt);
\draw[color_29791,line width=0.7pt]
(77.8pt, -613.611pt) -- (87.3pt, -613.611pt)
;
\end{tikzpicture}
\begin{picture}(-5,0)(2.5,0)
\put(87.3,-617.011){\fontsize{12}{1}\usefont{T1}{cmr}{m}{n}\selectfont\color{color_29791} }
\end{picture}
\begin{tikzpicture}[overlay]
\path(0pt,0pt);
\draw[color_29791,line width=0.7pt]
(87.3pt, -613.611pt) -- (93.3pt, -613.611pt)
;
\end{tikzpicture}
\begin{picture}(-5,0)(2.5,0)
\put(95.8,-617.011){\fontsize{12}{1}\usefont{T1}{cmr}{m}{n}\selectfont\color{color_29791}Si comportano come se fossero connesse all'interfaccia dell'host attraverso un }
\end{picture}
\begin{tikzpicture}[overlay]
\path(0pt,0pt);
\draw[color_29791,line width=0.7pt]
(95.8pt, -613.611pt) -- (471.3pt, -613.611pt)
;
\end{tikzpicture}
\begin{picture}(-5,0)(2.5,0)
\put(95.8,-630.811){\fontsize{12}{1}\usefont{T1}{cmr}{m}{n}\selectfont\color{color_29791}bridge/hub, quindi come se attestate sulla stessa LAN dell'host. Accesso materiale alla }
\end{picture}
\begin{tikzpicture}[overlay]
\path(0pt,0pt);
\draw[color_29791,line width=0.7pt]
(95.8pt, -627.411pt) -- (512.4pt, -627.411pt)
;
\end{tikzpicture}
\begin{picture}(-5,0)(2.5,0)
\put(95.8,-644.611){\fontsize{12}{1}\usefont{T1}{cmr}{m}{n}\selectfont\color{color_29791}stessa rete offre grande vantaggio in termini di flessibilità , ma va fatta attenzione }
\end{picture}
\begin{tikzpicture}[overlay]
\path(0pt,0pt);
\draw[color_29791,line width=0.7pt]
(95.8pt, -641.211pt) -- (489.7pt, -641.211pt)
;
\end{tikzpicture}
\begin{picture}(-5,0)(2.5,0)
\put(95.8,-658.411){\fontsize{12}{1}\usefont{T1}{cmr}{m}{n}\selectfont\color{color_29791}all'isolamento (la VM ha accesso diretto alla stessa rete delle macchina fisica: visibile }
\end{picture}
\begin{tikzpicture}[overlay]
\path(0pt,0pt);
\draw[color_29791,line width=0.7pt]
(95.8pt, -655.011pt) -- (509.8pt, -655.011pt)
;
\end{tikzpicture}
\begin{picture}(-5,0)(2.5,0)
\put(95.8,-672.211){\fontsize{12}{1}\usefont{T1}{cmr}{m}{n}\selectfont\color{color_29791}dall'esterno come se fosse una macchina fisica connessa alla stessa rete dell'host).}
\end{picture}
\begin{tikzpicture}[overlay]
\path(0pt,0pt);
\draw[color_29791,line width=0.7pt]
(95.8pt, -668.811pt) -- (487.2pt, -668.811pt)
;
\end{tikzpicture}
\begin{picture}(-5,0)(2.5,0)
\put(95.8,-693.011){\fontsize{12}{1}\usefont{T1}{cmr}{m}{n}\selectfont\color{color_29791}▪}
\end{picture}
\begin{tikzpicture}[overlay]
\path(0pt,0pt);
\draw[color_29791,line width=0.7pt]
(95.8pt, -689.611pt) -- (104.7pt, -689.611pt)
;
\end{tikzpicture}
\begin{picture}(-5,0)(2.5,0)
\put(104.7,-693.011){\fontsize{12}{1}\usefont{T1}{cmr}{m}{n}\selectfont\color{color_29791} }
\end{picture}
\begin{tikzpicture}[overlay]
\path(0pt,0pt);
\draw[color_29791,line width=0.7pt]
(104.7pt, -689.611pt) -- (110.7pt, -689.611pt)
;
\end{tikzpicture}
\begin{picture}(-5,0)(2.5,0)
\put(113.8,-693.011){\fontsize{12}{1}\usefont{T1}{cmr}{m}{n}\selectfont\color{color_29791}Vagrant può assegnare un'interfaccia bridged ad una VM con direttiva}
\end{picture}
\begin{tikzpicture}[overlay]
\path(0pt,0pt);
\draw[color_29791,line width=0.7pt]
(113.8pt, -689.611pt) -- (449.2pt, -689.611pt)
;
\end{tikzpicture}
\begin{picture}(-5,0)(2.5,0)
\put(113.7,-706.811){\fontsize{12}{1}\usefont{T1}{cmr}{m}{n}\selectfont\color{color_29791}           }
\end{picture}
\begin{tikzpicture}[overlay]
\path(0pt,0pt);
\draw[color_29791,line width=0.7pt]
(113.7pt, -703.411pt) -- (149.2pt, -703.411pt)
;
\end{tikzpicture}
\begin{picture}(-5,0)(2.5,0)
\put(149.3,-706.811){\fontsize{12}{1}\usefont{T1}{cmr}{m}{it}\selectfont\color{color_29791}config.vm.network "public\_network"}
\end{picture}
\begin{tikzpicture}[overlay]
\path(0pt,0pt);
\draw[color_29791,line width=0.7pt]
(149.3pt, -703.411pt) -- (325.6pt, -703.411pt)
;
\end{tikzpicture}
\begin{picture}(-5,0)(2.5,0)
\put(326.4,-706.811){\fontsize{12}{1}\usefont{T1}{cmr}{m}{it}\selectfont\color{color_29791}           }
\end{picture}
\begin{tikzpicture}[overlay]
\path(0pt,0pt);
\draw[color_29791,line width=0.7pt]
(326.4pt, -703.411pt) -- (361.9pt, -703.411pt)
;
\end{tikzpicture}
\begin{picture}(-5,0)(2.5,0)
\put(362,-706.811){\fontsize{12}{1}\usefont{T1}{cmr}{m}{n}\selectfont\color{color_29791}Nel vagrantfile, entro il ciclo di }
\end{picture}
\begin{tikzpicture}[overlay]
\path(0pt,0pt);
\draw[color_29791,line width=0.7pt]
(362pt, -703.411pt) -- (516.3pt, -703.411pt)
;
\end{tikzpicture}
\begin{picture}(-5,0)(2.5,0)
\put(113.8,-720.611){\fontsize{12}{1}\usefont{T1}{cmr}{m}{n}\selectfont\color{color_29791}configurazione }
\end{picture}
\begin{tikzpicture}[overlay]
\path(0pt,0pt);
\draw[color_29791,line width=0.7pt]
(113.8pt, -717.211pt) -- (188.7pt, -717.211pt)
;
\end{tikzpicture}
\begin{picture}(-5,0)(2.5,0)
\put(188.7,-720.611){\fontsize{12}{1}\usefont{T1}{cmr}{m}{n}\selectfont\color{color_29791}          }
\end{picture}
\begin{tikzpicture}[overlay]
\path(0pt,0pt);
\draw[color_29791,line width=0.7pt]
(188.7pt, -717.211pt) -- (220.1pt, -717.211pt)
;
\end{tikzpicture}
\begin{picture}(-5,0)(2.5,0)
\put(220.2,-720.611){\fontsize{12}{1}\usefont{T1}{cmr}{m}{n}\selectfont\color{color_29791}do|config| }
\end{picture}
\begin{tikzpicture}[overlay]
\path(0pt,0pt);
\draw[color_29791,line width=0.7pt]
(220.2pt, -717.211pt) -- (270.6pt, -717.211pt)
;
\end{tikzpicture}
\begin{picture}(-5,0)(2.5,0)
\put(270.5,-720.611){\fontsize{12}{1}\usefont{T1}{cmr}{m}{n}\selectfont\color{color_29791}      }
\end{picture}
\begin{tikzpicture}[overlay]
\path(0pt,0pt);
\draw[color_29791,line width=0.7pt]
(270.5pt, -717.211pt) -- (291pt, -717.211pt)
;
\end{tikzpicture}
\begin{picture}(-5,0)(2.5,0)
\put(291.1,-720.611){\fontsize{12}{1}\usefont{T1}{cmr}{m}{n}\selectfont\color{color_29791}se c'è questa direttiva, viene matchata con }
\end{picture}
\begin{tikzpicture}[overlay]
\path(0pt,0pt);
\draw[color_29791,line width=0.7pt]
(291.1pt, -717.211pt) -- (495.1pt, -717.211pt)
;
\end{tikzpicture}
\begin{picture}(-5,0)(2.5,0)
\put(113.8,-734.411){\fontsize{12}{1}\usefont{T1}{cmr}{m}{n}\selectfont\color{color_29791}interfaccia bridged di virtualbox. A default assegna in automatico tutti i parametri, }
\end{picture}
\begin{tikzpicture}[overlay]
\path(0pt,0pt);
\draw[color_29791,line width=0.7pt]
(113.8pt, -731.011pt) -- (511pt, -731.011pt)
;
\end{tikzpicture}
\begin{picture}(-5,0)(2.5,0)
\put(113.8,-748.211){\fontsize{12}{1}\usefont{T1}{cmr}{m}{n}\selectfont\color{color_29791}ma volendo con}
\end{picture}
\begin{tikzpicture}[overlay]
\path(0pt,0pt);
\draw[color_29791,line width=0.7pt]
(113.8pt, -744.811pt) -- (190.4pt, -744.811pt)
;
\end{tikzpicture}
\newpage
\begin{tikzpicture}[overlay]\path(0pt,0pt);\end{tikzpicture}
\begin{picture}(-5,0)(2.5,0)
\put(113.8,-85.01099){\fontsize{12}{1}\usefont{T1}{cmr}{m}{n}\selectfont\color{color_29791}1.}
\end{picture}
\begin{tikzpicture}[overlay]
\path(0pt,0pt);
\draw[color_29791,line width=0.7pt]
(113.8pt, -81.61096pt) -- (122.8pt, -81.61096pt)
;
\end{tikzpicture}
\begin{picture}(-5,0)(2.5,0)
\put(122.7,-85.01099){\fontsize{12}{1}\usefont{T1}{cmr}{m}{n}\selectfont\color{color_29791}   }
\end{picture}
\begin{tikzpicture}[overlay]
\path(0pt,0pt);
\draw[color_29791,line width=0.7pt]
(122.7pt, -81.61096pt) -- (131.7pt, -81.61096pt)
;
\end{tikzpicture}
\begin{picture}(-5,0)(2.5,0)
\put(131.8,-85.01099){\fontsize{12}{1}\usefont{T1}{cmr}{m}{it}\selectfont\color{color_29791}config.vm.network "public\_network", ip: "192.168.0.17"}
\end{picture}
\begin{tikzpicture}[overlay]
\path(0pt,0pt);
\draw[color_29791,line width=0.7pt]
(131.8pt, -81.61096pt) -- (403.6pt, -81.61096pt)
;
\end{tikzpicture}
\begin{picture}(-5,0)(2.5,0)
\put(403.5,-85.01099){\fontsize{12}{1}\usefont{T1}{cmr}{m}{it}\selectfont\color{color_29791}   }
\end{picture}
\begin{tikzpicture}[overlay]
\path(0pt,0pt);
\draw[color_29791,line width=0.7pt]
(403.5pt, -81.61096pt) -- (415.3pt, -81.61096pt)
;
\end{tikzpicture}
\begin{picture}(-5,0)(2.5,0)
\put(415.4,-85.01099){\fontsize{12}{1}\usefont{T1}{cmr}{m}{n}\selectfont\color{color_29791}si può assegnare IP }
\end{picture}
\begin{tikzpicture}[overlay]
\path(0pt,0pt);
\draw[color_29791,line width=0.7pt]
(415.4pt, -81.61096pt) -- (510.2pt, -81.61096pt)
;
\end{tikzpicture}
\begin{picture}(-5,0)(2.5,0)
\put(131.8,-98.81097){\fontsize{12}{1}\usefont{T1}{cmr}{m}{n}\selectfont\color{color_29791}statico che verrà dato alla VM in fase di provisioning; usando vagrant per }
\end{picture}
\begin{tikzpicture}[overlay]
\path(0pt,0pt);
\draw[color_29791,line width=0.7pt]
(131.8pt, -95.41095pt) -- (487.8pt, -95.41095pt)
;
\end{tikzpicture}
\begin{picture}(-5,0)(2.5,0)
\put(131.8,-112.611){\fontsize{12}{1}\usefont{T1}{cmr}{m}{n}\selectfont\color{color_29791}configurare il guest, assegnandogli direttamente un indirizzo IP }
\end{picture}
\begin{tikzpicture}[overlay]
\path(0pt,0pt);
\draw[color_29791,line width=0.7pt]
(131.8pt, -109.211pt) -- (438.9pt, -109.211pt)
;
\end{tikzpicture}
\begin{picture}(-5,0)(2.5,0)
\put(113.8,-133.411){\fontsize{12}{1}\usefont{T1}{cmr}{m}{it}\selectfont\color{color_29791}2.}
\end{picture}
\begin{tikzpicture}[overlay]
\path(0pt,0pt);
\draw[color_29791,line width=0.7pt]
(113.8pt, -130.011pt) -- (122.8pt, -130.011pt)
;
\end{tikzpicture}
\begin{picture}(-5,0)(2.5,0)
\put(122.7,-133.411){\fontsize{12}{1}\usefont{T1}{cmr}{m}{it}\selectfont\color{color_29791}   }
\end{picture}
\begin{tikzpicture}[overlay]
\path(0pt,0pt);
\draw[color_29791,line width=0.7pt]
(122.7pt, -130.011pt) -- (131.7pt, -130.011pt)
;
\end{tikzpicture}
\begin{picture}(-5,0)(2.5,0)
\put(131.8,-133.411){\fontsize{12}{1}\usefont{T1}{cmr}{m}{it}\selectfont\color{color_29791}config.vm.network "public\_network", bridge: "eth0"}
\end{picture}
\begin{tikzpicture}[overlay]
\path(0pt,0pt);
\draw[color_29791,line width=0.7pt]
(131.8pt, -130.011pt) -- (383.2pt, -130.011pt)
;
\end{tikzpicture}
\begin{picture}(-5,0)(2.5,0)
\put(383.2,-133.411){\fontsize{12}{1}\usefont{T1}{cmr}{m}{it}\selectfont\color{color_29791}          }
\end{picture}
\begin{tikzpicture}[overlay]
\path(0pt,0pt);
\draw[color_29791,line width=0.7pt]
(383.2pt, -130.011pt) -- (415.4pt, -130.011pt)
;
\end{tikzpicture}
\begin{picture}(-5,0)(2.5,0)
\put(415.4,-133.411){\fontsize{12}{1}\usefont{T1}{cmr}{m}{n}\selectfont\color{color_29791}si può selezionare a }
\end{picture}
\begin{tikzpicture}[overlay]
\path(0pt,0pt);
\draw[color_29791,line width=0.7pt]
(415.4pt, -130.011pt) -- (512.7pt, -130.011pt)
;
\end{tikzpicture}
\begin{picture}(-5,0)(2.5,0)
\put(131.8,-147.211){\fontsize{12}{1}\usefont{T1}{cmr}{m}{n}\selectfont\color{color_29791}quale interfaccia host agganciarsi}
\end{picture}
\begin{tikzpicture}[overlay]
\path(0pt,0pt);
\draw[color_29791,line width=0.7pt]
(131.8pt, -143.811pt) -- (292pt, -143.811pt)
;
\end{tikzpicture}
\begin{picture}(-5,0)(2.5,0)
\put(113.8,-168.011){\fontsize{12}{1}\usefont{T1}{cmr}{m}{it}\selectfont\color{color_29791}3.}
\end{picture}
\begin{tikzpicture}[overlay]
\path(0pt,0pt);
\draw[color_29791,line width=0.7pt]
(113.8pt, -164.611pt) -- (122.8pt, -164.611pt)
;
\end{tikzpicture}
\begin{picture}(-5,0)(2.5,0)
\put(122.7,-168.011){\fontsize{12}{1}\usefont{T1}{cmr}{m}{it}\selectfont\color{color_29791}   }
\end{picture}
\begin{tikzpicture}[overlay]
\path(0pt,0pt);
\draw[color_29791,line width=0.7pt]
(122.7pt, -164.611pt) -- (131.7pt, -164.611pt)
;
\end{tikzpicture}
\begin{picture}(-5,0)(2.5,0)
\put(131.8,-168.011){\fontsize{12}{1}\usefont{T1}{cmr}{m}{it}\selectfont\color{color_29791}config.vm.network "public\_network", auto\_config: false}
\end{picture}
\begin{tikzpicture}[overlay]
\path(0pt,0pt);
\draw[color_29791,line width=0.7pt]
(131.8pt, -164.611pt) -- (401.1pt, -164.611pt)
;
\end{tikzpicture}
\begin{picture}(-5,0)(2.5,0)
\put(401.1,-168.011){\fontsize{12}{1}\usefont{T1}{cmr}{m}{it}\selectfont\color{color_29791}    }
\end{picture}
\begin{tikzpicture}[overlay]
\path(0pt,0pt);
\draw[color_29791,line width=0.7pt]
(401.1pt, -164.611pt) -- (415.4pt, -164.611pt)
;
\end{tikzpicture}
\begin{picture}(-5,0)(2.5,0)
\put(415.4,-168.011){\fontsize{12}{1}\usefont{T1}{cmr}{m}{n}\selectfont\color{color_29791}si può disabilitare }
\end{picture}
\begin{tikzpicture}[overlay]
\path(0pt,0pt);
\draw[color_29791,line width=0.7pt]
(415.4pt, -164.611pt) -- (503.7pt, -164.611pt)
;
\end{tikzpicture}
\begin{picture}(-5,0)(2.5,0)
\put(131.8,-181.811){\fontsize{12}{1}\usefont{T1}{cmr}{m}{n}\selectfont\color{color_29791}automatismo per lasciare che il guest si occupi della configurazione, utile ad }
\end{picture}
\begin{tikzpicture}[overlay]
\path(0pt,0pt);
\draw[color_29791,line width=0.7pt]
(131.8pt, -178.4109pt) -- (501.3pt, -178.4109pt)
;
\end{tikzpicture}
\begin{picture}(-5,0)(2.5,0)
\put(131.8,-195.611){\fontsize{12}{1}\usefont{T1}{cmr}{m}{n}\selectfont\color{color_29791}esempio nel caso nella VM ci sia un client DHCP: servirà un server DCHP sulla }
\end{picture}
\begin{tikzpicture}[overlay]
\path(0pt,0pt);
\draw[color_29791,line width=0.7pt]
(131.8pt, -192.211pt) -- (519.3pt, -192.211pt)
;
\end{tikzpicture}
\begin{picture}(-5,0)(2.5,0)
\put(131.8,-209.411){\fontsize{12}{1}\usefont{T1}{cmr}{m}{n}\selectfont\color{color_29791}rete reale che fornisca configurazone alla VM.}
\end{picture}
\begin{tikzpicture}[overlay]
\path(0pt,0pt);
\draw[color_29791,line width=0.7pt]
(131.8pt, -206.011pt) -- (354.4pt, -206.011pt)
;
\end{tikzpicture}
\begin{picture}(-5,0)(2.5,0)
\put(59.8,-230.211){\fontsize{12}{1}\usefont{T1}{cmr}{m}{n}\selectfont\color{color_29791}•}
\end{picture}
\begin{tikzpicture}[overlay]
\path(0pt,0pt);
\draw[color_29791,line width=0.7pt]
(59.8pt, -226.811pt) -- (64.10001pt, -226.811pt)
;
\end{tikzpicture}
\begin{picture}(-5,0)(2.5,0)
\put(64,-230.211){\fontsize{12}{1}\usefont{T1}{cmr}{m}{n}\selectfont\color{color_29791}  }
\end{picture}
\begin{tikzpicture}[overlay]
\path(0pt,0pt);
\draw[color_29791,line width=0.7pt]
(64pt, -226.811pt) -- (77.7pt, -226.811pt)
;
\end{tikzpicture}
\begin{picture}(-5,0)(2.5,0)
\put(77.8,-230.211){\fontsize{12}{1}\usefont{T1}{cmr}{b}{n}\selectfont\color{color_29791}Host-only}
\end{picture}
\begin{tikzpicture}[overlay]
\path(0pt,0pt);
\draw[color_29791,line width=0.7pt]
(77.8pt, -226.811pt) -- (127.8pt, -226.811pt)
;
\end{tikzpicture}
\begin{picture}(-5,0)(2.5,0)
\put(77.8,-251.011){\fontsize{12}{1}\usefont{T1}{cmr}{m}{n}\selectfont\color{color_29791}◦}
\end{picture}
\begin{tikzpicture}[overlay]
\path(0pt,0pt);
\draw[color_29791,line width=0.7pt]
(77.8pt, -247.611pt) -- (87.3pt, -247.611pt)
;
\end{tikzpicture}
\begin{picture}(-5,0)(2.5,0)
\put(87.3,-251.011){\fontsize{12}{1}\usefont{T1}{cmr}{m}{n}\selectfont\color{color_29791} }
\end{picture}
\begin{tikzpicture}[overlay]
\path(0pt,0pt);
\draw[color_29791,line width=0.7pt]
(87.3pt, -247.611pt) -- (93.3pt, -247.611pt)
;
\end{tikzpicture}
\begin{picture}(-5,0)(2.5,0)
\put(95.8,-251.011){\fontsize{12}{1}\usefont{T1}{cmr}{m}{n}\selectfont\color{color_29791}Hypervisor genera interfaccia virtuale sull'host e le assegna un IP di una specifica }
\end{picture}
\begin{tikzpicture}[overlay]
\path(0pt,0pt);
\draw[color_29791,line width=0.7pt]
(95.8pt, -247.611pt) -- (490.3pt, -247.611pt)
;
\end{tikzpicture}
\begin{picture}(-5,0)(2.5,0)
\put(95.8,-264.811){\fontsize{12}{1}\usefont{T1}{cmr}{m}{n}\selectfont\color{color_29791}subnet, poi connette alla LAN virtuale la VM, così che questa possa comunicare solo }
\end{picture}
\begin{tikzpicture}[overlay]
\path(0pt,0pt);
\draw[color_29791,line width=0.7pt]
(95.8pt, -261.4109pt) -- (506.7pt, -261.4109pt)
;
\end{tikzpicture}
\begin{picture}(-5,0)(2.5,0)
\put(95.8,-278.611){\fontsize{12}{1}\usefont{T1}{cmr}{m}{n}\selectfont\color{color_29791}con l'host. }
\end{picture}
\begin{tikzpicture}[overlay]
\path(0pt,0pt);
\draw[color_29791,line width=0.7pt]
(95.8pt, -275.211pt) -- (147.6pt, -275.211pt)
;
\end{tikzpicture}
\begin{picture}(-5,0)(2.5,0)
\put(95.8,-299.411){\fontsize{12}{1}\usefont{T1}{cmr}{m}{n}\selectfont\color{color_29791}▪}
\end{picture}
\begin{tikzpicture}[overlay]
\path(0pt,0pt);
\draw[color_29791,line width=0.7pt]
(95.8pt, -296.011pt) -- (104.7pt, -296.011pt)
;
\end{tikzpicture}
\begin{picture}(-5,0)(2.5,0)
\put(104.7,-299.411){\fontsize{12}{1}\usefont{T1}{cmr}{m}{n}\selectfont\color{color_29791} }
\end{picture}
\begin{tikzpicture}[overlay]
\path(0pt,0pt);
\draw[color_29791,line width=0.7pt]
(104.7pt, -296.011pt) -- (110.7pt, -296.011pt)
;
\end{tikzpicture}
\begin{picture}(-5,0)(2.5,0)
\put(113.8,-299.411){\fontsize{12}{1}\usefont{T1}{cmr}{m}{n}\selectfont\color{color_29791}Vagrant assegna interfaccia host-only a VM con direttiva}
\end{picture}
\begin{tikzpicture}[overlay]
\path(0pt,0pt);
\draw[color_29791,line width=0.7pt]
(113.8pt, -296.011pt) -- (387.1pt, -296.011pt)
;
\end{tikzpicture}
\begin{picture}(-5,0)(2.5,0)
\put(387.1,-299.411){\fontsize{12}{1}\usefont{T1}{cmr}{m}{n}\selectfont\color{color_29791}   }
\end{picture}
\begin{tikzpicture}[overlay]
\path(0pt,0pt);
\draw[color_29791,line width=0.7pt]
(387.1pt, -296.011pt) -- (397.4pt, -296.011pt)
;
\end{tikzpicture}
\begin{picture}(-5,0)(2.5,0)
\put(397.4,-299.411){\fontsize{12}{1}\usefont{T1}{cmr}{m}{it}\selectfont\color{color_29791}config.vm.network }
\end{picture}
\begin{tikzpicture}[overlay]
\path(0pt,0pt);
\draw[color_29791,line width=0.7pt]
(397.4pt, -296.011pt) -- (489pt, -296.011pt)
;
\end{tikzpicture}
\begin{picture}(-5,0)(2.5,0)
\put(113.8,-313.211){\fontsize{12}{1}\usefont{T1}{cmr}{m}{it}\selectfont\color{color_29791}"private\_network"}
\end{picture}
\begin{tikzpicture}[overlay]
\path(0pt,0pt);
\draw[color_29791,line width=0.7pt]
(113.8pt, -309.811pt) -- (202.5pt, -309.811pt)
;
\end{tikzpicture}
\begin{picture}(-5,0)(2.5,0)
\put(202.4,-313.211){\fontsize{12}{1}\usefont{T1}{cmr}{m}{it}\selectfont\color{color_29791}     }
\end{picture}
\begin{tikzpicture}[overlay]
\path(0pt,0pt);
\draw[color_29791,line width=0.7pt]
(202.4pt, -309.811pt) -- (220.1pt, -309.811pt)
;
\end{tikzpicture}
\begin{picture}(-5,0)(2.5,0)
\put(113.8,-334.011){\fontsize{12}{1}\usefont{T1}{cmr}{m}{n}\selectfont\color{color_29791}1.}
\end{picture}
\begin{tikzpicture}[overlay]
\path(0pt,0pt);
\draw[color_29791,line width=0.7pt]
(113.8pt, -330.611pt) -- (122.8pt, -330.611pt)
;
\end{tikzpicture}
\begin{picture}(-5,0)(2.5,0)
\put(122.7,-334.011){\fontsize{12}{1}\usefont{T1}{cmr}{m}{n}\selectfont\color{color_29791}   }
\end{picture}
\begin{tikzpicture}[overlay]
\path(0pt,0pt);
\draw[color_29791,line width=0.7pt]
(122.7pt, -330.611pt) -- (131.7pt, -330.611pt)
;
\end{tikzpicture}
\begin{picture}(-5,0)(2.5,0)
\put(131.8,-334.011){\fontsize{12}{1}\usefont{T1}{cmr}{m}{it}\selectfont\color{color_29791}config.vm.network "private\_network", name: “vboxnet3”}
\end{picture}
\begin{tikzpicture}[overlay]
\path(0pt,0pt);
\draw[color_29791,line width=0.7pt]
(131.8pt, -330.611pt) -- (407.8pt, -330.611pt)
;
\end{tikzpicture}
\begin{picture}(-5,0)(2.5,0)
\put(407.7,-334.011){\fontsize{12}{1}\usefont{T1}{cmr}{m}{n}\selectfont\color{color_29791}  }
\end{picture}
\begin{tikzpicture}[overlay]
\path(0pt,0pt);
\draw[color_29791,line width=0.7pt]
(407.7pt, -330.611pt) -- (415.3pt, -330.611pt)
;
\end{tikzpicture}
\begin{picture}(-5,0)(2.5,0)
\put(415.4,-334.011){\fontsize{12}{1}\usefont{T1}{cmr}{m}{n}\selectfont\color{color_29791}si può creare più di }
\end{picture}
\begin{tikzpicture}[overlay]
\path(0pt,0pt);
\draw[color_29791,line width=0.7pt]
(415.4pt, -330.611pt) -- (510.3pt, -330.611pt)
;
\end{tikzpicture}
\begin{picture}(-5,0)(2.5,0)
\put(131.8,-347.811){\fontsize{12}{1}\usefont{T1}{cmr}{m}{n}\selectfont\color{color_29791}una rete host-only specificando il nome}
\end{picture}
\begin{tikzpicture}[overlay]
\path(0pt,0pt);
\draw[color_29791,line width=0.7pt]
(131.8pt, -344.411pt) -- (321.4pt, -344.411pt)
;
\end{tikzpicture}
\begin{picture}(-5,0)(2.5,0)
\put(113.8,-368.611){\fontsize{12}{1}\usefont{T1}{cmr}{m}{n}\selectfont\color{color_29791}2.}
\end{picture}
\begin{tikzpicture}[overlay]
\path(0pt,0pt);
\draw[color_29791,line width=0.7pt]
(113.8pt, -365.211pt) -- (122.8pt, -365.211pt)
;
\end{tikzpicture}
\begin{picture}(-5,0)(2.5,0)
\put(122.7,-368.611){\fontsize{12}{1}\usefont{T1}{cmr}{m}{n}\selectfont\color{color_29791}   }
\end{picture}
\begin{tikzpicture}[overlay]
\path(0pt,0pt);
\draw[color_29791,line width=0.7pt]
(122.7pt, -365.211pt) -- (131.7pt, -365.211pt)
;
\end{tikzpicture}
\begin{picture}(-5,0)(2.5,0)
\put(131.8,-368.611){\fontsize{12}{1}\usefont{T1}{cmr}{m}{n}\selectfont\color{color_29791}valgono opzioni }
\end{picture}
\begin{tikzpicture}[overlay]
\path(0pt,0pt);
\draw[color_29791,line width=0.7pt]
(131.8pt, -365.211pt) -- (212.4pt, -365.211pt)
;
\end{tikzpicture}
\begin{picture}(-5,0)(2.5,0)
\put(212.5,-368.611){\fontsize{12}{1}\usefont{T1}{cmr}{m}{it}\selectfont\color{color_29791}ip }
\end{picture}
\begin{tikzpicture}[overlay]
\path(0pt,0pt);
\draw[color_29791,line width=0.7pt]
(212.5pt, -365.211pt) -- (224.8pt, -365.211pt)
;
\end{tikzpicture}
\begin{picture}(-5,0)(2.5,0)
\put(224.8,-368.611){\fontsize{12}{1}\usefont{T1}{cmr}{m}{n}\selectfont\color{color_29791}e }
\end{picture}
\begin{tikzpicture}[overlay]
\path(0pt,0pt);
\draw[color_29791,line width=0.7pt]
(224.8pt, -365.211pt) -- (233.1pt, -365.211pt)
;
\end{tikzpicture}
\begin{picture}(-5,0)(2.5,0)
\put(233.2,-368.611){\fontsize{12}{1}\usefont{T1}{cmr}{m}{it}\selectfont\color{color_29791}auto\_config}
\end{picture}
\begin{tikzpicture}[overlay]
\path(0pt,0pt);
\draw[color_29791,line width=0.7pt]
(233.2pt, -365.211pt) -- (290.5pt, -365.211pt)
;
\end{tikzpicture}
\begin{picture}(-5,0)(2.5,0)
\put(290.5,-368.611){\fontsize{12}{1}\usefont{T1}{cmr}{m}{n}\selectfont\color{color_29791} come per bridged}
\end{picture}
\begin{tikzpicture}[overlay]
\path(0pt,0pt);
\draw[color_29791,line width=0.7pt]
(290.5pt, -365.211pt) -- (377.4pt, -365.211pt)
;
\end{tikzpicture}
\begin{picture}(-5,0)(2.5,0)
\put(59.8,-389.411){\fontsize{12}{1}\usefont{T1}{cmr}{m}{n}\selectfont\color{color_29791}•}
\end{picture}
\begin{tikzpicture}[overlay]
\path(0pt,0pt);
\draw[color_29791,line width=0.7pt]
(59.8pt, -386.011pt) -- (64.10001pt, -386.011pt)
;
\end{tikzpicture}
\begin{picture}(-5,0)(2.5,0)
\put(64,-389.411){\fontsize{12}{1}\usefont{T1}{cmr}{m}{n}\selectfont\color{color_29791}  }
\end{picture}
\begin{tikzpicture}[overlay]
\path(0pt,0pt);
\draw[color_29791,line width=0.7pt]
(64pt, -386.011pt) -- (77.7pt, -386.011pt)
;
\end{tikzpicture}
\begin{picture}(-5,0)(2.5,0)
\put(77.8,-389.411){\fontsize{12}{1}\usefont{T1}{cmr}{b}{n}\selectfont\color{color_29791}Internal}
\end{picture}
\begin{tikzpicture}[overlay]
\path(0pt,0pt);
\draw[color_29791,line width=0.7pt]
(77.8pt, -386.011pt) -- (119.8pt, -386.011pt)
;
\end{tikzpicture}
\begin{picture}(-5,0)(2.5,0)
\put(77.8,-410.211){\fontsize{12}{1}\usefont{T1}{cmr}{m}{n}\selectfont\color{color_29791}◦}
\end{picture}
\begin{tikzpicture}[overlay]
\path(0pt,0pt);
\draw[color_29791,line width=0.7pt]
(77.8pt, -406.811pt) -- (87.3pt, -406.811pt)
;
\end{tikzpicture}
\begin{picture}(-5,0)(2.5,0)
\put(87.3,-410.211){\fontsize{12}{1}\usefont{T1}{cmr}{m}{n}\selectfont\color{color_29791} }
\end{picture}
\begin{tikzpicture}[overlay]
\path(0pt,0pt);
\draw[color_29791,line width=0.7pt]
(87.3pt, -406.811pt) -- (93.3pt, -406.811pt)
;
\end{tikzpicture}
\begin{picture}(-5,0)(2.5,0)
\put(95.8,-410.211){\fontsize{12}{1}\usefont{T1}{cmr}{m}{n}\selectfont\color{color_29791}Hypervisore assegna queste interfacce a LAN completamente virtuale, e fa sì che solo }
\end{picture}
\begin{tikzpicture}[overlay]
\path(0pt,0pt);
\draw[color_29791,line width=0.7pt]
(95.8pt, -406.811pt) -- (511.6pt, -406.811pt)
;
\end{tikzpicture}
\begin{picture}(-5,0)(2.5,0)
\put(95.8,-424.011){\fontsize{12}{1}\usefont{T1}{cmr}{m}{n}\selectfont\color{color_29791}interfacce della stessa internal network possano comunicare tra loro}
\end{picture}
\begin{tikzpicture}[overlay]
\path(0pt,0pt);
\draw[color_29791,line width=0.7pt]
(95.8pt, -420.611pt) -- (421pt, -420.611pt)
;
\end{tikzpicture}
\begin{picture}(-5,0)(2.5,0)
\put(95.8,-444.811){\fontsize{12}{1}\usefont{T1}{cmr}{m}{n}\selectfont\color{color_29791}▪}
\end{picture}
\begin{tikzpicture}[overlay]
\path(0pt,0pt);
\draw[color_29791,line width=0.7pt]
(95.8pt, -441.411pt) -- (104.7pt, -441.411pt)
;
\end{tikzpicture}
\begin{picture}(-5,0)(2.5,0)
\put(104.7,-444.811){\fontsize{12}{1}\usefont{T1}{cmr}{m}{n}\selectfont\color{color_29791} }
\end{picture}
\begin{tikzpicture}[overlay]
\path(0pt,0pt);
\draw[color_29791,line width=0.7pt]
(104.7pt, -441.411pt) -- (110.7pt, -441.411pt)
;
\end{tikzpicture}
\begin{picture}(-5,0)(2.5,0)
\put(113.8,-444.811){\fontsize{12}{1}\usefont{T1}{cmr}{m}{n}\selectfont\color{color_29791}Vagrant le tratta come caso speciale di private\_network}
\end{picture}
\begin{tikzpicture}[overlay]
\path(0pt,0pt);
\draw[color_29791,line width=0.7pt]
(113.8pt, -441.411pt) -- (379.3pt, -441.411pt)
;
\end{tikzpicture}
\begin{picture}(-5,0)(2.5,0)
\put(379.3,-444.811){\fontsize{12}{1}\usefont{T1}{cmr}{m}{n}\selectfont\color{color_29791}      }
\end{picture}
\begin{tikzpicture}[overlay]
\path(0pt,0pt);
\draw[color_29791,line width=0.7pt]
(379.3pt, -441.411pt) -- (397.4pt, -441.411pt)
;
\end{tikzpicture}
\begin{picture}(-5,0)(2.5,0)
\put(397.4,-444.811){\fontsize{12}{1}\usefont{T1}{cmr}{m}{it}\selectfont\color{color_29791}config.vm.network }
\end{picture}
\begin{tikzpicture}[overlay]
\path(0pt,0pt);
\draw[color_29791,line width=0.7pt]
(397.4pt, -441.411pt) -- (489pt, -441.411pt)
;
\end{tikzpicture}
\begin{picture}(-5,0)(2.5,0)
\put(113.8,-458.611){\fontsize{12}{1}\usefont{T1}{cmr}{m}{it}\selectfont\color{color_29791}"private\_network", virtualbox\_\_intnet: "LAN1"}
\end{picture}
\begin{tikzpicture}[overlay]
\path(0pt,0pt);
\draw[color_29791,line width=0.7pt]
(113.8pt, -455.211pt) -- (342.2pt, -455.211pt)
;
\end{tikzpicture}
\begin{picture}(-5,0)(2.5,0)
\put(113.8,-479.411){\fontsize{12}{1}\usefont{T1}{cmr}{m}{n}\selectfont\color{color_29791}1.}
\end{picture}
\begin{tikzpicture}[overlay]
\path(0pt,0pt);
\draw[color_29791,line width=0.7pt]
(113.8pt, -476.011pt) -- (122.8pt, -476.011pt)
;
\end{tikzpicture}
\begin{picture}(-5,0)(2.5,0)
\put(122.7,-479.411){\fontsize{12}{1}\usefont{T1}{cmr}{m}{n}\selectfont\color{color_29791}   }
\end{picture}
\begin{tikzpicture}[overlay]
\path(0pt,0pt);
\draw[color_29791,line width=0.7pt]
(122.7pt, -476.011pt) -- (131.7pt, -476.011pt)
;
\end{tikzpicture}
\begin{picture}(-5,0)(2.5,0)
\put(131.8,-479.411){\fontsize{12}{1}\usefont{T1}{cmr}{m}{n}\selectfont\color{color_29791}Valgono opzioni }
\end{picture}
\begin{tikzpicture}[overlay]
\path(0pt,0pt);
\draw[color_29791,line width=0.7pt]
(131.8pt, -476.011pt) -- (213.8pt, -476.011pt)
;
\end{tikzpicture}
\begin{picture}(-5,0)(2.5,0)
\put(213.8,-479.411){\fontsize{12}{1}\usefont{T1}{cmr}{m}{it}\selectfont\color{color_29791}ip}
\end{picture}
\begin{tikzpicture}[overlay]
\path(0pt,0pt);
\draw[color_29791,line width=0.7pt]
(213.8pt, -476.011pt) -- (223.1pt, -476.011pt)
;
\end{tikzpicture}
\begin{picture}(-5,0)(2.5,0)
\put(223.2,-479.411){\fontsize{12}{1}\usefont{T1}{cmr}{m}{n}\selectfont\color{color_29791} e }
\end{picture}
\begin{tikzpicture}[overlay]
\path(0pt,0pt);
\draw[color_29791,line width=0.7pt]
(223.2pt, -476.011pt) -- (234.5pt, -476.011pt)
;
\end{tikzpicture}
\begin{picture}(-5,0)(2.5,0)
\put(234.5,-479.411){\fontsize{12}{1}\usefont{T1}{cmr}{m}{it}\selectfont\color{color_29791}auto\_config }
\end{picture}
\begin{tikzpicture}[overlay]
\path(0pt,0pt);
\draw[color_29791,line width=0.7pt]
(234.5pt, -476.011pt) -- (294.8pt, -476.011pt)
;
\end{tikzpicture}
\begin{picture}(-5,0)(2.5,0)
\put(294.9,-479.411){\fontsize{12}{1}\usefont{T1}{cmr}{m}{n}\selectfont\color{color_29791}(queste sono opzioni specifica del provider }
\end{picture}
\begin{tikzpicture}[overlay]
\path(0pt,0pt);
\draw[color_29791,line width=0.7pt]
(294.9pt, -476.011pt) -- (504.1pt, -476.011pt)
;
\end{tikzpicture}
\begin{picture}(-5,0)(2.5,0)
\put(131.8,-493.211){\fontsize{12}{1}\usefont{T1}{cmr}{m}{n}\selectfont\color{color_29791}Vbox, (dal punto di vista di vagrant, il provider è chi fornisce supporto alla }
\end{picture}
\begin{tikzpicture}[overlay]
\path(0pt,0pt);
\draw[color_29791,line width=0.7pt]
(131.8pt, -489.811pt) -- (495pt, -489.811pt)
;
\end{tikzpicture}
\begin{picture}(-5,0)(2.5,0)
\put(131.8,-507.011){\fontsize{12}{1}\usefont{T1}{cmr}{m}{n}\selectfont\color{color_29791}virtualizzazione), potrebbero essere diverse con altri provider)}
\end{picture}
\begin{tikzpicture}[overlay]
\path(0pt,0pt);
\draw[color_29791,line width=0.7pt]
(131.8pt, -503.611pt) -- (430.6pt, -503.611pt)
;
\end{tikzpicture}
\begin{picture}(-5,0)(2.5,0)
\put(41.8,-536.811){\fontsize{14.1}{1}\usefont{T1}{cmr}{b}{n}\selectfont\color{color_29791}Reti globali}
\put(41.8,-557.011){\fontsize{12}{1}\usefont{T1}{cmr}{m}{n}\selectfont\color{color_29791}Le isole sono interconnesse da apparti che fanno da ponte (computer specializzati detti router o }
\put(41.8,-570.811){\fontsize{12}{1}\usefont{T1}{cmr}{m}{n}\selectfont\color{color_29791}gateway) spesso realizzati con tecnologie diverse da quelle dell'isola. L'obiettivo di IP è rendere }
\put(41.8,-584.611){\fontsize{12}{1}\usefont{T1}{cmr}{m}{n}\selectfont\color{color_29791}possibile il dialogo tra network a prescindere dall'implementazione, è realizzato per lavorare }
\put(41.8,-598.411){\fontsize{12}{1}\usefont{T1}{cmr}{m}{n}\selectfont\color{color_29791}indifferentemente su tecnologie diverse. LAN con switch separati, interconnesse mediante un }
\put(41.8,-612.211){\fontsize{12}{1}\usefont{T1}{cmr}{m}{n}\selectfont\color{color_29791}router, hanno possibilità di broadcast separato e possono contare su maggiore sicurezza (dovendo }
\put(41.8,-626.011){\fontsize{12}{1}\usefont{T1}{cmr}{m}{n}\selectfont\color{color_29791}ogni host comunicare con il router per raggiungere le altre LAN interconnesse). I router (gateway) }
\put(41.8,-639.811){\fontsize{12}{1}\usefont{T1}{cmr}{m}{n}\selectfont\color{color_29791}devono poter parlare sulle tecnologie di entrambi le network. Per capire se un dato pacchetto in }
\put(41.8,-653.611){\fontsize{12}{1}\usefont{T1}{cmr}{m}{n}\selectfont\color{color_29791}uscita da un host deve andare sulla rete locale o se deve passare da un gateway, ogni nodo ha una }
\put(41.8,-667.411){\fontsize{12}{1}\usefont{T1}{cmr}{m}{n}\selectfont\color{color_29791}base dati con le destinazioni (route) possibili, su una macchina linux è consultabile con ip route }
\put(41.8,-681.211){\fontsize{12}{1}\usefont{T1}{cmr}{m}{n}\selectfont\color{color_29791}(abbreviato ip r). }
\put(41.8,-702.011){\fontsize{12}{1}\usefont{T1}{cmr}{m}{n}\selectfont\color{color_29791}La scelta del percorso su cui inviare i dati è detta routing. I router formano una struttura }
\put(41.8,-715.811){\fontsize{12}{1}\usefont{T1}{cmr}{m}{n}\selectfont\color{color_29791}interconnessa, i datagrammi passano da uno all'altro fino a raggiungere quello che può consegnarli }
\put(41.8,-729.611){\fontsize{12}{1}\usefont{T1}{cmr}{m}{n}\selectfont\color{color_29791}direttamente al destinatario: L'instradamento di un pacchetto si dice diretto quando IP sorgente e IP }
\put(41.8,-743.411){\fontsize{12}{1}\usefont{T1}{cmr}{m}{n}\selectfont\color{color_29791}destinatario sono sulla stessa rete fisica, indiretto in caso contrario. Quindi da mittente a destinatario}
\put(41.8,-757.211){\fontsize{12}{1}\usefont{T1}{cmr}{m}{n}\selectfont\color{color_29791}c'è sempre almeno una consegna diretta, e possono esserci zero (sorgente-destinazione sulla stessa }
\put(41.8,-771.011){\fontsize{12}{1}\usefont{T1}{cmr}{m}{n}\selectfont\color{color_29791}rete fisica) o più (numero variabile di hop tra router) consegne indirette.}
\end{picture}
\newpage
\begin{tikzpicture}[overlay]\path(0pt,0pt);\end{tikzpicture}
\begin{picture}(-5,0)(2.5,0)
\put(41.8,-85.01099){\fontsize{12}{1}\usefont{T1}{cmr}{m}{n}\selectfont\color{color_29791}Sulla LAN le macchine si possono parlare elettronicamente con i MAC address. Su Internet le }
\put(41.8,-98.81097){\fontsize{12}{1}\usefont{T1}{cmr}{m}{n}\selectfont\color{color_29791}macchine possono parlare facendo dei salti fino al nodo destinazione.}
\put(41.8,-119.611){\fontsize{12}{1}\usefont{T1}{cmr}{m}{n}\selectfont\color{color_29791}A livello di trasporto, ulteriore indirizzamento in base alla porta; i pacchetti vengono spediti ad una }
\put(41.8,-133.411){\fontsize{12}{1}\usefont{T1}{cmr}{m}{n}\selectfont\color{color_29791}porta perché una data applicazione li riceva, discriminando tra applicazioni in esecuzione sullo }
\put(41.8,-147.211){\fontsize{12}{1}\usefont{T1}{cmr}{m}{n}\selectfont\color{color_29791}stesso host.}
\put(41.8,-177.011){\fontsize{14.1}{1}\usefont{T1}{cmr}{b}{n}\selectfont\color{color_29791}NAT}
\put(41.8,-197.211){\fontsize{12}{1}\usefont{T1}{cmr}{m}{n}\selectfont\color{color_29791}La tecnica NAT (Network Address Translation) permette di usare un solo indirizzo pubblico per }
\put(41.8,-211.011){\fontsize{12}{1}\usefont{T1}{cmr}{m}{n}\selectfont\color{color_29791}realizzare una rete privata anche di grandi dimensioni. La sua efficacia si basa sul fatto che la }
\put(41.8,-224.811){\fontsize{12}{1}\usefont{T1}{cmr}{m}{n}\selectfont\color{color_29791}maggior parte degli host è solo client e non server, quindi originano richieste e devono poter essere }
\put(41.8,-238.611){\fontsize{12}{1}\usefont{T1}{cmr}{m}{n}\selectfont\color{color_29791}raggiunti dalle risposte ma non devono poter essere raggiunti da richieste. }
\put(41.8,-259.411){\fontsize{12}{1}\usefont{T1}{cmr}{m}{n}\selectfont\color{color_29791}Posto un solo indirizzo pubblico fornito dal provider; nella rete locale il router che fa NAT si }
\put(41.8,-273.211){\fontsize{12}{1}\usefont{T1}{cmr}{m}{n}\selectfont\color{color_29791}occupa, per le richieste in uscita, di sostituire indirizzo locale di origine con quello globale fornito }
\put(41.8,-287.011){\fontsize{12}{1}\usefont{T1}{cmr}{m}{n}\selectfont\color{color_29791}dal provider, e per la risposta si ricorderà che deve consegnarla a quella una specifica macchina. }
\put(41.8,-300.811){\fontsize{12}{1}\usefont{T1}{cmr}{m}{n}\selectfont\color{color_29791}Quindi molti IP sorgente vengono sostituiti da unico IP pubblico del router. Per connessioni che }
\put(41.8,-314.611){\fontsize{12}{1}\usefont{T1}{cmr}{m}{n}\selectfont\color{color_29791}differiscono unicamente per l'IP sorgente, il router può anche fare sostituzione della porta sorgente }
\put(41.8,-328.411){\fontsize{12}{1}\usefont{T1}{cmr}{m}{n}\selectfont\color{color_29791}per distinguerle (anche in questo caso memorizza traslazione per riconoscere destinatario risposte).}
\put(41.8,-349.211){\fontsize{12}{1}\usefont{T1}{cmr}{m}{n}\selectfont\color{color_29791}Gli IP della rete risultano completamente nascosti, scegliendoli arbitrariamente si potrebbe violare }
\put(41.8,-363.011){\fontsize{12}{1}\usefont{T1}{cmr}{m}{n}\selectfont\color{color_29791}principio di univocità o tentare di connettersi a sé stessi. Per evitare il problema dell'oscuramento di }
\put(41.8,-376.811){\fontsize{12}{1}\usefont{T1}{cmr}{m}{n}\selectfont\color{color_29791}IP validi, per standard si definiscono tre intervalli di indirizzi che non possono essere usati su }
\put(41.8,-390.611){\fontsize{12}{1}\usefont{T1}{cmr}{m}{n}\selectfont\color{color_29791}Internet:}
\put(59.8,-411.411){\fontsize{12}{1}\usefont{T1}{cmr}{m}{n}\selectfont\color{color_29791}•10.0.0.0/8(da 10.0.0.0 a 10.255.255.255)}
\put(59.8,-432.211){\fontsize{12}{1}\usefont{T1}{cmr}{m}{n}\selectfont\color{color_29791}•172.16.0.0/16 – 172.31.0.0/16(da 172.16.0.0 a 172.31.255.255 = 172.16.0.0/12)}
\put(59.8,-453.011){\fontsize{12}{1}\usefont{T1}{cmr}{m}{n}\selectfont\color{color_29791}•192.168.0.0/24 – 192.168.255.0/24(da 192.168.0.0 a 192.168.255.255 = 192.168.0.0/16)}
\put(41.8,-473.811){\fontsize{12}{1}\usefont{T1}{cmr}{m}{n}\selectfont\color{color_29791}Questi indirizzi privati sono usati solo localmente, dietro il proprio router che fa NAT.}
\put(41.8,-494.611){\fontsize{12}{1}\usefont{T1}{cmr}{m}{n}\selectfont\color{color_217499}Il NAT può fornire una semplice conversione di indirizzo IP (statica o dinamica), le conversioni }
\put(41.8,-508.411){\fontsize{12}{1}\usefont{T1}{cmr}{m}{n}\selectfont\color{color_217499}contemporanee sono limitate dal numero di IP pubblici a destinazione del gateway NAT. Può inoltre}
\put(41.8,-522.211){\fontsize{12}{1}\usefont{T1}{cmr}{m}{n}\selectfont\color{color_217499}fornire conversione di IP + porta TCP o UDP (noto come PAT), in questo caso le conversioni }
\put(41.8,-536.011){\fontsize{12}{1}\usefont{T1}{cmr}{m}{n}\selectfont\color{color_217499}contemporanee sono possibili anche con un unico indirizzo IP pubblico del gateway NAT. }
\put(41.8,-556.811){\fontsize{12}{1}\usefont{T1}{cmr}{m}{n}\selectfont\color{color_217499}Il NAT si applica di solito da rete privata verso rete pubblica (Outbound NAT): si tiene memoria }
\put(41.8,-570.611){\fontsize{12}{1}\usefont{T1}{cmr}{m}{n}\selectfont\color{color_217499}delle connessioni e/o dei flussi di traffico (intrinsecamente stateful}
\end{picture}
\begin{tikzpicture}[overlay]
\path(0pt,0pt);
\draw[color_217499,line width=0.7pt]
(246.5pt, -571.711pt) -- (360.8pt, -571.711pt)
;
\end{tikzpicture}
\begin{picture}(-5,0)(2.5,0)
\put(360.8,-570.611){\fontsize{12}{1}\usefont{T1}{cmr}{m}{n}\selectfont\color{color_217499}), le traduzioni in corso sono }
\put(41.8,-584.411){\fontsize{12}{1}\usefont{T1}{cmr}{m}{n}\selectfont\color{color_217499}messe in cache (traduzioni con tempo di vita limitato), si preoccupa di fare conversione inversa }
\put(41.8,-598.211){\fontsize{12}{1}\usefont{T1}{cmr}{m}{n}\selectfont\color{color_217499}quando arrivano pacchetti in direzione opposta relativi ad un flusso già attivo. }
\put(41.8,-619.011){\fontsize{12}{1}\usefont{T1}{cmr}{m}{n}\selectfont\color{color_217499}È anche possibile contattare dalla rete pubblica un host sulla rete privata (Bi-directional NAT), ma }
\put(41.8,-632.811){\fontsize{12}{1}\usefont{T1}{cmr}{m}{n}\selectfont\color{color_217499}bisogna configurare in modo esplicito il NAT (port-forwarding), e riutilizzare le porte se indirizzi }
\put(41.8,-646.611){\fontsize{12}{1}\usefont{T1}{cmr}{m}{n}\selectfont\color{color_217499}sono limitati.}
\put(41.8,-667.411){\fontsize{12}{1}\usefont{T1}{cmr}{m}{n}\selectfont\color{color_29791}Per usare una rete di client con un solo IP pubblico si modifica l'IP sorgente (SNAT, Source NAT), }
\put(41.8,-681.211){\fontsize{12}{1}\usefont{T1}{cmr}{m}{n}\selectfont\color{color_29791}con il traffico che fluisce attraverso il default gateway in modo trasparente ed automatico. Lo stesso}
\put(41.8,-695.011){\fontsize{12}{1}\usefont{T1}{cmr}{m}{n}\selectfont\color{color_29791}gateway permette di rendere raggiungibili dall'esterno degli host della rete privata modificando }
\put(41.8,-708.811){\fontsize{12}{1}\usefont{T1}{cmr}{m}{n}\selectfont\color{color_29791}l'indirizzo della destinazione, quando riceve richieste su di una specifica porta del proprio IP }
\put(41.8,-722.611){\fontsize{12}{1}\usefont{T1}{cmr}{m}{n}\selectfont\color{color_29791}pubblico (DNAT, Destination NAT): la mappatura tra porta (servizio) e host interno a cui inoltrare }
\put(41.8,-736.411){\fontsize{12}{1}\usefont{T1}{cmr}{m}{n}\selectfont\color{color_29791}la richiesta va configurata in modo esplicito (port forwarding, scelta specifica della porta da usare: }
\put(41.8,-750.211){\fontsize{12}{1}\usefont{T1}{cmr}{m}{n}\selectfont\color{color_29791}si dice al router di sostituire indirizzo di destinazione globale dal suo (del router) con quello di un }
\put(41.8,-764.011){\fontsize{12}{1}\usefont{T1}{cmr}{m}{n}\selectfont\color{color_29791}particolare host interno). }
\end{picture}
\newpage
\begin{tikzpicture}[overlay]\path(0pt,0pt);\end{tikzpicture}
\begin{picture}(-5,0)(2.5,0)
\put(59.8,-105.811){\fontsize{12}{1}\usefont{T1}{cmr}{m}{n}\selectfont\color{color_29791}•}
\end{picture}
\begin{tikzpicture}[overlay]
\path(0pt,0pt);
\draw[color_29791,line width=0.7pt]
(59.8pt, -102.4109pt) -- (64.10001pt, -102.4109pt)
;
\end{tikzpicture}
\begin{picture}(-5,0)(2.5,0)
\put(64,-105.811){\fontsize{12}{1}\usefont{T1}{cmr}{m}{n}\selectfont\color{color_29791}  }
\end{picture}
\begin{tikzpicture}[overlay]
\path(0pt,0pt);
\draw[color_29791,line width=0.7pt]
(64pt, -102.4109pt) -- (77.7pt, -102.4109pt)
;
\end{tikzpicture}
\begin{picture}(-5,0)(2.5,0)
\put(77.8,-105.811){\fontsize{12}{1}\usefont{T1}{cmr}{m}{n}\selectfont\color{color_29791}Vagrant stesso da fa SNAT-Router: }
\end{picture}
\begin{tikzpicture}[overlay]
\path(0pt,0pt);
\draw[color_29791,line width=0.7pt]
(77.8pt, -102.4109pt) -- (248.3pt, -102.4109pt)
;
\end{tikzpicture}
\begin{picture}(-5,0)(2.5,0)
\put(77.8,-126.611){\fontsize{12}{1}\usefont{T1}{cmr}{m}{n}\selectfont\color{color_29791}◦}
\end{picture}
\begin{tikzpicture}[overlay]
\path(0pt,0pt);
\draw[color_29791,line width=0.7pt]
(77.8pt, -123.211pt) -- (87.3pt, -123.211pt)
;
\end{tikzpicture}
\begin{picture}(-5,0)(2.5,0)
\put(87.3,-126.611){\fontsize{12}{1}\usefont{T1}{cmr}{m}{n}\selectfont\color{color_29791} }
\end{picture}
\begin{tikzpicture}[overlay]
\path(0pt,0pt);
\draw[color_29791,line width=0.7pt]
(87.3pt, -123.211pt) -- (93.3pt, -123.211pt)
;
\end{tikzpicture}
\begin{picture}(-5,0)(2.5,0)
\put(95.8,-126.611){\fontsize{12}{1}\usefont{T1}{cmr}{m}{n}\selectfont\color{color_29791}la VM viene configurata per usare come default gateway un'interfaccia virtuale che }
\end{picture}
\begin{tikzpicture}[overlay]
\path(0pt,0pt);
\draw[color_29791,line width=0.7pt]
(95.8pt, -123.211pt) -- (498.2pt, -123.211pt)
;
\end{tikzpicture}
\begin{picture}(-5,0)(2.5,0)
\put(95.8,-140.411){\fontsize{12}{1}\usefont{T1}{cmr}{m}{n}\selectfont\color{color_29791}consegna i pacchetti al processo VirtualBox. Questo li propaga all'host come se li avesse }
\end{picture}
\begin{tikzpicture}[overlay]
\path(0pt,0pt);
\draw[color_29791,line width=0.7pt]
(95.8pt, -137.011pt) -- (523.2pt, -137.011pt)
;
\end{tikzpicture}
\begin{picture}(-5,0)(2.5,0)
\put(95.8,-154.211){\fontsize{12}{1}\usefont{T1}{cmr}{m}{n}\selectfont\color{color_29791}generati lui, e quindi vengono etichettati con IP sorgente dell'host.}
\end{picture}
\begin{tikzpicture}[overlay]
\path(0pt,0pt);
\draw[color_29791,line width=0.7pt]
(95.8pt, -150.811pt) -- (412.8pt, -150.811pt)
;
\end{tikzpicture}
\begin{picture}(-5,0)(2.5,0)
\put(95.8,-175.011){\fontsize{12}{1}\usefont{T1}{cmr}{m}{n}\selectfont\color{color_29791}▪}
\end{picture}
\begin{tikzpicture}[overlay]
\path(0pt,0pt);
\draw[color_29791,line width=0.7pt]
(95.8pt, -171.611pt) -- (104.7pt, -171.611pt)
;
\end{tikzpicture}
\begin{picture}(-5,0)(2.5,0)
\put(104.7,-175.011){\fontsize{12}{1}\usefont{T1}{cmr}{m}{n}\selectfont\color{color_29791} }
\end{picture}
\begin{tikzpicture}[overlay]
\path(0pt,0pt);
\draw[color_29791,line width=0.7pt]
(104.7pt, -171.611pt) -- (110.7pt, -171.611pt)
;
\end{tikzpicture}
\begin{picture}(-5,0)(2.5,0)
\put(113.8,-175.011){\fontsize{12}{1}\usefont{T1}{cmr}{m}{n}\selectfont\color{color_29791}I parametri dell'interfaccia sono assegnati automaticamente, ma si possono }
\end{picture}
\begin{tikzpicture}[overlay]
\path(0pt,0pt);
\draw[color_29791,line width=0.7pt]
(113.8pt, -171.611pt) -- (475.1pt, -171.611pt)
;
\end{tikzpicture}
\begin{picture}(-5,0)(2.5,0)
\put(113.8,-188.811){\fontsize{12}{1}\usefont{T1}{cmr}{m}{n}\selectfont\color{color_29791}configurare con }
\end{picture}
\begin{tikzpicture}[overlay]
\path(0pt,0pt);
\draw[color_29791,line width=0.7pt]
(113.8pt, -185.4109pt) -- (192.4pt, -185.4109pt)
;
\end{tikzpicture}
\begin{picture}(-5,0)(2.5,0)
\put(192.5,-188.811){\fontsize{12}{1}\usefont{T1}{cmr}{m}{it}\selectfont\color{color_29791}config.vm.base\_mac}
\end{picture}
\begin{tikzpicture}[overlay]
\path(0pt,0pt);
\draw[color_29791,line width=0.7pt]
(192.5pt, -185.4109pt) -- (290.4pt, -185.4109pt)
;
\end{tikzpicture}
\begin{picture}(-5,0)(2.5,0)
\put(291.1,-188.811){\fontsize{12}{1}\usefont{T1}{cmr}{m}{n}\selectfont\color{color_29791}e}
\end{picture}
\begin{tikzpicture}[overlay]
\path(0pt,0pt);
\draw[color_29791,line width=0.7pt]
(291.1pt, -185.4109pt) -- (296.4pt, -185.4109pt)
;
\end{tikzpicture}
\begin{picture}(-5,0)(2.5,0)
\put(296.3,-188.811){\fontsize{12}{1}\usefont{T1}{cmr}{m}{n}\selectfont\color{color_29791}          }
\end{picture}
\begin{tikzpicture}[overlay]
\path(0pt,0pt);
\draw[color_29791,line width=0.7pt]
(296.3pt, -185.4109pt) -- (326.4pt, -185.4109pt)
;
\end{tikzpicture}
\begin{picture}(-5,0)(2.5,0)
\put(326.5,-188.811){\fontsize{12}{1}\usefont{T1}{cmr}{m}{it}\selectfont\color{color_29791}config.vm.base\_address}
\end{picture}
\begin{tikzpicture}[overlay]
\path(0pt,0pt);
\draw[color_29791,line width=0.7pt]
(326.5pt, -185.4109pt) -- (441.3pt, -185.4109pt)
;
\end{tikzpicture}
\begin{picture}(-5,0)(2.5,0)
\put(59.8,-209.611){\fontsize{12}{1}\usefont{T1}{cmr}{m}{n}\selectfont\color{color_29791}•}
\end{picture}
\begin{tikzpicture}[overlay]
\path(0pt,0pt);
\draw[color_29791,line width=0.7pt]
(59.8pt, -206.211pt) -- (64.10001pt, -206.211pt)
;
\end{tikzpicture}
\begin{picture}(-5,0)(2.5,0)
\put(64,-209.611){\fontsize{12}{1}\usefont{T1}{cmr}{m}{n}\selectfont\color{color_29791}  }
\end{picture}
\begin{tikzpicture}[overlay]
\path(0pt,0pt);
\draw[color_29791,line width=0.7pt]
(64pt, -206.211pt) -- (77.7pt, -206.211pt)
;
\end{tikzpicture}
\begin{picture}(-5,0)(2.5,0)
\put(77.8,-209.611){\fontsize{12}{1}\usefont{T1}{cmr}{m}{n}\selectfont\color{color_29791}Si può configurare VirtualBox perché si comporti anche da DNAT-Router:}
\end{picture}
\begin{tikzpicture}[overlay]
\path(0pt,0pt);
\draw[color_29791,line width=0.7pt]
(77.8pt, -206.211pt) -- (434.6pt, -206.211pt)
;
\end{tikzpicture}
\begin{picture}(-5,0)(2.5,0)
\put(77.8,-230.411){\fontsize{12}{1}\usefont{T1}{cmr}{m}{n}\selectfont\color{color_29791}◦}
\end{picture}
\begin{tikzpicture}[overlay]
\path(0pt,0pt);
\draw[color_29791,line width=0.7pt]
(77.8pt, -227.011pt) -- (87.3pt, -227.011pt)
;
\end{tikzpicture}
\begin{picture}(-5,0)(2.5,0)
\put(87.3,-230.411){\fontsize{12}{1}\usefont{T1}{cmr}{m}{n}\selectfont\color{color_29791} }
\end{picture}
\begin{tikzpicture}[overlay]
\path(0pt,0pt);
\draw[color_29791,line width=0.7pt]
(87.3pt, -227.011pt) -- (93.3pt, -227.011pt)
;
\end{tikzpicture}
\begin{picture}(-5,0)(2.5,0)
\put(95.8,-230.411){\fontsize{12}{1}\usefont{T1}{cmr}{m}{n}\selectfont\color{color_29791}Il processo Vbox si mette in ascolto su una porta TCP o UDP dell'host, il traffico entrant }
\end{picture}
\begin{tikzpicture}[overlay]
\path(0pt,0pt);
\draw[color_29791,line width=0.7pt]
(95.8pt, -227.011pt) -- (522.3pt, -227.011pt)
;
\end{tikzpicture}
\begin{picture}(-5,0)(2.5,0)
\put(95.8,-244.211){\fontsize{12}{1}\usefont{T1}{cmr}{m}{n}\selectfont\color{color_29791}viene modificato assegnando come destinazione l'IP della scheda virtuale NAT del guest }
\end{picture}
\begin{tikzpicture}[overlay]
\path(0pt,0pt);
\draw[color_29791,line width=0.7pt]
(95.8pt, -240.811pt) -- (522.4pt, -240.811pt)
;
\end{tikzpicture}
\begin{picture}(-5,0)(2.5,0)
\put(95.8,-258.011){\fontsize{12}{1}\usefont{T1}{cmr}{m}{n}\selectfont\color{color_29791}(impostando port-forwarding: vagrant fa questo da porta 2222 dell'host a porta 22 del }
\end{picture}
\begin{tikzpicture}[overlay]
\path(0pt,0pt);
\draw[color_29791,line width=0.7pt]
(95.8pt, -254.611pt) -- (508.2pt, -254.611pt)
;
\end{tikzpicture}
\begin{picture}(-5,0)(2.5,0)
\put(95.8,-271.811){\fontsize{12}{1}\usefont{T1}{cmr}{m}{n}\selectfont\color{color_29791}guest, per permettere ssh).}
\end{picture}
\begin{tikzpicture}[overlay]
\path(0pt,0pt);
\draw[color_29791,line width=0.7pt]
(95.8pt, -268.4109pt) -- (222.1pt, -268.4109pt)
;
\end{tikzpicture}
\begin{picture}(-5,0)(2.5,0)
\put(95.8,-292.611){\fontsize{12}{1}\usefont{T1}{cmr}{m}{n}\selectfont\color{color_29791}▪}
\end{picture}
\begin{tikzpicture}[overlay]
\path(0pt,0pt);
\draw[color_29791,line width=0.7pt]
(95.8pt, -289.211pt) -- (104.7pt, -289.211pt)
;
\end{tikzpicture}
\begin{picture}(-5,0)(2.5,0)
\put(104.7,-292.611){\fontsize{12}{1}\usefont{T1}{cmr}{m}{n}\selectfont\color{color_29791} }
\end{picture}
\begin{tikzpicture}[overlay]
\path(0pt,0pt);
\draw[color_29791,line width=0.7pt]
(104.7pt, -289.211pt) -- (110.7pt, -289.211pt)
;
\end{tikzpicture}
\begin{picture}(-5,0)(2.5,0)
\put(113.8,-292.611){\fontsize{12}{1}\usefont{T1}{cmr}{m}{n}\selectfont\color{color_29791}Le mappature si configurano con, ad esempio }
\end{picture}
\begin{tikzpicture}[overlay]
\path(0pt,0pt);
\draw[color_29791,line width=0.7pt]
(113.8pt, -289.211pt) -- (335.7pt, -289.211pt)
;
\end{tikzpicture}
\begin{picture}(-5,0)(2.5,0)
\put(335.6,-292.611){\fontsize{12}{1}\usefont{T1}{cmr}{m}{n}\selectfont\color{color_29791}        }
\end{picture}
\begin{tikzpicture}[overlay]
\path(0pt,0pt);
\draw[color_29791,line width=0.7pt]
(335.6pt, -289.211pt) -- (361.9pt, -289.211pt)
;
\end{tikzpicture}
\begin{picture}(-5,0)(2.5,0)
\put(362,-292.611){\fontsize{12}{1}\usefont{T1}{cmr}{m}{it}\selectfont\color{color_29791}config.vm.network }
\end{picture}
\begin{tikzpicture}[overlay]
\path(0pt,0pt);
\draw[color_29791,line width=0.7pt]
(362pt, -289.211pt) -- (453.6pt, -289.211pt)
;
\end{tikzpicture}
\begin{picture}(-5,0)(2.5,0)
\put(113.8,-306.411){\fontsize{12}{1}\usefont{T1}{cmr}{m}{it}\selectfont\color{color_29791}"forwarded\_port", guest: 80, host: 8080}
\end{picture}
\begin{tikzpicture}[overlay]
\path(0pt,0pt);
\draw[color_29791,line width=0.7pt]
(113.8pt, -303.011pt) -- (306.7pt, -303.011pt)
;
\end{tikzpicture}
\begin{picture}(-5,0)(2.5,0)
\put(95.8,-327.211){\fontsize{12}{1}\usefont{T1}{cmr}{m}{n}\selectfont\color{color_29791}▪}
\end{picture}
\begin{tikzpicture}[overlay]
\path(0pt,0pt);
\draw[color_29791,line width=0.7pt]
(95.8pt, -323.811pt) -- (104.7pt, -323.811pt)
;
\end{tikzpicture}
\begin{picture}(-5,0)(2.5,0)
\put(104.7,-327.211){\fontsize{12}{1}\usefont{T1}{cmr}{m}{n}\selectfont\color{color_29791} }
\end{picture}
\begin{tikzpicture}[overlay]
\path(0pt,0pt);
\draw[color_29791,line width=0.7pt]
(104.7pt, -323.811pt) -- (110.7pt, -323.811pt)
;
\end{tikzpicture}
\begin{picture}(-5,0)(2.5,0)
\put(113.8,-327.211){\fontsize{12}{1}\usefont{T1}{cmr}{m}{n}\selectfont\color{color_29791}il parametro opzionale }
\end{picture}
\begin{tikzpicture}[overlay]
\path(0pt,0pt);
\draw[color_29791,line width=0.7pt]
(113.8pt, -323.811pt) -- (224.7pt, -323.811pt)
;
\end{tikzpicture}
\begin{picture}(-5,0)(2.5,0)
\put(224.8,-327.211){\fontsize{12}{1}\usefont{T1}{cmr}{m}{it}\selectfont\color{color_29791}host\_ip }
\end{picture}
\begin{tikzpicture}[overlay]
\path(0pt,0pt);
\draw[color_29791,line width=0.7pt]
(224.8pt, -323.811pt) -- (263.1pt, -323.811pt)
;
\end{tikzpicture}
\begin{picture}(-5,0)(2.5,0)
\put(263.1,-327.211){\fontsize{12}{1}\usefont{T1}{cmr}{m}{n}\selectfont\color{color_29791}può essere usato per limitare raggiungibilità della }
\end{picture}
\begin{tikzpicture}[overlay]
\path(0pt,0pt);
\draw[color_29791,line width=0.7pt]
(263.1pt, -323.811pt) -- (503.3pt, -323.811pt)
;
\end{tikzpicture}
\begin{picture}(-5,0)(2.5,0)
\put(113.8,-341.011){\fontsize{12}{1}\usefont{T1}{cmr}{m}{n}\selectfont\color{color_29791}porta, es.}
\end{picture}
\begin{tikzpicture}[overlay]
\path(0pt,0pt);
\draw[color_29791,line width=0.7pt]
(113.8pt, -337.611pt) -- (157.4pt, -337.611pt)
;
\end{tikzpicture}
\begin{picture}(-5,0)(2.5,0)
\put(157.4,-341.011){\fontsize{12}{1}\usefont{T1}{cmr}{m}{n}\selectfont\color{color_29791}         }
\end{picture}
\begin{tikzpicture}[overlay]
\path(0pt,0pt);
\draw[color_29791,line width=0.7pt]
(157.4pt, -337.611pt) -- (184.7pt, -337.611pt)
;
\end{tikzpicture}
\begin{picture}(-5,0)(2.5,0)
\put(184.7,-341.011){\fontsize{12}{1}\usefont{T1}{cmr}{m}{it}\selectfont\color{color_29791}host\_ip: "127.0.0.1"}
\end{picture}
\begin{tikzpicture}[overlay]
\path(0pt,0pt);
\draw[color_29791,line width=0.7pt]
(184.7pt, -337.611pt) -- (282.1pt, -337.611pt)
;
\end{tikzpicture}
\begin{picture}(-5,0)(2.5,0)
\put(41.8,-370.811){\fontsize{14.1}{1}\usefont{T1}{cmr}{b}{n}\selectfont\color{color_29791}Configurazione}
\put(41.8,-391.011){\fontsize{12}{1}\usefont{T1}{cmr}{m}{n}\selectfont\color{color_29791}La configurazione di un'interfaccia di rete richiede come minimo IP e netmask; in più se la rete è }
\put(41.8,-404.811){\fontsize{12}{1}\usefont{T1}{cmr}{m}{n}\selectfont\color{color_29791}connessa ad altre, gateway specifici e default gateway. Se è disponibile un sistema di risoluzione dei}
\put(41.8,-418.611){\fontsize{12}{1}\usefont{T1}{cmr}{m}{n}\selectfont\color{color_29791}nomi, indirizzi dei server DNS e domini di default per costruire i FQDN.}
\put(41.8,-439.411){\fontsize{12}{1}\usefont{T1}{cmr}{m}{n}\selectfont\color{color_29791}Queste informazioni possono essere assegnate manualmente, da un server DHCP o localmente in }
\put(41.8,-453.211){\fontsize{12}{1}\usefont{T1}{cmr}{m}{n}\selectfont\color{color_29791}modo automatico. Ancora una volta ci sono sia comandi per la modifica istantanea (runtime) della }
\put(41.8,-467.011){\fontsize{12}{1}\usefont{T1}{cmr}{m}{n}\selectfont\color{color_29791}configurazione, sia altri per modifiche persistenti da applicare all'avvio. }
\put(41.8,-487.811){\fontsize{12}{1}\usefont{T1}{cmr}{m}{n}\selectfont\color{color_29791}L'approccio classico prevede editing di file di testo, seguendo lo standard di qualsiasi altro servizio. }
\put(41.8,-501.611){\fontsize{12}{1}\usefont{T1}{cmr}{m}{n}\selectfont\color{color_29791}Le modifiche sono persistenti, ma è necessario che venga preso in considerazione il file (riavvio }
\put(41.8,-515.411){\fontsize{12}{1}\usefont{T1}{cmr}{m}{n}\selectfont\color{color_29791}networking con systemctl restart networking o equivalenti). In molte distro c'è NetworkManager, }
\put(41.8,-529.211){\fontsize{12}{1}\usefont{T1}{cmr}{m}{n}\selectfont\color{color_29791}che automatizza la rilevazione di nuove interfacce. Si può pilotare da GUI o con nmcli, e può }
\put(41.8,-543.011){\fontsize{12}{1}\usefont{T1}{cmr}{m}{n}\selectfont\color{color_29791}sovrascrivere le modifiche runtime in qualsiasi momento, avendo un approccio dinamico. Il sistema}
\put(41.8,-556.811){\fontsize{12}{1}\usefont{T1}{cmr}{m}{n}\selectfont\color{color_29791}di networking ha un approccio più semplice, tradizionale, con meno feature avanzate, ma ha grande }
\put(41.8,-570.611){\fontsize{12}{1}\usefont{T1}{cmr}{m}{n}\selectfont\color{color_29791}pregio del determinismo: legge il file di configurazione e finché non si cambia il file e si riavvia, }
\put(41.8,-584.411){\fontsize{12}{1}\usefont{T1}{cmr}{m}{n}\selectfont\color{color_29791}non cambia niente. Invece networkmanager viene risvegliato da eventi (es. rilevazione nuove }
\put(41.8,-598.211){\fontsize{12}{1}\usefont{T1}{cmr}{m}{n}\selectfont\color{color_29791}schede di rete) e dinamicamente applica modifiche: importante sapere bene dove sono i file di }
\put(41.8,-612.011){\fontsize{12}{1}\usefont{T1}{cmr}{m}{n}\selectfont\color{color_29791}configurazione per poter intervenire run-time.}
\put(59.8,-632.811){\fontsize{12}{1}\usefont{T1}{cmr}{m}{n}\selectfont\color{color_29791}•Configurazione runtime}
\put(77.8,-653.611){\fontsize{12}{1}\usefont{T1}{cmr}{m}{n}\selectfont\color{color_29791}◦suite iproute2 formata da comando ip e sottocomandi, permette controllo completo di }
\put(95.8,-667.411){\fontsize{12}{1}\usefont{T1}{cmr}{m}{n}\selectfont\color{color_29791}tutti gli aspetti più avanzati (link-layer, interfacce virtuali, tunnel…).}
\put(95.8,-688.211){\fontsize{12}{1}\usefont{T1}{cmr}{m}{n}\selectfont\color{color_29791}▪sottocomando address (a) manipola interfacce:}
\put(113.8,-709.011){\fontsize{12}{1}\usefont{T1}{cmr}{m}{n}\selectfont\color{color_29791}•visualizzazioneip a}
\put(113.8,-729.811){\fontsize{12}{1}\usefont{T1}{cmr}{m}{n}\selectfont\color{color_29791}•assegnazione ip a add <address>/<mask> dev <interface>}
\put(113.8,-750.611){\fontsize{12}{1}\usefont{T1}{cmr}{m}{n}\selectfont\color{color_29791}•rimozioneip a del <address>/<mask> dev <interface>}
\put(95.8,-771.411){\fontsize{12}{1}\usefont{T1}{cmr}{m}{n}\selectfont\color{color_29791}▪sottocomando route (r) manipola tabelle di indirizzamento:}
\end{picture}
\newpage
\begin{tikzpicture}[overlay]\path(0pt,0pt);\end{tikzpicture}
\begin{picture}(-5,0)(2.5,0)
\put(113.8,-85.01099){\fontsize{12}{1}\usefont{T1}{cmr}{m}{n}\selectfont\color{color_29791}•visualizzazioneip r}
\put(113.8,-105.811){\fontsize{12}{1}\usefont{T1}{cmr}{m}{n}\selectfont\color{color_29791}•routing via gatewayip r add <dst\_net>/<mask> via <gw\_addr>(passa }
\put(131.8,-119.611){\fontsize{12}{1}\usefont{T1}{cmr}{m}{n}\selectfont\color{color_29791}attraverso un router)}
\put(113.8,-140.411){\fontsize{12}{1}\usefont{T1}{cmr}{m}{n}\selectfont\color{color_29791}•routing via devip r add <dst\_net>/<mask> dev <interface>(passa }
\put(131.8,-154.211){\fontsize{12}{1}\usefont{T1}{cmr}{m}{n}\selectfont\color{color_29791}attraverso interfaccia fisica (device))}
\put(113.8,-175.011){\fontsize{12}{1}\usefont{T1}{cmr}{m}{n}\selectfont\color{color_29791}•rimozioneip a del <address>/<mask>(is this a correct?)}
\put(59.8,-195.811){\fontsize{12}{1}\usefont{T1}{cmr}{m}{n}\selectfont\color{color_29791}•Configurazione persistente (classica, DEBIAN)}
\put(77.8,-216.611){\fontsize{12}{1}\usefont{T1}{cmr}{m}{n}\selectfont\color{color_29791}◦file /etc/network/interfacesvedi man 5 interfaces}
\put(77.8,-237.411){\fontsize{12}{1}\usefont{T1}{cmr}{m}{n}\selectfont\color{color_29791}◦snippet nella cartella/etc/network/interfaces.d/}
\put(77.8,-258.211){\fontsize{12}{1}\usefont{T1}{cmr}{m}{n}\selectfont\color{color_29791}◦Esempio:auto eth0\# attiva con ifup -a}
\put(167.8,-279.011){\fontsize{12}{1}\usefont{T1}{cmr}{m}{it}\selectfont\color{color_29791}iface eth0 inet static\#con dhcp al posto di static, non serve altro}
\put(167.8,-299.811){\fontsize{12}{1}\usefont{T1}{cmr}{m}{it}\selectfont\color{color_29791}address 192.168.56.203}
\put(167.8,-320.611){\fontsize{12}{1}\usefont{T1}{cmr}{m}{it}\selectfont\color{color_29791}netmask 255.255.255.0\# se omesso, class-based ip}
\put(112.7,-341.411){\fontsize{12}{1}\usefont{T1}{cmr}{m}{n}\selectfont\color{color_29791}Opzionalmente}
\put(167.8,-362.211){\fontsize{12}{1}\usefont{T1}{cmr}{m}{it}\selectfont\color{color_29791}gateway 192.168.56.1\# uno solo, non per interfaccia}
\put(167.8,-383.011){\fontsize{12}{1}\usefont{T1}{cmr}{m}{it}\selectfont\color{color_29791}up /path/to/command arguments\# eseguito dopo configurazione}
\put(41.8,-403.811){\fontsize{12}{1}\usefont{T1}{cmr}{m}{n}\selectfont\color{color_217499}Abilitazione packet forwarding: il kernel non inoltra i pacchetti a meno che non impostiamo il flag }
\put(41.8,-417.611){\fontsize{12}{1}\usefont{T1}{cmr}{m}{n}\selectfont\color{color_217499}per dirgli di farlo.}
\put(59.8,-438.411){\fontsize{12}{1}\usefont{T1}{cmr}{m}{n}\selectfont\color{color_29791}•nel file /etc/sysctl.conf , decommentando la riga \#net.ipv4.ip\_forward=1e poi }
\put(77.8,-452.211){\fontsize{12}{1}\usefont{T1}{cmr}{m}{n}\selectfont\color{color_217499}dando sysctl -p per applicare le modifiche}
\put(77.8,-473.011){\fontsize{12}{1}\usefont{T1}{cmr}{m}{n}\selectfont\color{color_217499}◦Commentare nuovamente l'opzione non resetta il valore, in quanto semplicemente si }
\put(95.8,-486.811){\fontsize{12}{1}\usefont{T1}{cmr}{m}{n}\selectfont\color{color_217499}copre l'istruzione. Se si vuole riportarlo a 0, è necessario farlo esplicitamente}
\put(59.8,-507.611){\fontsize{12}{1}\usefont{T1}{cmr}{m}{n}\selectfont\color{color_29791}•Configurazione instantanea (non persistente) ??? ??? sysctl -w net.ipv4.ip\_forward=1}
\put(41.8,-528.411){\fontsize{12}{1}\usefont{T1}{cmr}{b}{n}\selectfont\color{color_29791}Tool per il monitoraggio:useremo i primi due tool per verificare la raggiungibilità delle reti che }
\put(41.8,-542.211){\fontsize{12}{1}\usefont{T1}{cmr}{m}{n}\selectfont\color{color_29791}configuriamo}
\put(59.8,-563.011){\fontsize{12}{1}\usefont{T1}{cmr}{m}{n}\selectfont\color{color_29791}•ping <IP>verifica base della connettività}
\put(59.8,-583.811){\fontsize{12}{1}\usefont{T1}{cmr}{m}{n}\selectfont\color{color_29791}•traceroute <IP>verifica del percorso dei pacchetti}
\put(59.8,-604.611){\fontsize{12}{1}\usefont{T1}{cmr}{m}{n}\selectfont\color{color_29791}•ssverifica stato delle connessioni (socket status)}
\put(77.8,-625.411){\fontsize{12}{1}\usefont{T1}{cmr}{m}{n}\selectfont\color{color_29791}◦-t / -u TCP/UDP only}
\put(77.8,-646.211){\fontsize{12}{1}\usefont{T1}{cmr}{m}{n}\selectfont\color{color_29791}◦-l / -astato LISTEN (default è ESTABLISHED)/ALL}
\put(77.8,-667.011){\fontsize{12}{1}\usefont{T1}{cmr}{m}{n}\selectfont\color{color_29791}◦-nnon risolve indirizzi/porte in nomi simbolici}
\put(77.8,-687.811){\fontsize{12}{1}\usefont{T1}{cmr}{m}{n}\selectfont\color{color_29791}◦-pmostra processi che usano la socket}
\put(77.8,-708.611){\fontsize{12}{1}\usefont{T1}{cmr}{m}{n}\selectfont\color{color_217499}◦-emostra info dettagliate sulle socket}
\put(59.8,-729.411){\fontsize{12}{1}\usefont{T1}{cmr}{m}{n}\selectfont\color{color_29791}•tcpdump e wireshark per intercettare contenuto dei pacchetti: lo stato delle socket è }
\put(77.8,-743.211){\fontsize{12}{1}\usefont{T1}{cmr}{m}{n}\selectfont\color{color_29791}su ogni endpoint, con questi due invece possiamo vedere proprio i pacchetti in transito.}
\put(77.8,-764.011){\fontsize{12}{1}\usefont{T1}{cmr}{m}{n}\selectfont\color{color_29791}◦tcpdump tcpdump -nvl }
\end{picture}
\newpage
\begin{tikzpicture}[overlay]\path(0pt,0pt);\end{tikzpicture}
\begin{picture}(-5,0)(2.5,0)
\put(95.8,-85.01099){\fontsize{12}{1}\usefont{T1}{cmr}{m}{n}\selectfont\color{color_29791}▪-n output numerico, non converte indirizzi in nomi}
\put(95.8,-105.811){\fontsize{12}{1}\usefont{T1}{cmr}{m}{n}\selectfont\color{color_29791}▪-v output verboso, meglio toglierlo quando si mette in uno script ma utile}
\put(113.8,-119.611){\fontsize{12}{1}\usefont{T1}{cmr}{m}{n}\selectfont\color{color_29791}per uso interattivo}
\put(95.8,-140.411){\fontsize{12}{1}\usefont{T1}{cmr}{m}{n}\selectfont\color{color_29791}▪-loutput line-buffered: fa output una linea alla volta, molto importante }
\put(113.8,-154.211){\fontsize{12}{1}\usefont{T1}{cmr}{m}{n}\selectfont\color{color_29791}metterlo}
\put(95.8,-175.011){\fontsize{12}{1}\usefont{T1}{cmr}{m}{n}\selectfont\color{color_29791}▪-c NUMesce appena ha ricevuto NUM pacchetti}
\put(95.8,-195.811){\fontsize{12}{1}\usefont{T1}{cmr}{m}{n}\selectfont\color{color_29791}▪-i [any|INAME] specifica su quale interfaccia ascoltare [tutte oppure INAME]}
\put(95.8,-216.611){\fontsize{12}{1}\usefont{T1}{cmr}{m}{n}\selectfont\color{color_29791}▪udpspecifica il protocollo}
\put(95.8,-237.411){\fontsize{12}{1}\usefont{T1}{cmr}{m}{n}\selectfont\color{color_29791}▪-w dumpa i pacchetti così come sono su un file invece di parsarli e }
\put(113.8,-251.211){\fontsize{12}{1}\usefont{T1}{cmr}{m}{n}\selectfont\color{color_29791}stamparli su stdout (utile per creare file PCAP)}
\put(95.8,-272.011){\fontsize{12}{1}\usefont{T1}{cmr}{m}{n}\selectfont\color{color_29791}▪poi con "and" concatena i filtri: (sicuro? lezione del 24-05 contiene riferimenti a }
\put(113.8,-285.811){\fontsize{12}{1}\usefont{T1}{cmr}{m}{n}\selectfont\color{color_29791}tcpdump) }
\put(95.8,-306.611){\fontsize{12}{1}\usefont{T1}{cmr}{m}{n}\selectfont\color{color_29791}▪con "src" imposta sorgente dei pacchetti (e con "net"? intende tutta una rete }
\put(113.8,-320.411){\fontsize{12}{1}\usefont{T1}{cmr}{m}{n}\selectfont\color{color_29791}[mettendo quindi anche netmask], e non un singolo host?)}
\put(95.8,-341.211){\fontsize{12}{1}\usefont{T1}{cmr}{m}{n}\selectfont\color{color_29791}▪con "dst" imposta destinazione dei pacchetti (con "net"? stesso dubbio di sopra. con }
\put(113.8,-355.011){\fontsize{12}{1}\usefont{T1}{cmr}{m}{n}\selectfont\color{color_29791}"port" specifica porta destinazione, vale anche per source volendo)}
\put(95.8,-375.811){\fontsize{12}{1}\usefont{T1}{cmr}{m}{n}\selectfont\color{color_29791}▪sudo tcpdump -nv -c 10 -i any udp and src net 127.0.0.0/24 and dst net 127.0.0.0/24 }
\put(113.8,-389.611){\fontsize{12}{1}\usefont{T1}{cmr}{m}{n}\selectfont\color{color_29791}and dst port 514}
\put(41.8,-425.611){\fontsize{17.5}{1}\usefont{T1}{cmr}{b}{n}\selectfont\color{color_29791}Servizi di rete di base}
\put(41.8,-455.511){\fontsize{14.1}{1}\usefont{T1}{cmr}{b}{n}\selectfont\color{color_29791}NSS}
\put(41.8,-475.711){\fontsize{12}{1}\usefont{T1}{cmr}{m}{n}\selectfont\color{color_29791}La risoluzione dei nomi di host ad indirizzi IP e viceversa è uno dei vari casi in cui il sistema }
\put(41.8,-489.511){\fontsize{12}{1}\usefont{T1}{cmr}{m}{n}\selectfont\color{color_29791}necessita di un dizionario di nomi. In GNU/Linux primo strato di risoluzione è NSS, che si occupa }
\put(41.8,-503.311){\fontsize{12}{1}\usefont{T1}{cmr}{m}{n}\selectfont\color{color_29791}della scelta della sorgente di informazioni. NSS (Name Server Switch) è parte della libreria C di }
\put(41.8,-517.111){\fontsize{12}{1}\usefont{T1}{cmr}{m}{n}\selectfont\color{color_29791}sistema, e supporta un set fisso di possibili DB; le categorie di nomi sono configurabili tramite il }
\put(41.8,-530.911){\fontsize{12}{1}\usefont{T1}{cmr}{m}{n}\selectfont\color{color_29791}file /etc/nsswitch.conf}
\put(41.8,-551.711){\fontsize{12}{1}\usefont{T1}{cmr}{m}{n}\selectfont\color{color_29791}Il contenuto del file segue la sintassi <entry>::= <database>":"[<source> [<criteria> ]]*}
\put(148.2,-572.511){\fontsize{12}{1}\usefont{T1}{cmr}{m}{it}\selectfont\color{color_29791}<criteria>::="[" <criterion> + "]"}
\put(148.2,-593.311){\fontsize{12}{1}\usefont{T1}{cmr}{m}{it}\selectfont\color{color_29791}<criterion>::=<status> "= " <action>}
\put(148.2,-614.111){\fontsize{12}{1}\usefont{T1}{cmr}{m}{it}\selectfont\color{color_29791}<status>::= "success" | "notfound" | "unavail" | "tryagain"}
\put(148.2,-634.911){\fontsize{12}{1}\usefont{T1}{cmr}{m}{it}\selectfont\color{color_29791}<action> ::= "return" | "continue"}
\put(41.8,-655.711){\fontsize{12}{1}\usefont{T1}{cmr}{m}{n}\selectfont\color{color_217499}A default, success (risposta ricevuta) ha come action return (ritorna un risultato ora), mentre }
\put(41.8,-669.511){\fontsize{12}{1}\usefont{T1}{cmr}{m}{n}\selectfont\color{color_217499}notfound (sorgente esiste ma non sa rispondere), unavail (sorgente è irragiungibile) e tryagain }
\put(41.8,-683.311){\fontsize{12}{1}\usefont{T1}{cmr}{m}{n}\selectfont\color{color_217499}(sorgente esiste ma è occupata) hanno continue (chiama la prossima funzione di lookup)}
\put(41.8,-704.111){\fontsize{12}{1}\usefont{T1}{cmr}{m}{n}\selectfont\color{color_29791}Esempi di entry:passwd: files nis ldap}
\put(148.2,-724.911){\fontsize{12}{1}\usefont{T1}{cmr}{m}{it}\selectfont\color{color_29791}group: files ldap}
\put(41.8,-745.711){\fontsize{12}{1}\usefont{T1}{cmr}{m}{n}\selectfont\color{color_29791}Esempio su record hosts:ldap [NOTFOUND=return] dns files}
\end{picture}
\newpage
\begin{tikzpicture}[overlay]\path(0pt,0pt);\end{tikzpicture}
\begin{picture}(-5,0)(2.5,0)
\put(41.8,-85.01099){\fontsize{12}{1}\usefont{T1}{cmr}{m}{n}\selectfont\color{color_29791}Il database hosts è quello dei nomi di host (siamo abituati a chiamarlo DNS ma è un'astrazione }
\put(41.8,-98.81097){\fontsize{12}{1}\usefont{T1}{cmr}{m}{n}\selectfont\color{color_29791}maggiore). Qui c'è scritto che per risolvere gli host, si va nella sorgente ldap, che può fare 4 cose: }
\put(59.8,-119.611){\fontsize{12}{1}\usefont{T1}{cmr}{m}{n}\selectfont\color{color_29791}•success,rispondere che lo trova, e la catena di chiamate torna all'origine col risultato}
\put(59.8,-140.411){\fontsize{12}{1}\usefont{T1}{cmr}{m}{n}\selectfont\color{color_29791}•unavail, risultato preliminare che vuol dire che ldap non riesce a rispondere: si passa a}
\put(77.8,-154.211){\fontsize{12}{1}\usefont{T1}{cmr}{m}{n}\selectfont\color{color_29791}controllare dns}
\put(59.8,-175.011){\fontsize{12}{1}\usefont{T1}{cmr}{m}{n}\selectfont\color{color_29791}•tryagain, riprovare più tardi}
\put(59.8,-195.811){\fontsize{12}{1}\usefont{T1}{cmr}{m}{n}\selectfont\color{color_29791}•not found,per com'è scritta la regola, se ldap dà not found, c'è il return e si ferma la }
\put(77.8,-209.611){\fontsize{12}{1}\usefont{T1}{cmr}{m}{n}\selectfont\color{color_29791}ricerca; quindi si dà massima autorità a ldap stesso.}
\put(41.8,-230.411){\fontsize{12}{1}\usefont{T1}{cmr}{m}{n}\selectfont\color{color_29791}Se <source> è files , si indica che NSS va a cercare un file locale con lo stesso nome della}
\put(41.8,-244.211){\fontsize{12}{1}\usefont{T1}{cmr}{m}{n}\selectfont\color{color_29791}entry. Per questo record di esempio, quindi, la sorgente di informazioni sarà il file/etc/hosts}
\put(112.7,-265.011){\fontsize{12}{1}\usefont{T1}{cmr}{m}{n}\selectfont\color{color_29791}/etc/hosts ha formato  <IP><FQDN>[<ALIAS>…]fully }
\put(41.8,-278.811){\fontsize{12}{1}\usefont{T1}{cmr}{m}{n}\selectfont\color{color_29791}qualified, con dominio per intero}
\put(112.7,-299.611){\fontsize{12}{1}\usefont{T1}{cmr}{m}{n}\selectfont\color{color_29791}esempio:8.8.8.8dns.google.comgnds}
\put(41.8,-320.411){\fontsize{12}{1}\usefont{T1}{cmr}{m}{n}\selectfont\color{color_29791}Altrimenti se <source> è dns , la sorgente di informazione è il DNS. L'interrogazione a }
\put(41.8,-334.211){\fontsize{12}{1}\usefont{T1}{cmr}{m}{n}\selectfont\color{color_29791}questo è svolta da un altro set di funzioni della libsys C, il resolver. Questo si configura attraverso }
\put(41.8,-348.011){\fontsize{12}{1}\usefont{T1}{cmr}{m}{n}\selectfont\color{color_29791}il file/etc/resolv.confrecord di esempionameserver137.204.58.1}
\put(290,-368.811){\fontsize{12}{1}\usefont{T1}{cmr}{m}{it}\selectfont\color{color_29791}domaindisi.unibo.it}
\put(290,-389.611){\fontsize{12}{1}\usefont{T1}{cmr}{m}{it}\selectfont\color{color_29791}searching.unibo.it}
\put(41.8,-410.411){\fontsize{12}{1}\usefont{T1}{cmr}{m}{n}\selectfont\color{color_29791}Per il principio di caching multilivello (per migliorare le prestazioni) del DNS, avremo che il primo }
\put(41.8,-424.211){\fontsize{12}{1}\usefont{T1}{cmr}{m}{n}\selectfont\color{color_29791}livello di cache è locale alla macchina: questa userà sé stessa come name resolver mediante un }
\put(41.8,-438.011){\fontsize{12}{1}\usefont{T1}{cmr}{m}{n}\selectfont\color{color_29791}server DNS locale. Guardando nel file fondamentale lato client  ( /etc/resolv.conf ) vediamo che è }
\put(41.8,-451.811){\fontsize{12}{1}\usefont{T1}{cmr}{m}{n}\selectfont\color{color_29791}indicato nameserver 127.X.X.Xpoiché tutti gli IP che iniziano per 127 puntano a localhost. }
\put(41.8,-472.611){\fontsize{12}{1}\usefont{T1}{cmr}{m}{n}\selectfont\color{color_29791}Il DNS è un protocollo C/S basato su UDP che risponde sulla porta 53, potremo verificare la }
\put(41.8,-486.411){\fontsize{12}{1}\usefont{T1}{cmr}{m}{n}\selectfont\color{color_29791}presenza del servizio consudo ss -naup | grep 127.X.X.X:53, il comando mostra che c'è una }
\put(41.8,-500.211){\fontsize{12}{1}\usefont{T1}{cmr}{m}{n}\selectfont\color{color_29791}socket unconnected (udp) legata alla porta 53, su cui gira dnsmasq pronto a ricevere.}
\put(41.8,-530.011){\fontsize{14.1}{1}\usefont{T1}{cmr}{b}{n}\selectfont\color{color_29791}Risoluzione nomi via NSS}
\put(41.8,-550.211){\fontsize{12}{1}\usefont{T1}{cmr}{m}{n}\selectfont\color{color_29791}Comando getent permette di interrogare i database del NSS: è il metodo per passare correttamente }
\put(41.8,-564.011){\fontsize{12}{1}\usefont{T1}{cmr}{m}{n}\selectfont\color{color_29791}attraverso lo strato di astrazione fornito da NSS, che si occupa di scegliere la fonte delle }
\put(41.8,-577.811){\fontsize{12}{1}\usefont{T1}{cmr}{m}{n}\selectfont\color{color_29791}informazioni restituite. Sintassi:}
\put(77.3,-598.611){\fontsize{12}{1}\usefont{T1}{cmr}{b}{it}\selectfont\color{color_29791}getent <dbname> <keyword>nome del db da interrogare e keyword su cui }
\put(41.8,-612.411){\fontsize{12}{1}\usefont{T1}{cmr}{m}{n}\selectfont\color{color_29791}fare la ricerca}
\put(41.8,-633.211){\fontsize{12}{1}\usefont{T1}{cmr}{m}{n}\selectfont\color{color_29791}Esempi:getent passwd lasse voglio vedere cosa c'è su etc/passwd, getent fa lookup di }
\put(41.8,-647.011){\fontsize{12}{1}\usefont{T1}{cmr}{m}{n}\selectfont\color{color_29791}fonte dati "passwd" alla ricerca di una entry che abbia come keyword "las", passando da NSS per }
\put(41.8,-660.811){\fontsize{12}{1}\usefont{T1}{cmr}{m}{n}\selectfont\color{color_29791}selezionare la fonte più adatta.}
\put(41.8,-690.611){\fontsize{14.1}{1}\usefont{T1}{cmr}{b}{n}\selectfont\color{color_29791}Risoluzione nomi DNS diretta}
\put(41.8,-710.811){\fontsize{12}{1}\usefont{T1}{cmr}{m}{n}\selectfont\color{color_29791}Per interrogare direttamente il DNS e avere più controllo sulle query, si usano host e dig, che non }
\put(41.8,-724.611){\fontsize{12}{1}\usefont{T1}{cmr}{m}{n}\selectfont\color{color_29791}considerano nsswitch e usano i nameserver di resolv.conf di default, ma possono interrogare un }
\put(41.8,-738.411){\fontsize{12}{1}\usefont{T1}{cmr}{m}{n}\selectfont\color{color_29791}server specifico. }
\put(59.8,-759.211){\fontsize{12}{1}\usefont{T1}{cmr}{m}{n}\selectfont\color{color_29791}•hostsi usa tipicamente per conversioni IP ← → nome}
\end{picture}
\newpage
\begin{tikzpicture}[overlay]\path(0pt,0pt);\end{tikzpicture}
\begin{picture}(-5,0)(2.5,0)
\put(77.8,-85.01099){\fontsize{12}{1}\usefont{T1}{cmr}{m}{n}\selectfont\color{color_29791}◦query di un nome:host www.unibo.it}
\end{picture}
\begin{tikzpicture}[overlay]
\path(0pt,0pt);
\draw[color_29919,line width=0.7pt]
(225.2pt, -86.11096pt) -- (288.3pt, -86.11096pt)
;
\end{tikzpicture}
\begin{picture}(-5,0)(2.5,0)
\put(308.5,-85.01099){\fontsize{12}{1}\usefont{T1}{cmr}{m}{n}\selectfont\color{color_29791}interroga il server indicato in }
\put(95.8,-98.81097){\fontsize{12}{1}\usefont{T1}{cmr}{m}{n}\selectfont\color{color_29791}/etc/resolv.conf e restituisce risultato, se lo trova}
\put(77.8,-119.611){\fontsize{12}{1}\usefont{T1}{cmr}{m}{n}\selectfont\color{color_29791}◦query di server specifico:host www.unibo.it}
\end{picture}
\begin{tikzpicture}[overlay]
\path(0pt,0pt);
\draw[color_29919,line width=0.7pt]
(260.6pt, -120.711pt) -- (323.7pt, -120.711pt)
;
\end{tikzpicture}
\begin{picture}(-5,0)(2.5,0)
\put(323.7,-119.611){\fontsize{12}{1}\usefont{T1}{cmr}{m}{it}\selectfont\color{color_29791} 8.8.8.8così salta il file resolv.conf e }
\put(95.8,-133.411){\fontsize{12}{1}\usefont{T1}{cmr}{m}{n}\selectfont\color{color_29791}va a chiedere direttamente al name server indicato (8.8.8.8)(citazione dei glue }
\put(95.8,-147.211){\fontsize{12}{1}\usefont{T1}{cmr}{m}{n}\selectfont\color{color_29791}record - da ricordare per quando si diventa matti sulla risoluzione DNS, da sistemisti)}
\put(59.8,-168.011){\fontsize{12}{1}\usefont{T1}{cmr}{m}{n}\selectfont\color{color_29791}•digper ottenere informazioni legate a un dominio, diverse da nomi host: i record ns }
\put(77.8,-181.811){\fontsize{12}{1}\usefont{T1}{cmr}{m}{n}\selectfont\color{color_29791}dicono qual è il name server di un dominio, i record mx quali sono i mail exchanger.}
\put(77.8,-202.611){\fontsize{12}{1}\usefont{T1}{cmr}{m}{n}\selectfont\color{color_29791}◦conoscere i Mail eXchanger:dig mx example.com}
\put(77.8,-223.411){\fontsize{12}{1}\usefont{T1}{cmr}{m}{n}\selectfont\color{color_29791}◦conoscere i Name Server:dig ns example.com}
\put(41.8,-253.211){\fontsize{14.1}{1}\usefont{T1}{cmr}{b}{n}\selectfont\color{color_29791}Zeroconf}
\end{picture}
\begin{tikzpicture}[overlay]
\path(0pt,0pt);
\draw[color_29791,line width=0.8pt]
(41.8pt, -249.111pt) -- (102.1pt, -249.111pt)
;
\end{tikzpicture}
\begin{picture}(-5,0)(2.5,0)
\put(41.8,-273.411){\fontsize{12}{1}\usefont{T1}{cmr}{m}{n}\selectfont\color{color_29791}Se DNS pemette di associare nomi e indirizzi, ed è quindi un database compilabile, a mano ma }
\end{picture}
\begin{tikzpicture}[overlay]
\path(0pt,0pt);
\draw[color_29791,line width=0.7pt]
(41.8pt, -270.011pt) -- (499.9pt, -270.011pt)
;
\end{tikzpicture}
\begin{picture}(-5,0)(2.5,0)
\put(41.8,-287.211){\fontsize{12}{1}\usefont{T1}{cmr}{m}{n}\selectfont\color{color_29791}anche in maniera automatica; lo }
\end{picture}
\begin{tikzpicture}[overlay]
\path(0pt,0pt);
\draw[color_29791,line width=0.7pt]
(41.8pt, -283.811pt) -- (198pt, -283.811pt)
;
\end{tikzpicture}
\begin{picture}(-5,0)(2.5,0)
\put(198.1,-287.211){\fontsize{12}{1}\usefont{T1}{cmr}{b}{n}\selectfont\color{color_29791}zeroconf }
\end{picture}
\begin{tikzpicture}[overlay]
\path(0pt,0pt);
\draw[color_29791,line width=0.7pt]
(198.1pt, -283.811pt) -- (244.8pt, -283.811pt)
;
\end{tikzpicture}
\begin{picture}(-5,0)(2.5,0)
\put(244.8,-287.211){\fontsize{12}{1}\usefont{T1}{cmr}{m}{n}\selectfont\color{color_29791}è uno standard per far funzionare un servizio appena }
\end{picture}
\begin{tikzpicture}[overlay]
\path(0pt,0pt);
\draw[color_29791,line width=0.7pt]
(244.8pt, -283.811pt) -- (499.7pt, -283.811pt)
;
\end{tikzpicture}
\begin{picture}(-5,0)(2.5,0)
\put(41.8,-301.011){\fontsize{12}{1}\usefont{T1}{cmr}{m}{n}\selectfont\color{color_29791}connesso, automaticamente, senza configurarlo a mano. Questo non considera la comunicazione }
\end{picture}
\begin{tikzpicture}[overlay]
\path(0pt,0pt);
\draw[color_29791,line width=0.7pt]
(41.8pt, -297.611pt) -- (506.9pt, -297.611pt)
;
\end{tikzpicture}
\begin{picture}(-5,0)(2.5,0)
\put(41.8,-314.811){\fontsize{12}{1}\usefont{T1}{cmr}{m}{n}\selectfont\color{color_29791}standard a livello applicativo con periferiche (come fa UpnP, Universal Plug and Play), ma i layer }
\end{picture}
\begin{tikzpicture}[overlay]
\path(0pt,0pt);
\draw[color_29791,line width=0.7pt]
(41.8pt, -311.4109pt) -- (513.6pt, -311.4109pt)
;
\end{tikzpicture}
\begin{picture}(-5,0)(2.5,0)
\put(41.8,-328.611){\fontsize{12}{1}\usefont{T1}{cmr}{m}{n}\selectfont\color{color_29791}comuni a tutti:}
\end{picture}
\begin{tikzpicture}[overlay]
\path(0pt,0pt);
\draw[color_29791,line width=0.7pt]
(41.8pt, -325.211pt) -- (111.8pt, -325.211pt)
;
\end{tikzpicture}
\begin{picture}(-5,0)(2.5,0)
\put(59.8,-349.411){\fontsize{12}{1}\usefont{T1}{cmr}{m}{n}\selectfont\color{color_29791}•}
\end{picture}
\begin{tikzpicture}[overlay]
\path(0pt,0pt);
\draw[color_29791,line width=0.7pt]
(59.8pt, -346.011pt) -- (64.10001pt, -346.011pt)
;
\end{tikzpicture}
\begin{picture}(-5,0)(2.5,0)
\put(64,-349.411){\fontsize{12}{1}\usefont{T1}{cmr}{m}{n}\selectfont\color{color_29791}  }
\end{picture}
\begin{tikzpicture}[overlay]
\path(0pt,0pt);
\draw[color_29791,line width=0.7pt]
(64pt, -346.011pt) -- (77.7pt, -346.011pt)
;
\end{tikzpicture}
\begin{picture}(-5,0)(2.5,0)
\put(77.8,-349.411){\fontsize{12}{1}\usefont{T1}{cmr}{b}{n}\selectfont\color{color_29791}link-local addressing:}
\end{picture}
\begin{tikzpicture}[overlay]
\path(0pt,0pt);
\draw[color_29791,line width=0.7pt]
(77.8pt, -346.011pt) -- (187.9pt, -346.011pt)
;
\end{tikzpicture}
\begin{picture}(-5,0)(2.5,0)
\put(187.9,-349.411){\fontsize{12}{1}\usefont{T1}{cmr}{b}{n}\selectfont\color{color_29791}          }
\end{picture}
\begin{tikzpicture}[overlay]
\path(0pt,0pt);
\draw[color_29791,line width=0.7pt]
(187.9pt, -346.011pt) -- (219.6pt, -346.011pt)
;
\end{tikzpicture}
\begin{picture}(-5,0)(2.5,0)
\put(219.6,-349.411){\fontsize{12}{1}\usefont{T1}{cmr}{m}{n}\selectfont\color{color_29791}per determinare automaticamente un indirizzo di rete, }
\end{picture}
\begin{tikzpicture}[overlay]
\path(0pt,0pt);
\draw[color_29791,line width=0.7pt]
(219.6pt, -346.011pt) -- (480.1pt, -346.011pt)
;
\end{tikzpicture}
\begin{picture}(-5,0)(2.5,0)
\put(77.8,-363.211){\fontsize{12}{1}\usefont{T1}{cmr}{m}{n}\selectfont\color{color_29791}assegnazione di indirizzi validi sulla LAN: gli indirizzi di layer 2 per definizione valgono }
\end{picture}
\begin{tikzpicture}[overlay]
\path(0pt,0pt);
\draw[color_29791,line width=0.7pt]
(77.8pt, -359.811pt) -- (510.3pt, -359.811pt)
;
\end{tikzpicture}
\begin{picture}(-5,0)(2.5,0)
\put(77.8,-377.011){\fontsize{12}{1}\usefont{T1}{cmr}{m}{n}\selectfont\color{color_29791}solo localmente (MAC della scheda di rete è univoco sulla LAN ma non globalmente, non }
\end{picture}
\begin{tikzpicture}[overlay]
\path(0pt,0pt);
\draw[color_29791,line width=0.7pt]
(77.8pt, -373.611pt) -- (514.3pt, -373.611pt)
;
\end{tikzpicture}
\begin{picture}(-5,0)(2.5,0)
\put(77.8,-390.811){\fontsize{12}{1}\usefont{T1}{cmr}{m}{n}\selectfont\color{color_29791}sono instradabili). Il link-local addressing si pone l'obiettivo di avere lo stesso approccio per }
\end{picture}
\begin{tikzpicture}[overlay]
\path(0pt,0pt);
\draw[color_29791,line width=0.7pt]
(77.8pt, -387.411pt) -- (523.5pt, -387.411pt)
;
\end{tikzpicture}
\begin{picture}(-5,0)(2.5,0)
\put(77.8,-404.611){\fontsize{12}{1}\usefont{T1}{cmr}{m}{n}\selectfont\color{color_29791}il layer 3, per generare spontaneamente una subnet; ottenendo un IP univoco senza farselo }
\end{picture}
\begin{tikzpicture}[overlay]
\path(0pt,0pt);
\draw[color_29791,line width=0.7pt]
(77.8pt, -401.211pt) -- (512.8pt, -401.211pt)
;
\end{tikzpicture}
\begin{picture}(-5,0)(2.5,0)
\put(77.8,-418.411){\fontsize{12}{1}\usefont{T1}{cmr}{m}{n}\selectfont\color{color_29791}consegnare da un nameserver o un DHCP.}
\end{picture}
\begin{tikzpicture}[overlay]
\path(0pt,0pt);
\draw[color_29791,line width=0.7pt]
(77.8pt, -415.011pt) -- (280pt, -415.011pt)
;
\end{tikzpicture}
\begin{picture}(-5,0)(2.5,0)
\put(77.8,-439.211){\fontsize{12}{1}\usefont{T1}{cmr}{m}{n}\selectfont\color{color_29791}◦}
\end{picture}
\begin{tikzpicture}[overlay]
\path(0pt,0pt);
\draw[color_29791,line width=0.7pt]
(77.8pt, -435.811pt) -- (87.3pt, -435.811pt)
;
\end{tikzpicture}
\begin{picture}(-5,0)(2.5,0)
\put(87.3,-439.211){\fontsize{12}{1}\usefont{T1}{cmr}{m}{n}\selectfont\color{color_29791} }
\end{picture}
\begin{tikzpicture}[overlay]
\path(0pt,0pt);
\draw[color_29791,line width=0.7pt]
(87.3pt, -435.811pt) -- (93.3pt, -435.811pt)
;
\end{tikzpicture}
\begin{picture}(-5,0)(2.5,0)
\put(95.8,-439.211){\fontsize{12}{1}\usefont{T1}{cmr}{m}{n}\selectfont\color{color_29791}Link-local IPv4:}
\end{picture}
\begin{tikzpicture}[overlay]
\path(0pt,0pt);
\draw[color_29791,line width=0.7pt]
(95.8pt, -435.811pt) -- (174.8pt, -435.811pt)
;
\end{tikzpicture}
\begin{picture}(-5,0)(2.5,0)
\put(174.7,-439.211){\fontsize{12}{1}\usefont{T1}{cmr}{m}{n}\selectfont\color{color_29791}         }
\end{picture}
\begin{tikzpicture}[overlay]
\path(0pt,0pt);
\draw[color_29791,line width=0.7pt]
(174.7pt, -435.811pt) -- (202.1pt, -435.811pt)
;
\end{tikzpicture}
\begin{picture}(-5,0)(2.5,0)
\put(202.2,-439.211){\fontsize{12}{1}\usefont{T1}{cmr}{m}{n}\selectfont\color{color_29791}riservata a questo scopo la classe 169.254/16, best practice per }
\end{picture}
\begin{tikzpicture}[overlay]
\path(0pt,0pt);
\draw[color_29791,line width=0.7pt]
(202.2pt, -435.811pt) -- (505.4pt, -435.811pt)
;
\end{tikzpicture}
\begin{picture}(-5,0)(2.5,0)
\put(95.8,-453.011){\fontsize{12}{1}\usefont{T1}{cmr}{m}{n}\selectfont\color{color_29791}l'uso degli indirizzi sono: evitare l'assegnamento a interfacce che hanno indirizzi }
\end{picture}
\begin{tikzpicture}[overlay]
\path(0pt,0pt);
\draw[color_29791,line width=0.7pt]
(95.8pt, -449.611pt) -- (484.9pt, -449.611pt)
;
\end{tikzpicture}
\begin{picture}(-5,0)(2.5,0)
\put(95.8,-466.811){\fontsize{12}{1}\usefont{T1}{cmr}{m}{n}\selectfont\color{color_29791}instradabili, non distribuirli con DHCP, non associarli stabilmente a nomi DNS (siccome}
\end{picture}
\begin{tikzpicture}[overlay]
\path(0pt,0pt);
\draw[color_29791,line width=0.7pt]
(95.8pt, -463.411pt) -- (521.4pt, -463.411pt)
;
\end{tikzpicture}
\begin{picture}(-5,0)(2.5,0)
\put(95.8,-480.611){\fontsize{12}{1}\usefont{T1}{cmr}{m}{n}\selectfont\color{color_29791}servono solo sessione per sessione, per parlare IP tra le macchine). Assegnazione viene }
\end{picture}
\begin{tikzpicture}[overlay]
\path(0pt,0pt);
\draw[color_29791,line width=0.7pt]
(95.8pt, -477.211pt) -- (516.2pt, -477.211pt)
;
\end{tikzpicture}
\begin{picture}(-5,0)(2.5,0)
\put(95.8,-494.411){\fontsize{12}{1}\usefont{T1}{cmr}{m}{n}\selectfont\color{color_29791}fatta solo se interfaccia non ha già indirizzo assegnato staticamente o via DHCP:}
\end{picture}
\begin{tikzpicture}[overlay]
\path(0pt,0pt);
\draw[color_29791,line width=0.7pt]
(95.8pt, -491.011pt) -- (482.9pt, -491.011pt)
;
\end{tikzpicture}
\begin{picture}(-5,0)(2.5,0)
\put(95.8,-515.211){\fontsize{12}{1}\usefont{T1}{cmr}{m}{n}\selectfont\color{color_29791}▪}
\end{picture}
\begin{tikzpicture}[overlay]
\path(0pt,0pt);
\draw[color_29791,line width=0.7pt]
(95.8pt, -511.811pt) -- (104.7pt, -511.811pt)
;
\end{tikzpicture}
\begin{picture}(-5,0)(2.5,0)
\put(104.7,-515.211){\fontsize{12}{1}\usefont{T1}{cmr}{m}{n}\selectfont\color{color_29791} }
\end{picture}
\begin{tikzpicture}[overlay]
\path(0pt,0pt);
\draw[color_29791,line width=0.7pt]
(104.7pt, -511.811pt) -- (110.7pt, -511.811pt)
;
\end{tikzpicture}
\begin{picture}(-5,0)(2.5,0)
\put(113.8,-515.211){\fontsize{12}{1}\usefont{T1}{cmr}{m}{n}\selectfont\color{color_29791}Viene scelto IP random nel range 169.254.1.0 – 168.254.254.255, con seed legato a }
\end{picture}
\begin{tikzpicture}[overlay]
\path(0pt,0pt);
\draw[color_29791,line width=0.7pt]
(113.8pt, -511.811pt) -- (517.2pt, -511.811pt)
;
\end{tikzpicture}
\begin{picture}(-5,0)(2.5,0)
\put(113.8,-529.011){\fontsize{12}{1}\usefont{T1}{cmr}{m}{n}\selectfont\color{color_29791}caratteristica univoca (es. MAC, per ridurre probabilità di conflitto e tendere a }
\end{picture}
\begin{tikzpicture}[overlay]
\path(0pt,0pt);
\draw[color_29791,line width=0.7pt]
(113.8pt, -525.611pt) -- (492.3pt, -525.611pt)
;
\end{tikzpicture}
\begin{picture}(-5,0)(2.5,0)
\put(113.8,-542.811){\fontsize{12}{1}\usefont{T1}{cmr}{m}{n}\selectfont\color{color_29791}riassegnare IP stabili). }
\end{picture}
\begin{tikzpicture}[overlay]
\path(0pt,0pt);
\draw[color_29791,line width=0.7pt]
(113.8pt, -539.411pt) -- (223.3pt, -539.411pt)
;
\end{tikzpicture}
\begin{picture}(-5,0)(2.5,0)
\put(95.8,-563.611){\fontsize{12}{1}\usefont{T1}{cmr}{m}{n}\selectfont\color{color_29791}▪}
\end{picture}
\begin{tikzpicture}[overlay]
\path(0pt,0pt);
\draw[color_29791,line width=0.7pt]
(95.8pt, -560.211pt) -- (104.7pt, -560.211pt)
;
\end{tikzpicture}
\begin{picture}(-5,0)(2.5,0)
\put(104.7,-563.611){\fontsize{12}{1}\usefont{T1}{cmr}{m}{n}\selectfont\color{color_29791} }
\end{picture}
\begin{tikzpicture}[overlay]
\path(0pt,0pt);
\draw[color_29791,line width=0.7pt]
(104.7pt, -560.211pt) -- (110.7pt, -560.211pt)
;
\end{tikzpicture}
\begin{picture}(-5,0)(2.5,0)
\put(113.8,-563.611){\fontsize{12}{1}\usefont{T1}{cmr}{m}{n}\selectfont\color{color_29791}Poi si verifica che qualcuno non abbia già l'indirizzo con ARP probe (IP disponibile)}
\end{picture}
\begin{tikzpicture}[overlay]
\path(0pt,0pt);
\draw[color_29791,line width=0.7pt]
(113.8pt, -560.211pt) -- (519.3pt, -560.211pt)
;
\end{tikzpicture}
\begin{picture}(-5,0)(2.5,0)
\put(95.8,-584.411){\fontsize{12}{1}\usefont{T1}{cmr}{m}{n}\selectfont\color{color_29791}▪}
\end{picture}
\begin{tikzpicture}[overlay]
\path(0pt,0pt);
\draw[color_29791,line width=0.7pt]
(95.8pt, -581.011pt) -- (104.7pt, -581.011pt)
;
\end{tikzpicture}
\begin{picture}(-5,0)(2.5,0)
\put(104.7,-584.411){\fontsize{12}{1}\usefont{T1}{cmr}{m}{n}\selectfont\color{color_29791} }
\end{picture}
\begin{tikzpicture}[overlay]
\path(0pt,0pt);
\draw[color_29791,line width=0.7pt]
(104.7pt, -581.011pt) -- (110.7pt, -581.011pt)
;
\end{tikzpicture}
\begin{picture}(-5,0)(2.5,0)
\put(113.8,-584.411){\fontsize{12}{1}\usefont{T1}{cmr}{m}{n}\selectfont\color{color_29791}Infine si annuncia acquisizione con gratuitous ARP}
\end{picture}
\begin{tikzpicture}[overlay]
\path(0pt,0pt);
\draw[color_29791,line width=0.7pt]
(113.8pt, -581.011pt) -- (359.7pt, -581.011pt)
;
\end{tikzpicture}
\begin{picture}(-5,0)(2.5,0)
\put(77.8,-605.211){\fontsize{12}{1}\usefont{T1}{cmr}{m}{n}\selectfont\color{color_29791}◦}
\end{picture}
\begin{tikzpicture}[overlay]
\path(0pt,0pt);
\draw[color_29791,line width=0.7pt]
(77.8pt, -601.811pt) -- (87.3pt, -601.811pt)
;
\end{tikzpicture}
\begin{picture}(-5,0)(2.5,0)
\put(87.3,-605.211){\fontsize{12}{1}\usefont{T1}{cmr}{m}{n}\selectfont\color{color_29791} }
\end{picture}
\begin{tikzpicture}[overlay]
\path(0pt,0pt);
\draw[color_29791,line width=0.7pt]
(87.3pt, -601.811pt) -- (93.3pt, -601.811pt)
;
\end{tikzpicture}
\begin{picture}(-5,0)(2.5,0)
\put(95.8,-605.211){\fontsize{12}{1}\usefont{T1}{cmr}{m}{n}\selectfont\color{color_29791}Link-local IPv6:}
\end{picture}
\begin{tikzpicture}[overlay]
\path(0pt,0pt);
\draw[color_29791,line width=0.7pt]
(95.8pt, -601.811pt) -- (174.8pt, -601.811pt)
;
\end{tikzpicture}
\begin{picture}(-5,0)(2.5,0)
\put(174.7,-605.211){\fontsize{12}{1}\usefont{T1}{cmr}{m}{n}\selectfont\color{color_29791}         }
\end{picture}
\begin{tikzpicture}[overlay]
\path(0pt,0pt);
\draw[color_29791,line width=0.7pt]
(174.7pt, -601.811pt) -- (202.1pt, -601.811pt)
;
\end{tikzpicture}
\begin{picture}(-5,0)(2.5,0)
\put(202.2,-605.211){\fontsize{12}{1}\usefont{T1}{cmr}{m}{n}\selectfont\color{color_29791}gli indirizzi IPv6 sono divisi in 64bit di Subnet Prefix e 64 bit di }
\end{picture}
\begin{tikzpicture}[overlay]
\path(0pt,0pt);
\draw[color_29791,line width=0.7pt]
(202.2pt, -601.811pt) -- (514.2pt, -601.811pt)
;
\end{tikzpicture}
\begin{picture}(-5,0)(2.5,0)
\put(95.8,-619.011){\fontsize{12}{1}\usefont{T1}{cmr}{m}{n}\selectfont\color{color_29791}Interface ID. IPv6 definisce un range per indirizzi link-local (logicamente equivale a }
\end{picture}
\begin{tikzpicture}[overlay]
\path(0pt,0pt);
\draw[color_29791,line width=0.7pt]
(95.8pt, -615.611pt) -- (504.6pt, -615.611pt)
;
\end{tikzpicture}
\begin{picture}(-5,0)(2.5,0)
\put(95.8,-632.811){\fontsize{12}{1}\usefont{T1}{cmr}{m}{n}\selectfont\color{color_29791}169.254/16), con prefix FE80::/10. (:: significa che ci sono solo zeri). Ogni interfaccia }
\end{picture}
\begin{tikzpicture}[overlay]
\path(0pt,0pt);
\draw[color_29791,line width=0.7pt]
(95.8pt, -629.411pt) -- (512.7pt, -629.411pt)
;
\end{tikzpicture}
\begin{picture}(-5,0)(2.5,0)
\put(95.8,-646.611){\fontsize{12}{1}\usefont{T1}{cmr}{m}{n}\selectfont\color{color_29791}può costruire un suo IPv6 link-local seguendo lo schema:}
\end{picture}
\begin{tikzpicture}[overlay]
\path(0pt,0pt);
\draw[color_29791,line width=0.7pt]
(95.8pt, -643.211pt) -- (371pt, -643.211pt)
;
\end{tikzpicture}
\begin{picture}(-5,0)(2.5,0)
\put(95.8,-667.411){\fontsize{12}{1}\usefont{T1}{cmr}{m}{n}\selectfont\color{color_29791}▪}
\end{picture}
\begin{tikzpicture}[overlay]
\path(0pt,0pt);
\draw[color_29791,line width=0.7pt]
(95.8pt, -664.011pt) -- (104.7pt, -664.011pt)
;
\end{tikzpicture}
\begin{picture}(-5,0)(2.5,0)
\put(104.7,-667.411){\fontsize{12}{1}\usefont{T1}{cmr}{m}{n}\selectfont\color{color_29791} }
\end{picture}
\begin{tikzpicture}[overlay]
\path(0pt,0pt);
\draw[color_29791,line width=0.7pt]
(104.7pt, -664.011pt) -- (110.7pt, -664.011pt)
;
\end{tikzpicture}
\begin{picture}(-5,0)(2.5,0)
\put(113.8,-667.411){\fontsize{12}{1}\usefont{T1}{cmr}{m}{n}\selectfont\color{color_29791}Si prende il MAC (, si allunga di 2 byte mettendo in mezzo i due byte fissi :ff:fe:}
\end{picture}
\begin{tikzpicture}[overlay]
\path(0pt,0pt);
\draw[color_29791,line width=0.7pt]
(113.8pt, -664.011pt) -- (501.5pt, -664.011pt)
;
\end{tikzpicture}
\begin{picture}(-5,0)(2.5,0)
\put(95.8,-688.211){\fontsize{12}{1}\usefont{T1}{cmr}{m}{n}\selectfont\color{color_29791}▪}
\end{picture}
\begin{tikzpicture}[overlay]
\path(0pt,0pt);
\draw[color_29791,line width=0.7pt]
(95.8pt, -684.811pt) -- (104.7pt, -684.811pt)
;
\end{tikzpicture}
\begin{picture}(-5,0)(2.5,0)
\put(104.7,-688.211){\fontsize{12}{1}\usefont{T1}{cmr}{m}{n}\selectfont\color{color_29791} }
\end{picture}
\begin{tikzpicture}[overlay]
\path(0pt,0pt);
\draw[color_29791,line width=0.7pt]
(104.7pt, -684.811pt) -- (110.7pt, -684.811pt)
;
\end{tikzpicture}
\begin{picture}(-5,0)(2.5,0)
\put(113.8,-688.211){\fontsize{12}{1}\usefont{T1}{cmr}{m}{n}\selectfont\color{color_29791}Si inverte il 7° bit del primo byte (trasformato in binario ovviamente)}
\end{picture}
\begin{tikzpicture}[overlay]
\path(0pt,0pt);
\draw[color_29791,line width=0.7pt]
(113.8pt, -684.811pt) -- (446.8pt, -684.811pt)
;
\end{tikzpicture}
\begin{picture}(-5,0)(2.5,0)
\put(95.8,-709.011){\fontsize{12}{1}\usefont{T1}{cmr}{m}{n}\selectfont\color{color_29791}▪}
\end{picture}
\begin{tikzpicture}[overlay]
\path(0pt,0pt);
\draw[color_29791,line width=0.7pt]
(95.8pt, -705.611pt) -- (104.7pt, -705.611pt)
;
\end{tikzpicture}
\begin{picture}(-5,0)(2.5,0)
\put(104.7,-709.011){\fontsize{12}{1}\usefont{T1}{cmr}{m}{n}\selectfont\color{color_29791} }
\end{picture}
\begin{tikzpicture}[overlay]
\path(0pt,0pt);
\draw[color_29791,line width=0.7pt]
(104.7pt, -705.611pt) -- (110.7pt, -705.611pt)
;
\end{tikzpicture}
\begin{picture}(-5,0)(2.5,0)
\put(113.8,-709.011){\fontsize{12}{1}\usefont{T1}{cmr}{m}{n}\selectfont\color{color_29791}Il risultato è l'Interface ID da concatenare a FE80:: (il subnet prefix)}
\end{picture}
\begin{tikzpicture}[overlay]
\path(0pt,0pt);
\draw[color_29791,line width=0.7pt]
(113.8pt, -705.611pt) -- (440.8pt, -705.611pt)
;
\end{tikzpicture}
\begin{picture}(-5,0)(2.5,0)
\put(77.8,-729.811){\fontsize{12}{1}\usefont{T1}{cmr}{m}{n}\selectfont\color{color_29791}◦}
\end{picture}
\begin{tikzpicture}[overlay]
\path(0pt,0pt);
\draw[color_29791,line width=0.7pt]
(77.8pt, -726.411pt) -- (87.3pt, -726.411pt)
;
\end{tikzpicture}
\begin{picture}(-5,0)(2.5,0)
\put(87.3,-729.811){\fontsize{12}{1}\usefont{T1}{cmr}{m}{n}\selectfont\color{color_29791} }
\end{picture}
\begin{tikzpicture}[overlay]
\path(0pt,0pt);
\draw[color_29791,line width=0.7pt]
(87.3pt, -726.411pt) -- (93.3pt, -726.411pt)
;
\end{tikzpicture}
\begin{picture}(-5,0)(2.5,0)
\put(95.8,-729.811){\fontsize{12}{1}\usefont{T1}{cmr}{m}{n}\selectfont\color{color_29791}IPv6 ha sistema flessibile per determinare se indirizzo è libero e valutare se rete locale è }
\end{picture}
\begin{tikzpicture}[overlay]
\path(0pt,0pt);
\draw[color_29791,line width=0.7pt]
(95.8pt, -726.411pt) -- (521.6pt, -726.411pt)
;
\end{tikzpicture}
\begin{picture}(-5,0)(2.5,0)
\put(95.8,-743.611){\fontsize{12}{1}\usefont{T1}{cmr}{m}{n}\selectfont\color{color_29791}raggiungibile dall'esterno: SLAAC è l'algoritmo per costruire indirizzi link-local validi }
\end{picture}
\begin{tikzpicture}[overlay]
\path(0pt,0pt);
\draw[color_29791,line width=0.7pt]
(95.8pt, -740.211pt) -- (515.3pt, -740.211pt)
;
\end{tikzpicture}
\begin{picture}(-5,0)(2.5,0)
\put(95.8,-757.411){\fontsize{12}{1}\usefont{T1}{cmr}{m}{n}\selectfont\color{color_29791}globalmente configurando host in automatico (se possibile). }
\end{picture}
\begin{tikzpicture}[overlay]
\path(0pt,0pt);
\draw[color_29791,line width=0.7pt]
(95.8pt, -754.011pt) -- (386.4pt, -754.011pt)
;
\end{tikzpicture}
\newpage
\begin{tikzpicture}[overlay]\path(0pt,0pt);\end{tikzpicture}
\begin{picture}(-5,0)(2.5,0)
\put(95.8,-85.01099){\fontsize{12}{1}\usefont{T1}{cmr}{m}{n}\selectfont\color{color_29791}▪}
\end{picture}
\begin{tikzpicture}[overlay]
\path(0pt,0pt);
\draw[color_29791,line width=0.7pt]
(95.8pt, -81.61096pt) -- (104.7pt, -81.61096pt)
;
\end{tikzpicture}
\begin{picture}(-5,0)(2.5,0)
\put(104.7,-85.01099){\fontsize{12}{1}\usefont{T1}{cmr}{m}{n}\selectfont\color{color_29791} }
\end{picture}
\begin{tikzpicture}[overlay]
\path(0pt,0pt);
\draw[color_29791,line width=0.7pt]
(104.7pt, -81.61096pt) -- (110.7pt, -81.61096pt)
;
\end{tikzpicture}
\begin{picture}(-5,0)(2.5,0)
\put(113.8,-85.01099){\fontsize{12}{1}\usefont{T1}{cmr}{m}{n}\selectfont\color{color_29791}Definizione token identificativo interfaccia (da MAC o valore random)}
\end{picture}
\begin{tikzpicture}[overlay]
\path(0pt,0pt);
\draw[color_29791,line width=0.7pt]
(113.8pt, -81.61096pt) -- (455pt, -81.61096pt)
;
\end{tikzpicture}
\begin{picture}(-5,0)(2.5,0)
\put(95.8,-105.811){\fontsize{12}{1}\usefont{T1}{cmr}{m}{n}\selectfont\color{color_29791}▪}
\end{picture}
\begin{tikzpicture}[overlay]
\path(0pt,0pt);
\draw[color_29791,line width=0.7pt]
(95.8pt, -102.4109pt) -- (104.7pt, -102.4109pt)
;
\end{tikzpicture}
\begin{picture}(-5,0)(2.5,0)
\put(104.7,-105.811){\fontsize{12}{1}\usefont{T1}{cmr}{m}{n}\selectfont\color{color_29791} }
\end{picture}
\begin{tikzpicture}[overlay]
\path(0pt,0pt);
\draw[color_29791,line width=0.7pt]
(104.7pt, -102.4109pt) -- (110.7pt, -102.4109pt)
;
\end{tikzpicture}
\begin{picture}(-5,0)(2.5,0)
\put(113.8,-105.811){\fontsize{12}{1}\usefont{T1}{cmr}{m}{n}\selectfont\color{color_29791}generazione indirizzo IPv6 link-local con prefix FE80::/10}
\end{picture}
\begin{tikzpicture}[overlay]
\path(0pt,0pt);
\draw[color_29791,line width=0.7pt]
(113.8pt, -102.4109pt) -- (395pt, -102.4109pt)
;
\end{tikzpicture}
\begin{picture}(-5,0)(2.5,0)
\put(95.8,-126.611){\fontsize{12}{1}\usefont{T1}{cmr}{m}{n}\selectfont\color{color_29791}▪}
\end{picture}
\begin{tikzpicture}[overlay]
\path(0pt,0pt);
\draw[color_29791,line width=0.7pt]
(95.8pt, -123.211pt) -- (104.7pt, -123.211pt)
;
\end{tikzpicture}
\begin{picture}(-5,0)(2.5,0)
\put(104.7,-126.611){\fontsize{12}{1}\usefont{T1}{cmr}{m}{n}\selectfont\color{color_29791} }
\end{picture}
\begin{tikzpicture}[overlay]
\path(0pt,0pt);
\draw[color_29791,line width=0.7pt]
(104.7pt, -123.211pt) -- (110.7pt, -123.211pt)
;
\end{tikzpicture}
\begin{picture}(-5,0)(2.5,0)
\put(113.8,-126.611){\fontsize{12}{1}\usefont{T1}{cmr}{m}{n}\selectfont\color{color_29791}invio di Neighbor Solicitation (algoritmo di scoperta dei vicini), se nessuno risponde }
\end{picture}
\begin{tikzpicture}[overlay]
\path(0pt,0pt);
\draw[color_29791,line width=0.7pt]
(113.8pt, -123.211pt) -- (523.4pt, -123.211pt)
;
\end{tikzpicture}
\begin{picture}(-5,0)(2.5,0)
\put(113.8,-140.411){\fontsize{12}{1}\usefont{T1}{cmr}{m}{n}\selectfont\color{color_29791}si avanza, altrimenti indirizzo è occupato e processo automatico si ferma}
\end{picture}
\begin{tikzpicture}[overlay]
\path(0pt,0pt);
\draw[color_29791,line width=0.7pt]
(113.8pt, -137.011pt) -- (462.6pt, -137.011pt)
;
\end{tikzpicture}
\begin{picture}(-5,0)(2.5,0)
\put(95.8,-161.211){\fontsize{12}{1}\usefont{T1}{cmr}{m}{n}\selectfont\color{color_29791}▪}
\end{picture}
\begin{tikzpicture}[overlay]
\path(0pt,0pt);
\draw[color_29791,line width=0.7pt]
(95.8pt, -157.811pt) -- (104.7pt, -157.811pt)
;
\end{tikzpicture}
\begin{picture}(-5,0)(2.5,0)
\put(104.7,-161.211){\fontsize{12}{1}\usefont{T1}{cmr}{m}{n}\selectfont\color{color_29791} }
\end{picture}
\begin{tikzpicture}[overlay]
\path(0pt,0pt);
\draw[color_29791,line width=0.7pt]
(104.7pt, -157.811pt) -- (110.7pt, -157.811pt)
;
\end{tikzpicture}
\begin{picture}(-5,0)(2.5,0)
\put(113.8,-161.211){\fontsize{12}{1}\usefont{T1}{cmr}{m}{n}\selectfont\color{color_29791}invio di Router Solicitation all'indirizzo multicast di tutti i router (serve per scoprire }
\end{picture}
\begin{tikzpicture}[overlay]
\path(0pt,0pt);
\draw[color_29791,line width=0.7pt]
(113.8pt, -157.811pt) -- (519.5pt, -157.811pt)
;
\end{tikzpicture}
\begin{picture}(-5,0)(2.5,0)
\put(113.8,-175.011){\fontsize{12}{1}\usefont{T1}{cmr}{m}{n}\selectfont\color{color_29791}se sulla rete ci sono router disposti a instradare il mio traffico). }
\end{picture}
\begin{tikzpicture}[overlay]
\path(0pt,0pt);
\draw[color_29791,line width=0.7pt]
(113.8pt, -171.611pt) -- (417.8pt, -171.611pt)
;
\end{tikzpicture}
\begin{picture}(-5,0)(2.5,0)
\put(113.8,-195.811){\fontsize{12}{1}\usefont{T1}{cmr}{m}{n}\selectfont\color{color_29791}1.}
\end{picture}
\begin{tikzpicture}[overlay]
\path(0pt,0pt);
\draw[color_29791,line width=0.7pt]
(113.8pt, -192.4109pt) -- (122.8pt, -192.4109pt)
;
\end{tikzpicture}
\begin{picture}(-5,0)(2.5,0)
\put(122.7,-195.811){\fontsize{12}{1}\usefont{T1}{cmr}{m}{n}\selectfont\color{color_29791}   }
\end{picture}
\begin{tikzpicture}[overlay]
\path(0pt,0pt);
\draw[color_29791,line width=0.7pt]
(122.7pt, -192.4109pt) -- (131.7pt, -192.4109pt)
;
\end{tikzpicture}
\begin{picture}(-5,0)(2.5,0)
\put(131.8,-195.811){\fontsize{12}{1}\usefont{T1}{cmr}{m}{n}\selectfont\color{color_29791}Se c'è risposta, si valuta la risposta: }
\end{picture}
\begin{tikzpicture}[overlay]
\path(0pt,0pt);
\draw[color_29791,line width=0.7pt]
(131.8pt, -192.4109pt) -- (304.5pt, -192.4109pt)
;
\end{tikzpicture}
\begin{picture}(-5,0)(2.5,0)
\put(131.8,-216.611){\fontsize{12}{1}\usefont{T1}{cmr}{m}{n}\selectfont\color{color_29791}1.}
\end{picture}
\begin{tikzpicture}[overlay]
\path(0pt,0pt);
\draw[color_29791,line width=0.7pt]
(131.8pt, -213.211pt) -- (140.8pt, -213.211pt)
;
\end{tikzpicture}
\begin{picture}(-5,0)(2.5,0)
\put(140.7,-216.611){\fontsize{12}{1}\usefont{T1}{cmr}{m}{n}\selectfont\color{color_29791}   }
\end{picture}
\begin{tikzpicture}[overlay]
\path(0pt,0pt);
\draw[color_29791,line width=0.7pt]
(140.7pt, -213.211pt) -- (149.7pt, -213.211pt)
;
\end{tikzpicture}
\begin{picture}(-5,0)(2.5,0)
\put(149.8,-216.611){\fontsize{12}{1}\usefont{T1}{cmr}{m}{n}\selectfont\color{color_29791}Risposta M:}
\end{picture}
\begin{tikzpicture}[overlay]
\path(0pt,0pt);
\draw[color_29791,line width=0.7pt]
(149.8pt, -213.211pt) -- (208.1pt, -213.211pt)
;
\end{tikzpicture}
\begin{picture}(-5,0)(2.5,0)
\put(208.1,-216.611){\fontsize{12}{1}\usefont{T1}{cmr}{m}{n}\selectfont\color{color_29791}    }
\end{picture}
\begin{tikzpicture}[overlay]
\path(0pt,0pt);
\draw[color_29791,line width=0.7pt]
(208.1pt, -213.211pt) -- (220.7pt, -213.211pt)
;
\end{tikzpicture}
\begin{picture}(-5,0)(2.5,0)
\put(220.7,-216.611){\fontsize{12}{1}\usefont{T1}{cmr}{m}{n}\selectfont\color{color_29791}è necessario DHCP, si va nello stesso stato di mancanza router}
\end{picture}
\begin{tikzpicture}[overlay]
\path(0pt,0pt);
\draw[color_29791,line width=0.7pt]
(220.7pt, -213.211pt) -- (520.2pt, -213.211pt)
;
\end{tikzpicture}
\begin{picture}(-5,0)(2.5,0)
\put(131.8,-237.411){\fontsize{12}{1}\usefont{T1}{cmr}{m}{n}\selectfont\color{color_29791}2.}
\end{picture}
\begin{tikzpicture}[overlay]
\path(0pt,0pt);
\draw[color_29791,line width=0.7pt]
(131.8pt, -234.011pt) -- (140.8pt, -234.011pt)
;
\end{tikzpicture}
\begin{picture}(-5,0)(2.5,0)
\put(140.7,-237.411){\fontsize{12}{1}\usefont{T1}{cmr}{m}{n}\selectfont\color{color_29791}   }
\end{picture}
\begin{tikzpicture}[overlay]
\path(0pt,0pt);
\draw[color_29791,line width=0.7pt]
(140.7pt, -234.011pt) -- (149.7pt, -234.011pt)
;
\end{tikzpicture}
\begin{picture}(-5,0)(2.5,0)
\put(149.8,-237.411){\fontsize{12}{1}\usefont{T1}{cmr}{m}{n}\selectfont\color{color_29791}Risposta A (+subnet prefix):}
\end{picture}
\begin{tikzpicture}[overlay]
\path(0pt,0pt);
\draw[color_29791,line width=0.7pt]
(149.8pt, -234.011pt) -- (285.5pt, -234.011pt)
;
\end{tikzpicture}
\begin{picture}(-5,0)(2.5,0)
\put(285.5,-237.411){\fontsize{12}{1}\usefont{T1}{cmr}{m}{n}\selectfont\color{color_29791}  }
\end{picture}
\begin{tikzpicture}[overlay]
\path(0pt,0pt);
\draw[color_29791,line width=0.7pt]
(285.5pt, -234.011pt) -- (291.6pt, -234.011pt)
;
\end{tikzpicture}
\begin{picture}(-5,0)(2.5,0)
\put(291.6,-237.411){\fontsize{12}{1}\usefont{T1}{cmr}{m}{n}\selectfont\color{color_29791}concatenazione del prefisso con indirizzo }
\end{picture}
\begin{tikzpicture}[overlay]
\path(0pt,0pt);
\draw[color_29791,line width=0.7pt]
(291.6pt, -234.011pt) -- (493.1pt, -234.011pt)
;
\end{tikzpicture}
\begin{picture}(-5,0)(2.5,0)
\put(149.8,-251.211){\fontsize{12}{1}\usefont{T1}{cmr}{m}{n}\selectfont\color{color_29791}globale ottenuto}
\end{picture}
\begin{tikzpicture}[overlay]
\path(0pt,0pt);
\draw[color_29791,line width=0.7pt]
(149.8pt, -247.811pt) -- (227.4pt, -247.811pt)
;
\end{tikzpicture}
\begin{picture}(-5,0)(2.5,0)
\put(113.8,-272.011){\fontsize{12}{1}\usefont{T1}{cmr}{m}{n}\selectfont\color{color_29791}2.}
\end{picture}
\begin{tikzpicture}[overlay]
\path(0pt,0pt);
\draw[color_29791,line width=0.7pt]
(113.8pt, -268.611pt) -- (122.8pt, -268.611pt)
;
\end{tikzpicture}
\begin{picture}(-5,0)(2.5,0)
\put(122.7,-272.011){\fontsize{12}{1}\usefont{T1}{cmr}{m}{n}\selectfont\color{color_29791}   }
\end{picture}
\begin{tikzpicture}[overlay]
\path(0pt,0pt);
\draw[color_29791,line width=0.7pt]
(122.7pt, -268.611pt) -- (131.7pt, -268.611pt)
;
\end{tikzpicture}
\begin{picture}(-5,0)(2.5,0)
\put(131.8,-272.011){\fontsize{12}{1}\usefont{T1}{cmr}{m}{n}\selectfont\color{color_29791}Se non c'è risposta (nessun router), si fa richiesta in broadcast per scoprire se c'è }
\end{picture}
\begin{tikzpicture}[overlay]
\path(0pt,0pt);
\draw[color_29791,line width=0.7pt]
(131.8pt, -268.611pt) -- (519.3pt, -268.611pt)
;
\end{tikzpicture}
\begin{picture}(-5,0)(2.5,0)
\put(131.8,-285.811){\fontsize{12}{1}\usefont{T1}{cmr}{m}{n}\selectfont\color{color_29791}server DHCPv6 che può darmi un indirizzo}
\end{picture}
\begin{tikzpicture}[overlay]
\path(0pt,0pt);
\draw[color_29791,line width=0.7pt]
(131.8pt, -282.4109pt) -- (340.4pt, -282.4109pt)
;
\end{tikzpicture}
\begin{picture}(-5,0)(2.5,0)
\put(131.8,-306.611){\fontsize{12}{1}\usefont{T1}{cmr}{m}{n}\selectfont\color{color_29791}1.}
\end{picture}
\begin{tikzpicture}[overlay]
\path(0pt,0pt);
\draw[color_29791,line width=0.7pt]
(131.8pt, -303.211pt) -- (140.8pt, -303.211pt)
;
\end{tikzpicture}
\begin{picture}(-5,0)(2.5,0)
\put(140.7,-306.611){\fontsize{12}{1}\usefont{T1}{cmr}{m}{n}\selectfont\color{color_29791}   }
\end{picture}
\begin{tikzpicture}[overlay]
\path(0pt,0pt);
\draw[color_29791,line width=0.7pt]
(140.7pt, -303.211pt) -- (149.7pt, -303.211pt)
;
\end{tikzpicture}
\begin{picture}(-5,0)(2.5,0)
\put(149.8,-306.611){\fontsize{12}{1}\usefont{T1}{cmr}{m}{n}\selectfont\color{color_29791}Risposta ricevuta, si assegna e usa l'indirizzo ricevuto}
\end{picture}
\begin{tikzpicture}[overlay]
\path(0pt,0pt);
\draw[color_29791,line width=0.7pt]
(149.8pt, -303.211pt) -- (407.8pt, -303.211pt)
;
\end{tikzpicture}
\begin{picture}(-5,0)(2.5,0)
\put(131.8,-327.411){\fontsize{12}{1}\usefont{T1}{cmr}{m}{n}\selectfont\color{color_29791}2.}
\end{picture}
\begin{tikzpicture}[overlay]
\path(0pt,0pt);
\draw[color_29791,line width=0.7pt]
(131.8pt, -324.011pt) -- (140.8pt, -324.011pt)
;
\end{tikzpicture}
\begin{picture}(-5,0)(2.5,0)
\put(140.7,-327.411){\fontsize{12}{1}\usefont{T1}{cmr}{m}{n}\selectfont\color{color_29791}   }
\end{picture}
\begin{tikzpicture}[overlay]
\path(0pt,0pt);
\draw[color_29791,line width=0.7pt]
(140.7pt, -324.011pt) -- (149.7pt, -324.011pt)
;
\end{tikzpicture}
\begin{picture}(-5,0)(2.5,0)
\put(149.8,-327.411){\fontsize{12}{1}\usefont{T1}{cmr}{m}{n}\selectfont\color{color_29791}Nessuna risposta, si mantiene indirizzo link-local.}
\end{picture}
\begin{tikzpicture}[overlay]
\path(0pt,0pt);
\draw[color_29791,line width=0.7pt]
(149.8pt, -324.011pt) -- (389.4pt, -324.011pt)
;
\end{tikzpicture}
\begin{picture}(-5,0)(2.5,0)
\put(59.8,-348.211){\fontsize{12}{1}\usefont{T1}{cmr}{m}{n}\selectfont\color{color_29791}•}
\end{picture}
\begin{tikzpicture}[overlay]
\path(0pt,0pt);
\draw[color_29791,line width=0.7pt]
(59.8pt, -344.811pt) -- (64.10001pt, -344.811pt)
;
\end{tikzpicture}
\begin{picture}(-5,0)(2.5,0)
\put(64,-348.211){\fontsize{12}{1}\usefont{T1}{cmr}{m}{n}\selectfont\color{color_29791}  }
\end{picture}
\begin{tikzpicture}[overlay]
\path(0pt,0pt);
\draw[color_29791,line width=0.7pt]
(64pt, -344.811pt) -- (77.7pt, -344.811pt)
;
\end{tikzpicture}
\begin{picture}(-5,0)(2.5,0)
\put(77.8,-348.211){\fontsize{12}{1}\usefont{T1}{cmr}{b}{n}\selectfont\color{color_29791}multicast DNS:}
\end{picture}
\begin{tikzpicture}[overlay]
\path(0pt,0pt);
\draw[color_29791,line width=0.7pt]
(77.8pt, -344.811pt) -- (156.1pt, -344.811pt)
;
\end{tikzpicture}
\begin{picture}(-5,0)(2.5,0)
\put(156.1,-348.211){\fontsize{12}{1}\usefont{T1}{cmr}{b}{n}\selectfont\color{color_29791}         }
\end{picture}
\begin{tikzpicture}[overlay]
\path(0pt,0pt);
\draw[color_29791,line width=0.7pt]
(156.1pt, -344.811pt) -- (184.1pt, -344.811pt)
;
\end{tikzpicture}
\begin{picture}(-5,0)(2.5,0)
\put(184.1,-348.211){\fontsize{12}{1}\usefont{T1}{cmr}{b}{n}\selectfont\color{color_29791}           }
\end{picture}
\begin{tikzpicture}[overlay]
\path(0pt,0pt);
\draw[color_29791,line width=0.7pt]
(184.1pt, -344.811pt) -- (219.6pt, -344.811pt)
;
\end{tikzpicture}
\begin{picture}(-5,0)(2.5,0)
\put(219.6,-348.211){\fontsize{12}{1}\usefont{T1}{cmr}{m}{n}\selectfont\color{color_29791}per la traduzione di nomi in indirizzi in mancanza di DNS }
\end{picture}
\begin{tikzpicture}[overlay]
\path(0pt,0pt);
\draw[color_29791,line width=0.7pt]
(219.6pt, -344.811pt) -- (501.2pt, -344.811pt)
;
\end{tikzpicture}
\begin{picture}(-5,0)(2.5,0)
\put(77.8,-362.011){\fontsize{12}{1}\usefont{T1}{cmr}{m}{n}\selectfont\color{color_29791}unicast configurato manualmente. L'associazione nomi-indirizzi viene gestita secondo }
\end{picture}
\begin{tikzpicture}[overlay]
\path(0pt,0pt);
\draw[color_29791,line width=0.7pt]
(77.8pt, -358.611pt) -- (494.1pt, -358.611pt)
;
\end{tikzpicture}
\begin{picture}(-5,0)(2.5,0)
\put(77.8,-375.811){\fontsize{12}{1}\usefont{T1}{cmr}{m}{n}\selectfont\color{color_29791}standard, che definisce il TLD }
\end{picture}
\begin{tikzpicture}[overlay]
\path(0pt,0pt);
\draw[color_29791,line width=0.7pt]
(77.8pt, -372.411pt) -- (226.2pt, -372.411pt)
;
\end{tikzpicture}
\begin{picture}(-5,0)(2.5,0)
\put(226.2,-375.811){\fontsize{12}{1}\usefont{T1}{cmr}{m}{it}\selectfont\color{color_29791}.local. }
\end{picture}
\begin{tikzpicture}[overlay]
\path(0pt,0pt);
\draw[color_29791,line width=0.7pt]
(226.2pt, -372.411pt) -- (259.2pt, -372.411pt)
;
\end{tikzpicture}
\begin{picture}(-5,0)(2.5,0)
\put(259.2,-375.811){\fontsize{12}{1}\usefont{T1}{cmr}{m}{n}\selectfont\color{color_29791}come riservato ad host appartenenti a una rete link-}
\end{picture}
\begin{tikzpicture}[overlay]
\path(0pt,0pt);
\draw[color_29791,line width=0.7pt]
(259.2pt, -372.411pt) -- (505.1pt, -372.411pt)
;
\end{tikzpicture}
\begin{picture}(-5,0)(2.5,0)
\put(77.8,-389.611){\fontsize{12}{1}\usefont{T1}{cmr}{m}{n}\selectfont\color{color_29791}local; e impone che le richieste di risoluzione per nomi che terminano in .local. siano inviate}
\end{picture}
\begin{tikzpicture}[overlay]
\path(0pt,0pt);
\draw[color_29791,line width=0.7pt]
(77.8pt, -386.211pt) -- (521.3pt, -386.211pt)
;
\end{tikzpicture}
\begin{picture}(-5,0)(2.5,0)
\put(77.8,-403.411){\fontsize{12}{1}\usefont{T1}{cmr}{m}{n}\selectfont\color{color_29791}all'indirizzo link-local di multicast 224.0.0.251 (IPv4) o FF02::FB (IPv6), porta 5353. }
\end{picture}
\begin{tikzpicture}[overlay]
\path(0pt,0pt);
\draw[color_29791,line width=0.7pt]
(77.8pt, -400.011pt) -- (492.6pt, -400.011pt)
;
\end{tikzpicture}
\begin{picture}(-5,0)(2.5,0)
\put(77.8,-417.211){\fontsize{12}{1}\usefont{T1}{cmr}{m}{n}\selectfont\color{color_29791}Inoltre lo standard raccomanda di strutturare i nomi in modo flat, sconsiglia domini di }
\end{picture}
\begin{tikzpicture}[overlay]
\path(0pt,0pt);
\draw[color_29791,line width=0.7pt]
(77.8pt, -413.811pt) -- (493.3pt, -413.811pt)
;
\end{tikzpicture}
\begin{picture}(-5,0)(2.5,0)
\put(77.8,-431.011){\fontsize{12}{1}\usefont{T1}{cmr}{m}{n}\selectfont\color{color_29791}secondo, terzo... livello.}
\end{picture}
\begin{tikzpicture}[overlay]
\path(0pt,0pt);
\draw[color_29791,line width=0.7pt]
(77.8pt, -427.611pt) -- (192.7pt, -427.611pt)
;
\end{tikzpicture}
\begin{picture}(-5,0)(2.5,0)
\put(59.8,-451.811){\fontsize{12}{1}\usefont{T1}{cmr}{m}{n}\selectfont\color{color_29791}•}
\end{picture}
\begin{tikzpicture}[overlay]
\path(0pt,0pt);
\draw[color_29791,line width=0.7pt]
(59.8pt, -448.411pt) -- (64.10001pt, -448.411pt)
;
\end{tikzpicture}
\begin{picture}(-5,0)(2.5,0)
\put(64,-451.811){\fontsize{12}{1}\usefont{T1}{cmr}{m}{n}\selectfont\color{color_29791}  }
\end{picture}
\begin{tikzpicture}[overlay]
\path(0pt,0pt);
\draw[color_29791,line width=0.7pt]
(64pt, -448.411pt) -- (77.7pt, -448.411pt)
;
\end{tikzpicture}
\begin{picture}(-5,0)(2.5,0)
\put(77.8,-451.811){\fontsize{12}{1}\usefont{T1}{cmr}{b}{n}\selectfont\color{color_29791}service discovery (DNS-SD):}
\end{picture}
\begin{tikzpicture}[overlay]
\path(0pt,0pt);
\draw[color_29791,line width=0.7pt]
(77.8pt, -448.411pt) -- (223.1pt, -448.411pt)
;
\end{tikzpicture}
\begin{picture}(-5,0)(2.5,0)
\put(223,-451.811){\fontsize{12}{1}\usefont{T1}{cmr}{b}{n}\selectfont\color{color_29791}          }
\end{picture}
\begin{tikzpicture}[overlay]
\path(0pt,0pt);
\draw[color_29791,line width=0.7pt]
(223pt, -448.411pt) -- (255pt, -448.411pt)
;
\end{tikzpicture}
\begin{picture}(-5,0)(2.5,0)
\put(255.1,-451.811){\fontsize{12}{1}\usefont{T1}{cmr}{m}{n}\selectfont\color{color_29791}basato su server DNS aggiornabile dinamicamente per }
\end{picture}
\begin{tikzpicture}[overlay]
\path(0pt,0pt);
\draw[color_29791,line width=0.7pt]
(255.1pt, -448.411pt) -- (519.3pt, -448.411pt)
;
\end{tikzpicture}
\begin{picture}(-5,0)(2.5,0)
\put(77.8,-465.611){\fontsize{12}{1}\usefont{T1}{cmr}{m}{n}\selectfont\color{color_29791}registrare servizi, per scoprire nomi di servizi anche senza un DNS indicato specificamente. }
\end{picture}
\begin{tikzpicture}[overlay]
\path(0pt,0pt);
\draw[color_29791,line width=0.7pt]
(77.8pt, -462.211pt) -- (521.3pt, -462.211pt)
;
\end{tikzpicture}
\begin{picture}(-5,0)(2.5,0)
\put(77.8,-479.411){\fontsize{12}{1}\usefont{T1}{cmr}{m}{n}\selectfont\color{color_29791}Offre la rilevazione automatica di servizi disponibili in rete, stabilendo un formato di entry }
\end{picture}
\begin{tikzpicture}[overlay]
\path(0pt,0pt);
\draw[color_29791,line width=0.7pt]
(77.8pt, -476.011pt) -- (516.4pt, -476.011pt)
;
\end{tikzpicture}
\begin{picture}(-5,0)(2.5,0)
\put(77.8,-493.211){\fontsize{12}{1}\usefont{T1}{cmr}{m}{n}\selectfont\color{color_29791}DNS per descrivere la collocazione in rete, i protocolli applicativi ed eventuali parametri da }
\end{picture}
\begin{tikzpicture}[overlay]
\path(0pt,0pt);
\draw[color_29791,line width=0.7pt]
(77.8pt, -489.811pt) -- (521.2pt, -489.811pt)
;
\end{tikzpicture}
\begin{picture}(-5,0)(2.5,0)
\put(77.8,-507.011){\fontsize{12}{1}\usefont{T1}{cmr}{m}{n}\selectfont\color{color_29791}utilizzare.}
\end{picture}
\begin{tikzpicture}[overlay]
\path(0pt,0pt);
\draw[color_29791,line width=0.7pt]
(77.8pt, -503.611pt) -- (125.4pt, -503.611pt)
;
\end{tikzpicture}
\begin{picture}(-5,0)(2.5,0)
\put(41.8,-527.811){\fontsize{12}{1}\usefont{T1}{cmr}{m}{n}\selectfont\color{color_29791}Implementazioni comuni di zeroconf sono Apple Bonjour ( mDNS e SD), e Microsoft APIPA (solo }
\end{picture}
\begin{tikzpicture}[overlay]
\path(0pt,0pt);
\draw[color_29791,line width=0.7pt]
(41.8pt, -524.411pt) -- (520pt, -524.411pt)
;
\end{tikzpicture}
\begin{picture}(-5,0)(2.5,0)
\put(41.8,-541.611){\fontsize{12}{1}\usefont{T1}{cmr}{m}{n}\selectfont\color{color_29791}link-local addressing).}
\end{picture}
\begin{tikzpicture}[overlay]
\path(0pt,0pt);
\draw[color_29791,line width=0.7pt]
(41.8pt, -538.211pt) -- (149.1pt, -538.211pt)
;
\end{tikzpicture}
\begin{picture}(-5,0)(2.5,0)
\put(41.8,-571.411){\fontsize{14.1}{1}\usefont{T1}{cmr}{b}{n}\selectfont\color{color_29791}Sincronizzazione}
\put(41.8,-591.611){\fontsize{12}{1}\usefont{T1}{cmr}{m}{n}\selectfont\color{color_29791}L'allineamento dell'ora di un sistema ad un orologio di riferimento è molto importante, ad esempio }
\put(41.8,-605.411){\fontsize{12}{1}\usefont{T1}{cmr}{m}{n}\selectfont\color{color_29791}per la diagnostica dei problemi (timestamp su log), o per i protocolli di autenticazione e }
\put(41.8,-619.211){\fontsize{12}{1}\usefont{T1}{cmr}{m}{n}\selectfont\color{color_29791}autorizzazione (spesso i protocolli di sicurezza si basano sulla validità temporale dei messaggi, }
\put(41.8,-633.011){\fontsize{12}{1}\usefont{T1}{cmr}{m}{n}\selectfont\color{color_29791}token: es. per sapere se c'è un dato utente dietro alla tastiera, si usa un token per certificare che sia }
\put(41.8,-646.811){\fontsize{12}{1}\usefont{T1}{cmr}{m}{n}\selectfont\color{color_29791}ancora là, ed è importante verificare il tempo in maniera corretta). E' possibile usare un protocollo, }
\put(41.8,-660.611){\fontsize{12}{1}\usefont{T1}{cmr}{m}{n}\selectfont\color{color_29791}detto NTP (Network Time Protocol), che compensa i ritardi di rete per ottenere informazioni precise}
\put(41.8,-674.411){\fontsize{12}{1}\usefont{T1}{cmr}{m}{n}\selectfont\color{color_29791}via internet. NTP è preciso (poche decine di ms di scarto su WAN, <1 ms su LAN, supporta sorgenti}
\put(41.8,-688.211){\fontsize{12}{1}\usefont{T1}{cmr}{m}{n}\selectfont\color{color_29791}HW come oscillatori e GPS); standard (portato su ogni architettura), scalabile e affidabile (è multi-}
\put(41.8,-702.011){\fontsize{12}{1}\usefont{T1}{cmr}{m}{n}\selectfont\color{color_29791}server, prevede stratificazione per permettere servizio a un grande numero di nodi (tipicamente non }
\put(41.8,-715.811){\fontsize{12}{1}\usefont{T1}{cmr}{m}{n}\selectfont\color{color_29791}più di 2-3 strati; inoltre si può avere peering -i server allo stesso strato verificano l'uno con l'altro di }
\put(41.8,-729.611){\fontsize{12}{1}\usefont{T1}{cmr}{m}{n}\selectfont\color{color_29791}essere allineati-).}
\end{picture}
\newpage
\begin{tikzpicture}[overlay]\path(0pt,0pt);\end{tikzpicture}
\begin{picture}(-5,0)(2.5,0)
\put(41.8,-87.01099){\fontsize{14.1}{1}\usefont{T1}{cmr}{b}{n}\selectfont\color{color_29791}Strumenti Linux per i servizi di rete}
\put(41.8,-107.211){\fontsize{12}{1}\usefont{T1}{cmr}{m}{n}\selectfont\color{color_29791}Linux offre diverse possibilità per la gestione dei servizi di rete:}
\end{picture}
\begin{tikzpicture}[overlay]
\path(0pt,0pt);
\draw[color_29791,line width=0.7pt]
(41.8pt, -103.811pt) -- (348.8pt, -103.811pt)
;
\end{tikzpicture}
\begin{picture}(-5,0)(2.5,0)
\put(348.7,-107.211){\fontsize{12}{1}\usefont{T1}{cmr}{m}{n}\selectfont\color{color_29791}    }
\end{picture}
\begin{tikzpicture}[overlay]
\path(0pt,0pt);
\draw[color_29791,line width=0.7pt]
(348.7pt, -103.811pt) -- (360.8pt, -103.811pt)
;
\end{tikzpicture}
\begin{picture}(-5,0)(2.5,0)
\put(360.9,-107.211){\fontsize{12}{1}\usefont{T1}{cmr}{m}{n}\selectfont\color{color_29791}(nelle slide, link ai vari manuali)}
\end{picture}
\begin{tikzpicture}[overlay]
\path(0pt,0pt);
\draw[color_29791,line width=0.7pt]
(360.9pt, -103.811pt) -- (517.5pt, -103.811pt)
;
\end{tikzpicture}
\begin{picture}(-5,0)(2.5,0)
\put(59.8,-128.011){\fontsize{12}{1}\usefont{T1}{cmr}{m}{n}\selectfont\color{color_29791}•}
\end{picture}
\begin{tikzpicture}[overlay]
\path(0pt,0pt);
\draw[color_29791,line width=0.7pt]
(59.8pt, -124.611pt) -- (64.10001pt, -124.611pt)
;
\end{tikzpicture}
\begin{picture}(-5,0)(2.5,0)
\put(64,-128.011){\fontsize{12}{1}\usefont{T1}{cmr}{m}{n}\selectfont\color{color_29791}  }
\end{picture}
\begin{tikzpicture}[overlay]
\path(0pt,0pt);
\draw[color_29791,line width=0.7pt]
(64pt, -124.611pt) -- (77.7pt, -124.611pt)
;
\end{tikzpicture}
\begin{picture}(-5,0)(2.5,0)
\put(77.8,-128.011){\fontsize{12}{1}\usefont{T1}{cmr}{m}{n}\selectfont\color{color_29791}Client side:}
\end{picture}
\begin{tikzpicture}[overlay]
\path(0pt,0pt);
\draw[color_29791,line width=0.7pt]
(77.8pt, -124.611pt) -- (132.8pt, -124.611pt)
;
\end{tikzpicture}
\begin{picture}(-5,0)(2.5,0)
\put(77.8,-148.811){\fontsize{12}{1}\usefont{T1}{cmr}{m}{n}\selectfont\color{color_29791}◦}
\end{picture}
\begin{tikzpicture}[overlay]
\path(0pt,0pt);
\draw[color_29791,line width=0.7pt]
(77.8pt, -145.4109pt) -- (87.3pt, -145.4109pt)
;
\end{tikzpicture}
\begin{picture}(-5,0)(2.5,0)
\put(87.3,-148.811){\fontsize{12}{1}\usefont{T1}{cmr}{m}{n}\selectfont\color{color_29791} }
\end{picture}
\begin{tikzpicture}[overlay]
\path(0pt,0pt);
\draw[color_29791,line width=0.7pt]
(87.3pt, -145.4109pt) -- (93.3pt, -145.4109pt)
;
\end{tikzpicture}
\begin{picture}(-5,0)(2.5,0)
\put(95.8,-148.811){\fontsize{12}{1}\usefont{T1}{cmr}{m}{n}\selectfont\color{color_29791}Avahi (link-local, mDNS, DNS-SD)}
\end{picture}
\begin{tikzpicture}[overlay]
\path(0pt,0pt);
\draw[color_29791,line width=0.7pt]
(95.8pt, -145.4109pt) -- (269.9pt, -145.4109pt)
;
\end{tikzpicture}
\begin{picture}(-5,0)(2.5,0)
\put(77.8,-169.611){\fontsize{12}{1}\usefont{T1}{cmr}{m}{n}\selectfont\color{color_29791}◦}
\end{picture}
\begin{tikzpicture}[overlay]
\path(0pt,0pt);
\draw[color_29791,line width=0.7pt]
(77.8pt, -166.211pt) -- (87.3pt, -166.211pt)
;
\end{tikzpicture}
\begin{picture}(-5,0)(2.5,0)
\put(87.3,-169.611){\fontsize{12}{1}\usefont{T1}{cmr}{m}{n}\selectfont\color{color_29791} }
\end{picture}
\begin{tikzpicture}[overlay]
\path(0pt,0pt);
\draw[color_29791,line width=0.7pt]
(87.3pt, -166.211pt) -- (93.3pt, -166.211pt)
;
\end{tikzpicture}
\begin{picture}(-5,0)(2.5,0)
\put(95.8,-169.611){\fontsize{12}{1}\usefont{T1}{cmr}{m}{n}\selectfont\color{color_29791}ISC dhcp (client DHCP per acquisire info su servizi di rete (e info simili))}
\end{picture}
\begin{tikzpicture}[overlay]
\path(0pt,0pt);
\draw[color_29791,line width=0.7pt]
(95.8pt, -166.211pt) -- (450.9pt, -166.211pt)
;
\end{tikzpicture}
\begin{picture}(-5,0)(2.5,0)
\put(77.8,-190.411){\fontsize{12}{1}\usefont{T1}{cmr}{m}{n}\selectfont\color{color_29791}◦}
\end{picture}
\begin{tikzpicture}[overlay]
\path(0pt,0pt);
\draw[color_29791,line width=0.7pt]
(77.8pt, -187.011pt) -- (87.3pt, -187.011pt)
;
\end{tikzpicture}
\begin{picture}(-5,0)(2.5,0)
\put(87.3,-190.411){\fontsize{12}{1}\usefont{T1}{cmr}{m}{n}\selectfont\color{color_29791} }
\end{picture}
\begin{tikzpicture}[overlay]
\path(0pt,0pt);
\draw[color_29791,line width=0.7pt]
(87.3pt, -187.011pt) -- (93.3pt, -187.011pt)
;
\end{tikzpicture}
\begin{picture}(-5,0)(2.5,0)
\put(95.8,-190.411){\fontsize{12}{1}\usefont{T1}{cmr}{m}{n}\selectfont\color{color_29791}systemd.network (ogni possibile modo di configurare la rete, incluso link-local)}
\end{picture}
\begin{tikzpicture}[overlay]
\path(0pt,0pt);
\draw[color_29791,line width=0.7pt]
(95.8pt, -187.011pt) -- (478.7pt, -187.011pt)
;
\end{tikzpicture}
\begin{picture}(-5,0)(2.5,0)
\put(77.8,-211.211){\fontsize{12}{1}\usefont{T1}{cmr}{m}{n}\selectfont\color{color_29791}◦}
\end{picture}
\begin{tikzpicture}[overlay]
\path(0pt,0pt);
\draw[color_29791,line width=0.7pt]
(77.8pt, -207.811pt) -- (87.3pt, -207.811pt)
;
\end{tikzpicture}
\begin{picture}(-5,0)(2.5,0)
\put(87.3,-211.211){\fontsize{12}{1}\usefont{T1}{cmr}{m}{n}\selectfont\color{color_29791} }
\end{picture}
\begin{tikzpicture}[overlay]
\path(0pt,0pt);
\draw[color_29791,line width=0.7pt]
(87.3pt, -207.811pt) -- (93.3pt, -207.811pt)
;
\end{tikzpicture}
\begin{picture}(-5,0)(2.5,0)
\put(95.8,-211.211){\fontsize{12}{1}\usefont{T1}{cmr}{m}{n}\selectfont\color{color_29791}systemd-resolved (DNS/mDNS resolver, sostituisce resolvconf)}
\end{picture}
\begin{tikzpicture}[overlay]
\path(0pt,0pt);
\draw[color_29791,line width=0.7pt]
(95.8pt, -207.811pt) -- (402.2pt, -207.811pt)
;
\end{tikzpicture}
\begin{picture}(-5,0)(2.5,0)
\put(77.8,-232.011){\fontsize{12}{1}\usefont{T1}{cmr}{m}{n}\selectfont\color{color_29791}◦}
\end{picture}
\begin{tikzpicture}[overlay]
\path(0pt,0pt);
\draw[color_29791,line width=0.7pt]
(77.8pt, -228.611pt) -- (87.3pt, -228.611pt)
;
\end{tikzpicture}
\begin{picture}(-5,0)(2.5,0)
\put(87.3,-232.011){\fontsize{12}{1}\usefont{T1}{cmr}{m}{n}\selectfont\color{color_29791} }
\end{picture}
\begin{tikzpicture}[overlay]
\path(0pt,0pt);
\draw[color_29791,line width=0.7pt]
(87.3pt, -228.611pt) -- (93.3pt, -228.611pt)
;
\end{tikzpicture}
\begin{picture}(-5,0)(2.5,0)
\put(95.8,-232.011){\fontsize{12}{1}\usefont{T1}{cmr}{m}{n}\selectfont\color{color_29791}ntpd /ntpdate}
\end{picture}
\begin{tikzpicture}[overlay]
\path(0pt,0pt);
\draw[color_29791,line width=0.7pt]
(95.8pt, -228.611pt) -- (158.8pt, -228.611pt)
;
\end{tikzpicture}
\begin{picture}(-5,0)(2.5,0)
\put(59.8,-252.811){\fontsize{12}{1}\usefont{T1}{cmr}{m}{n}\selectfont\color{color_29791}•}
\end{picture}
\begin{tikzpicture}[overlay]
\path(0pt,0pt);
\draw[color_29791,line width=0.7pt]
(59.8pt, -249.4109pt) -- (64.10001pt, -249.4109pt)
;
\end{tikzpicture}
\begin{picture}(-5,0)(2.5,0)
\put(64,-252.811){\fontsize{12}{1}\usefont{T1}{cmr}{m}{n}\selectfont\color{color_29791}  }
\end{picture}
\begin{tikzpicture}[overlay]
\path(0pt,0pt);
\draw[color_29791,line width=0.7pt]
(64pt, -249.4109pt) -- (77.7pt, -249.4109pt)
;
\end{tikzpicture}
\begin{picture}(-5,0)(2.5,0)
\put(77.8,-252.811){\fontsize{12}{1}\usefont{T1}{cmr}{m}{n}\selectfont\color{color_29791}Server side:}
\end{picture}
\begin{tikzpicture}[overlay]
\path(0pt,0pt);
\draw[color_29791,line width=0.7pt]
(77.8pt, -249.4109pt) -- (134.8pt, -249.4109pt)
;
\end{tikzpicture}
\begin{picture}(-5,0)(2.5,0)
\put(77.8,-273.611){\fontsize{12}{1}\usefont{T1}{cmr}{m}{n}\selectfont\color{color_29791}◦}
\end{picture}
\begin{tikzpicture}[overlay]
\path(0pt,0pt);
\draw[color_29791,line width=0.7pt]
(77.8pt, -270.211pt) -- (87.3pt, -270.211pt)
;
\end{tikzpicture}
\begin{picture}(-5,0)(2.5,0)
\put(87.3,-273.611){\fontsize{12}{1}\usefont{T1}{cmr}{m}{n}\selectfont\color{color_29791} }
\end{picture}
\begin{tikzpicture}[overlay]
\path(0pt,0pt);
\draw[color_29791,line width=0.7pt]
(87.3pt, -270.211pt) -- (93.3pt, -270.211pt)
;
\end{tikzpicture}
\begin{picture}(-5,0)(2.5,0)
\put(95.8,-273.611){\fontsize{12}{1}\usefont{T1}{cmr}{m}{n}\selectfont\color{color_29791}dnsmasq (DCHP, DNS) }
\end{picture}
\begin{tikzpicture}[overlay]
\path(0pt,0pt);
\draw[color_29791,line width=0.7pt]
(95.8pt, -270.211pt) -- (212.5pt, -270.211pt)
;
\end{tikzpicture}
\begin{picture}(-5,0)(2.5,0)
\put(95.8,-294.411){\fontsize{12}{1}\usefont{T1}{cmr}{m}{n}\selectfont\color{color_29791}▪}
\end{picture}
\begin{tikzpicture}[overlay]
\path(0pt,0pt);
\draw[color_29791,line width=0.7pt]
(95.8pt, -291.011pt) -- (104.7pt, -291.011pt)
;
\end{tikzpicture}
\begin{picture}(-5,0)(2.5,0)
\put(104.7,-294.411){\fontsize{12}{1}\usefont{T1}{cmr}{m}{n}\selectfont\color{color_29791} }
\end{picture}
\begin{tikzpicture}[overlay]
\path(0pt,0pt);
\draw[color_29791,line width=0.7pt]
(104.7pt, -291.011pt) -- (110.7pt, -291.011pt)
;
\end{tikzpicture}
\begin{picture}(-5,0)(2.5,0)
\put(113.8,-294.411){\fontsize{12}{1}\usefont{T1}{cmr}{m}{n}\selectfont\color{color_29791}inoltre PXE (Pre-boot eXecution Environment) e TFTP (in laboratorio dell'uni c'è un}
\end{picture}
\begin{tikzpicture}[overlay]
\path(0pt,0pt);
\draw[color_29791,line width=0.7pt]
(113.8pt, -291.011pt) -- (521pt, -291.011pt)
;
\end{tikzpicture}
\begin{picture}(-5,0)(2.5,0)
\put(113.8,-308.211){\fontsize{12}{1}\usefont{T1}{cmr}{m}{n}\selectfont\color{color_29791}server dnsmasq che permette di avere avvio sistemi diskless mediante PXE, TFTP }
\end{picture}
\begin{tikzpicture}[overlay]
\path(0pt,0pt);
\draw[color_29791,line width=0.7pt]
(113.8pt, -304.811pt) -- (510.7pt, -304.811pt)
;
\end{tikzpicture}
\begin{picture}(-5,0)(2.5,0)
\put(113.8,-322.011){\fontsize{12}{1}\usefont{T1}{cmr}{m}{n}\selectfont\color{color_29791}viene usato per trasferire l'iso)}
\end{picture}
\begin{tikzpicture}[overlay]
\path(0pt,0pt);
\draw[color_29791,line width=0.7pt]
(113.8pt, -318.611pt) -- (259.2pt, -318.611pt)
;
\end{tikzpicture}
\begin{picture}(-5,0)(2.5,0)
\put(77.8,-342.811){\fontsize{12}{1}\usefont{T1}{cmr}{m}{n}\selectfont\color{color_29791}◦}
\end{picture}
\begin{tikzpicture}[overlay]
\path(0pt,0pt);
\draw[color_29791,line width=0.7pt]
(77.8pt, -339.411pt) -- (87.3pt, -339.411pt)
;
\end{tikzpicture}
\begin{picture}(-5,0)(2.5,0)
\put(87.3,-342.811){\fontsize{12}{1}\usefont{T1}{cmr}{m}{n}\selectfont\color{color_29791} }
\end{picture}
\begin{tikzpicture}[overlay]
\path(0pt,0pt);
\draw[color_29791,line width=0.7pt]
(87.3pt, -339.411pt) -- (93.3pt, -339.411pt)
;
\end{tikzpicture}
\begin{picture}(-5,0)(2.5,0)
\put(95.8,-342.811){\fontsize{12}{1}\usefont{T1}{cmr}{m}{n}\selectfont\color{color_29791}systemd.dnnsd (DNS-SD)}
\end{picture}
\begin{tikzpicture}[overlay]
\path(0pt,0pt);
\draw[color_29791,line width=0.7pt]
(95.8pt, -339.411pt) -- (221.1pt, -339.411pt)
;
\end{tikzpicture}
\begin{picture}(-5,0)(2.5,0)
\put(77.8,-363.611){\fontsize{12}{1}\usefont{T1}{cmr}{m}{n}\selectfont\color{color_29791}◦}
\end{picture}
\begin{tikzpicture}[overlay]
\path(0pt,0pt);
\draw[color_29791,line width=0.7pt]
(77.8pt, -360.211pt) -- (87.3pt, -360.211pt)
;
\end{tikzpicture}
\begin{picture}(-5,0)(2.5,0)
\put(87.3,-363.611){\fontsize{12}{1}\usefont{T1}{cmr}{m}{n}\selectfont\color{color_29791} }
\end{picture}
\begin{tikzpicture}[overlay]
\path(0pt,0pt);
\draw[color_29791,line width=0.7pt]
(87.3pt, -360.211pt) -- (93.3pt, -360.211pt)
;
\end{tikzpicture}
\begin{picture}(-5,0)(2.5,0)
\put(95.8,-363.611){\fontsize{12}{1}\usefont{T1}{cmr}{m}{n}\selectfont\color{color_29791}ntpd}
\end{picture}
\begin{tikzpicture}[overlay]
\path(0pt,0pt);
\draw[color_29791,line width=0.7pt]
(95.8pt, -360.211pt) -- (117.1pt, -360.211pt)
;
\end{tikzpicture}
\begin{picture}(-5,0)(2.5,0)
\put(41.8,-384.411){\fontsize{12}{1}\usefont{T1}{cmr}{m}{n}\selectfont\color{color_29791}Ci occuperemo principalmente di configurare }
\end{picture}
\begin{tikzpicture}[overlay]
\path(0pt,0pt);
\draw[color_29791,line width=0.7pt]
(41.8pt, -381.011pt) -- (263.4pt, -381.011pt)
;
\end{tikzpicture}
\begin{picture}(-5,0)(2.5,0)
\put(263.4,-384.411){\fontsize{12}{1}\usefont{T1}{cmr}{b}{n}\selectfont\color{color_29791}dnsmasq}
\end{picture}
\begin{tikzpicture}[overlay]
\path(0pt,0pt);
\draw[color_29791,line width=0.7pt]
(263.4pt, -381.011pt) -- (308.7pt, -381.011pt)
;
\end{tikzpicture}
\begin{picture}(-5,0)(2.5,0)
\put(308.8,-384.411){\fontsize{12}{1}\usefont{T1}{cmr}{m}{n}\selectfont\color{color_29791}: su reti di dimensioni ridotte è una scelta }
\end{picture}
\begin{tikzpicture}[overlay]
\path(0pt,0pt);
\draw[color_29791,line width=0.7pt]
(308.8pt, -381.011pt) -- (509.7pt, -381.011pt)
;
\end{tikzpicture}
\begin{picture}(-5,0)(2.5,0)
\put(41.8,-398.211){\fontsize{12}{1}\usefont{T1}{cmr}{m}{n}\selectfont\color{color_29791}pratica per fornire i servizi necessari all'avvio zeroconf. Configurazione generale si fa su file }
\end{picture}
\begin{tikzpicture}[overlay]
\path(0pt,0pt);
\draw[color_29791,line width=0.7pt]
(41.8pt, -394.811pt) -- (488.7pt, -394.811pt)
;
\end{tikzpicture}
\begin{picture}(-5,0)(2.5,0)
\put(41.7,-412.011){\fontsize{12}{1}\usefont{T1}{cmr}{m}{n}\selectfont\color{color_29791}           }
\end{picture}
\begin{tikzpicture}[overlay]
\path(0pt,0pt);
\draw[color_29791,line width=0.7pt]
(41.7pt, -408.611pt) -- (77.2pt, -408.611pt)
;
\end{tikzpicture}
\begin{picture}(-5,0)(2.5,0)
\put(77.3,-412.011){\fontsize{12}{1}\usefont{T1}{cmr}{b}{it}\selectfont\color{color_29791}/etc/dnsmasq.conf}
\end{picture}
\begin{tikzpicture}[overlay]
\path(0pt,0pt);
\draw[color_29791,line width=0.7pt]
(77.3pt, -408.611pt) -- (166.3pt, -408.611pt)
;
\end{tikzpicture}
\begin{picture}(-5,0)(2.5,0)
\put(166.2,-412.011){\fontsize{12}{1}\usefont{T1}{cmr}{m}{n}\selectfont\color{color_29791}     }
\end{picture}
\begin{tikzpicture}[overlay]
\path(0pt,0pt);
\draw[color_29791,line width=0.7pt]
(166.2pt, -408.611pt) -- (183.6pt, -408.611pt)
;
\end{tikzpicture}
\begin{picture}(-5,0)(2.5,0)
\put(183.6,-412.011){\fontsize{12}{1}\usefont{T1}{cmr}{m}{n}\selectfont\color{color_29791}, dove possiamo specificare su quali interfacce esporre dei servizi.}
\end{picture}
\begin{tikzpicture}[overlay]
\path(0pt,0pt);
\draw[color_29791,line width=0.7pt]
(183.6pt, -408.611pt) -- (501.1pt, -408.611pt)
;
\end{tikzpicture}
\begin{picture}(-5,0)(2.5,0)
\put(41.7,-432.811){\fontsize{12}{1}\usefont{T1}{cmr}{m}{n}\selectfont\color{color_29791}           }
\end{picture}
\begin{tikzpicture}[overlay]
\path(0pt,0pt);
\draw[color_29791,line width=0.7pt]
(41.7pt, -429.411pt) -- (77.2pt, -429.411pt)
;
\end{tikzpicture}
\begin{picture}(-5,0)(2.5,0)
\put(77.3,-432.811){\fontsize{12}{1}\usefont{T1}{cmr}{m}{n}\selectfont\color{color_29791}(se all'esame, ci viene chiesto di configurare dnsmasq, consegnare un file con SOLO le righe}
\end{picture}
\begin{tikzpicture}[overlay]
\path(0pt,0pt);
\draw[color_29791,line width=0.7pt]
(77.3pt, -429.411pt) -- (522.6pt, -429.411pt)
;
\end{tikzpicture}
\begin{picture}(-5,0)(2.5,0)
\put(41.8,-446.611){\fontsize{12}{1}\usefont{T1}{cmr}{m}{n}\selectfont\color{color_29791}utili, invece di quello con 8k righe dove controllare quali sono state scommentate e quali sono }
\end{picture}
\begin{tikzpicture}[overlay]
\path(0pt,0pt);
\draw[color_29791,line width=0.7pt]
(41.8pt, -443.211pt) -- (496pt, -443.211pt)
;
\end{tikzpicture}
\begin{picture}(-5,0)(2.5,0)
\put(41.8,-460.411){\fontsize{12}{1}\usefont{T1}{cmr}{m}{n}\selectfont\color{color_29791}commentate...)}
\end{picture}
\begin{tikzpicture}[overlay]
\path(0pt,0pt);
\draw[color_29791,line width=0.7pt]
(41.8pt, -457.011pt) -- (113.4pt, -457.011pt)
;
\end{tikzpicture}
\begin{picture}(-5,0)(2.5,0)
\put(41.8,-481.211){\fontsize{12}{1}\usefont{T1}{cmr}{m}{n}\selectfont\color{color_29791}Opzioni base:}
\end{picture}
\begin{tikzpicture}[overlay]
\path(0pt,0pt);
\draw[color_29791,line width=0.7pt]
(41.8pt, -477.811pt) -- (108.1pt, -477.811pt)
;
\end{tikzpicture}
\begin{picture}(-5,0)(2.5,0)
\put(59.8,-502.011){\fontsize{12}{1}\usefont{T1}{cmr}{m}{n}\selectfont\color{color_29791}•}
\end{picture}
\begin{tikzpicture}[overlay]
\path(0pt,0pt);
\draw[color_29791,line width=0.7pt]
(59.8pt, -498.611pt) -- (64.10001pt, -498.611pt)
;
\end{tikzpicture}
\begin{picture}(-5,0)(2.5,0)
\put(64,-502.011){\fontsize{12}{1}\usefont{T1}{cmr}{m}{n}\selectfont\color{color_29791}  }
\end{picture}
\begin{tikzpicture}[overlay]
\path(0pt,0pt);
\draw[color_29791,line width=0.7pt]
(64pt, -498.611pt) -- (77.7pt, -498.611pt)
;
\end{tikzpicture}
\begin{picture}(-5,0)(2.5,0)
\put(77.8,-502.011){\fontsize{12}{1}\usefont{T1}{cmr}{m}{it}\selectfont\color{color_29791}bind-interfaces}
\end{picture}
\begin{tikzpicture}[overlay]
\path(0pt,0pt);
\draw[color_29791,line width=0.7pt]
(77.8pt, -498.611pt) -- (150.4pt, -498.611pt)
;
\end{tikzpicture}
\begin{picture}(-5,0)(2.5,0)
\put(150.4,-502.011){\fontsize{12}{1}\usefont{T1}{cmr}{m}{it}\selectfont\color{color_29791}           }
\end{picture}
\begin{tikzpicture}[overlay]
\path(0pt,0pt);
\draw[color_29791,line width=0.7pt]
(150.4pt, -498.611pt) -- (184.1pt, -498.611pt)
;
\end{tikzpicture}
\begin{picture}(-5,0)(2.5,0)
\put(184.2,-502.011){\fontsize{12}{1}\usefont{T1}{cmr}{m}{n}\selectfont\color{color_29791}evita conflitti in caso si vogliano usare più istanze di dnsmasq per reti }
\end{picture}
\begin{tikzpicture}[overlay]
\path(0pt,0pt);
\draw[color_29791,line width=0.7pt]
(184.2pt, -498.611pt) -- (521.7pt, -498.611pt)
;
\end{tikzpicture}
\begin{picture}(-5,0)(2.5,0)
\put(77.8,-515.811){\fontsize{12}{1}\usefont{T1}{cmr}{m}{n}\selectfont\color{color_29791}diverse connesse al server}
\end{picture}
\begin{tikzpicture}[overlay]
\path(0pt,0pt);
\draw[color_29791,line width=0.7pt]
(77.8pt, -512.411pt) -- (202.7pt, -512.411pt)
;
\end{tikzpicture}
\begin{picture}(-5,0)(2.5,0)
\put(59.8,-536.611){\fontsize{12}{1}\usefont{T1}{cmr}{m}{n}\selectfont\color{color_29791}•}
\end{picture}
\begin{tikzpicture}[overlay]
\path(0pt,0pt);
\draw[color_29791,line width=0.7pt]
(59.8pt, -533.211pt) -- (64.10001pt, -533.211pt)
;
\end{tikzpicture}
\begin{picture}(-5,0)(2.5,0)
\put(64,-536.611){\fontsize{12}{1}\usefont{T1}{cmr}{m}{n}\selectfont\color{color_29791}  }
\end{picture}
\begin{tikzpicture}[overlay]
\path(0pt,0pt);
\draw[color_29791,line width=0.7pt]
(64pt, -533.211pt) -- (77.7pt, -533.211pt)
;
\end{tikzpicture}
\begin{picture}(-5,0)(2.5,0)
\put(77.8,-536.611){\fontsize{12}{1}\usefont{T1}{cmr}{m}{it}\selectfont\color{color_29791}interface=<interface name> }
\end{picture}
\begin{tikzpicture}[overlay]
\path(0pt,0pt);
\draw[color_29791,line width=0.7pt]
(77.8pt, -533.211pt) -- (219.4pt, -533.211pt)
;
\end{tikzpicture}
\begin{picture}(-5,0)(2.5,0)
\put(219.4,-536.611){\fontsize{12}{1}\usefont{T1}{cmr}{m}{n}\selectfont\color{color_29791}e}
\end{picture}
\begin{tikzpicture}[overlay]
\path(0pt,0pt);
\draw[color_29791,line width=0.7pt]
(219.4pt, -533.211pt) -- (224.7pt, -533.211pt)
;
\end{tikzpicture}
\begin{picture}(-5,0)(2.5,0)
\put(59.8,-557.411){\fontsize{12}{1}\usefont{T1}{cmr}{m}{n}\selectfont\color{color_29791}•}
\end{picture}
\begin{tikzpicture}[overlay]
\path(0pt,0pt);
\draw[color_29791,line width=0.7pt]
(59.8pt, -554.011pt) -- (64.10001pt, -554.011pt)
;
\end{tikzpicture}
\begin{picture}(-5,0)(2.5,0)
\put(64,-557.411){\fontsize{12}{1}\usefont{T1}{cmr}{m}{n}\selectfont\color{color_29791}  }
\end{picture}
\begin{tikzpicture}[overlay]
\path(0pt,0pt);
\draw[color_29791,line width=0.7pt]
(64pt, -554.011pt) -- (77.7pt, -554.011pt)
;
\end{tikzpicture}
\begin{picture}(-5,0)(2.5,0)
\put(77.8,-557.411){\fontsize{12}{1}\usefont{T1}{cmr}{m}{it}\selectfont\color{color_29791}listen-address=<ipaddr>}
\end{picture}
\begin{tikzpicture}[overlay]
\path(0pt,0pt);
\draw[color_29791,line width=0.7pt]
(77.8pt, -554.011pt) -- (201pt, -554.011pt)
;
\end{tikzpicture}
\begin{picture}(-5,0)(2.5,0)
\put(200.9,-557.411){\fontsize{12}{1}\usefont{T1}{cmr}{m}{it}\selectfont\color{color_29791}      }
\end{picture}
\begin{tikzpicture}[overlay]
\path(0pt,0pt);
\draw[color_29791,line width=0.7pt]
(200.9pt, -554.011pt) -- (219.5pt, -554.011pt)
;
\end{tikzpicture}
\begin{picture}(-5,0)(2.5,0)
\put(219.6,-557.411){\fontsize{12}{1}\usefont{T1}{cmr}{m}{n}\selectfont\color{color_29791}mettono dnsmasq in ascolto solo sull'interfaccia o indirizzo }
\end{picture}
\begin{tikzpicture}[overlay]
\path(0pt,0pt);
\draw[color_29791,line width=0.7pt]
(219.6pt, -554.011pt) -- (507pt, -554.011pt)
;
\end{tikzpicture}
\begin{picture}(-5,0)(2.5,0)
\put(77.8,-571.211){\fontsize{12}{1}\usefont{T1}{cmr}{m}{n}\selectfont\color{color_29791}indicati (anche più di una/uno)}
\end{picture}
\begin{tikzpicture}[overlay]
\path(0pt,0pt);
\draw[color_29791,line width=0.7pt]
(77.8pt, -567.811pt) -- (225.1pt, -567.811pt)
;
\end{tikzpicture}
\begin{picture}(-5,0)(2.5,0)
\put(59.8,-592.011){\fontsize{12}{1}\usefont{T1}{cmr}{m}{n}\selectfont\color{color_29791}•}
\end{picture}
\begin{tikzpicture}[overlay]
\path(0pt,0pt);
\draw[color_29791,line width=0.7pt]
(59.8pt, -588.611pt) -- (64.10001pt, -588.611pt)
;
\end{tikzpicture}
\begin{picture}(-5,0)(2.5,0)
\put(64,-592.011){\fontsize{12}{1}\usefont{T1}{cmr}{m}{n}\selectfont\color{color_29791}  }
\end{picture}
\begin{tikzpicture}[overlay]
\path(0pt,0pt);
\draw[color_29791,line width=0.7pt]
(64pt, -588.611pt) -- (77.7pt, -588.611pt)
;
\end{tikzpicture}
\begin{picture}(-5,0)(2.5,0)
\put(77.8,-592.011){\fontsize{12}{1}\usefont{T1}{cmr}{m}{it}\selectfont\color{color_29791}user }
\end{picture}
\begin{tikzpicture}[overlay]
\path(0pt,0pt);
\draw[color_29791,line width=0.7pt]
(77.8pt, -588.611pt) -- (101.4pt, -588.611pt)
;
\end{tikzpicture}
\begin{picture}(-5,0)(2.5,0)
\put(101.5,-592.011){\fontsize{12}{1}\usefont{T1}{cmr}{m}{n}\selectfont\color{color_29791}/ }
\end{picture}
\begin{tikzpicture}[overlay]
\path(0pt,0pt);
\draw[color_29791,line width=0.7pt]
(101.5pt, -588.611pt) -- (107.8pt, -588.611pt)
;
\end{tikzpicture}
\begin{picture}(-5,0)(2.5,0)
\put(107.8,-592.011){\fontsize{12}{1}\usefont{T1}{cmr}{m}{it}\selectfont\color{color_29791}group }
\end{picture}
\begin{tikzpicture}[overlay]
\path(0pt,0pt);
\draw[color_29791,line width=0.7pt]
(107.8pt, -588.611pt) -- (139pt, -588.611pt)
;
\end{tikzpicture}
\begin{picture}(-5,0)(2.5,0)
\put(139.1,-592.011){\fontsize{12}{1}\usefont{T1}{cmr}{m}{n}\selectfont\color{color_29791}/ }
\end{picture}
\begin{tikzpicture}[overlay]
\path(0pt,0pt);
\draw[color_29791,line width=0.7pt]
(139.1pt, -588.611pt) -- (145.4pt, -588.611pt)
;
\end{tikzpicture}
\begin{picture}(-5,0)(2.5,0)
\put(145.4,-592.011){\fontsize{12}{1}\usefont{T1}{cmr}{m}{it}\selectfont\color{color_29791}pid}
\end{picture}
\begin{tikzpicture}[overlay]
\path(0pt,0pt);
\draw[color_29791,line width=0.7pt]
(145.4pt, -588.611pt) -- (160.7pt, -588.611pt)
;
\end{tikzpicture}
\begin{picture}(-5,0)(2.5,0)
\put(160.8,-592.011){\fontsize{12}{1}\usefont{T1}{cmr}{m}{n}\selectfont\color{color_29791} }
\end{picture}
\begin{tikzpicture}[overlay]
\path(0pt,0pt);
\draw[color_29791,line width=0.7pt]
(160.8pt, -588.611pt) -- (163.8pt, -588.611pt)
;
\end{tikzpicture}
\begin{picture}(-5,0)(2.5,0)
\put(163.7,-592.011){\fontsize{12}{1}\usefont{T1}{cmr}{m}{n}\selectfont\color{color_29791}      }
\end{picture}
\begin{tikzpicture}[overlay]
\path(0pt,0pt);
\draw[color_29791,line width=0.7pt]
(163.7pt, -588.611pt) -- (184.1pt, -588.611pt)
;
\end{tikzpicture}
\begin{picture}(-5,0)(2.5,0)
\put(184.2,-592.011){\fontsize{12}{1}\usefont{T1}{cmr}{m}{n}\selectfont\color{color_29791}utente e gruppo UNIX del processo, file in cui salvare il PID}
\end{picture}
\begin{tikzpicture}[overlay]
\path(0pt,0pt);
\draw[color_29791,line width=0.7pt]
(184.2pt, -588.611pt) -- (475.4pt, -588.611pt)
;
\end{tikzpicture}
\begin{picture}(-5,0)(2.5,0)
\put(41.8,-612.811){\fontsize{12}{1}\usefont{T1}{cmr}{b}{n}\selectfont\color{color_29791}dnsmasq }
\end{picture}
\begin{tikzpicture}[overlay]
\path(0pt,0pt);
\draw[color_29791,line width=0.7pt]
(41.8pt, -609.411pt) -- (90.1pt, -609.411pt)
;
\end{tikzpicture}
\begin{picture}(-5,0)(2.5,0)
\put(90.2,-612.811){\fontsize{12}{1}\usefont{T1}{cmr}{m}{n}\selectfont\color{color_29791}può fare da DCHP, il server DHCP è disabilitato se non sono specificate le opzioni }
\end{picture}
\begin{tikzpicture}[overlay]
\path(0pt,0pt);
\draw[color_29791,line width=0.7pt]
(90.2pt, -609.411pt) -- (489.6pt, -609.411pt)
;
\end{tikzpicture}
\begin{picture}(-5,0)(2.5,0)
\put(41.8,-626.611){\fontsize{12}{1}\usefont{T1}{cmr}{m}{n}\selectfont\color{color_29791}descritte di seguito (altrimenti eroga indirizzi DHCP in un certo range o per un certo host):}
\end{picture}
\begin{tikzpicture}[overlay]
\path(0pt,0pt);
\draw[color_29791,line width=0.7pt]
(41.8pt, -623.211pt) -- (476.9pt, -623.211pt)
;
\end{tikzpicture}
\begin{picture}(-5,0)(2.5,0)
\put(59.8,-647.411){\fontsize{12}{1}\usefont{T1}{cmr}{m}{n}\selectfont\color{color_29791}•}
\end{picture}
\begin{tikzpicture}[overlay]
\path(0pt,0pt);
\draw[color_29791,line width=0.7pt]
(59.8pt, -644.011pt) -- (64.10001pt, -644.011pt)
;
\end{tikzpicture}
\begin{picture}(-5,0)(2.5,0)
\put(64,-647.411){\fontsize{12}{1}\usefont{T1}{cmr}{m}{n}\selectfont\color{color_29791}  }
\end{picture}
\begin{tikzpicture}[overlay]
\path(0pt,0pt);
\draw[color_29791,line width=0.7pt]
(64pt, -644.011pt) -- (77.7pt, -644.011pt)
;
\end{tikzpicture}
\begin{picture}(-5,0)(2.5,0)
\put(77.8,-647.411){\fontsize{12}{1}\usefont{T1}{cmr}{m}{it}\selectfont\color{color_29791}dhcp-range=<start-addr>[,<end-addr>|<mode>][,<netmask>[,<broadcast>]][,<lease }
\end{picture}
\begin{tikzpicture}[overlay]
\path(0pt,0pt);
\draw[color_29791,line width=0.7pt]
(77.8pt, -644.011pt) -- (511.4pt, -644.011pt)
;
\end{tikzpicture}
\begin{picture}(-5,0)(2.5,0)
\put(77.8,-661.211){\fontsize{12}{1}\usefont{T1}{cmr}{m}{it}\selectfont\color{color_29791}time>]}
\end{picture}
\begin{tikzpicture}[overlay]
\path(0pt,0pt);
\draw[color_29791,line width=0.7pt]
(77.8pt, -657.811pt) -- (111.2pt, -657.811pt)
;
\end{tikzpicture}
\begin{picture}(-5,0)(2.5,0)
\put(113.2,-661.211){\fontsize{12}{1}\usefont{T1}{cmr}{m}{n}\selectfont\color{color_29791}           }
\end{picture}
\begin{tikzpicture}[overlay]
\path(0pt,0pt);
\draw[color_29791,line width=0.7pt]
(113.2pt, -657.811pt) -- (148.7pt, -657.811pt)
;
\end{tikzpicture}
\begin{picture}(-5,0)(2.5,0)
\put(77.8,-682.011){\fontsize{12}{1}\usefont{T1}{cmr}{m}{n}\selectfont\color{color_29791}◦}
\end{picture}
\begin{tikzpicture}[overlay]
\path(0pt,0pt);
\draw[color_29791,line width=0.7pt]
(77.8pt, -678.611pt) -- (87.3pt, -678.611pt)
;
\end{tikzpicture}
\begin{picture}(-5,0)(2.5,0)
\put(87.3,-682.011){\fontsize{12}{1}\usefont{T1}{cmr}{m}{n}\selectfont\color{color_29791} }
\end{picture}
\begin{tikzpicture}[overlay]
\path(0pt,0pt);
\draw[color_29791,line width=0.7pt]
(87.3pt, -678.611pt) -- (93.3pt, -678.611pt)
;
\end{tikzpicture}
\begin{picture}(-5,0)(2.5,0)
\put(95.8,-682.011){\fontsize{12}{1}\usefont{T1}{cmr}{m}{n}\selectfont\color{color_29791}questo specifica la rete da assegnare mediante indirizzo inizio, fine e netmask. Mettendo }
\end{picture}
\begin{tikzpicture}[overlay]
\path(0pt,0pt);
\draw[color_29791,line width=0.7pt]
(95.8pt, -678.611pt) -- (523.2pt, -678.611pt)
;
\end{tikzpicture}
\begin{picture}(-5,0)(2.5,0)
\put(95.8,-695.811){\fontsize{12}{1}\usefont{T1}{cmr}{m}{n}\selectfont\color{color_29791}più di una riga, si specificano range per più reti/interfacce}
\end{picture}
\begin{tikzpicture}[overlay]
\path(0pt,0pt);
\draw[color_29791,line width=0.7pt]
(95.8pt, -692.411pt) -- (373pt, -692.411pt)
;
\end{tikzpicture}
\begin{picture}(-5,0)(2.5,0)
\put(77.8,-716.611){\fontsize{12}{1}\usefont{T1}{cmr}{m}{n}\selectfont\color{color_29791}◦}
\end{picture}
\begin{tikzpicture}[overlay]
\path(0pt,0pt);
\draw[color_29791,line width=0.7pt]
(77.8pt, -713.211pt) -- (87.3pt, -713.211pt)
;
\end{tikzpicture}
\begin{picture}(-5,0)(2.5,0)
\put(87.3,-716.611){\fontsize{12}{1}\usefont{T1}{cmr}{m}{n}\selectfont\color{color_29791} }
\end{picture}
\begin{tikzpicture}[overlay]
\path(0pt,0pt);
\draw[color_29791,line width=0.7pt]
(87.3pt, -713.211pt) -- (93.3pt, -713.211pt)
;
\end{tikzpicture}
\begin{picture}(-5,0)(2.5,0)
\put(95.8,-716.611){\fontsize{12}{1}\usefont{T1}{cmr}{m}{n}\selectfont\color{color_29791}indirizzi tra <start-addr> e <end-addr>, se specificato imposta il <lease time>}
\end{picture}
\begin{tikzpicture}[overlay]
\path(0pt,0pt);
\draw[color_29791,line width=0.7pt]
(95.8pt, -713.211pt) -- (467.9pt, -713.211pt)
;
\end{tikzpicture}
\begin{picture}(-5,0)(2.5,0)
\put(77.8,-737.411){\fontsize{12}{1}\usefont{T1}{cmr}{m}{n}\selectfont\color{color_29791}◦}
\end{picture}
\begin{tikzpicture}[overlay]
\path(0pt,0pt);
\draw[color_29791,line width=0.7pt]
(77.8pt, -734.011pt) -- (87.3pt, -734.011pt)
;
\end{tikzpicture}
\begin{picture}(-5,0)(2.5,0)
\put(87.3,-737.411){\fontsize{12}{1}\usefont{T1}{cmr}{m}{n}\selectfont\color{color_29791} }
\end{picture}
\begin{tikzpicture}[overlay]
\path(0pt,0pt);
\draw[color_29791,line width=0.7pt]
(87.3pt, -734.011pt) -- (93.3pt, -734.011pt)
;
\end{tikzpicture}
\begin{picture}(-5,0)(2.5,0)
\put(95.8,-737.411){\fontsize{12}{1}\usefont{T1}{cmr}{m}{n}\selectfont\color{color_29791}la <netmask> è opzionale per reti connesse direttamente al server}
\end{picture}
\begin{tikzpicture}[overlay]
\path(0pt,0pt);
\draw[color_29791,line width=0.7pt]
(95.8pt, -734.011pt) -- (409.5pt, -734.011pt)
;
\end{tikzpicture}
\newpage
\begin{tikzpicture}[overlay]\path(0pt,0pt);\end{tikzpicture}
\begin{picture}(-5,0)(2.5,0)
\put(77.8,-85.01099){\fontsize{12}{1}\usefont{T1}{cmr}{m}{n}\selectfont\color{color_29791}◦}
\end{picture}
\begin{tikzpicture}[overlay]
\path(0pt,0pt);
\draw[color_29791,line width=0.7pt]
(77.8pt, -81.61096pt) -- (87.3pt, -81.61096pt)
;
\end{tikzpicture}
\begin{picture}(-5,0)(2.5,0)
\put(87.3,-85.01099){\fontsize{12}{1}\usefont{T1}{cmr}{m}{n}\selectfont\color{color_29791} }
\end{picture}
\begin{tikzpicture}[overlay]
\path(0pt,0pt);
\draw[color_29791,line width=0.7pt]
(87.3pt, -81.61096pt) -- (93.3pt, -81.61096pt)
;
\end{tikzpicture}
\begin{picture}(-5,0)(2.5,0)
\put(95.8,-85.01099){\fontsize{12}{1}\usefont{T1}{cmr}{m}{n}\selectfont\color{color_29791}al posto di <end addr>, <mode> può essere }
\end{picture}
\begin{tikzpicture}[overlay]
\path(0pt,0pt);
\draw[color_29791,line width=0.7pt]
(95.8pt, -81.61096pt) -- (306.5pt, -81.61096pt)
;
\end{tikzpicture}
\begin{picture}(-5,0)(2.5,0)
\put(306.5,-85.01099){\fontsize{12}{1}\usefont{T1}{cmr}{m}{it}\selectfont\color{color_29791}static}
\end{picture}
\begin{tikzpicture}[overlay]
\path(0pt,0pt);
\draw[color_29791,line width=0.7pt]
(306.5pt, -81.61096pt) -- (332.5pt, -81.61096pt)
;
\end{tikzpicture}
\begin{picture}(-5,0)(2.5,0)
\put(332.5,-85.01099){\fontsize{12}{1}\usefont{T1}{cmr}{m}{n}\selectfont\color{color_29791} per abilitare il server sulla rete indicata}
\end{picture}
\begin{tikzpicture}[overlay]
\path(0pt,0pt);
\draw[color_29791,line width=0.7pt]
(332.5pt, -81.61096pt) -- (522.7pt, -81.61096pt)
;
\end{tikzpicture}
\begin{picture}(-5,0)(2.5,0)
\put(95.8,-98.81097){\fontsize{12}{1}\usefont{T1}{cmr}{m}{n}\selectfont\color{color_29791}senza servire indirizzi dinamici, ma solo quelli specificati con opzione }
\end{picture}
\begin{tikzpicture}[overlay]
\path(0pt,0pt);
\draw[color_29791,line width=0.7pt]
(95.8pt, -95.41095pt) -- (436.6pt, -95.41095pt)
;
\end{tikzpicture}
\begin{picture}(-5,0)(2.5,0)
\put(436.7,-98.81097){\fontsize{12}{1}\usefont{T1}{cmr}{m}{it}\selectfont\color{color_29791}dhcp-host }
\end{picture}
\begin{tikzpicture}[overlay]
\path(0pt,0pt);
\draw[color_29791,line width=0.7pt]
(436.7pt, -95.41095pt) -- (487pt, -95.41095pt)
;
\end{tikzpicture}
\begin{picture}(-5,0)(2.5,0)
\put(487,-98.81097){\fontsize{12}{1}\usefont{T1}{cmr}{m}{n}\selectfont\color{color_29791}(di }
\end{picture}
\begin{tikzpicture}[overlay]
\path(0pt,0pt);
\draw[color_29791,line width=0.7pt]
(487pt, -95.41095pt) -- (503.3pt, -95.41095pt)
;
\end{tikzpicture}
\begin{picture}(-5,0)(2.5,0)
\put(95.8,-112.611){\fontsize{12}{1}\usefont{T1}{cmr}{m}{n}\selectfont\color{color_29791}seguito)}
\end{picture}
\begin{tikzpicture}[overlay]
\path(0pt,0pt);
\draw[color_29791,line width=0.7pt]
(95.8pt, -109.211pt) -- (134.4pt, -109.211pt)
;
\end{tikzpicture}
\begin{picture}(-5,0)(2.5,0)
\put(59.8,-133.411){\fontsize{12}{1}\usefont{T1}{cmr}{m}{n}\selectfont\color{color_29791}•}
\end{picture}
\begin{tikzpicture}[overlay]
\path(0pt,0pt);
\draw[color_29791,line width=0.7pt]
(59.8pt, -130.011pt) -- (64.10001pt, -130.011pt)
;
\end{tikzpicture}
\begin{picture}(-5,0)(2.5,0)
\put(64,-133.411){\fontsize{12}{1}\usefont{T1}{cmr}{m}{n}\selectfont\color{color_29791}  }
\end{picture}
\begin{tikzpicture}[overlay]
\path(0pt,0pt);
\draw[color_29791,line width=0.7pt]
(64pt, -130.011pt) -- (77.7pt, -130.011pt)
;
\end{tikzpicture}
\begin{picture}(-5,0)(2.5,0)
\put(77.8,-133.411){\fontsize{12}{1}\usefont{T1}{cmr}{m}{it}\selectfont\color{color_29791}dhcp-host=[<hwaddr>][,<ipaddr>][,<hostname>][,<lease\_time>][,ignore]}
\end{picture}
\begin{tikzpicture}[overlay]
\path(0pt,0pt);
\draw[color_29791,line width=0.7pt]
(77.8pt, -130.011pt) -- (453.5pt, -130.011pt)
;
\end{tikzpicture}
\begin{picture}(-5,0)(2.5,0)
\put(77.8,-154.211){\fontsize{12}{1}\usefont{T1}{cmr}{m}{n}\selectfont\color{color_29791}◦}
\end{picture}
\begin{tikzpicture}[overlay]
\path(0pt,0pt);
\draw[color_29791,line width=0.7pt]
(77.8pt, -150.811pt) -- (87.3pt, -150.811pt)
;
\end{tikzpicture}
\begin{picture}(-5,0)(2.5,0)
\put(87.3,-154.211){\fontsize{12}{1}\usefont{T1}{cmr}{m}{n}\selectfont\color{color_29791} }
\end{picture}
\begin{tikzpicture}[overlay]
\path(0pt,0pt);
\draw[color_29791,line width=0.7pt]
(87.3pt, -150.811pt) -- (93.3pt, -150.811pt)
;
\end{tikzpicture}
\begin{picture}(-5,0)(2.5,0)
\put(95.8,-154.211){\fontsize{12}{1}\usefont{T1}{cmr}{m}{n}\selectfont\color{color_29791}questo caso serve per assegnare indirizzi statici (sempre stesso indirizzo per stessa }
\end{picture}
\begin{tikzpicture}[overlay]
\path(0pt,0pt);
\draw[color_29791,line width=0.7pt]
(95.8pt, -150.811pt) -- (493.6pt, -150.811pt)
;
\end{tikzpicture}
\begin{picture}(-5,0)(2.5,0)
\put(95.8,-168.011){\fontsize{12}{1}\usefont{T1}{cmr}{m}{n}\selectfont\color{color_29791}macchina)}
\end{picture}
\begin{tikzpicture}[overlay]
\path(0pt,0pt);
\draw[color_29791,line width=0.7pt]
(95.8pt, -164.611pt) -- (145.7pt, -164.611pt)
;
\end{tikzpicture}
\begin{picture}(-5,0)(2.5,0)
\put(77.8,-188.811){\fontsize{12}{1}\usefont{T1}{cmr}{m}{n}\selectfont\color{color_29791}◦}
\end{picture}
\begin{tikzpicture}[overlay]
\path(0pt,0pt);
\draw[color_29791,line width=0.7pt]
(77.8pt, -185.4109pt) -- (87.3pt, -185.4109pt)
;
\end{tikzpicture}
\begin{picture}(-5,0)(2.5,0)
\put(87.3,-188.811){\fontsize{12}{1}\usefont{T1}{cmr}{m}{n}\selectfont\color{color_29791} }
\end{picture}
\begin{tikzpicture}[overlay]
\path(0pt,0pt);
\draw[color_29791,line width=0.7pt]
(87.3pt, -185.4109pt) -- (93.3pt, -185.4109pt)
;
\end{tikzpicture}
\begin{picture}(-5,0)(2.5,0)
\put(95.8,-188.811){\fontsize{12}{1}\usefont{T1}{cmr}{m}{n}\selectfont\color{color_29791}assegna <hostname>,<ipaddr> e <lease time> stabili all'host con MAC=<hwaddr>}
\end{picture}
\begin{tikzpicture}[overlay]
\path(0pt,0pt);
\draw[color_29791,line width=0.7pt]
(95.8pt, -185.4109pt) -- (492.4pt, -185.4109pt)
;
\end{tikzpicture}
\begin{picture}(-5,0)(2.5,0)
\put(77.8,-209.611){\fontsize{12}{1}\usefont{T1}{cmr}{m}{n}\selectfont\color{color_29791}◦}
\end{picture}
\begin{tikzpicture}[overlay]
\path(0pt,0pt);
\draw[color_29791,line width=0.7pt]
(77.8pt, -206.211pt) -- (87.3pt, -206.211pt)
;
\end{tikzpicture}
\begin{picture}(-5,0)(2.5,0)
\put(87.3,-209.611){\fontsize{12}{1}\usefont{T1}{cmr}{m}{n}\selectfont\color{color_29791} }
\end{picture}
\begin{tikzpicture}[overlay]
\path(0pt,0pt);
\draw[color_29791,line width=0.7pt]
(87.3pt, -206.211pt) -- (93.3pt, -206.211pt)
;
\end{tikzpicture}
\begin{picture}(-5,0)(2.5,0)
\put(95.8,-209.611){\fontsize{12}{1}\usefont{T1}{cmr}{m}{n}\selectfont\color{color_29791}con }
\end{picture}
\begin{tikzpicture}[overlay]
\path(0pt,0pt);
\draw[color_29791,line width=0.7pt]
(95.8pt, -206.211pt) -- (116.1pt, -206.211pt)
;
\end{tikzpicture}
\begin{picture}(-5,0)(2.5,0)
\put(116.2,-209.611){\fontsize{12}{1}\usefont{T1}{cmr}{m}{it}\selectfont\color{color_29791}ignore}
\end{picture}
\begin{tikzpicture}[overlay]
\path(0pt,0pt);
\draw[color_29791,line width=0.7pt]
(116.2pt, -206.211pt) -- (147.1pt, -206.211pt)
;
\end{tikzpicture}
\begin{picture}(-5,0)(2.5,0)
\put(147.1,-209.611){\fontsize{12}{1}\usefont{T1}{cmr}{m}{n}\selectfont\color{color_29791} non fornirà mai un lease all'host indicato}
\end{picture}
\begin{tikzpicture}[overlay]
\path(0pt,0pt);
\draw[color_29791,line width=0.7pt]
(147.1pt, -206.211pt) -- (345.5pt, -206.211pt)
;
\end{tikzpicture}
\begin{picture}(-5,0)(2.5,0)
\put(59.8,-230.411){\fontsize{12}{1}\usefont{T1}{cmr}{m}{n}\selectfont\color{color_29791}•}
\end{picture}
\begin{tikzpicture}[overlay]
\path(0pt,0pt);
\draw[color_29791,line width=0.7pt]
(59.8pt, -227.011pt) -- (64.10001pt, -227.011pt)
;
\end{tikzpicture}
\begin{picture}(-5,0)(2.5,0)
\put(64,-230.411){\fontsize{12}{1}\usefont{T1}{cmr}{m}{n}\selectfont\color{color_29791}  }
\end{picture}
\begin{tikzpicture}[overlay]
\path(0pt,0pt);
\draw[color_29791,line width=0.7pt]
(64pt, -227.011pt) -- (77.7pt, -227.011pt)
;
\end{tikzpicture}
\begin{picture}(-5,0)(2.5,0)
\put(77.8,-230.411){\fontsize{12}{1}\usefont{T1}{cmr}{m}{it}\selectfont\color{color_29791}dhcp-hostfile}
\end{picture}
\begin{tikzpicture}[overlay]
\path(0pt,0pt);
\draw[color_29791,line width=0.7pt]
(77.8pt, -227.011pt) -- (140.4pt, -227.011pt)
;
\end{tikzpicture}
\begin{picture}(-5,0)(2.5,0)
\put(140.5,-230.411){\fontsize{12}{1}\usefont{T1}{cmr}{m}{n}\selectfont\color{color_29791} permette di specificare una directory di file contenente informazioni formattate}
\end{picture}
\begin{tikzpicture}[overlay]
\path(0pt,0pt);
\draw[color_29791,line width=0.7pt]
(140.5pt, -227.011pt) -- (523pt, -227.011pt)
;
\end{tikzpicture}
\begin{picture}(-5,0)(2.5,0)
\put(77.8,-244.211){\fontsize{12}{1}\usefont{T1}{cmr}{m}{n}\selectfont\color{color_29791}come la parte a destra dell'= di }
\end{picture}
\begin{tikzpicture}[overlay]
\path(0pt,0pt);
\draw[color_29791,line width=0.7pt]
(77.8pt, -240.811pt) -- (227.6pt, -240.811pt)
;
\end{tikzpicture}
\begin{picture}(-5,0)(2.5,0)
\put(227.7,-244.211){\fontsize{12}{1}\usefont{T1}{cmr}{m}{it}\selectfont\color{color_29791}dhcp-host}
\end{picture}
\begin{tikzpicture}[overlay]
\path(0pt,0pt);
\draw[color_29791,line width=0.7pt]
(227.7pt, -240.811pt) -- (275pt, -240.811pt)
;
\end{tikzpicture}
\begin{picture}(-5,0)(2.5,0)
\put(59.8,-265.011){\fontsize{12}{1}\usefont{T1}{cmr}{m}{n}\selectfont\color{color_29791}•}
\end{picture}
\begin{tikzpicture}[overlay]
\path(0pt,0pt);
\draw[color_29791,line width=0.7pt]
(59.8pt, -261.611pt) -- (64.10001pt, -261.611pt)
;
\end{tikzpicture}
\begin{picture}(-5,0)(2.5,0)
\put(64,-265.011){\fontsize{12}{1}\usefont{T1}{cmr}{m}{n}\selectfont\color{color_29791}  }
\end{picture}
\begin{tikzpicture}[overlay]
\path(0pt,0pt);
\draw[color_29791,line width=0.7pt]
(64pt, -261.611pt) -- (77.7pt, -261.611pt)
;
\end{tikzpicture}
\begin{picture}(-5,0)(2.5,0)
\put(77.8,-265.011){\fontsize{12}{1}\usefont{T1}{cmr}{m}{it}\selectfont\color{color_29791}dhcp-option=[<opt>|option:<opt-name>|option6:<opt>|option6:<opt-name>],}
\end{picture}
\begin{tikzpicture}[overlay]
\path(0pt,0pt);
\draw[color_29791,line width=0.7pt]
(77.8pt, -261.611pt) -- (468.2pt, -261.611pt)
;
\end{tikzpicture}
\begin{picture}(-5,0)(2.5,0)
\put(77.8,-278.811){\fontsize{12}{1}\usefont{T1}{cmr}{m}{it}\selectfont\color{color_29791}[<value>[,<value>]]}
\end{picture}
\begin{tikzpicture}[overlay]
\path(0pt,0pt);
\draw[color_29791,line width=0.7pt]
(77.8pt, -275.4109pt) -- (183.8pt, -275.4109pt)
;
\end{tikzpicture}
\begin{picture}(-5,0)(2.5,0)
\put(184.1,-278.811){\fontsize{12}{1}\usefont{T1}{cmr}{m}{it}\selectfont\color{color_29791}           }
\end{picture}
\begin{tikzpicture}[overlay]
\path(0pt,0pt);
\draw[color_29791,line width=0.7pt]
(184.1pt, -275.4109pt) -- (219.6pt, -275.4109pt)
;
\end{tikzpicture}
\begin{picture}(-5,0)(2.5,0)
\put(219.6,-278.811){\fontsize{12}{1}\usefont{T1}{cmr}{m}{n}\selectfont\color{color_29791}uso più comune:}
\end{picture}
\begin{tikzpicture}[overlay]
\path(0pt,0pt);
\draw[color_29791,line width=0.7pt]
(219.6pt, -275.4109pt) -- (298.9pt, -275.4109pt)
;
\end{tikzpicture}
\begin{picture}(-5,0)(2.5,0)
\put(298.9,-278.811){\fontsize{12}{1}\usefont{T1}{cmr}{m}{it}\selectfont\color{color_29791} }
\end{picture}
\begin{tikzpicture}[overlay]
\path(0pt,0pt);
\draw[color_29791,line width=0.7pt]
(298.9pt, -275.4109pt) -- (301.9pt, -275.4109pt)
;
\end{tikzpicture}
\begin{picture}(-5,0)(2.5,0)
\put(301.8,-278.811){\fontsize{12}{1}\usefont{T1}{cmr}{m}{it}\selectfont\color{color_29791}        }
\end{picture}
\begin{tikzpicture}[overlay]
\path(0pt,0pt);
\draw[color_29791,line width=0.7pt]
(301.8pt, -275.4109pt) -- (325.9pt, -275.4109pt)
;
\end{tikzpicture}
\begin{picture}(-5,0)(2.5,0)
\put(326,-278.811){\fontsize{12}{1}\usefont{T1}{cmr}{m}{it}\selectfont\color{color_29791}dhcp-option=<opt>,<value>}
\end{picture}
\begin{tikzpicture}[overlay]
\path(0pt,0pt);
\draw[color_29791,line width=0.7pt]
(326pt, -275.4109pt) -- (468.8pt, -275.4109pt)
;
\end{tikzpicture}
\begin{picture}(-5,0)(2.5,0)
\put(77.8,-299.611){\fontsize{12}{1}\usefont{T1}{cmr}{m}{n}\selectfont\color{color_29791}◦}
\end{picture}
\begin{tikzpicture}[overlay]
\path(0pt,0pt);
\draw[color_29791,line width=0.7pt]
(77.8pt, -296.211pt) -- (87.3pt, -296.211pt)
;
\end{tikzpicture}
\begin{picture}(-5,0)(2.5,0)
\put(87.3,-299.611){\fontsize{12}{1}\usefont{T1}{cmr}{m}{n}\selectfont\color{color_29791} }
\end{picture}
\begin{tikzpicture}[overlay]
\path(0pt,0pt);
\draw[color_29791,line width=0.7pt]
(87.3pt, -296.211pt) -- (93.3pt, -296.211pt)
;
\end{tikzpicture}
\begin{picture}(-5,0)(2.5,0)
\put(95.8,-299.611){\fontsize{12}{1}\usefont{T1}{cmr}{m}{n}\selectfont\color{color_29791}dnsmasq fornisce un sacco di info, come convenzione si indica il gateway (solitamente il}
\end{picture}
\begin{tikzpicture}[overlay]
\path(0pt,0pt);
\draw[color_29791,line width=0.7pt]
(95.8pt, -296.211pt) -- (522.3pt, -296.211pt)
;
\end{tikzpicture}
\begin{picture}(-5,0)(2.5,0)
\put(95.8,-313.411){\fontsize{12}{1}\usefont{T1}{cmr}{m}{n}\selectfont\color{color_29791}primo o ultimo indirizzo del range) con opzione -3, per indicare rotta di default. Es. riga }
\end{picture}
\begin{tikzpicture}[overlay]
\path(0pt,0pt);
\draw[color_29791,line width=0.7pt]
(95.8pt, -310.011pt) -- (521.3pt, -310.011pt)
;
\end{tikzpicture}
\begin{picture}(-5,0)(2.5,0)
\put(95.8,-327.211){\fontsize{12}{1}\usefont{T1}{cmr}{m}{n}\selectfont\color{color_29791}del file /etc/dnsmasq.conf}
\end{picture}
\begin{tikzpicture}[overlay]
\path(0pt,0pt);
\draw[color_29791,line width=0.7pt]
(95.8pt, -323.811pt) -- (219.4pt, -323.811pt)
;
\end{tikzpicture}
\begin{picture}(-5,0)(2.5,0)
\put(219.3,-327.211){\fontsize{12}{1}\usefont{T1}{cmr}{m}{n}\selectfont\color{color_29791}      }
\end{picture}
\begin{tikzpicture}[overlay]
\path(0pt,0pt);
\draw[color_29791,line width=0.7pt]
(219.3pt, -323.811pt) -- (237.5pt, -323.811pt)
;
\end{tikzpicture}
\begin{picture}(-5,0)(2.5,0)
\put(237.5,-327.211){\fontsize{12}{1}\usefont{T1}{cmr}{m}{n}\selectfont\color{color_29791}           }
\end{picture}
\begin{tikzpicture}[overlay]
\path(0pt,0pt);
\draw[color_29791,line width=0.7pt]
(237.5pt, -323.811pt) -- (273pt, -323.811pt)
;
\end{tikzpicture}
\begin{picture}(-5,0)(2.5,0)
\put(273.1,-327.211){\fontsize{12}{1}\usefont{T1}{cmr}{m}{n}\selectfont\color{color_29791}dhcp-option=3,1.2.3.4}
\end{picture}
\begin{tikzpicture}[overlay]
\path(0pt,0pt);
\draw[color_29791,line width=0.7pt]
(273.1pt, -323.811pt) -- (379.8pt, -323.811pt)
;
\end{tikzpicture}
\begin{picture}(-5,0)(2.5,0)
\put(379.7,-327.211){\fontsize{12}{1}\usefont{T1}{cmr}{m}{n}\selectfont\color{color_29791}           }
\end{picture}
\begin{tikzpicture}[overlay]
\path(0pt,0pt);
\draw[color_29791,line width=0.7pt]
(379.7pt, -323.811pt) -- (414.8pt, -323.811pt)
;
\end{tikzpicture}
\begin{picture}(-5,0)(2.5,0)
\put(414.9,-327.211){\fontsize{12}{1}\usefont{T1}{cmr}{m}{n}\selectfont\color{color_29791}indica che il gateway }
\end{picture}
\begin{tikzpicture}[overlay]
\path(0pt,0pt);
\draw[color_29791,line width=0.7pt]
(414.9pt, -323.811pt) -- (519.5pt, -323.811pt)
;
\end{tikzpicture}
\begin{picture}(-5,0)(2.5,0)
\put(95.8,-341.011){\fontsize{12}{1}\usefont{T1}{cmr}{m}{n}\selectfont\color{color_29791}è su 1.2.3.4}
\end{picture}
\begin{tikzpicture}[overlay]
\path(0pt,0pt);
\draw[color_29791,line width=0.7pt]
(95.8pt, -337.611pt) -- (150.8pt, -337.611pt)
;
\end{tikzpicture}
\begin{picture}(-5,0)(2.5,0)
\put(77.8,-361.811){\fontsize{12}{1}\usefont{T1}{cmr}{m}{n}\selectfont\color{color_29791}◦}
\end{picture}
\begin{tikzpicture}[overlay]
\path(0pt,0pt);
\draw[color_29791,line width=0.7pt]
(77.8pt, -358.411pt) -- (87.3pt, -358.411pt)
;
\end{tikzpicture}
\begin{picture}(-5,0)(2.5,0)
\put(87.3,-361.811){\fontsize{12}{1}\usefont{T1}{cmr}{m}{n}\selectfont\color{color_29791} }
\end{picture}
\begin{tikzpicture}[overlay]
\path(0pt,0pt);
\draw[color_29791,line width=0.7pt]
(87.3pt, -358.411pt) -- (93.3pt, -358.411pt)
;
\end{tikzpicture}
\begin{picture}(-5,0)(2.5,0)
\put(95.8,-361.811){\fontsize{12}{1}\usefont{T1}{cmr}{m}{n}\selectfont\color{color_29791}Opzioni comuni:}
\end{picture}
\begin{tikzpicture}[overlay]
\path(0pt,0pt);
\draw[color_29791,line width=0.7pt]
(95.8pt, -358.411pt) -- (176.8pt, -358.411pt)
;
\end{tikzpicture}
\begin{picture}(-5,0)(2.5,0)
\put(95.8,-382.611){\fontsize{12}{1}\usefont{T1}{cmr}{m}{n}\selectfont\color{color_29791}▪}
\end{picture}
\begin{tikzpicture}[overlay]
\path(0pt,0pt);
\draw[color_29791,line width=0.7pt]
(95.8pt, -379.211pt) -- (104.7pt, -379.211pt)
;
\end{tikzpicture}
\begin{picture}(-5,0)(2.5,0)
\put(104.7,-382.611){\fontsize{12}{1}\usefont{T1}{cmr}{m}{n}\selectfont\color{color_29791} }
\end{picture}
\begin{tikzpicture}[overlay]
\path(0pt,0pt);
\draw[color_29791,line width=0.7pt]
(104.7pt, -379.211pt) -- (110.7pt, -379.211pt)
;
\end{tikzpicture}
\begin{picture}(-5,0)(2.5,0)
\put(113.8,-382.611){\fontsize{12}{1}\usefont{T1}{cmr}{m}{n}\selectfont\color{color_29791}1 }
\end{picture}
\begin{tikzpicture}[overlay]
\path(0pt,0pt);
\draw[color_29791,line width=0.7pt]
(113.8pt, -379.211pt) -- (122.8pt, -379.211pt)
;
\end{tikzpicture}
\begin{picture}(-5,0)(2.5,0)
\put(122.7,-382.611){\fontsize{12}{1}\usefont{T1}{cmr}{m}{n}\selectfont\color{color_29791}        }
\end{picture}
\begin{tikzpicture}[overlay]
\path(0pt,0pt);
\draw[color_29791,line width=0.7pt]
(122.7pt, -379.211pt) -- (149.2pt, -379.211pt)
;
\end{tikzpicture}
\begin{picture}(-5,0)(2.5,0)
\put(149.3,-382.611){\fontsize{12}{1}\usefont{T1}{cmr}{m}{n}\selectfont\color{color_29791}netmask}
\end{picture}
\begin{tikzpicture}[overlay]
\path(0pt,0pt);
\draw[color_29791,line width=0.7pt]
(149.3pt, -379.211pt) -- (189.3pt, -379.211pt)
;
\end{tikzpicture}
\begin{picture}(-5,0)(2.5,0)
\put(95.8,-403.411){\fontsize{12}{1}\usefont{T1}{cmr}{m}{n}\selectfont\color{color_29791}▪}
\end{picture}
\begin{tikzpicture}[overlay]
\path(0pt,0pt);
\draw[color_29791,line width=0.7pt]
(95.8pt, -400.011pt) -- (104.7pt, -400.011pt)
;
\end{tikzpicture}
\begin{picture}(-5,0)(2.5,0)
\put(104.7,-403.411){\fontsize{12}{1}\usefont{T1}{cmr}{m}{n}\selectfont\color{color_29791} }
\end{picture}
\begin{tikzpicture}[overlay]
\path(0pt,0pt);
\draw[color_29791,line width=0.7pt]
(104.7pt, -400.011pt) -- (110.7pt, -400.011pt)
;
\end{tikzpicture}
\begin{picture}(-5,0)(2.5,0)
\put(113.8,-403.411){\fontsize{12}{1}\usefont{T1}{cmr}{m}{n}\selectfont\color{color_29791}2 }
\end{picture}
\begin{tikzpicture}[overlay]
\path(0pt,0pt);
\draw[color_29791,line width=0.7pt]
(113.8pt, -400.011pt) -- (122.8pt, -400.011pt)
;
\end{tikzpicture}
\begin{picture}(-5,0)(2.5,0)
\put(122.7,-403.411){\fontsize{12}{1}\usefont{T1}{cmr}{m}{n}\selectfont\color{color_29791}        }
\end{picture}
\begin{tikzpicture}[overlay]
\path(0pt,0pt);
\draw[color_29791,line width=0.7pt]
(122.7pt, -400.011pt) -- (149.2pt, -400.011pt)
;
\end{tikzpicture}
\begin{picture}(-5,0)(2.5,0)
\put(149.3,-403.411){\fontsize{12}{1}\usefont{T1}{cmr}{m}{n}\selectfont\color{color_29791}fuso orario (offset da UTC)}
\end{picture}
\begin{tikzpicture}[overlay]
\path(0pt,0pt);
\draw[color_29791,line width=0.7pt]
(149.3pt, -400.011pt) -- (281pt, -400.011pt)
;
\end{tikzpicture}
\begin{picture}(-5,0)(2.5,0)
\put(95.8,-424.211){\fontsize{12}{1}\usefont{T1}{cmr}{m}{n}\selectfont\color{color_29791}▪}
\end{picture}
\begin{tikzpicture}[overlay]
\path(0pt,0pt);
\draw[color_29791,line width=0.7pt]
(95.8pt, -420.811pt) -- (104.7pt, -420.811pt)
;
\end{tikzpicture}
\begin{picture}(-5,0)(2.5,0)
\put(104.7,-424.211){\fontsize{12}{1}\usefont{T1}{cmr}{m}{n}\selectfont\color{color_29791} }
\end{picture}
\begin{tikzpicture}[overlay]
\path(0pt,0pt);
\draw[color_29791,line width=0.7pt]
(104.7pt, -420.811pt) -- (110.7pt, -420.811pt)
;
\end{tikzpicture}
\begin{picture}(-5,0)(2.5,0)
\put(113.8,-424.211){\fontsize{12}{1}\usefont{T1}{cmr}{m}{n}\selectfont\color{color_29791}3 }
\end{picture}
\begin{tikzpicture}[overlay]
\path(0pt,0pt);
\draw[color_29791,line width=0.7pt]
(113.8pt, -420.811pt) -- (122.8pt, -420.811pt)
;
\end{tikzpicture}
\begin{picture}(-5,0)(2.5,0)
\put(122.7,-424.211){\fontsize{12}{1}\usefont{T1}{cmr}{m}{n}\selectfont\color{color_29791}        }
\end{picture}
\begin{tikzpicture}[overlay]
\path(0pt,0pt);
\draw[color_29791,line width=0.7pt]
(122.7pt, -420.811pt) -- (149.2pt, -420.811pt)
;
\end{tikzpicture}
\begin{picture}(-5,0)(2.5,0)
\put(149.3,-424.211){\fontsize{12}{1}\usefont{T1}{cmr}{m}{n}\selectfont\color{color_29791}default gateway}
\end{picture}
\begin{tikzpicture}[overlay]
\path(0pt,0pt);
\draw[color_29791,line width=0.7pt]
(149.3pt, -420.811pt) -- (225.6pt, -420.811pt)
;
\end{tikzpicture}
\begin{picture}(-5,0)(2.5,0)
\put(95.8,-445.011){\fontsize{12}{1}\usefont{T1}{cmr}{m}{n}\selectfont\color{color_29791}▪}
\end{picture}
\begin{tikzpicture}[overlay]
\path(0pt,0pt);
\draw[color_29791,line width=0.7pt]
(95.8pt, -441.611pt) -- (104.7pt, -441.611pt)
;
\end{tikzpicture}
\begin{picture}(-5,0)(2.5,0)
\put(104.7,-445.011){\fontsize{12}{1}\usefont{T1}{cmr}{m}{n}\selectfont\color{color_29791} }
\end{picture}
\begin{tikzpicture}[overlay]
\path(0pt,0pt);
\draw[color_29791,line width=0.7pt]
(104.7pt, -441.611pt) -- (110.7pt, -441.611pt)
;
\end{tikzpicture}
\begin{picture}(-5,0)(2.5,0)
\put(113.8,-445.011){\fontsize{12}{1}\usefont{T1}{cmr}{m}{n}\selectfont\color{color_29791}4 }
\end{picture}
\begin{tikzpicture}[overlay]
\path(0pt,0pt);
\draw[color_29791,line width=0.7pt]
(113.8pt, -441.611pt) -- (122.8pt, -441.611pt)
;
\end{tikzpicture}
\begin{picture}(-5,0)(2.5,0)
\put(122.7,-445.011){\fontsize{12}{1}\usefont{T1}{cmr}{m}{n}\selectfont\color{color_29791}        }
\end{picture}
\begin{tikzpicture}[overlay]
\path(0pt,0pt);
\draw[color_29791,line width=0.7pt]
(122.7pt, -441.611pt) -- (149.2pt, -441.611pt)
;
\end{tikzpicture}
\begin{picture}(-5,0)(2.5,0)
\put(149.3,-445.011){\fontsize{12}{1}\usefont{T1}{cmr}{m}{n}\selectfont\color{color_29791}time server}
\end{picture}
\begin{tikzpicture}[overlay]
\path(0pt,0pt);
\draw[color_29791,line width=0.7pt]
(149.3pt, -441.611pt) -- (202.9pt, -441.611pt)
;
\end{tikzpicture}
\begin{picture}(-5,0)(2.5,0)
\put(95.8,-465.811){\fontsize{12}{1}\usefont{T1}{cmr}{m}{n}\selectfont\color{color_29791}▪}
\end{picture}
\begin{tikzpicture}[overlay]
\path(0pt,0pt);
\draw[color_29791,line width=0.7pt]
(95.8pt, -462.411pt) -- (104.7pt, -462.411pt)
;
\end{tikzpicture}
\begin{picture}(-5,0)(2.5,0)
\put(104.7,-465.811){\fontsize{12}{1}\usefont{T1}{cmr}{m}{n}\selectfont\color{color_29791} }
\end{picture}
\begin{tikzpicture}[overlay]
\path(0pt,0pt);
\draw[color_29791,line width=0.7pt]
(104.7pt, -462.411pt) -- (110.7pt, -462.411pt)
;
\end{tikzpicture}
\begin{picture}(-5,0)(2.5,0)
\put(113.8,-465.811){\fontsize{12}{1}\usefont{T1}{cmr}{m}{n}\selectfont\color{color_29791}6 }
\end{picture}
\begin{tikzpicture}[overlay]
\path(0pt,0pt);
\draw[color_29791,line width=0.7pt]
(113.8pt, -462.411pt) -- (122.8pt, -462.411pt)
;
\end{tikzpicture}
\begin{picture}(-5,0)(2.5,0)
\put(122.7,-465.811){\fontsize{12}{1}\usefont{T1}{cmr}{m}{n}\selectfont\color{color_29791}        }
\end{picture}
\begin{tikzpicture}[overlay]
\path(0pt,0pt);
\draw[color_29791,line width=0.7pt]
(122.7pt, -462.411pt) -- (149.2pt, -462.411pt)
;
\end{tikzpicture}
\begin{picture}(-5,0)(2.5,0)
\put(149.3,-465.811){\fontsize{12}{1}\usefont{T1}{cmr}{m}{n}\selectfont\color{color_29791}DNS server}
\end{picture}
\begin{tikzpicture}[overlay]
\path(0pt,0pt);
\draw[color_29791,line width=0.7pt]
(149.3pt, -462.411pt) -- (205.6pt, -462.411pt)
;
\end{tikzpicture}
\begin{picture}(-5,0)(2.5,0)
\put(95.8,-486.611){\fontsize{12}{1}\usefont{T1}{cmr}{m}{n}\selectfont\color{color_29791}▪}
\end{picture}
\begin{tikzpicture}[overlay]
\path(0pt,0pt);
\draw[color_29791,line width=0.7pt]
(95.8pt, -483.211pt) -- (104.7pt, -483.211pt)
;
\end{tikzpicture}
\begin{picture}(-5,0)(2.5,0)
\put(104.7,-486.611){\fontsize{12}{1}\usefont{T1}{cmr}{m}{n}\selectfont\color{color_29791} }
\end{picture}
\begin{tikzpicture}[overlay]
\path(0pt,0pt);
\draw[color_29791,line width=0.7pt]
(104.7pt, -483.211pt) -- (110.7pt, -483.211pt)
;
\end{tikzpicture}
\begin{picture}(-5,0)(2.5,0)
\put(113.8,-486.611){\fontsize{12}{1}\usefont{T1}{cmr}{m}{n}\selectfont\color{color_29791}12}
\end{picture}
\begin{tikzpicture}[overlay]
\path(0pt,0pt);
\draw[color_29791,line width=0.7pt]
(113.8pt, -483.211pt) -- (125.8pt, -483.211pt)
;
\end{tikzpicture}
\begin{picture}(-5,0)(2.5,0)
\put(125.7,-486.611){\fontsize{12}{1}\usefont{T1}{cmr}{m}{n}\selectfont\color{color_29791}       }
\end{picture}
\begin{tikzpicture}[overlay]
\path(0pt,0pt);
\draw[color_29791,line width=0.7pt]
(125.7pt, -483.211pt) -- (149.2pt, -483.211pt)
;
\end{tikzpicture}
\begin{picture}(-5,0)(2.5,0)
\put(149.3,-486.611){\fontsize{12}{1}\usefont{T1}{cmr}{m}{n}\selectfont\color{color_29791}host name}
\end{picture}
\begin{tikzpicture}[overlay]
\path(0pt,0pt);
\draw[color_29791,line width=0.7pt]
(149.3pt, -483.211pt) -- (198.3pt, -483.211pt)
;
\end{tikzpicture}
\begin{picture}(-5,0)(2.5,0)
\put(95.8,-507.411){\fontsize{12}{1}\usefont{T1}{cmr}{m}{n}\selectfont\color{color_29791}▪}
\end{picture}
\begin{tikzpicture}[overlay]
\path(0pt,0pt);
\draw[color_29791,line width=0.7pt]
(95.8pt, -504.011pt) -- (104.7pt, -504.011pt)
;
\end{tikzpicture}
\begin{picture}(-5,0)(2.5,0)
\put(104.7,-507.411){\fontsize{12}{1}\usefont{T1}{cmr}{m}{n}\selectfont\color{color_29791} }
\end{picture}
\begin{tikzpicture}[overlay]
\path(0pt,0pt);
\draw[color_29791,line width=0.7pt]
(104.7pt, -504.011pt) -- (110.7pt, -504.011pt)
;
\end{tikzpicture}
\begin{picture}(-5,0)(2.5,0)
\put(113.8,-507.411){\fontsize{12}{1}\usefont{T1}{cmr}{m}{n}\selectfont\color{color_29791}15}
\end{picture}
\begin{tikzpicture}[overlay]
\path(0pt,0pt);
\draw[color_29791,line width=0.7pt]
(113.8pt, -504.011pt) -- (125.8pt, -504.011pt)
;
\end{tikzpicture}
\begin{picture}(-5,0)(2.5,0)
\put(125.7,-507.411){\fontsize{12}{1}\usefont{T1}{cmr}{m}{n}\selectfont\color{color_29791}       }
\end{picture}
\begin{tikzpicture}[overlay]
\path(0pt,0pt);
\draw[color_29791,line width=0.7pt]
(125.7pt, -504.011pt) -- (149.2pt, -504.011pt)
;
\end{tikzpicture}
\begin{picture}(-5,0)(2.5,0)
\put(149.3,-507.411){\fontsize{12}{1}\usefont{T1}{cmr}{m}{n}\selectfont\color{color_29791}domain name}
\end{picture}
\begin{tikzpicture}[overlay]
\path(0pt,0pt);
\draw[color_29791,line width=0.7pt]
(149.3pt, -504.011pt) -- (214.3pt, -504.011pt)
;
\end{tikzpicture}
\begin{picture}(-5,0)(2.5,0)
\put(95.8,-528.211){\fontsize{12}{1}\usefont{T1}{cmr}{m}{n}\selectfont\color{color_29791}▪}
\end{picture}
\begin{tikzpicture}[overlay]
\path(0pt,0pt);
\draw[color_29791,line width=0.7pt]
(95.8pt, -524.811pt) -- (104.7pt, -524.811pt)
;
\end{tikzpicture}
\begin{picture}(-5,0)(2.5,0)
\put(104.7,-528.211){\fontsize{12}{1}\usefont{T1}{cmr}{m}{n}\selectfont\color{color_29791} }
\end{picture}
\begin{tikzpicture}[overlay]
\path(0pt,0pt);
\draw[color_29791,line width=0.7pt]
(104.7pt, -524.811pt) -- (110.7pt, -524.811pt)
;
\end{tikzpicture}
\begin{picture}(-5,0)(2.5,0)
\put(113.8,-528.211){\fontsize{12}{1}\usefont{T1}{cmr}{m}{n}\selectfont\color{color_29791}121}
\end{picture}
\begin{tikzpicture}[overlay]
\path(0pt,0pt);
\draw[color_29791,line width=0.7pt]
(113.8pt, -524.811pt) -- (131.8pt, -524.811pt)
;
\end{tikzpicture}
\begin{picture}(-5,0)(2.5,0)
\put(131.7,-528.211){\fontsize{12}{1}\usefont{T1}{cmr}{m}{n}\selectfont\color{color_29791}     }
\end{picture}
\begin{tikzpicture}[overlay]
\path(0pt,0pt);
\draw[color_29791,line width=0.7pt]
(131.7pt, -524.811pt) -- (149.2pt, -524.811pt)
;
\end{tikzpicture}
\begin{picture}(-5,0)(2.5,0)
\put(149.3,-528.211){\fontsize{12}{1}\usefont{T1}{cmr}{m}{n}\selectfont\color{color_29791}static route (parametro:network/netmask,gateway)}
\end{picture}
\begin{tikzpicture}[overlay]
\path(0pt,0pt);
\draw[color_29791,line width=0.7pt]
(149.3pt, -524.811pt) -- (390.8pt, -524.811pt)
;
\end{tikzpicture}
\begin{picture}(-5,0)(2.5,0)
\put(390.7,-528.211){\fontsize{12}{1}\usefont{T1}{cmr}{m}{n}\selectfont\color{color_29791}  }
\end{picture}
\begin{tikzpicture}[overlay]
\path(0pt,0pt);
\draw[color_29791,line width=0.7pt]
(390.7pt, -524.811pt) -- (397.3pt, -524.811pt)
;
\end{tikzpicture}
\begin{picture}(-5,0)(2.5,0)
\put(113.8,-549.011){\fontsize{12}{1}\usefont{T1}{cmr}{m}{n}\selectfont\color{color_29791}1.}
\end{picture}
\begin{tikzpicture}[overlay]
\path(0pt,0pt);
\draw[color_29791,line width=0.7pt]
(113.8pt, -545.611pt) -- (122.8pt, -545.611pt)
;
\end{tikzpicture}
\begin{picture}(-5,0)(2.5,0)
\put(122.7,-549.011){\fontsize{12}{1}\usefont{T1}{cmr}{m}{n}\selectfont\color{color_29791}   }
\end{picture}
\begin{tikzpicture}[overlay]
\path(0pt,0pt);
\draw[color_29791,line width=0.7pt]
(122.7pt, -545.611pt) -- (131.7pt, -545.611pt)
;
\end{tikzpicture}
\begin{picture}(-5,0)(2.5,0)
\put(131.8,-549.011){\fontsize{12}{1}\usefont{T1}{cmr}{m}{n}\selectfont\color{color_29791}possibile definire anche rotte statiche: es. per andare alla rete XXXX si passa }
\end{picture}
\begin{tikzpicture}[overlay]
\path(0pt,0pt);
\draw[color_29791,line width=0.7pt]
(131.8pt, -545.611pt) -- (504.9pt, -545.611pt)
;
\end{tikzpicture}
\begin{picture}(-5,0)(2.5,0)
\put(131.8,-562.811){\fontsize{12}{1}\usefont{T1}{cmr}{m}{n}\selectfont\color{color_29791}COMUNQUE da quel router lì}
\end{picture}
\begin{tikzpicture}[overlay]
\path(0pt,0pt);
\draw[color_29791,line width=0.7pt]
(131.8pt, -559.411pt) -- (280.4pt, -559.411pt)
;
\end{tikzpicture}
\begin{picture}(-5,0)(2.5,0)
\put(41.8,-583.611){\fontsize{12}{1}\usefont{T1}{cmr}{b}{n}\selectfont\color{color_29791}dnsmasq }
\end{picture}
\begin{tikzpicture}[overlay]
\path(0pt,0pt);
\draw[color_29791,line width=0.7pt]
(41.8pt, -580.211pt) -- (90.1pt, -580.211pt)
;
\end{tikzpicture}
\begin{picture}(-5,0)(2.5,0)
\put(90.2,-583.611){\fontsize{12}{1}\usefont{T1}{cmr}{m}{n}\selectfont\color{color_29791}può fare da DNS, opzioni base sono}
\end{picture}
\begin{tikzpicture}[overlay]
\path(0pt,0pt);
\draw[color_29791,line width=0.7pt]
(90.2pt, -580.211pt) -- (263.1pt, -580.211pt)
;
\end{tikzpicture}
\begin{picture}(-5,0)(2.5,0)
\put(59.8,-604.411){\fontsize{12}{1}\usefont{T1}{cmr}{m}{n}\selectfont\color{color_29791}•}
\end{picture}
\begin{tikzpicture}[overlay]
\path(0pt,0pt);
\draw[color_29791,line width=0.7pt]
(59.8pt, -601.011pt) -- (64.10001pt, -601.011pt)
;
\end{tikzpicture}
\begin{picture}(-5,0)(2.5,0)
\put(64,-604.411){\fontsize{12}{1}\usefont{T1}{cmr}{m}{n}\selectfont\color{color_29791}  }
\end{picture}
\begin{tikzpicture}[overlay]
\path(0pt,0pt);
\draw[color_29791,line width=0.7pt]
(64pt, -601.011pt) -- (77.7pt, -601.011pt)
;
\end{tikzpicture}
\begin{picture}(-5,0)(2.5,0)
\put(77.8,-604.411){\fontsize{12}{1}\usefont{T1}{cmr}{m}{it}\selectfont\color{color_29791}port=<dns server port>}
\end{picture}
\begin{tikzpicture}[overlay]
\path(0pt,0pt);
\draw[color_29791,line width=0.7pt]
(77.8pt, -601.011pt) -- (194.7pt, -601.011pt)
;
\end{tikzpicture}
\begin{picture}(-5,0)(2.5,0)
\put(77.8,-625.211){\fontsize{12}{1}\usefont{T1}{cmr}{m}{n}\selectfont\color{color_29791}◦}
\end{picture}
\begin{tikzpicture}[overlay]
\path(0pt,0pt);
\draw[color_29791,line width=0.7pt]
(77.8pt, -621.811pt) -- (87.3pt, -621.811pt)
;
\end{tikzpicture}
\begin{picture}(-5,0)(2.5,0)
\put(87.3,-625.211){\fontsize{12}{1}\usefont{T1}{cmr}{m}{n}\selectfont\color{color_29791} }
\end{picture}
\begin{tikzpicture}[overlay]
\path(0pt,0pt);
\draw[color_29791,line width=0.7pt]
(87.3pt, -621.811pt) -- (93.3pt, -621.811pt)
;
\end{tikzpicture}
\begin{picture}(-5,0)(2.5,0)
\put(95.8,-625.211){\fontsize{12}{1}\usefont{T1}{cmr}{m}{n}\selectfont\color{color_29791}di default 53, se impostato a 0 disabilita il server DNS}
\end{picture}
\begin{tikzpicture}[overlay]
\path(0pt,0pt);
\draw[color_29791,line width=0.7pt]
(95.8pt, -621.811pt) -- (356pt, -621.811pt)
;
\end{tikzpicture}
\begin{picture}(-5,0)(2.5,0)
\put(59.8,-646.011){\fontsize{12}{1}\usefont{T1}{cmr}{m}{n}\selectfont\color{color_29791}•}
\end{picture}
\begin{tikzpicture}[overlay]
\path(0pt,0pt);
\draw[color_29791,line width=0.7pt]
(59.8pt, -642.611pt) -- (64.10001pt, -642.611pt)
;
\end{tikzpicture}
\begin{picture}(-5,0)(2.5,0)
\put(64,-646.011){\fontsize{12}{1}\usefont{T1}{cmr}{m}{n}\selectfont\color{color_29791}  }
\end{picture}
\begin{tikzpicture}[overlay]
\path(0pt,0pt);
\draw[color_29791,line width=0.7pt]
(64pt, -642.611pt) -- (77.7pt, -642.611pt)
;
\end{tikzpicture}
\begin{picture}(-5,0)(2.5,0)
\put(77.8,-646.011){\fontsize{12}{1}\usefont{T1}{cmr}{m}{it}\selectfont\color{color_29791}local-service}
\end{picture}
\begin{tikzpicture}[overlay]
\path(0pt,0pt);
\draw[color_29791,line width=0.7pt]
(77.8pt, -642.611pt) -- (139.7pt, -642.611pt)
;
\end{tikzpicture}
\begin{picture}(-5,0)(2.5,0)
\put(77.8,-666.811){\fontsize{12}{1}\usefont{T1}{cmr}{m}{n}\selectfont\color{color_29791}◦}
\end{picture}
\begin{tikzpicture}[overlay]
\path(0pt,0pt);
\draw[color_29791,line width=0.7pt]
(77.8pt, -663.411pt) -- (87.3pt, -663.411pt)
;
\end{tikzpicture}
\begin{picture}(-5,0)(2.5,0)
\put(87.3,-666.811){\fontsize{12}{1}\usefont{T1}{cmr}{m}{n}\selectfont\color{color_29791} }
\end{picture}
\begin{tikzpicture}[overlay]
\path(0pt,0pt);
\draw[color_29791,line width=0.7pt]
(87.3pt, -663.411pt) -- (93.3pt, -663.411pt)
;
\end{tikzpicture}
\begin{picture}(-5,0)(2.5,0)
\put(95.8,-666.811){\fontsize{12}{1}\usefont{T1}{cmr}{m}{n}\selectfont\color{color_29791}accetta query DNS solo dagli host delle subnet locali al server: tipica di default, previene}
\end{picture}
\begin{tikzpicture}[overlay]
\path(0pt,0pt);
\draw[color_29791,line width=0.7pt]
(95.8pt, -663.411pt) -- (522.6pt, -663.411pt)
;
\end{tikzpicture}
\begin{picture}(-5,0)(2.5,0)
\put(95.8,-680.611){\fontsize{12}{1}\usefont{T1}{cmr}{m}{n}\selectfont\color{color_29791}DNS amplification attacks}
\end{picture}
\begin{tikzpicture}[overlay]
\path(0pt,0pt);
\draw[color_29791,line width=0.7pt]
(95.8pt, -677.211pt) -- (223.1pt, -677.211pt)
;
\end{tikzpicture}
\begin{picture}(-5,0)(2.5,0)
\put(77.8,-701.411){\fontsize{12}{1}\usefont{T1}{cmr}{m}{n}\selectfont\color{color_29791}◦}
\end{picture}
\begin{tikzpicture}[overlay]
\path(0pt,0pt);
\draw[color_29791,line width=0.7pt]
(77.8pt, -698.011pt) -- (87.3pt, -698.011pt)
;
\end{tikzpicture}
\begin{picture}(-5,0)(2.5,0)
\put(87.3,-701.411){\fontsize{12}{1}\usefont{T1}{cmr}{m}{n}\selectfont\color{color_29791} }
\end{picture}
\begin{tikzpicture}[overlay]
\path(0pt,0pt);
\draw[color_29791,line width=0.7pt]
(87.3pt, -698.011pt) -- (93.3pt, -698.011pt)
;
\end{tikzpicture}
\begin{picture}(-5,0)(2.5,0)
\put(95.8,-701.411){\fontsize{12}{1}\usefont{T1}{cmr}{m}{n}\selectfont\color{color_29791}ha effetto solo se non è specificata nessuna delle opzioni seguenti}
\end{picture}
\begin{tikzpicture}[overlay]
\path(0pt,0pt);
\draw[color_29791,line width=0.7pt]
(95.8pt, -698.011pt) -- (410.1pt, -698.011pt)
;
\end{tikzpicture}
\begin{picture}(-5,0)(2.5,0)
\put(95.8,-722.211){\fontsize{12}{1}\usefont{T1}{cmr}{m}{n}\selectfont\color{color_29791}▪}
\end{picture}
\begin{tikzpicture}[overlay]
\path(0pt,0pt);
\draw[color_29791,line width=0.7pt]
(95.8pt, -718.811pt) -- (104.7pt, -718.811pt)
;
\end{tikzpicture}
\begin{picture}(-5,0)(2.5,0)
\put(104.7,-722.211){\fontsize{12}{1}\usefont{T1}{cmr}{m}{n}\selectfont\color{color_29791} }
\end{picture}
\begin{tikzpicture}[overlay]
\path(0pt,0pt);
\draw[color_29791,line width=0.7pt]
(104.7pt, -718.811pt) -- (110.7pt, -718.811pt)
;
\end{tikzpicture}
\begin{picture}(-5,0)(2.5,0)
\put(113.8,-722.211){\fontsize{12}{1}\usefont{T1}{cmr}{m}{it}\selectfont\color{color_29791}interface }
\end{picture}
\begin{tikzpicture}[overlay]
\path(0pt,0pt);
\draw[color_29791,line width=0.7pt]
(113.8pt, -718.811pt) -- (159.4pt, -718.811pt)
;
\end{tikzpicture}
\begin{picture}(-5,0)(2.5,0)
\put(95.8,-743.011){\fontsize{12}{1}\usefont{T1}{cmr}{m}{n}\selectfont\color{color_29791}▪}
\end{picture}
\begin{tikzpicture}[overlay]
\path(0pt,0pt);
\draw[color_29791,line width=0.7pt]
(95.8pt, -739.611pt) -- (104.7pt, -739.611pt)
;
\end{tikzpicture}
\begin{picture}(-5,0)(2.5,0)
\put(104.7,-743.011){\fontsize{12}{1}\usefont{T1}{cmr}{m}{n}\selectfont\color{color_29791} }
\end{picture}
\begin{tikzpicture}[overlay]
\path(0pt,0pt);
\draw[color_29791,line width=0.7pt]
(104.7pt, -739.611pt) -- (110.7pt, -739.611pt)
;
\end{tikzpicture}
\begin{picture}(-5,0)(2.5,0)
\put(113.8,-743.011){\fontsize{12}{1}\usefont{T1}{cmr}{m}{it}\selectfont\color{color_29791}except-interface}
\end{picture}
\begin{tikzpicture}[overlay]
\path(0pt,0pt);
\draw[color_29791,line width=0.7pt]
(113.8pt, -739.611pt) -- (191.1pt, -739.611pt)
;
\end{tikzpicture}
\begin{picture}(-5,0)(2.5,0)
\put(95.8,-763.811){\fontsize{12}{1}\usefont{T1}{cmr}{m}{n}\selectfont\color{color_29791}▪}
\end{picture}
\begin{tikzpicture}[overlay]
\path(0pt,0pt);
\draw[color_29791,line width=0.7pt]
(95.8pt, -760.411pt) -- (104.7pt, -760.411pt)
;
\end{tikzpicture}
\begin{picture}(-5,0)(2.5,0)
\put(104.7,-763.811){\fontsize{12}{1}\usefont{T1}{cmr}{m}{n}\selectfont\color{color_29791} }
\end{picture}
\begin{tikzpicture}[overlay]
\path(0pt,0pt);
\draw[color_29791,line width=0.7pt]
(104.7pt, -760.411pt) -- (110.7pt, -760.411pt)
;
\end{tikzpicture}
\begin{picture}(-5,0)(2.5,0)
\put(113.8,-763.811){\fontsize{12}{1}\usefont{T1}{cmr}{m}{it}\selectfont\color{color_29791}listen-address}
\end{picture}
\begin{tikzpicture}[overlay]
\path(0pt,0pt);
\draw[color_29791,line width=0.7pt]
(113.8pt, -760.411pt) -- (180.7pt, -760.411pt)
;
\end{tikzpicture}
\newpage
\begin{tikzpicture}[overlay]\path(0pt,0pt);\end{tikzpicture}
\begin{picture}(-5,0)(2.5,0)
\put(95.8,-85.01099){\fontsize{12}{1}\usefont{T1}{cmr}{m}{n}\selectfont\color{color_29791}▪}
\end{picture}
\begin{tikzpicture}[overlay]
\path(0pt,0pt);
\draw[color_29791,line width=0.7pt]
(95.8pt, -81.61096pt) -- (104.7pt, -81.61096pt)
;
\end{tikzpicture}
\begin{picture}(-5,0)(2.5,0)
\put(104.7,-85.01099){\fontsize{12}{1}\usefont{T1}{cmr}{m}{n}\selectfont\color{color_29791} }
\end{picture}
\begin{tikzpicture}[overlay]
\path(0pt,0pt);
\draw[color_29791,line width=0.7pt]
(104.7pt, -81.61096pt) -- (110.7pt, -81.61096pt)
;
\end{tikzpicture}
\begin{picture}(-5,0)(2.5,0)
\put(113.8,-85.01099){\fontsize{12}{1}\usefont{T1}{cmr}{m}{it}\selectfont\color{color_29791}auth-server options}
\end{picture}
\begin{tikzpicture}[overlay]
\path(0pt,0pt);
\draw[color_29791,line width=0.7pt]
(113.8pt, -81.61096pt) -- (207.4pt, -81.61096pt)
;
\end{tikzpicture}
\begin{picture}(-5,0)(2.5,0)
\put(41.8,-105.811){\fontsize{12}{1}\usefont{T1}{cmr}{m}{n}\selectfont\color{color_29791}Non fa da resolver ricorsivo, deve appoggiarsi ad uno esterno (prende coppie nome-indirizzo da }
\end{picture}
\begin{tikzpicture}[overlay]
\path(0pt,0pt);
\draw[color_29791,line width=0.7pt]
(41.8pt, -102.4109pt) -- (505.3pt, -102.4109pt)
;
\end{tikzpicture}
\begin{picture}(-5,0)(2.5,0)
\put(41.8,-119.611){\fontsize{12}{1}\usefont{T1}{cmr}{m}{n}\selectfont\color{color_29791}qualcun altro, è un cached DNS e non un DNS autorevole): di default prende indirizzi dei }
\end{picture}
\begin{tikzpicture}[overlay]
\path(0pt,0pt);
\draw[color_29791,line width=0.7pt]
(41.8pt, -116.211pt) -- (474.6pt, -116.211pt)
;
\end{tikzpicture}
\begin{picture}(-5,0)(2.5,0)
\put(41.8,-133.411){\fontsize{12}{1}\usefont{T1}{cmr}{m}{n}\selectfont\color{color_29791}nameserver upstream da }
\end{picture}
\begin{tikzpicture}[overlay]
\path(0pt,0pt);
\draw[color_29791,line width=0.7pt]
(41.8pt, -130.011pt) -- (161.4pt, -130.011pt)
;
\end{tikzpicture}
\begin{picture}(-5,0)(2.5,0)
\put(161.3,-133.411){\fontsize{12}{1}\usefont{T1}{cmr}{m}{n}\selectfont\color{color_29791}       }
\end{picture}
\begin{tikzpicture}[overlay]
\path(0pt,0pt);
\draw[color_29791,line width=0.7pt]
(161.3pt, -130.011pt) -- (183.5pt, -130.011pt)
;
\end{tikzpicture}
\begin{picture}(-5,0)(2.5,0)
\put(183.6,-133.411){\fontsize{12}{1}\usefont{T1}{cmr}{m}{n}\selectfont\color{color_29791}/etc/resolv.conf}
\end{picture}
\begin{tikzpicture}[overlay]
\path(0pt,0pt);
\draw[color_29791,line width=0.7pt]
(183.6pt, -130.011pt) -- (257.1pt, -130.011pt)
;
\end{tikzpicture}
\begin{picture}(-5,0)(2.5,0)
\put(257,-133.411){\fontsize{12}{1}\usefont{T1}{cmr}{m}{n}\selectfont\color{color_29791}          }
\end{picture}
\begin{tikzpicture}[overlay]
\path(0pt,0pt);
\draw[color_29791,line width=0.7pt]
(257pt, -130.011pt) -- (289.9pt, -130.011pt)
;
\end{tikzpicture}
\begin{picture}(-5,0)(2.5,0)
\put(290,-133.411){\fontsize{12}{1}\usefont{T1}{cmr}{m}{n}\selectfont\color{color_29791}(che viene aggiornato in automatico dai demoni }
\end{picture}
\begin{tikzpicture}[overlay]
\path(0pt,0pt);
\draw[color_29791,line width=0.7pt]
(290pt, -130.011pt) -- (522.2pt, -130.011pt)
;
\end{tikzpicture}
\begin{picture}(-5,0)(2.5,0)
\put(41.8,-147.211){\fontsize{12}{1}\usefont{T1}{cmr}{m}{n}\selectfont\color{color_29791}es. dhcp e dnsmasq si accorge automaticamente delle modifiche).}
\end{picture}
\begin{tikzpicture}[overlay]
\path(0pt,0pt);
\draw[color_29791,line width=0.7pt]
(41.8pt, -143.811pt) -- (355.4pt, -143.811pt)
;
\end{tikzpicture}
\begin{picture}(-5,0)(2.5,0)
\put(41.8,-168.011){\fontsize{12}{1}\usefont{T1}{cmr}{m}{n}\selectfont\color{color_29791}Per usarlo localmente, bisogna aggiungere }
\end{picture}
\begin{tikzpicture}[overlay]
\path(0pt,0pt);
\draw[color_29791,line width=0.7pt]
(41.8pt, -164.611pt) -- (248.4pt, -164.611pt)
;
\end{tikzpicture}
\begin{picture}(-5,0)(2.5,0)
\put(248.3,-168.011){\fontsize{12}{1}\usefont{T1}{cmr}{m}{n}\selectfont\color{color_29791}  }
\end{picture}
\begin{tikzpicture}[overlay]
\path(0pt,0pt);
\draw[color_29791,line width=0.7pt]
(248.3pt, -164.611pt) -- (254.4pt, -164.611pt)
;
\end{tikzpicture}
\begin{picture}(-5,0)(2.5,0)
\put(254.5,-168.011){\fontsize{12}{1}\usefont{T1}{cmr}{m}{it}\selectfont\color{color_29791}nameserver 127.0.0.1 }
\end{picture}
\begin{tikzpicture}[overlay]
\path(0pt,0pt);
\draw[color_29791,line width=0.7pt]
(254.5pt, -164.611pt) -- (361.4pt, -164.611pt)
;
\end{tikzpicture}
\begin{picture}(-5,0)(2.5,0)
\put(361.4,-168.011){\fontsize{12}{1}\usefont{T1}{cmr}{m}{it}\selectfont\color{color_29791}           }
\end{picture}
\begin{tikzpicture}[overlay]
\path(0pt,0pt);
\draw[color_29791,line width=0.7pt]
(361.4pt, -164.611pt) -- (396.3pt, -164.611pt)
;
\end{tikzpicture}
\begin{picture}(-5,0)(2.5,0)
\put(396.3,-168.011){\fontsize{12}{1}\usefont{T1}{cmr}{m}{n}\selectfont\color{color_29791}al file /etc/resolv.conf, }
\end{picture}
\begin{tikzpicture}[overlay]
\path(0pt,0pt);
\draw[color_29791,line width=0.7pt]
(396.3pt, -164.611pt) -- (506.4pt, -164.611pt)
;
\end{tikzpicture}
\begin{picture}(-5,0)(2.5,0)
\put(41.8,-181.811){\fontsize{12}{1}\usefont{T1}{cmr}{m}{n}\selectfont\color{color_29791}inoltre è necessario configurare gli upstream server perché non ci sia loop di richiesta a sé stesso:}
\end{picture}
\begin{tikzpicture}[overlay]
\path(0pt,0pt);
\draw[color_29791,line width=0.7pt]
(41.8pt, -178.4109pt) -- (508.2pt, -178.4109pt)
;
\end{tikzpicture}
\begin{picture}(-5,0)(2.5,0)
\put(59.8,-202.611){\fontsize{12}{1}\usefont{T1}{cmr}{m}{n}\selectfont\color{color_29791}•}
\end{picture}
\begin{tikzpicture}[overlay]
\path(0pt,0pt);
\draw[color_29791,line width=0.7pt]
(59.8pt, -199.211pt) -- (64.10001pt, -199.211pt)
;
\end{tikzpicture}
\begin{picture}(-5,0)(2.5,0)
\put(64,-202.611){\fontsize{12}{1}\usefont{T1}{cmr}{m}{n}\selectfont\color{color_29791}  }
\end{picture}
\begin{tikzpicture}[overlay]
\path(0pt,0pt);
\draw[color_29791,line width=0.7pt]
(64pt, -199.211pt) -- (77.7pt, -199.211pt)
;
\end{tikzpicture}
\begin{picture}(-5,0)(2.5,0)
\put(77.8,-202.611){\fontsize{12}{1}\usefont{T1}{cmr}{m}{n}\selectfont\color{color_29791}con}
\end{picture}
\begin{tikzpicture}[overlay]
\path(0pt,0pt);
\draw[color_29791,line width=0.7pt]
(77.8pt, -199.211pt) -- (95.10001pt, -199.211pt)
;
\end{tikzpicture}
\begin{picture}(-5,0)(2.5,0)
\put(95.1,-202.611){\fontsize{12}{1}\usefont{T1}{cmr}{m}{n}\selectfont\color{color_29791}      }
\end{picture}
\begin{tikzpicture}[overlay]
\path(0pt,0pt);
\draw[color_29791,line width=0.7pt]
(95.1pt, -199.211pt) -- (113.2pt, -199.211pt)
;
\end{tikzpicture}
\begin{picture}(-5,0)(2.5,0)
\put(113.3,-202.611){\fontsize{12}{1}\usefont{T1}{cmr}{m}{it}\selectfont\color{color_29791}resolv-file=<file>}
\end{picture}
\begin{tikzpicture}[overlay]
\path(0pt,0pt);
\draw[color_29791,line width=0.7pt]
(113.3pt, -199.211pt) -- (201.1pt, -199.211pt)
;
\end{tikzpicture}
\begin{picture}(-5,0)(2.5,0)
\put(77.8,-223.411){\fontsize{12}{1}\usefont{T1}{cmr}{m}{n}\selectfont\color{color_29791}◦}
\end{picture}
\begin{tikzpicture}[overlay]
\path(0pt,0pt);
\draw[color_29791,line width=0.7pt]
(77.8pt, -220.011pt) -- (87.3pt, -220.011pt)
;
\end{tikzpicture}
\begin{picture}(-5,0)(2.5,0)
\put(87.3,-223.411){\fontsize{12}{1}\usefont{T1}{cmr}{m}{n}\selectfont\color{color_29791} }
\end{picture}
\begin{tikzpicture}[overlay]
\path(0pt,0pt);
\draw[color_29791,line width=0.7pt]
(87.3pt, -220.011pt) -- (93.3pt, -220.011pt)
;
\end{tikzpicture}
\begin{picture}(-5,0)(2.5,0)
\put(95.8,-223.411){\fontsize{12}{1}\usefont{T1}{cmr}{m}{n}\selectfont\color{color_29791}sopprime l'uso di /etc/resolv.conf}
\end{picture}
\begin{tikzpicture}[overlay]
\path(0pt,0pt);
\draw[color_29791,line width=0.7pt]
(95.8pt, -220.011pt) -- (254.4pt, -220.011pt)
;
\end{tikzpicture}
\begin{picture}(-5,0)(2.5,0)
\put(59.8,-244.211){\fontsize{12}{1}\usefont{T1}{cmr}{m}{n}\selectfont\color{color_29791}•}
\end{picture}
\begin{tikzpicture}[overlay]
\path(0pt,0pt);
\draw[color_29791,line width=0.7pt]
(59.8pt, -240.811pt) -- (64.10001pt, -240.811pt)
;
\end{tikzpicture}
\begin{picture}(-5,0)(2.5,0)
\put(64,-244.211){\fontsize{12}{1}\usefont{T1}{cmr}{m}{n}\selectfont\color{color_29791}  }
\end{picture}
\begin{tikzpicture}[overlay]
\path(0pt,0pt);
\draw[color_29791,line width=0.7pt]
(64pt, -240.811pt) -- (77.7pt, -240.811pt)
;
\end{tikzpicture}
\begin{picture}(-5,0)(2.5,0)
\put(77.8,-244.211){\fontsize{12}{1}\usefont{T1}{cmr}{m}{n}\selectfont\color{color_29791}con}
\end{picture}
\begin{tikzpicture}[overlay]
\path(0pt,0pt);
\draw[color_29791,line width=0.7pt]
(77.8pt, -240.811pt) -- (95.10001pt, -240.811pt)
;
\end{tikzpicture}
\begin{picture}(-5,0)(2.5,0)
\put(95.1,-244.211){\fontsize{12}{1}\usefont{T1}{cmr}{m}{n}\selectfont\color{color_29791}      }
\end{picture}
\begin{tikzpicture}[overlay]
\path(0pt,0pt);
\draw[color_29791,line width=0.7pt]
(95.1pt, -240.811pt) -- (113.2pt, -240.811pt)
;
\end{tikzpicture}
\begin{picture}(-5,0)(2.5,0)
\put(113.3,-244.211){\fontsize{12}{1}\usefont{T1}{cmr}{m}{it}\selectfont\color{color_29791}server=[/<domain>/]<ipaddr>}
\end{picture}
\begin{tikzpicture}[overlay]
\path(0pt,0pt);
\draw[color_29791,line width=0.7pt]
(113.3pt, -240.811pt) -- (267.8pt, -240.811pt)
;
\end{tikzpicture}
\begin{picture}(-5,0)(2.5,0)
\put(77.8,-265.011){\fontsize{12}{1}\usefont{T1}{cmr}{m}{n}\selectfont\color{color_29791}◦}
\end{picture}
\begin{tikzpicture}[overlay]
\path(0pt,0pt);
\draw[color_29791,line width=0.7pt]
(77.8pt, -261.611pt) -- (87.3pt, -261.611pt)
;
\end{tikzpicture}
\begin{picture}(-5,0)(2.5,0)
\put(87.3,-265.011){\fontsize{12}{1}\usefont{T1}{cmr}{m}{n}\selectfont\color{color_29791} }
\end{picture}
\begin{tikzpicture}[overlay]
\path(0pt,0pt);
\draw[color_29791,line width=0.7pt]
(87.3pt, -261.611pt) -- (93.3pt, -261.611pt)
;
\end{tikzpicture}
\begin{picture}(-5,0)(2.5,0)
\put(95.8,-265.011){\fontsize{12}{1}\usefont{T1}{cmr}{m}{n}\selectfont\color{color_29791}si deve aggiungere }
\end{picture}
\begin{tikzpicture}[overlay]
\path(0pt,0pt);
\draw[color_29791,line width=0.7pt]
(95.8pt, -261.611pt) -- (188.7pt, -261.611pt)
;
\end{tikzpicture}
\begin{picture}(-5,0)(2.5,0)
\put(188.7,-265.011){\fontsize{12}{1}\usefont{T1}{cmr}{m}{n}\selectfont\color{color_29791}    }
\end{picture}
\begin{tikzpicture}[overlay]
\path(0pt,0pt);
\draw[color_29791,line width=0.7pt]
(188.7pt, -261.611pt) -- (202.1pt, -261.611pt)
;
\end{tikzpicture}
\begin{picture}(-5,0)(2.5,0)
\put(202.2,-265.011){\fontsize{12}{1}\usefont{T1}{cmr}{m}{it}\selectfont\color{color_29791}no-resolv}
\end{picture}
\begin{tikzpicture}[overlay]
\path(0pt,0pt);
\draw[color_29791,line width=0.7pt]
(202.2pt, -261.611pt) -- (247.1pt, -261.611pt)
;
\end{tikzpicture}
\begin{picture}(-5,0)(2.5,0)
\put(247,-265.011){\fontsize{12}{1}\usefont{T1}{cmr}{m}{it}\selectfont\color{color_29791}        }
\end{picture}
\begin{tikzpicture}[overlay]
\path(0pt,0pt);
\draw[color_29791,line width=0.7pt]
(247pt, -261.611pt) -- (273pt, -261.611pt)
;
\end{tikzpicture}
\begin{picture}(-5,0)(2.5,0)
\put(273.1,-265.011){\fontsize{12}{1}\usefont{T1}{cmr}{m}{it}\selectfont\color{color_29791}, }
\end{picture}
\begin{tikzpicture}[overlay]
\path(0pt,0pt);
\draw[color_29791,line width=0.7pt]
(273.1pt, -261.611pt) -- (279.1pt, -261.611pt)
;
\end{tikzpicture}
\begin{picture}(-5,0)(2.5,0)
\put(279.1,-265.011){\fontsize{12}{1}\usefont{T1}{cmr}{m}{n}\selectfont\color{color_29791}per evitare uso di /etc/resolv.conf}
\end{picture}
\begin{tikzpicture}[overlay]
\path(0pt,0pt);
\draw[color_29791,line width=0.7pt]
(279.1pt, -261.611pt) -- (438.6pt, -261.611pt)
;
\end{tikzpicture}
\begin{picture}(-5,0)(2.5,0)
\put(41.8,-285.811){\fontsize{12}{1}\usefont{T1}{cmr}{b}{n}\selectfont\color{color_29791}ntpd è client e/o server in funzione della configurazione, fatta su /etc/ntp.conf.}
\put(41.8,-306.611){\fontsize{12}{1}\usefont{T1}{cmr}{m}{n}\selectfont\color{color_29791}Criteri di sicurezza da usare con NTP: non avrebbe senso lasciarlo aperto; viaggia su UDP quindi }
\put(41.8,-320.411){\fontsize{12}{1}\usefont{T1}{cmr}{m}{n}\selectfont\color{color_29791}qualcuno potrebbe inviare un pacchetto per far cambiare l'ora: mediante direttive si specifica di }
\put(41.8,-334.211){\fontsize{12}{1}\usefont{T1}{cmr}{m}{n}\selectfont\color{color_29791}accettare l'ora solo da alcuni server.}
\put(41.8,-355.011){\fontsize{12}{1}\usefont{T1}{cmr}{m}{n}\selectfont\color{color_29791}Il tool ntpdate permette di sincronizzare l'orologio locale ad un server NTP, non rimpiazza ntpd, che}
\put(41.8,-368.811){\fontsize{12}{1}\usefont{T1}{cmr}{m}{n}\selectfont\color{color_29791}usa algoritmi sofisticati per compensare errori e ritardi dei pacchetti dai server. (dice che tutti }
\end{picture}
\begin{tikzpicture}[overlay]
\path(0pt,0pt);
\draw[color_29791,line width=0.7pt]
(422.7pt, -365.411pt) -- (491.6pt, -365.411pt)
;
\end{tikzpicture}
\begin{picture}(-5,0)(2.5,0)
\put(41.8,-382.611){\fontsize{12}{1}\usefont{T1}{cmr}{m}{n}\selectfont\color{color_29791}parametri i configurazione li vedremo in laboratorio ma così non è stato?)}
\end{picture}
\begin{tikzpicture}[overlay]
\path(0pt,0pt);
\draw[color_29791,line width=0.7pt]
(41.8pt, -379.211pt) -- (396pt, -379.211pt)
;
\end{tikzpicture}
\begin{picture}(-5,0)(2.5,0)
\put(396,-382.611){\fontsize{12}{1}\usefont{T1}{cmr}{m}{n}\selectfont\color{color_29791}. Senza parametri usa i }
\put(41.8,-396.411){\fontsize{12}{1}\usefont{T1}{cmr}{m}{n}\selectfont\color{color_29791}server in /etc/ntp.conf(ntpd non deve essere attivo), ma accetta come parametro un server specifico.}
\put(41.8,-410.211){\fontsize{12}{1}\usefont{T1}{cmr}{m}{n}\selectfont\color{color_29791}L'ora viene modificata con step se errore > 0.5sec, con slew (mediante adjtime() )  se <0.5 sec. }
\put(41.8,-431.011){\fontsize{12}{1}\usefont{T1}{cmr}{m}{n}\selectfont\color{color_29791}Lato client: (non mostrato???)}
\end{picture}
\begin{tikzpicture}[overlay]
\path(0pt,0pt);
\draw[color_29791,line width=0.7pt]
(99.8pt, -427.611pt) -- (186.7pt, -427.611pt)
;
\end{tikzpicture}
\begin{picture}(-5,0)(2.5,0)
\put(186.8,-431.011){\fontsize{12}{1}\usefont{T1}{cmr}{m}{n}\selectfont\color{color_29791} 17-06 FOR ALL I KNOW, THAT'S WHATS NEEDED FOR DHCP?}
\put(59.8,-451.811){\fontsize{12}{1}\usefont{T1}{cmr}{m}{n}\selectfont\color{color_29791}•DHCP}
\put(77.8,-472.611){\fontsize{12}{1}\usefont{T1}{cmr}{m}{n}\selectfont\color{color_29791}◦Pacchetto isc-dhcp-clientfornisce comando dhclient, che lanciato senza }
\put(95.8,-486.411){\fontsize{12}{1}\usefont{T1}{cmr}{m}{n}\selectfont\color{color_29791}parametri avvia un demone che tenta di configurare tutte le interfacce. Di solito è }
\put(95.8,-500.211){\fontsize{12}{1}\usefont{T1}{cmr}{m}{n}\selectfont\color{color_29791}avviato da interfaces (/etc/network/interfaces) con }
\put(239.8,-521.011){\fontsize{12}{1}\usefont{T1}{cmr}{m}{it}\selectfont\color{color_29791}auto <ifname> }
\put(239.8,-541.811){\fontsize{12}{1}\usefont{T1}{cmr}{m}{it}\selectfont\color{color_29791}iface <ifname> inet dhcp}
\put(77.8,-562.611){\fontsize{12}{1}\usefont{T1}{cmr}{m}{n}\selectfont\color{color_29791}◦File /etc/dhcp/dhclient.conf per parametri impostabili}
\put(77.8,-583.411){\fontsize{12}{1}\usefont{T1}{cmr}{m}{n}\selectfont\color{color_29791}◦Hook per esecuzione automatica di script al cambio di stato interfaccia, nelle dir. }
\put(95.8,-604.211){\fontsize{12}{1}\usefont{T1}{cmr}{m}{n}\selectfont\color{color_29791}▪/etc/dchp/dhclient-enter-hooks.d/*}
\put(95.8,-625.011){\fontsize{12}{1}\usefont{T1}{cmr}{m}{n}\selectfont\color{color_29791}▪/etc/dhcp/dhclient-exit-hooks.d/*}
\put(59.8,-645.811){\fontsize{12}{1}\usefont{T1}{cmr}{m}{n}\selectfont\color{color_29791}•}
\end{picture}
\begin{tikzpicture}[overlay]
\path(0pt,0pt);
\draw[color_29791,line width=0.7pt]
(59.8pt, -642.411pt) -- (64.10001pt, -642.411pt)
;
\end{tikzpicture}
\begin{picture}(-5,0)(2.5,0)
\put(64,-645.811){\fontsize{12}{1}\usefont{T1}{cmr}{m}{n}\selectfont\color{color_29791}  }
\end{picture}
\begin{tikzpicture}[overlay]
\path(0pt,0pt);
\draw[color_29791,line width=0.7pt]
(64pt, -642.411pt) -- (77.7pt, -642.411pt)
;
\end{tikzpicture}
\begin{picture}(-5,0)(2.5,0)
\put(77.8,-645.811){\fontsize{12}{1}\usefont{T1}{cmr}{m}{n}\selectfont\color{color_29791}zeroconf}
\end{picture}
\begin{tikzpicture}[overlay]
\path(0pt,0pt);
\draw[color_29791,line width=0.7pt]
(77.8pt, -642.411pt) -- (119.8pt, -642.411pt)
;
\end{tikzpicture}
\begin{picture}(-5,0)(2.5,0)
\put(77.8,-666.611){\fontsize{12}{1}\usefont{T1}{cmr}{m}{n}\selectfont\color{color_29791}◦}
\end{picture}
\begin{tikzpicture}[overlay]
\path(0pt,0pt);
\draw[color_29791,line width=0.7pt]
(77.8pt, -663.211pt) -- (87.3pt, -663.211pt)
;
\end{tikzpicture}
\begin{picture}(-5,0)(2.5,0)
\put(87.3,-666.611){\fontsize{12}{1}\usefont{T1}{cmr}{m}{n}\selectfont\color{color_29791} }
\end{picture}
\begin{tikzpicture}[overlay]
\path(0pt,0pt);
\draw[color_29791,line width=0.7pt]
(87.3pt, -663.211pt) -- (93.3pt, -663.211pt)
;
\end{tikzpicture}
\begin{picture}(-5,0)(2.5,0)
\put(95.8,-666.611){\fontsize{12}{1}\usefont{T1}{cmr}{m}{n}\selectfont\color{color_29791}framework }
\end{picture}
\begin{tikzpicture}[overlay]
\path(0pt,0pt);
\draw[color_29791,line width=0.7pt]
(95.8pt, -663.211pt) -- (151.4pt, -663.211pt)
;
\end{tikzpicture}
\begin{picture}(-5,0)(2.5,0)
\put(151.4,-666.611){\fontsize{12}{1}\usefont{T1}{cmr}{m}{n}\selectfont\color{color_29791}     }
\end{picture}
\begin{tikzpicture}[overlay]
\path(0pt,0pt);
\draw[color_29791,line width=0.7pt]
(151.4pt, -663.211pt) -- (166.7pt, -663.211pt)
;
\end{tikzpicture}
\begin{picture}(-5,0)(2.5,0)
\put(166.7,-666.611){\fontsize{12}{1}\usefont{T1}{cmr}{m}{it}\selectfont\color{color_29791}avahi}
\end{picture}
\begin{tikzpicture}[overlay]
\path(0pt,0pt);
\draw[color_29791,line width=0.7pt]
(166.7pt, -663.211pt) -- (193.3pt, -663.211pt)
;
\end{tikzpicture}
\begin{picture}(-5,0)(2.5,0)
\put(193.3,-666.611){\fontsize{12}{1}\usefont{T1}{cmr}{m}{it}\selectfont\color{color_29791}  }
\end{picture}
\begin{tikzpicture}[overlay]
\path(0pt,0pt);
\draw[color_29791,line width=0.7pt]
(193.3pt, -663.211pt) -- (202.1pt, -663.211pt)
;
\end{tikzpicture}
\begin{picture}(-5,0)(2.5,0)
\put(202.2,-666.611){\fontsize{12}{1}\usefont{T1}{cmr}{m}{n}\selectfont\color{color_29791}offre stack completo mDNS/DNS-SD con API per integrazione }
\end{picture}
\begin{tikzpicture}[overlay]
\path(0pt,0pt);
\draw[color_29791,line width=0.7pt]
(202.2pt, -663.211pt) -- (509.2pt, -663.211pt)
;
\end{tikzpicture}
\begin{picture}(-5,0)(2.5,0)
\put(95.8,-680.411){\fontsize{12}{1}\usefont{T1}{cmr}{m}{n}\selectfont\color{color_29791}delle funzionalità con qualsiasi programma C; }
\end{picture}
\begin{tikzpicture}[overlay]
\path(0pt,0pt);
\draw[color_29791,line width=0.7pt]
(95.8pt, -677.011pt) -- (320.4pt, -677.011pt)
;
\end{tikzpicture}
\begin{picture}(-5,0)(2.5,0)
\put(320.4,-680.411){\fontsize{12}{1}\usefont{T1}{cmr}{m}{n}\selectfont\color{color_29791}un demone}
\end{picture}
\begin{tikzpicture}[overlay]
\path(0pt,0pt);
\draw[color_29791,line width=0.7pt]
(320.4pt, -681.511pt) -- (373.4pt, -681.511pt)
;
\draw[color_29791,line width=0.7pt]
(320.4pt, -677.011pt) -- (373.4pt, -677.011pt)
;
\end{tikzpicture}
\begin{picture}(-5,0)(2.5,0)
\put(373.4,-680.411){\fontsize{12}{1}\usefont{T1}{cmr}{m}{n}\selectfont\color{color_29791} }
\end{picture}
\begin{tikzpicture}[overlay]
\path(0pt,0pt);
\draw[color_29791,line width=0.7pt]
(373.4pt, -677.011pt) -- (376.4pt, -677.011pt)
;
\end{tikzpicture}
\begin{picture}(-5,0)(2.5,0)
\put(376.4,-680.411){\fontsize{12}{1}\usefont{T1}{cmr}{m}{n}\selectfont\color{color_29791}per gestire registrazione nuovi}
\end{picture}
\begin{tikzpicture}[overlay]
\path(0pt,0pt);
\draw[color_29791,line width=0.7pt]
(376.4pt, -677.011pt) -- (522pt, -677.011pt)
;
\end{tikzpicture}
\begin{picture}(-5,0)(2.5,0)
\put(95.8,-694.211){\fontsize{12}{1}\usefont{T1}{cmr}{m}{n}\selectfont\color{color_29791}servizi in modo orchestrato da qualsiasi programma non scritto in C (via D-bus); un }
\end{picture}
\begin{tikzpicture}[overlay]
\path(0pt,0pt);
\draw[color_29791,line width=0.7pt]
(95.8pt, -690.811pt) -- (501pt, -690.811pt)
;
\end{tikzpicture}
\begin{picture}(-5,0)(2.5,0)
\put(95.8,-708.011){\fontsize{12}{1}\usefont{T1}{cmr}{m}{n}\selectfont\color{color_29791}client/wrapper C che semplifica uso di D-Bus; adattatori per integrare le API di avahi }
\end{picture}
\begin{tikzpicture}[overlay]
\path(0pt,0pt);
\draw[color_29791,line width=0.7pt]
(95.8pt, -704.611pt) -- (508.3pt, -704.611pt)
;
\end{tikzpicture}
\begin{picture}(-5,0)(2.5,0)
\put(95.8,-721.811){\fontsize{12}{1}\usefont{T1}{cmr}{m}{n}\selectfont\color{color_29791}negli event loop dei sistemi grafici come GNOME e KDE}
\end{picture}
\begin{tikzpicture}[overlay]
\path(0pt,0pt);
\draw[color_29791,line width=0.7pt]
(95.8pt, -718.411pt) -- (374pt, -718.411pt)
;
\end{tikzpicture}
\begin{picture}(-5,0)(2.5,0)
\put(77.8,-742.611){\fontsize{12}{1}\usefont{T1}{cmr}{m}{n}\selectfont\color{color_29791}◦}
\end{picture}
\begin{tikzpicture}[overlay]
\path(0pt,0pt);
\draw[color_29791,line width=0.7pt]
(77.8pt, -739.211pt) -- (87.3pt, -739.211pt)
;
\end{tikzpicture}
\begin{picture}(-5,0)(2.5,0)
\put(87.3,-742.611){\fontsize{12}{1}\usefont{T1}{cmr}{m}{n}\selectfont\color{color_29791} }
\end{picture}
\begin{tikzpicture}[overlay]
\path(0pt,0pt);
\draw[color_29791,line width=0.7pt]
(87.3pt, -739.211pt) -- (93.3pt, -739.211pt)
;
\end{tikzpicture}
\begin{picture}(-5,0)(2.5,0)
\put(95.8,-742.611){\fontsize{12}{1}\usefont{T1}{cmr}{m}{n}\selectfont\color{color_29791}il demone è responsabile ad esempio della scoperta automatica di stampanti in una rete }
\end{picture}
\begin{tikzpicture}[overlay]
\path(0pt,0pt);
\draw[color_29791,line width=0.7pt]
(95.8pt, -739.211pt) -- (515.6pt, -739.211pt)
;
\end{tikzpicture}
\begin{picture}(-5,0)(2.5,0)
\put(95.8,-756.411){\fontsize{12}{1}\usefont{T1}{cmr}{m}{n}\selectfont\color{color_29791}locale}
\end{picture}
\begin{tikzpicture}[overlay]
\path(0pt,0pt);
\draw[color_29791,line width=0.7pt]
(95.8pt, -753.011pt) -- (124.4pt, -753.011pt)
;
\end{tikzpicture}
\newpage
\begin{tikzpicture}[overlay]\path(0pt,0pt);\end{tikzpicture}
\begin{picture}(-5,0)(2.5,0)
\put(77.8,-85.01099){\fontsize{12}{1}\usefont{T1}{cmr}{m}{n}\selectfont\color{color_29791}◦}
\end{picture}
\begin{tikzpicture}[overlay]
\path(0pt,0pt);
\draw[color_29791,line width=0.7pt]
(77.8pt, -81.61096pt) -- (87.3pt, -81.61096pt)
;
\end{tikzpicture}
\begin{picture}(-5,0)(2.5,0)
\put(87.3,-85.01099){\fontsize{12}{1}\usefont{T1}{cmr}{m}{n}\selectfont\color{color_29791} }
\end{picture}
\begin{tikzpicture}[overlay]
\path(0pt,0pt);
\draw[color_29791,line width=0.7pt]
(87.3pt, -81.61096pt) -- (93.3pt, -81.61096pt)
;
\end{tikzpicture}
\begin{picture}(-5,0)(2.5,0)
\put(95.8,-85.01099){\fontsize{12}{1}\usefont{T1}{cmr}{m}{n}\selectfont\color{color_29791}sono disponibili pacchetti con strumenti per svolgere funzioni singole specifiche}
\end{picture}
\begin{tikzpicture}[overlay]
\path(0pt,0pt);
\draw[color_29791,line width=0.7pt]
(95.8pt, -81.61096pt) -- (482pt, -81.61096pt)
;
\end{tikzpicture}
\begin{picture}(-5,0)(2.5,0)
\put(59.8,-105.811){\fontsize{12}{1}\usefont{T1}{cmr}{m}{n}\selectfont\color{color_29791}•}
\end{picture}
\begin{tikzpicture}[overlay]
\path(0pt,0pt);
\draw[color_29791,line width=0.7pt]
(59.8pt, -102.4109pt) -- (64.10001pt, -102.4109pt)
;
\end{tikzpicture}
\begin{picture}(-5,0)(2.5,0)
\put(64,-105.811){\fontsize{12}{1}\usefont{T1}{cmr}{m}{n}\selectfont\color{color_29791}  }
\end{picture}
\begin{tikzpicture}[overlay]
\path(0pt,0pt);
\draw[color_29791,line width=0.7pt]
(64pt, -102.4109pt) -- (77.7pt, -102.4109pt)
;
\end{tikzpicture}
\begin{picture}(-5,0)(2.5,0)
\put(77.8,-105.811){\fontsize{12}{1}\usefont{T1}{cmr}{m}{n}\selectfont\color{color_29791}link local}
\end{picture}
\begin{tikzpicture}[overlay]
\path(0pt,0pt);
\draw[color_29791,line width=0.7pt]
(77.8pt, -102.4109pt) -- (122.8pt, -102.4109pt)
;
\end{tikzpicture}
\begin{picture}(-5,0)(2.5,0)
\put(77.8,-126.611){\fontsize{12}{1}\usefont{T1}{cmr}{m}{n}\selectfont\color{color_29791}◦}
\end{picture}
\begin{tikzpicture}[overlay]
\path(0pt,0pt);
\draw[color_29791,line width=0.7pt]
(77.8pt, -123.211pt) -- (87.3pt, -123.211pt)
;
\end{tikzpicture}
\begin{picture}(-5,0)(2.5,0)
\put(87.3,-126.611){\fontsize{12}{1}\usefont{T1}{cmr}{m}{n}\selectfont\color{color_29791} }
\end{picture}
\begin{tikzpicture}[overlay]
\path(0pt,0pt);
\draw[color_29791,line width=0.7pt]
(87.3pt, -123.211pt) -- (93.3pt, -123.211pt)
;
\end{tikzpicture}
\begin{picture}(-5,0)(2.5,0)
\put(95.8,-126.611){\fontsize{12}{1}\usefont{T1}{cmr}{m}{n}\selectfont\color{color_29791}pacchetto }
\end{picture}
\begin{tikzpicture}[overlay]
\path(0pt,0pt);
\draw[color_29791,line width=0.7pt]
(95.8pt, -123.211pt) -- (144.7pt, -123.211pt)
;
\end{tikzpicture}
\begin{picture}(-5,0)(2.5,0)
\put(144.7,-126.611){\fontsize{12}{1}\usefont{T1}{cmr}{m}{n}\selectfont\color{color_29791}       }
\end{picture}
\begin{tikzpicture}[overlay]
\path(0pt,0pt);
\draw[color_29791,line width=0.7pt]
(144.7pt, -123.211pt) -- (166.7pt, -123.211pt)
;
\end{tikzpicture}
\begin{picture}(-5,0)(2.5,0)
\put(166.7,-126.611){\fontsize{12}{1}\usefont{T1}{cmr}{m}{it}\selectfont\color{color_29791}avahi-autoipd}
\end{picture}
\begin{tikzpicture}[overlay]
\path(0pt,0pt);
\draw[color_29791,line width=0.7pt]
(166.7pt, -123.211pt) -- (234pt, -123.211pt)
;
\end{tikzpicture}
\begin{picture}(-5,0)(2.5,0)
\put(234,-126.611){\fontsize{12}{1}\usefont{T1}{cmr}{m}{it}\selectfont\color{color_29791} }
\end{picture}
\begin{tikzpicture}[overlay]
\path(0pt,0pt);
\draw[color_29791,line width=0.7pt]
(234pt, -123.211pt) -- (237pt, -123.211pt)
;
\end{tikzpicture}
\begin{picture}(-5,0)(2.5,0)
\put(237.6,-126.611){\fontsize{12}{1}\usefont{T1}{cmr}{m}{n}\selectfont\color{color_29791}fornisce comando omonimo che implementa IPv4 Link }
\end{picture}
\begin{tikzpicture}[overlay]
\path(0pt,0pt);
\draw[color_29791,line width=0.7pt]
(237.6pt, -123.211pt) -- (505.8pt, -123.211pt)
;
\end{tikzpicture}
\begin{picture}(-5,0)(2.5,0)
\put(95.8,-140.411){\fontsize{12}{1}\usefont{T1}{cmr}{m}{n}\selectfont\color{color_29791}Local, mediante demone indipendente, oppure, nel file }
\end{picture}
\begin{tikzpicture}[overlay]
\path(0pt,0pt);
\draw[color_29791,line width=0.7pt]
(95.8pt, -137.011pt) -- (361pt, -137.011pt)
;
\end{tikzpicture}
\begin{picture}(-5,0)(2.5,0)
\put(361.1,-140.411){\fontsize{12}{1}\usefont{T1}{cmr}{m}{it}\selectfont\color{color_29791}/etc/network/interfaces }
\end{picture}
\begin{tikzpicture}[overlay]
\path(0pt,0pt);
\draw[color_29791,line width=0.7pt]
(361.1pt, -137.011pt) -- (474pt, -137.011pt)
;
\end{tikzpicture}
\begin{picture}(-5,0)(2.5,0)
\put(473.9,-140.411){\fontsize{12}{1}\usefont{T1}{cmr}{m}{it}\selectfont\color{color_29791}   }
\end{picture}
\begin{tikzpicture}[overlay]
\path(0pt,0pt);
\draw[color_29791,line width=0.7pt]
(473.9pt, -137.011pt) -- (485.7pt, -137.011pt)
;
\end{tikzpicture}
\begin{picture}(-5,0)(2.5,0)
\put(185.8,-161.211){\fontsize{12}{1}\usefont{T1}{cmr}{m}{it}\selectfont\color{color_29791}auto <ifname>}
\end{picture}
\begin{tikzpicture}[overlay]
\path(0pt,0pt);
\draw[color_29791,line width=0.7pt]
(185.8pt, -157.811pt) -- (259pt, -157.811pt)
;
\end{tikzpicture}
\begin{picture}(-5,0)(2.5,0)
\put(185.8,-182.011){\fontsize{12}{1}\usefont{T1}{cmr}{m}{it}\selectfont\color{color_29791}iface <ifname> inet ipv4all}
\end{picture}
\begin{tikzpicture}[overlay]
\path(0pt,0pt);
\draw[color_29791,line width=0.7pt]
(185.8pt, -178.611pt) -- (318.3pt, -178.611pt)
;
\end{tikzpicture}
\begin{picture}(-5,0)(2.5,0)
\put(77.8,-202.811){\fontsize{12}{1}\usefont{T1}{cmr}{m}{n}\selectfont\color{color_29791}◦}
\end{picture}
\begin{tikzpicture}[overlay]
\path(0pt,0pt);
\draw[color_29791,line width=0.7pt]
(77.8pt, -199.4109pt) -- (87.3pt, -199.4109pt)
;
\end{tikzpicture}
\begin{picture}(-5,0)(2.5,0)
\put(87.3,-202.811){\fontsize{12}{1}\usefont{T1}{cmr}{m}{n}\selectfont\color{color_29791} }
\end{picture}
\begin{tikzpicture}[overlay]
\path(0pt,0pt);
\draw[color_29791,line width=0.7pt]
(87.3pt, -199.4109pt) -- (93.3pt, -199.4109pt)
;
\end{tikzpicture}
\begin{picture}(-5,0)(2.5,0)
\put(95.8,-202.811){\fontsize{12}{1}\usefont{T1}{cmr}{m}{n}\selectfont\color{color_29791}ad ogni cambio stato dell'interfaccia invoca }
\end{picture}
\begin{tikzpicture}[overlay]
\path(0pt,0pt);
\draw[color_29791,line width=0.7pt]
(95.8pt, -199.4109pt) -- (307.2pt, -199.4109pt)
;
\end{tikzpicture}
\begin{picture}(-5,0)(2.5,0)
\put(308.4,-202.811){\fontsize{12}{1}\usefont{T1}{cmr}{m}{n}\selectfont\color{color_29791}           }
\end{picture}
\begin{tikzpicture}[overlay]
\path(0pt,0pt);
\draw[color_29791,line width=0.7pt]
(308.4pt, -199.4109pt) -- (343.9pt, -199.4109pt)
;
\end{tikzpicture}
\begin{picture}(-5,0)(2.5,0)
\put(344,-202.811){\fontsize{12}{1}\usefont{T1}{cmr}{m}{n}\selectfont\color{color_29791}/etc/avahi/avahi-autoipd.action}
\end{picture}
\begin{tikzpicture}[overlay]
\path(0pt,0pt);
\draw[color_29791,line width=0.7pt]
(344pt, -199.4109pt) -- (492.2pt, -199.4109pt)
;
\end{tikzpicture}
\begin{picture}(-5,0)(2.5,0)
\put(77.8,-223.611){\fontsize{12}{1}\usefont{T1}{cmr}{m}{n}\selectfont\color{color_29791}◦}
\end{picture}
\begin{tikzpicture}[overlay]
\path(0pt,0pt);
\draw[color_29791,line width=0.7pt]
(77.8pt, -220.211pt) -- (87.3pt, -220.211pt)
;
\end{tikzpicture}
\begin{picture}(-5,0)(2.5,0)
\put(87.3,-223.611){\fontsize{12}{1}\usefont{T1}{cmr}{m}{n}\selectfont\color{color_29791} }
\end{picture}
\begin{tikzpicture}[overlay]
\path(0pt,0pt);
\draw[color_29791,line width=0.7pt]
(87.3pt, -220.211pt) -- (93.3pt, -220.211pt)
;
\end{tikzpicture}
\begin{picture}(-5,0)(2.5,0)
\put(95.8,-223.611){\fontsize{12}{1}\usefont{T1}{cmr}{m}{n}\selectfont\color{color_29791}può essere usato come fallback se DCHP fallisce (come plugin per }
\end{picture}
\begin{tikzpicture}[overlay]
\path(0pt,0pt);
\draw[color_29791,line width=0.7pt]
(95.8pt, -220.211pt) -- (418.2pt, -220.211pt)
;
\end{tikzpicture}
\begin{picture}(-5,0)(2.5,0)
\put(418.3,-223.611){\fontsize{12}{1}\usefont{T1}{cmr}{b}{n}\selectfont\color{color_29791}dhclient}
\end{picture}
\begin{tikzpicture}[overlay]
\path(0pt,0pt);
\draw[color_29791,line width=0.7pt]
(418.3pt, -220.211pt) -- (459.6pt, -220.211pt)
;
\end{tikzpicture}
\begin{picture}(-5,0)(2.5,0)
\put(459.6,-223.611){\fontsize{12}{1}\usefont{T1}{cmr}{m}{n}\selectfont\color{color_29791}, usando }
\end{picture}
\begin{tikzpicture}[overlay]
\path(0pt,0pt);
\draw[color_29791,line width=0.7pt]
(459.6pt, -220.211pt) -- (502.6pt, -220.211pt)
;
\end{tikzpicture}
\begin{picture}(-5,0)(2.5,0)
\put(95.8,-237.411){\fontsize{12}{1}\usefont{T1}{cmr}{m}{n}\selectfont\color{color_29791}hook nelle cartelle sopra indicate)}
\end{picture}
\begin{tikzpicture}[overlay]
\path(0pt,0pt);
\draw[color_29791,line width=0.7pt]
(95.8pt, -234.011pt) -- (258.4pt, -234.011pt)
;
\end{tikzpicture}
\begin{picture}(-5,0)(2.5,0)
\put(41.8,-267.211){\fontsize{14.1}{1}\usefont{T1}{cmr}{b}{n}\selectfont\color{color_217499}Sicurezza di rete}
\put(41.8,-287.411){\fontsize{12}{1}\usefont{T1}{cmr}{m}{n}\selectfont\color{color_217499}Sono possibili vari tipi di attacchi che sfruttano le debolezze dei protocolli di rete (sia a livello }
\put(41.8,-301.211){\fontsize{12}{1}\usefont{T1}{cmr}{m}{n}\selectfont\color{color_217499}applicativo, che a livello di rete), ad esempio dirottamenti del traffico attraverso sistemi }
\put(41.8,-315.011){\fontsize{12}{1}\usefont{T1}{cmr}{m}{n}\selectfont\color{color_217499}compromessi.}
\put(59.8,-335.811){\fontsize{12}{1}\usefont{T1}{cmr}{m}{n}\selectfont\color{color_217499}•DNS spoofing: Quando un utente effettua una query DNS, l’attaccante la cattura e manda }
\put(77.8,-349.611){\fontsize{12}{1}\usefont{T1}{cmr}{m}{n}\selectfont\color{color_217499}alla vittima una risposta fasulla, differente da quella che avrebbe fornito il DNS. Questo }
\put(77.8,-363.411){\fontsize{12}{1}\usefont{T1}{cmr}{m}{n}\selectfont\color{color_217499}attacco può anche essere portato a termine tramite pharming, quanto la vittima visita un sito }
\put(77.8,-377.211){\fontsize{12}{1}\usefont{T1}{cmr}{m}{n}\selectfont\color{color_217499}compromesso uno script esegue una riconfigurazione del DNS locale del router redirigendo }
\put(77.8,-391.011){\fontsize{12}{1}\usefont{T1}{cmr}{m}{n}\selectfont\color{color_217499}tutte le successive query DNS verso un name server scelto dall’attaccante. Questo attacco è }
\put(77.8,-404.811){\fontsize{12}{1}\usefont{T1}{cmr}{m}{n}\selectfont\color{color_217499}molto semplice ma richiede l’accesso ad un name server e la possibilità di modificare }
\put(77.8,-418.611){\fontsize{12}{1}\usefont{T1}{cmr}{m}{n}\selectfont\color{color_217499}direttamente alcuni record. L’impatto di questa categoria di attacchi può essere mitigato }
\put(77.8,-432.411){\fontsize{12}{1}\usefont{T1}{cmr}{m}{n}\selectfont\color{color_217499}tramite l’uso di HTTPS.}
\put(59.8,-453.211){\fontsize{12}{1}\usefont{T1}{cmr}{m}{n}\selectfont\color{color_217499}•HTTPS (Sicurezza a livello applicativo)}
\put(77.8,-474.011){\fontsize{12}{1}\usefont{T1}{cmr}{m}{n}\selectfont\color{color_217499}◦Https è un protocollo per la comunicazione sicura attraverso reti internet, consiste }
\put(95.8,-487.811){\fontsize{12}{1}\usefont{T1}{cmr}{m}{n}\selectfont\color{color_217499}nell’utilizzo del protocollo HTTP all’interno di una connessione cifrata da TLS (o dal }
\put(95.8,-501.611){\fontsize{12}{1}\usefont{T1}{cmr}{m}{n}\selectfont\color{color_217499}predecessore SSL) fornendo autenticazione dei siti web visitati, protezione della privacy }
\put(95.8,-515.411){\fontsize{12}{1}\usefont{T1}{cmr}{m}{n}\selectfont\color{color_217499}durante la comunicazione ed integrità dei dati. Il protocollo garantisce una protezione }
\put(95.8,-529.211){\fontsize{12}{1}\usefont{T1}{cmr}{m}{n}\selectfont\color{color_217499}accettabile da eavesdropper e da attacchi della tipologia man in the middle. Grazie a TLS}
\put(95.8,-543.011){\fontsize{12}{1}\usefont{T1}{cmr}{m}{n}\selectfont\color{color_217499}vengono cifrati tutti i dati contenuti nei messaggi HTTP, quali URL, parametri della }
\put(95.8,-556.811){\fontsize{12}{1}\usefont{T1}{cmr}{m}{n}\selectfont\color{color_217499}query, cookies, header della connessione. Quando un browser si connette ad un server, la}
\put(95.8,-570.611){\fontsize{12}{1}\usefont{T1}{cmr}{m}{n}\selectfont\color{color_217499}verifica della prova fornita dal server (un attaccante può provare a spedire certificato }
\put(95.8,-584.411){\fontsize{12}{1}\usefont{T1}{cmr}{m}{n}\selectfont\color{color_217499}diverso, con sua chiave pubblica, che renderebbe inutile la prova crittografica) avviene }
\put(95.8,-598.211){\fontsize{12}{1}\usefont{T1}{cmr}{m}{n}\selectfont\color{color_217499}mediante i certificati conservati nel Certificate Store.}
\put(77.8,-619.011){\fontsize{12}{1}\usefont{T1}{cmr}{m}{n}\selectfont\color{color_217499}◦Certification Authority / Certificati digitali Una certification authority è un soggetto }
\put(95.8,-632.811){\fontsize{12}{1}\usefont{T1}{cmr}{m}{n}\selectfont\color{color_217499}terzo fidato abilitato ad emettere un certificato digitale, ovvero un documento elettronico}
\put(95.8,-646.611){\fontsize{12}{1}\usefont{T1}{cmr}{m}{n}\selectfont\color{color_217499}che attesta l’associazione univoca tra una chiave pubblica e l’identità di un soggetto. }
\put(95.8,-660.411){\fontsize{12}{1}\usefont{T1}{cmr}{m}{n}\selectfont\color{color_217499}L’infrastruttura PKI è costituita da varie CA organizzate gerarchicamente al cui vertice si}
\put(95.8,-674.211){\fontsize{12}{1}\usefont{T1}{cmr}{m}{n}\selectfont\color{color_217499}trova una CA-root, il cui certificato solitamente è auto-firmato, che certifica le sub-CA. }
\put(95.8,-688.011){\fontsize{12}{1}\usefont{T1}{cmr}{m}{n}\selectfont\color{color_217499}Una CA ha, tra i suoi compiti, il rilascio dei certificati, previa identificazione e verifica }
\put(95.8,-701.811){\fontsize{12}{1}\usefont{T1}{cmr}{m}{n}\selectfont\color{color_217499}del richiedente, la manutenzione del registro delle chiavi, la revoca/sospensione dei }
\put(95.8,-715.611){\fontsize{12}{1}\usefont{T1}{cmr}{m}{n}\selectfont\color{color_217499}certificati in caso di abusi/falsificazioni, la pubblicazione di liste sempre aggiornate di }
\put(95.8,-729.411){\fontsize{12}{1}\usefont{T1}{cmr}{m}{n}\selectfont\color{color_217499}certificati. All’interno di un certificato digitale sono contenute varie informazioni tra cui }
\put(95.8,-743.211){\fontsize{12}{1}\usefont{T1}{cmr}{m}{n}\selectfont\color{color_217499}la chiave pubblica del proprietario del certificato, l’identità del proprietario. Esse sono }
\put(95.8,-757.011){\fontsize{12}{1}\usefont{T1}{cmr}{m}{n}\selectfont\color{color_217499}digitalmente firmate da parte della CA per garantirne l’autenticità e l’integrità. Il formato}
\end{picture}
\newpage
\begin{tikzpicture}[overlay]\path(0pt,0pt);\end{tikzpicture}
\begin{picture}(-5,0)(2.5,0)
\put(95.8,-85.01099){\fontsize{12}{1}\usefont{T1}{cmr}{m}{n}\selectfont\color{color_217499}più comune per i certificati è definito dallo standard X.509, che però, essendo molto }
\put(95.8,-98.81097){\fontsize{12}{1}\usefont{T1}{cmr}{m}{n}\selectfont\color{color_217499}generale, ha la necessità di essere ulteriormente specificato per i vari casi d’uso.}
\put(77.8,-119.611){\fontsize{12}{1}\usefont{T1}{cmr}{m}{n}\selectfont\color{color_217499}◦SSL/TLS: Sono protocolli crittografici progettati per fornire sicurezza nelle }
\put(95.8,-133.411){\fontsize{12}{1}\usefont{T1}{cmr}{m}{n}\selectfont\color{color_217499}comunicazioni attraverso reti internet, in particolare per fornire caratteristiche di }
\put(95.8,-147.211){\fontsize{12}{1}\usefont{T1}{cmr}{m}{n}\selectfont\color{color_217499}confidenzialità ed integrità dei dati. La connessione è privata grazie all’uso della }
\put(95.8,-161.011){\fontsize{12}{1}\usefont{T1}{cmr}{m}{n}\selectfont\color{color_217499}crittografia simmetrica, la cui chiave viene scambiata durante l’handshake iniziale (a sua}
\put(95.8,-174.811){\fontsize{12}{1}\usefont{T1}{cmr}{m}{n}\selectfont\color{color_217499}volta protetto dalla crittografia asimmetrica). Inoltre grazie all’utilizzo della crittografia }
\put(95.8,-188.611){\fontsize{12}{1}\usefont{T1}{cmr}{m}{n}\selectfont\color{color_217499}asimmetrica durante l’handshake iniziale viene garantita l’autenticazione delle parti }
\put(95.8,-202.411){\fontsize{12}{1}\usefont{T1}{cmr}{m}{n}\selectfont\color{color_217499}comunicanti. L’handshake è composto da più fasi:}
\put(113.8,-223.211){\fontsize{12}{1}\usefont{T1}{cmr}{m}{n}\selectfont\color{color_217499}• Negoziazione: vengono concordate le caratteristiche della comunicazione, come }
\put(113.8,-237.011){\fontsize{12}{1}\usefont{T1}{cmr}{m}{n}\selectfont\color{color_217499}protocollo di comunicazione, protocollo crittografico}
\put(113.8,-257.811){\fontsize{12}{1}\usefont{T1}{cmr}{m}{n}\selectfont\color{color_217499}• Scambio dei certificati}
\put(113.8,-278.611){\fontsize{12}{1}\usefont{T1}{cmr}{m}{n}\selectfont\color{color_217499}• Inizializzazione della connessione cifrata}
\put(95.8,-299.411){\fontsize{12}{1}\usefont{T1}{cmr}{m}{n}\selectfont\color{color_217499}A causa di alcune vulnerabilità, sia a livello di protocollo sia a livello di }
\put(95.8,-313.211){\fontsize{12}{1}\usefont{T1}{cmr}{m}{n}\selectfont\color{color_217499}implementazione, SSL è stato sostituito con TLS.}
\put(59.8,-334.011){\fontsize{12}{1}\usefont{T1}{cmr}{m}{n}\selectfont\color{color_217499}•Sicurezza a livello di rete: Il protocollo IP non può garantire nessuna proprietà di sicurezza }
\put(77.8,-347.811){\fontsize{12}{1}\usefont{T1}{cmr}{m}{n}\selectfont\color{color_217499}per nessuna parte del pacchetto.}
\put(77.8,-368.611){\fontsize{12}{1}\usefont{T1}{cmr}{m}{n}\selectfont\color{color_217499}◦IP hijacking: Si opera in modo tale da informare internet che la rotta per una data }
\put(95.8,-382.411){\fontsize{12}{1}\usefont{T1}{cmr}{m}{n}\selectfont\color{color_217499}subnet passa attraverso la propria rete. Questo è possibile perché chiunque teoricamente }
\put(95.8,-396.211){\fontsize{12}{1}\usefont{T1}{cmr}{m}{n}\selectfont\color{color_217499}può diffondere informazioni circa la viabilità delle rotte in internet e permette di }
\put(95.8,-410.011){\fontsize{12}{1}\usefont{T1}{cmr}{m}{n}\selectfont\color{color_217499}realizzare vari tipi di attacco più o meno dannosi (DoS, man in the middle, spamma e }
\put(95.8,-423.811){\fontsize{12}{1}\usefont{T1}{cmr}{m}{n}\selectfont\color{color_217499}fuggi). }
\put(77.8,-444.611){\fontsize{12}{1}\usefont{T1}{cmr}{m}{n}\selectfont\color{color_217499}◦IPSec: IPSec non è un protocollo singolo ma un insieme di algoritmi per la sicurezza ed }
\put(95.8,-458.411){\fontsize{12}{1}\usefont{T1}{cmr}{m}{n}\selectfont\color{color_217499}un framework per la negoziazione di algoritmi che permettono di ottenere autenticazione}
\put(95.8,-472.211){\fontsize{12}{1}\usefont{T1}{cmr}{m}{n}\selectfont\color{color_217499}e crittografia direttamente a livello di rete. IPSec ha varie applicazioni tra cui }
\put(95.8,-486.011){\fontsize{12}{1}\usefont{T1}{cmr}{m}{n}\selectfont\color{color_217499}l’interconnessione sicura di reti remote attraverso internet e l’accesso sicuro di client ad }
\put(95.8,-499.811){\fontsize{12}{1}\usefont{T1}{cmr}{m}{n}\selectfont\color{color_217499}una rete privata. Rispetto ad altre soluzioni IPSec presenta il vantaggio di essere }
\put(95.8,-513.611){\fontsize{12}{1}\usefont{T1}{cmr}{m}{n}\selectfont\color{color_217499}trasparente alle applicazioni, tuttavia lo stack di IPSec è differente da quello standard di }
\put(95.8,-527.411){\fontsize{12}{1}\usefont{T1}{cmr}{m}{n}\selectfont\color{color_217499}TCP/IP, pertanto la macchina va configurata per poter usare questo stack specifico }
\put(95.8,-541.211){\fontsize{12}{1}\usefont{T1}{cmr}{m}{n}\selectfont\color{color_217499}invece di quello standard. IPSec è basato su tre componenti:}
\put(113.8,-562.011){\fontsize{12}{1}\usefont{T1}{cmr}{m}{n}\selectfont\color{color_217499}• AH (Authentication Header): fornisce un servizio di autenticazione dei pacchetti}
\put(113.8,-582.811){\fontsize{12}{1}\usefont{T1}{cmr}{m}{n}\selectfont\color{color_217499}• ESP (Encapsulating Security Protocol): fornisce un servizio di autenticazione e }
\put(113.8,-596.611){\fontsize{12}{1}\usefont{T1}{cmr}{m}{n}\selectfont\color{color_217499}cifratura dei pacchetti}
\put(113.8,-617.411){\fontsize{12}{1}\usefont{T1}{cmr}{m}{n}\selectfont\color{color_217499}• IKE (Internet Key Exchange): fornisce un servizio di negoziazione dei parametri }
\put(113.8,-631.211){\fontsize{12}{1}\usefont{T1}{cmr}{m}{n}\selectfont\color{color_217499}necessari al funzionamento dei precedenti componenti}
\put(95.8,-652.011){\fontsize{12}{1}\usefont{T1}{cmr}{m}{n}\selectfont\color{color_217499}Nel funzionamento di IPSec giocano un ruolo chiave le Security Association (SA) che }
\put(95.8,-665.811){\fontsize{12}{1}\usefont{T1}{cmr}{m}{n}\selectfont\color{color_217499}identificano un canale di comunicazione unidirezionale diretto dal nodo locale verso un }
\put(95.8,-679.611){\fontsize{12}{1}\usefont{T1}{cmr}{m}{n}\selectfont\color{color_217499}altro nodo e definito destinatario, protocollo utilizzato (AH o ESP), modalità di }
\put(95.8,-693.411){\fontsize{12}{1}\usefont{T1}{cmr}{m}{n}\selectfont\color{color_217499}funzionamento (tunnel o trasporto) ed un Security Parameter Index (SPI, usato per }
\put(95.8,-707.211){\fontsize{12}{1}\usefont{T1}{cmr}{m}{n}\selectfont\color{color_217499}distinguere tutti i canali aventi gli stessi estremi e lo stesso protocollo). Essendo una SA }
\put(95.8,-721.011){\fontsize{12}{1}\usefont{T1}{cmr}{m}{n}\selectfont\color{color_217499}unidirezionale ne servono sempre due per definire un canale di comunicazione }
\put(95.8,-734.811){\fontsize{12}{1}\usefont{T1}{cmr}{m}{n}\selectfont\color{color_217499}bidirezionale. La modalità di funzionamento transport prevede la comunicazione diretta }
\put(95.8,-748.611){\fontsize{12}{1}\usefont{T1}{cmr}{m}{n}\selectfont\color{color_217499}tra due stazioni, ciascuna delle quali tratta i pacchetti e li spedisce tramite IPSec; la }
\put(95.8,-762.411){\fontsize{12}{1}\usefont{T1}{cmr}{m}{n}\selectfont\color{color_217499}modalità tunnel invece prevede la presenza di almeno un router (il cosiddetto Security }
\end{picture}
\newpage
\begin{tikzpicture}[overlay]\path(0pt,0pt);\end{tikzpicture}
\begin{picture}(-5,0)(2.5,0)
\put(95.8,-85.01099){\fontsize{12}{1}\usefont{T1}{cmr}{m}{n}\selectfont\color{color_217499}Gateway) che faccia da tramite per la comunicazione con le macchine poste nella rete }
\put(95.8,-98.81097){\fontsize{12}{1}\usefont{T1}{cmr}{m}{n}\selectfont\color{color_217499}per la quale agisce da gateway, questa modalità è quella tipica delle VPN.}
\put(95.8,-119.611){\fontsize{12}{1}\usefont{T1}{cmr}{m}{n}\selectfont\color{color_217499}Il protocollo AH serve solamente per autenticare i pacchetti (eccezion fatta per i campi }
\put(95.8,-133.411){\fontsize{12}{1}\usefont{T1}{cmr}{m}{n}\selectfont\color{color_217499}variabili come TTL). In caso venga usato in modalità trasporto ci si limita a frapporre tra}
\put(95.8,-147.211){\fontsize{12}{1}\usefont{T1}{cmr}{m}{n}\selectfont\color{color_217499}l’header IP ed il payload l’intestazione di AH, che contiene tutti i dati relativi }
\put(95.8,-161.011){\fontsize{12}{1}\usefont{T1}{cmr}{m}{n}\selectfont\color{color_217499}all’autenticazione del pacchetto; il pacchetto risultante mantiene l’header originale, }
\put(95.8,-174.811){\fontsize{12}{1}\usefont{T1}{cmr}{m}{n}\selectfont\color{color_217499}anch’esso autenticato nei campi non variabili. In caso venga usato in modalità tunnel }
\put(95.8,-188.611){\fontsize{12}{1}\usefont{T1}{cmr}{m}{n}\selectfont\color{color_217499}invece si crea una nuova intestazione IP, che conterrà solamente gli estremi del tunnel, e }
\put(95.8,-202.411){\fontsize{12}{1}\usefont{T1}{cmr}{m}{n}\selectfont\color{color_217499}si autentica integralmente il pacchetto originale (i campi variabili della nuova }
\put(95.8,-216.211){\fontsize{12}{1}\usefont{T1}{cmr}{m}{n}\selectfont\color{color_217499}intestazione non vengono autenticati), la nuova intestazione viene posta in testa al }
\put(95.8,-230.011){\fontsize{12}{1}\usefont{T1}{cmr}{m}{n}\selectfont\color{color_217499}pacchetto, seguita dell’intestazione di AH e infine dal pacchetto originale replicato }
\put(95.8,-243.811){\fontsize{12}{1}\usefont{T1}{cmr}{m}{n}\selectfont\color{color_217499}integralmente; quando il pacchetto verrà ricevuto dal security gateway esso provvederà a}
\put(95.8,-257.611){\fontsize{12}{1}\usefont{T1}{cmr}{m}{n}\selectfont\color{color_217499}controllare l’autenticità del pacchetto ricevuto, a rimuovere l’intestazione usata per la }
\put(95.8,-271.411){\fontsize{12}{1}\usefont{T1}{cmr}{m}{n}\selectfont\color{color_217499}trasmissione ed infine ad inoltrare il pacchetto al destinatario. }
\put(95.8,-292.211){\fontsize{12}{1}\usefont{T1}{cmr}{m}{n}\selectfont\color{color_217499}Qualora si voglia cifrare il contenuto del pacchetto invece occorre ricorrere all’uso del }
\put(95.8,-306.011){\fontsize{12}{1}\usefont{T1}{cmr}{m}{n}\selectfont\color{color_217499}protocollo ESP. In caso si usi la modalità trasporto ci si limita a cifrare il payload del }
\put(95.8,-319.811){\fontsize{12}{1}\usefont{T1}{cmr}{m}{n}\selectfont\color{color_217499}pacchetto IP interponendo tra l’header IP ed il payload l’intestazione di ESP; }
\put(95.8,-333.611){\fontsize{12}{1}\usefont{T1}{cmr}{m}{n}\selectfont\color{color_217499}contestualmente viene fornita autenticazione sull’intero contenuto del pacchetto (ad }
\put(95.8,-347.411){\fontsize{12}{1}\usefont{T1}{cmr}{m}{n}\selectfont\color{color_217499}esclusione dell’header IP). In caso venga usata la modalità tunnel invece, si genera una }
\put(95.8,-361.211){\fontsize{12}{1}\usefont{T1}{cmr}{m}{n}\selectfont\color{color_217499}nuova intestazione IP ed il pacchetto originale viene cifrato integralmente, in questo }
\put(95.8,-375.011){\fontsize{12}{1}\usefont{T1}{cmr}{m}{n}\selectfont\color{color_217499}modo non si potranno vedere gli indirizzi originali (che possono così essere anche }
\put(95.8,-388.811){\fontsize{12}{1}\usefont{T1}{cmr}{m}{n}\selectfont\color{color_217499}indirizzi di reti private), contestualmente viene fornita autenticazione per quanto }
\put(95.8,-402.611){\fontsize{12}{1}\usefont{T1}{cmr}{m}{n}\selectfont\color{color_217499}riguarda l’intero pacchetto originale e l’intestazione ESP. In ogni caso in coda al }
\put(95.8,-416.411){\fontsize{12}{1}\usefont{T1}{cmr}{m}{n}\selectfont\color{color_217499}pacchetto vengono aggiunti due footer. Sebbene IPSec sia uno standard usato da molto }
\put(95.8,-430.211){\fontsize{12}{1}\usefont{T1}{cmr}{m}{n}\selectfont\color{color_217499}tempo per la creazione di VPN la sua diffusione è molto limitata a causa dei problemi }
\put(95.8,-444.011){\fontsize{12}{1}\usefont{T1}{cmr}{m}{n}\selectfont\color{color_217499}derivanti dall’interazione con il NAT. AH semplicemente non può funzionare in presenza}
\put(95.8,-457.811){\fontsize{12}{1}\usefont{T1}{cmr}{m}{n}\selectfont\color{color_217499}di NAT in quanto esso autentica l’intero pacchetto, compresi gli indirizzi di sorgente e }
\put(95.8,-471.611){\fontsize{12}{1}\usefont{T1}{cmr}{m}{n}\selectfont\color{color_217499}destinazione, che vengono modificati all’attraversamento di un NAT portando ad un }
\put(95.8,-485.411){\fontsize{12}{1}\usefont{T1}{cmr}{m}{n}\selectfont\color{color_217499}fallimento del controllo sull’integrità del pacchetto. Con ESP il problema è più sottile e }
\put(95.8,-499.211){\fontsize{12}{1}\usefont{T1}{cmr}{m}{n}\selectfont\color{color_217499}riguarda il checksum dei pacchetti TCP ed UDP, in questo caso non vi sono problemi di }
\put(95.8,-513.011){\fontsize{12}{1}\usefont{T1}{cmr}{m}{n}\selectfont\color{color_217499}autenticazione in quanto l’header IP esterno non viene mai autenticato, tuttavia il }
\put(95.8,-526.811){\fontsize{12}{1}\usefont{T1}{cmr}{m}{n}\selectfont\color{color_217499}checksum del pacchetto è calcolato sugli indirizzi originali e quindi se il pacchetto è }
\put(95.8,-540.611){\fontsize{12}{1}\usefont{T1}{cmr}{m}{n}\selectfont\color{color_217499}cifrato con IPSec questo valore non può essere aggiornato facendo fallire i controlli sugli}
\put(95.8,-554.411){\fontsize{12}{1}\usefont{T1}{cmr}{m}{n}\selectfont\color{color_217499}errori a destinazione. Per risolvere questi problemi è stata proposta un’alternativa al }
\put(95.8,-568.211){\fontsize{12}{1}\usefont{T1}{cmr}{m}{n}\selectfont\color{color_217499}NAT classico chiamata NAT-T che prevede che ogni volta che un pacchetto raggiunge }
\put(95.8,-582.011){\fontsize{12}{1}\usefont{T1}{cmr}{m}{n}\selectfont\color{color_217499}un NAT esso venga incapsulato all’interno di un pacchetto UDP per poi venir inviato }
\put(95.8,-595.811){\fontsize{12}{1}\usefont{T1}{cmr}{m}{n}\selectfont\color{color_217499}tramite lo stack TCP/IP standard. Questa soluzione tuttavia pone dei problemi per quanto}
\put(95.8,-609.611){\fontsize{12}{1}\usefont{T1}{cmr}{m}{n}\selectfont\color{color_217499}riguarda l’efficienza e rende l’utilizzo di IPSec equivalente a livello pratico a quello di }
\put(95.8,-623.411){\fontsize{12}{1}\usefont{T1}{cmr}{m}{n}\selectfont\color{color_217499}altre soluzioni come OpenVPN.}
\put(59.8,-644.211){\fontsize{12}{1}\usefont{T1}{cmr}{m}{n}\selectfont\color{color_217499}•VPN: Una Virtual Private Network è una rete di telecomunicazioni privata, instaurata tra }
\put(77.8,-658.011){\fontsize{12}{1}\usefont{T1}{cmr}{m}{n}\selectfont\color{color_217499}soggetti che utilizzano, come tecnologia di trasporto, un protocollo di trasmissione pubblico }
\put(77.8,-671.811){\fontsize{12}{1}\usefont{T1}{cmr}{m}{n}\selectfont\color{color_217499}e condiviso, come ad esempio la rete Internet. Lo scopo delle reti VPN è quello di offrire }
\put(77.8,-685.611){\fontsize{12}{1}\usefont{T1}{cmr}{m}{n}\selectfont\color{color_217499}alle aziende, a un costo minore, le stesse possibilità delle linee private a noleggio, ma }
\put(77.8,-699.411){\fontsize{12}{1}\usefont{T1}{cmr}{m}{n}\selectfont\color{color_217499}sfruttando reti condivise pubbliche: si può vedere dunque una VPN come l’estensione a }
\put(77.8,-713.211){\fontsize{12}{1}\usefont{T1}{cmr}{m}{n}\selectfont\color{color_217499}livello geografico di una rete locale privata aziendale sicura che colleghi tra loro siti interni }
\put(77.8,-727.011){\fontsize{12}{1}\usefont{T1}{cmr}{m}{n}\selectfont\color{color_217499}all’azienda stessa variamente dislocati su un ampio territorio, sfruttando l’instradamento }
\put(77.8,-740.811){\fontsize{12}{1}\usefont{T1}{cmr}{m}{n}\selectfont\color{color_217499}tramite IP per il trasporto su scala geografica e realizzando di fatto una rete LAN, detta }
\put(77.8,-754.611){\fontsize{12}{1}\usefont{T1}{cmr}{m}{n}\selectfont\color{color_217499}appunto "virtuale" e "privata", equivalente a un’infrastruttura fisica di rete (ossia con }
\put(77.8,-768.411){\fontsize{12}{1}\usefont{T1}{cmr}{m}{n}\selectfont\color{color_217499}collegamenti fisici) dedicata. }
\end{picture}
\newpage
\begin{tikzpicture}[overlay]\path(0pt,0pt);\end{tikzpicture}
\begin{picture}(-5,0)(2.5,0)
\put(77.8,-85.01099){\fontsize{12}{1}\usefont{T1}{cmr}{m}{n}\selectfont\color{color_217499}Tuttavia l’uso di una VPN non è trasparente all’utente in quanto è necessario istruire il }
\put(77.8,-98.81097){\fontsize{12}{1}\usefont{T1}{cmr}{m}{n}\selectfont\color{color_217499}sistema affinché invii i dati attraverso un’interfaccia di rete virtuale offerta dal programma }
\put(77.8,-112.611){\fontsize{12}{1}\usefont{T1}{cmr}{m}{n}\selectfont\color{color_217499}che gestisce la VPN stessa. Per l’autenticazione degli host si può ricorrere ad una modalità }
\put(77.8,-126.411){\fontsize{12}{1}\usefont{T1}{cmr}{m}{n}\selectfont\color{color_217499}"statica", con scambio manuale delle chiavi crittografiche, oppure ai protocolli SSL/TLS, }
\put(77.8,-140.211){\fontsize{12}{1}\usefont{T1}{cmr}{m}{n}\selectfont\color{color_217499}affidandosi all’architettura PKI per lo scambio delle chiavi.}
\put(59.8,-161.011){\fontsize{12}{1}\usefont{T1}{cmr}{m}{n}\selectfont\color{color_217499}•Firewall: Un firewall può essere paragonato ad una porta blindata che divide l’interno (da }
\put(77.8,-174.811){\fontsize{12}{1}\usefont{T1}{cmr}{m}{n}\selectfont\color{color_217499}proteggere) dall’esterno (da cui provengono delle minacce), in questo senso un firewall }
\put(77.8,-188.611){\fontsize{12}{1}\usefont{T1}{cmr}{m}{n}\selectfont\color{color_217499}permette di bloccare gli accessi indebiti alla rete. Così come una porta blindata un firewall }
\put(77.8,-202.411){\fontsize{12}{1}\usefont{T1}{cmr}{m}{n}\selectfont\color{color_217499}può bloccare certe minacce, ma diventa inutile nel momento in cui vengono lasciate altre }
\put(77.8,-216.211){\fontsize{12}{1}\usefont{T1}{cmr}{m}{n}\selectfont\color{color_217499}falle aperte, pertanto è buona norma porlo al di sopra di un sistema sicuro in modo da ridurre}
\put(77.8,-230.011){\fontsize{12}{1}\usefont{T1}{cmr}{m}{n}\selectfont\color{color_217499}le vulnerabilità del firewall stesso. Le tecniche di controllo prevedono l’esame di:}
\put(77.8,-250.811){\fontsize{12}{1}\usefont{T1}{cmr}{m}{n}\selectfont\color{color_217499}• Traffico: esaminare di indirizzi, porte ed altri indicatori relativi al tipo di indirizzo che si }
\put(77.8,-264.611){\fontsize{12}{1}\usefont{T1}{cmr}{m}{n}\selectfont\color{color_217499}vuole esplicitamente autorizzare.}
\put(77.8,-285.411){\fontsize{12}{1}\usefont{T1}{cmr}{m}{n}\selectfont\color{color_217499}• Direzione: discriminare il traffico in base alla direzione a parità di indirizzi e porte}
\put(77.8,-306.211){\fontsize{12}{1}\usefont{T1}{cmr}{m}{n}\selectfont\color{color_217499}• Utenti: differenziare il traffico in base a chi lo genera}
\put(77.8,-327.011){\fontsize{12}{1}\usefont{T1}{cmr}{m}{n}\selectfont\color{color_217499}• Comportamento: valutare come sono usati i servizi ammessi per identificare le anomalie}
\put(77.8,-347.811){\fontsize{12}{1}\usefont{T1}{cmr}{m}{n}\selectfont\color{color_217499}◦Tipologie: Si identificano tre tipi fondamentali di firewall:}
\put(113.8,-368.611){\fontsize{12}{1}\usefont{T1}{cmr}{b}{n}\selectfont\color{color_217499}1. Packet filter: sono in grado di esaminare unicamente l’header dei pacchetti }
\put(113.8,-382.411){\fontsize{12}{1}\usefont{T1}{cmr}{m}{n}\selectfont\color{color_217499}applicando in serie un insieme di regole, generalmente raggruppate in degli elenchi }
\put(113.8,-396.211){\fontsize{12}{1}\usefont{T1}{cmr}{m}{n}\selectfont\color{color_217499}corrispondenti a punti di controllo differenti (pacchetti in ingresso, in uscita, ...), del }
\put(113.8,-410.011){\fontsize{12}{1}\usefont{T1}{cmr}{m}{n}\selectfont\color{color_217499}tipo "se condizione allora azione". Normalmente la prima condizione ad essere }
\put(113.8,-423.811){\fontsize{12}{1}\usefont{T1}{cmr}{m}{n}\selectfont\color{color_217499}verificata decide il destino del pacchetto ed interrompe la scansione dell’elenco. Di }
\put(113.8,-437.611){\fontsize{12}{1}\usefont{T1}{cmr}{m}{n}\selectfont\color{color_217499}base le azioni sono accettare il pacchetto o scartarlo, ma possono anche essere }
\put(113.8,-451.411){\fontsize{12}{1}\usefont{T1}{cmr}{m}{n}\selectfont\color{color_217499}intraprese azioni più complesse come loggare i dettagli del pacchetto oppure }
\put(113.8,-465.211){\fontsize{12}{1}\usefont{T1}{cmr}{m}{n}\selectfont\color{color_217499}modificarlo in qualche modo. Se nessuna regola è applicabile viene attivata una }
\put(113.8,-479.011){\fontsize{12}{1}\usefont{T1}{cmr}{m}{n}\selectfont\color{color_217499}policy di default. Questo tipo di firewall porta con sé alcuni vantaggi, ad esempio è }
\put(113.8,-492.811){\fontsize{12}{1}\usefont{T1}{cmr}{m}{n}\selectfont\color{color_217499}molto veloce e semplice (tanto che è implementato in moltissimi router) ed è }
\put(113.8,-506.611){\fontsize{12}{1}\usefont{T1}{cmr}{m}{n}\selectfont\color{color_217499}trasparente all’utente (non occorre modificare il sistema in nessun modo, il firewall }
\put(113.8,-520.411){\fontsize{12}{1}\usefont{T1}{cmr}{m}{n}\selectfont\color{color_217499}si occupa autonomamente di scansionare il traffico di rete); tuttavia le regole sono }
\put(113.8,-534.211){\fontsize{12}{1}\usefont{T1}{cmr}{m}{n}\selectfont\color{color_217499}spesso di basso livello, facendo sì che per ottenere comportamenti complessi occorra}
\put(113.8,-548.011){\fontsize{12}{1}\usefont{T1}{cmr}{m}{n}\selectfont\color{color_217499}costruire set di regole molti articolati, e mancano del supporto alla gestione degli }
\put(113.8,-561.811){\fontsize{12}{1}\usefont{T1}{cmr}{m}{n}\selectfont\color{color_217499}utenti. Inoltre i PF presentano alcune vulnerabilità e limitazioni:}
\put(131.8,-582.611){\fontsize{12}{1}\usefont{T1}{cmr}{m}{n}\selectfont\color{color_217499}• i pacchetti frammentati possono comportarsi in maniera non prevedibile (si }
\put(131.8,-596.411){\fontsize{12}{1}\usefont{T1}{cmr}{m}{n}\selectfont\color{color_217499}potrebbe riassemblare ogni pacchetto all’interno del firewall, ma questo fa sì che }
\put(131.8,-610.211){\fontsize{12}{1}\usefont{T1}{cmr}{m}{n}\selectfont\color{color_217499}il carico computazionale cresca molto rapidamente)}
\put(131.8,-631.011){\fontsize{12}{1}\usefont{T1}{cmr}{m}{n}\selectfont\color{color_217499}• poiché i controlli vengono effettuati solamente sulla base dei dati presenti }
\put(131.8,-644.811){\fontsize{12}{1}\usefont{T1}{cmr}{m}{n}\selectfont\color{color_217499}all’interno degli header i PF soffrono in presenza di spoofing. Per limitare i }
\put(131.8,-658.611){\fontsize{12}{1}\usefont{T1}{cmr}{m}{n}\selectfont\color{color_217499}problemi è necessario controllare la coerenza tra gli indirizzi e le interfacce e }
\put(131.8,-672.411){\fontsize{12}{1}\usefont{T1}{cmr}{m}{n}\selectfont\color{color_217499}controllare certi indirizzi particolari (indirizzi multicast, illegali, broadcast, }
\put(131.8,-686.211){\fontsize{12}{1}\usefont{T1}{cmr}{m}{n}\selectfont\color{color_217499}riservati, …)}
\put(131.8,-707.011){\fontsize{12}{1}\usefont{T1}{cmr}{m}{n}\selectfont\color{color_217499}• il filtraggio non è in grado di lavorare in presenza di protocolli che negoziano }
\put(131.8,-720.811){\fontsize{12}{1}\usefont{T1}{cmr}{m}{n}\selectfont\color{color_217499}l’apertura di porte dinamicamente (ad esempio connessione dati FTP)}
\put(131.8,-741.611){\fontsize{12}{1}\usefont{T1}{cmr}{m}{n}\selectfont\color{color_217499}• non è possibile implementare difese contro attacchi data-driven, poiché }
\put(131.8,-755.411){\fontsize{12}{1}\usefont{T1}{cmr}{m}{n}\selectfont\color{color_217499}viaggiano all’interno del payload del pacchetti}
\end{picture}
\newpage
\begin{tikzpicture}[overlay]\path(0pt,0pt);\end{tikzpicture}
\begin{picture}(-5,0)(2.5,0)
\put(113.8,-85.01099){\fontsize{12}{1}\usefont{T1}{cmr}{m}{n}\selectfont\color{color_217499}Tipicamente i PF sono stateless, tuttavia in alcuni casi è possibile usufruire di PF }
\put(113.8,-98.81097){\fontsize{12}{1}\usefont{T1}{cmr}{m}{n}\selectfont\color{color_217499}stateful; essi sono in grado di analizzare il traffico sulla base dei pacchetti già visti. È}
\put(113.8,-112.611){\fontsize{12}{1}\usefont{T1}{cmr}{m}{n}\selectfont\color{color_217499}possibile anche avere dei Multilayer protocol inspection firewall che sono in grado id}
\put(113.8,-126.411){\fontsize{12}{1}\usefont{T1}{cmr}{m}{n}\selectfont\color{color_217499}garantire la coerenza del protocollo tenendo traccia dell’intera storia della }
\put(113.8,-140.211){\fontsize{12}{1}\usefont{T1}{cmr}{m}{n}\selectfont\color{color_217499}connessione}
\put(113.8,-161.011){\fontsize{12}{1}\usefont{T1}{cmr}{b}{n}\selectfont\color{color_217499}2. Application-level gateway (anche chiamato proxy server): può essere definito }
\put(113.8,-174.811){\fontsize{12}{1}\usefont{T1}{cmr}{m}{n}\selectfont\color{color_217499}come un man in the middle" buono, in grado di propagare il traffico verso i server }
\put(113.8,-188.611){\fontsize{12}{1}\usefont{T1}{cmr}{m}{n}\selectfont\color{color_217499}effettivi. In questo modo è in grado di comprendere i protocolli applicativi e di }
\put(113.8,-202.411){\fontsize{12}{1}\usefont{T1}{cmr}{m}{n}\selectfont\color{color_217499}controllare anche il payload (ad esempio per bloccare spam/virus per quanto riguarda}
\put(113.8,-216.211){\fontsize{12}{1}\usefont{T1}{cmr}{m}{n}\selectfont\color{color_217499}i server di posta o per bloccare malware/phishing per quanto riguarda i server web). }
\put(113.8,-230.011){\fontsize{12}{1}\usefont{T1}{cmr}{m}{n}\selectfont\color{color_217499}Tuttavia i firewall appartenenti a questa tipologia sono moltopiù pesanti di un PF e }
\put(113.8,-243.811){\fontsize{12}{1}\usefont{T1}{cmr}{m}{n}\selectfont\color{color_217499}specifici per un singolo protocollo applicativo. Poiché agiscono come server nei }
\put(113.8,-257.611){\fontsize{12}{1}\usefont{T1}{cmr}{m}{n}\selectfont\color{color_217499}confronti dei client non è detto che siano trasparenti, ma è possibile che sia }
\put(113.8,-271.411){\fontsize{12}{1}\usefont{T1}{cmr}{m}{n}\selectfont\color{color_217499}necessario un minimo livello di configurazione.}
\put(113.8,-292.211){\fontsize{12}{1}\usefont{T1}{cmr}{b}{n}\selectfont\color{color_217499}3. Ciruit-level-gateway: spezzano la connessione a livello di trasporto diventando }
\put(113.8,-306.011){\fontsize{12}{1}\usefont{T1}{cmr}{m}{n}\selectfont\color{color_217499}endpoint del traffico ed inoltrando i payload senza esaminarli. Tipicamente sono }
\put(113.8,-319.811){\fontsize{12}{1}\usefont{T1}{cmr}{m}{n}\selectfont\color{color_217499}utilizzati per determinare quale sia il traffico ammissibile dall’interno verso }
\put(113.8,-333.611){\fontsize{12}{1}\usefont{T1}{cmr}{m}{n}\selectfont\color{color_217499}l’esterno. I firewall appartenenti a questa categoria hanno il vantaggio di poter essere}
\put(113.8,-347.411){\fontsize{12}{1}\usefont{T1}{cmr}{m}{n}\selectfont\color{color_217499}configurati per essere trasparenti agli utenti per autorizzare solamente le connessioni }
\put(113.8,-361.211){\fontsize{12}{1}\usefont{T1}{cmr}{m}{n}\selectfont\color{color_217499}fidate, inoltre possono agire da intermediario generico (non sono limitati ad un }
\put(113.8,-375.011){\fontsize{12}{1}\usefont{T1}{cmr}{m}{n}\selectfont\color{color_217499}singolo protocollo) e possono essere estesi con altri tipi di firewall per ottenere }
\put(113.8,-388.811){\fontsize{12}{1}\usefont{T1}{cmr}{m}{n}\selectfont\color{color_217499}servizi più complessi. Tuttavia le regole sono limitate ad indirizzi, porte ed utenti.}
\put(77.8,-409.611){\fontsize{12}{1}\usefont{T1}{cmr}{m}{n}\selectfont\color{color_217499}◦SOCKS: SOCKS (Socket Secure) è un protocollo di livello 5 per lo scambio di }
\put(95.8,-423.411){\fontsize{12}{1}\usefont{T1}{cmr}{m}{n}\selectfont\color{color_217499}pacchetto tra un client ed un server per mezzo di un proxy server. In aggiunta fornisce }
\put(95.8,-437.211){\fontsize{12}{1}\usefont{T1}{cmr}{m}{n}\selectfont\color{color_217499}l’autenticazione dei pacchetti, pertanto solamente gli utenti autorizzati sono in grado di }
\put(95.8,-451.011){\fontsize{12}{1}\usefont{T1}{cmr}{m}{n}\selectfont\color{color_217499}accedere al server. Questo protocollo è lo standard de facto per l’implementazione dei }
\put(95.8,-464.811){\fontsize{12}{1}\usefont{T1}{cmr}{m}{n}\selectfont\color{color_217499}CLG (Curcuit-level-gateway).}
\put(77.8,-485.611){\fontsize{12}{1}\usefont{T1}{cmr}{m}{n}\selectfont\color{color_217499}◦Tor: Il protocollo di Tor permette di realizzare connessioni cifrate in cui il legame tra chi}
\put(95.8,-499.411){\fontsize{12}{1}\usefont{T1}{cmr}{m}{n}\selectfont\color{color_217499}effettua richieste e il contenuto delle stesse è profondamente oscurato. Per prima cosa si }
\put(95.8,-513.211){\fontsize{12}{1}\usefont{T1}{cmr}{m}{n}\selectfont\color{color_217499}ottiene un elenco di nodi Tor da un directory server, successivamente ne vengono scelti }
\put(95.8,-527.011){\fontsize{12}{1}\usefont{T1}{cmr}{m}{n}\selectfont\color{color_217499}alcuni casualmente per creare un percorso tra mittente e destinatario su cui inviare un }
\put(95.8,-540.811){\fontsize{12}{1}\usefont{T1}{cmr}{m}{n}\selectfont\color{color_217499}messaggio cifrato a "cipolla", ad ogni hop viene rimosso uno strato di cifratura fino a }
\put(95.8,-554.611){\fontsize{12}{1}\usefont{T1}{cmr}{m}{n}\selectfont\color{color_217499}che il server non è in grado di leggere il traffico in chiaro. I nodi Tor formano un overlay}
\put(95.8,-568.411){\fontsize{12}{1}\usefont{T1}{cmr}{m}{n}\selectfont\color{color_217499}rispetto alla rete internet, pertanto è possibile che tra un nodo e l’altro il traffico }
\put(95.8,-582.211){\fontsize{12}{1}\usefont{T1}{cmr}{m}{n}\selectfont\color{color_217499}attraversi vari router, tuttavia questo non presenta problemi grazie ai vari livelli di }
\put(95.8,-596.011){\fontsize{12}{1}\usefont{T1}{cmr}{m}{n}\selectfont\color{color_217499}crittografia.}
\put(95.8,-616.811){\fontsize{12}{1}\usefont{T1}{cmr}{m}{n}\selectfont\color{color_217499}Ogni relay conosce solamente il nodo prima di lui e quello dopo, pertanto viene reso }
\put(95.8,-630.611){\fontsize{12}{1}\usefont{T1}{cmr}{m}{n}\selectfont\color{color_217499}molto difficile scoprire chi sia il vero mittente. Tuttavia il protocollo presenta alcune }
\put(95.8,-644.411){\fontsize{12}{1}\usefont{T1}{cmr}{m}{n}\selectfont\color{color_217499}debolezze intrinseche:}
\put(113.8,-665.211){\fontsize{12}{1}\usefont{T1}{cmr}{m}{n}\selectfont\color{color_217499}• è possibile correlare il traffico dei vari nodi per capire quale percorso facciano i }
\put(113.8,-679.011){\fontsize{12}{1}\usefont{T1}{cmr}{m}{n}\selectfont\color{color_217499}pacchetti, sebbene siano cifrati}
\put(113.8,-699.811){\fontsize{12}{1}\usefont{T1}{cmr}{m}{n}\selectfont\color{color_217499}• gli exit node vedono il traffico in chiaro, è sebbene non siano in grado di capire }
\put(113.8,-713.611){\fontsize{12}{1}\usefont{T1}{cmr}{m}{n}\selectfont\color{color_217499}automaticamente chi sia il mittente i payload dei pacchetti potrebbe contenere }
\put(113.8,-727.411){\fontsize{12}{1}\usefont{T1}{cmr}{m}{n}\selectfont\color{color_217499}informazioni ben più rilevanti ai fini dell’identificazione}
\put(113.8,-748.211){\fontsize{12}{1}\usefont{T1}{cmr}{m}{n}\selectfont\color{color_217499}• è sufficiente che all’interno della rete vi sia anche un solo nodo compromesso per }
\put(113.8,-762.011){\fontsize{12}{1}\usefont{T1}{cmr}{m}{n}\selectfont\color{color_217499}compromettere l’intera rete}
\end{picture}
\newpage
\begin{tikzpicture}[overlay]\path(0pt,0pt);\end{tikzpicture}
\begin{picture}(-5,0)(2.5,0)
\put(41.8,-90.211){\fontsize{17.5}{1}\usefont{T1}{cmr}{b}{n}\selectfont\color{color_217499}Packet Filtering}
\put(41.8,-111.111){\fontsize{12}{1}\usefont{T1}{cmr}{b}{n}\selectfont\color{color_217499}netstat è il comando utile a verificare le informazioni relative alle connessioni di rete.}
\put(59.8,-131.911){\fontsize{12}{1}\usefont{T1}{cmr}{m}{n}\selectfont\color{color_217499}•netstat -na --ip visualizza le informazioni relative alle connessioni TCP attive}
\put(59.8,-152.711){\fontsize{12}{1}\usefont{T1}{cmr}{m}{n}\selectfont\color{color_217499}•netstat -svisualizza una serie di dati di tipo statistico sul traffico relativo ai }
\put(77.8,-166.511){\fontsize{12}{1}\usefont{T1}{cmr}{m}{n}\selectfont\color{color_217499}diversi protocolli}
\put(41.8,-187.311){\fontsize{12}{1}\usefont{T1}{cmr}{b}{n}\selectfont\color{color_217499}nmap è il comando per fare port scanning su un host e visualizzare le porte TCP corrispondenti a }
\put(41.8,-201.111){\fontsize{12}{1}\usefont{T1}{cmr}{m}{n}\selectfont\color{color_217499}processi in ascolto (server). Il port scanning consiste nell'invio di un SYN verso ogni porta: se si }
\put(41.8,-214.911){\fontsize{12}{1}\usefont{T1}{cmr}{m}{n}\selectfont\color{color_217499}riceve un SYN+ACK la porta è attiva, se si riceve un RST+ACK la porta è chiusa. Se non si riceve }
\put(41.8,-228.711){\fontsize{12}{1}\usefont{T1}{cmr}{m}{n}\selectfont\color{color_217499}nulla, la porta è filtrata.}
\put(41.8,-249.511){\fontsize{12}{1}\usefont{T1}{cmr}{m}{n}\selectfont\color{color_217499}Un firewall è un filtro software che serve a proteggersi da accessi indesiderati provenienti }
\put(41.8,-263.311){\fontsize{12}{1}\usefont{T1}{cmr}{m}{n}\selectfont\color{color_217499}dall'esterno della rete, può essere un programma locale all'host (protegge questo da attacchi esterni e}
\put(41.8,-277.111){\fontsize{12}{1}\usefont{T1}{cmr}{m}{n}\selectfont\color{color_217499}la rete da attacchi generati da potenziali virus presenti sull'host) o una macchina dedicata che filtra }
\put(41.8,-290.911){\fontsize{12}{1}\usefont{T1}{cmr}{m}{n}\selectfont\color{color_217499}tutto il traffico in/out su una rete locale.}
\put(41.8,-311.711){\fontsize{12}{1}\usefont{T1}{cmr}{m}{n}\selectfont\color{color_217499}Il traffico tra la rete locale ed Internet deve essere filtrato dal firewall, solo quello autorizzato deve }
\put(41.8,-325.511){\fontsize{12}{1}\usefont{T1}{cmr}{m}{n}\selectfont\color{color_217499}attraversarlo. In fase di configurazione di un firewall, per prima cosa si decide la politica di default }
\put(41.8,-339.311){\fontsize{12}{1}\usefont{T1}{cmr}{m}{n}\selectfont\color{color_217499}per i servizi di rete:}
\put(59.8,-360.111){\fontsize{12}{1}\usefont{T1}{cmr}{m}{n}\selectfont\color{color_217499}•default deny:tutti i servizi non esplicitamente permessi sono negati}
\put(59.8,-380.911){\fontsize{12}{1}\usefont{T1}{cmr}{m}{n}\selectfont\color{color_217499}•default allow:tutti i servizi non esplicitamente negati sono permessi}
\put(41.8,-401.711){\fontsize{12}{1}\usefont{T1}{cmr}{m}{n}\selectfont\color{color_217499}Un firewall può essere implementato come }
\put(59.8,-422.511){\fontsize{12}{1}\usefont{T1}{cmr}{m}{n}\selectfont\color{color_217499}•Packet filter:si pone tra rete locale e Internet, filtrando i datagrammi IP da trasferire}
\put(77.8,-436.311){\fontsize{12}{1}\usefont{T1}{cmr}{m}{n}\selectfont\color{color_217499}sulle interfacce scartandoli in base a interfaccia (source e/o dest), MAC e/o IP (source o }
\put(77.8,-450.111){\fontsize{12}{1}\usefont{T1}{cmr}{m}{n}\selectfont\color{color_217499}dest), tipo di servizio (campo PROTOCOL o porta TCP/UDP). Può essere stateful/stateless }
\put(77.8,-463.911){\fontsize{12}{1}\usefont{T1}{cmr}{m}{n}\selectfont\color{color_217499}nel caso in cui tenga memoria o meno delle connessioni e/o dei flussi di traffico in corso. }
\put(59.8,-484.711){\fontsize{12}{1}\usefont{T1}{cmr}{m}{n}\selectfont\color{color_217499}•Application layer firewall o proxy server:filtra i dati in base ai protocolli }
\put(77.8,-498.511){\fontsize{12}{1}\usefont{T1}{cmr}{m}{n}\selectfont\color{color_217499}applicativi (HTTP, FTP…) e l'accesso alla rete esterna è possibile solo attraverso il proxy.}
\put(41.8,-519.311){\fontsize{12}{1}\usefont{T1}{cmr}{b}{n}\selectfont\color{color_217499}IPTABLES implementa uno stateful packet filter}
\end{picture}
\begin{tikzpicture}[overlay]
\path(0pt,0pt);
\draw[color_217499,line width=0.7pt]
(183.9pt, -520.411pt) -- (279.8pt, -520.411pt)
;
\end{tikzpicture}
\begin{picture}(-5,0)(2.5,0)
\put(279.9,-519.311){\fontsize{12}{1}\usefont{T1}{cmr}{m}{n}\selectfont\color{color_217499} nei kernel Linux 2.4 e successivi, lavora a livello }
\put(41.8,-533.111){\fontsize{12}{1}\usefont{T1}{cmr}{m}{n}\selectfont\color{color_217499}di kernel ed ha il controllo dei pacchetti IP in transito sulle interfacce di rete (loopback compreso). I}
\put(41.8,-546.911){\fontsize{12}{1}\usefont{T1}{cmr}{m}{n}\selectfont\color{color_217499}pacchetti processati sono sottoposti a diverse modalità di elaborazione chiamate table, ciascuna }
\put(41.8,-560.711){\fontsize{12}{1}\usefont{T1}{cmr}{m}{n}\selectfont\color{color_217499}delle quali è composta da gruppi di regole dette chain. IPTABLES definisce quattro table principali}
\put(59.8,-581.511){\fontsize{12}{1}\usefont{T1}{cmr}{m}{n}\selectfont\color{color_217499}•filterfiltraggio di pacchetti}
\put(59.8,-602.311){\fontsize{12}{1}\usefont{T1}{cmr}{m}{n}\selectfont\color{color_217499}•natsostituzione di indirizzi IP}
\put(59.8,-623.111){\fontsize{12}{1}\usefont{T1}{cmr}{m}{n}\selectfont\color{color_217499}•manglemanipolazione ulteriore di pacchetti (TOS, TTL..)}
\put(59.8,-643.911){\fontsize{12}{1}\usefont{T1}{cmr}{m}{n}\selectfont\color{color_217499}•rawesclusione di pacchetti dal connection tracking}
\put(41.8,-664.711){\fontsize{12}{1}\usefont{T1}{cmr}{m}{n}\selectfont\color{color_217499}•Tabella FILTER:contiene le vere funzionalità di firewall. Si occupa di filtrare i pacchetti in}
\put(59.8,-678.511){\fontsize{12}{1}\usefont{T1}{cmr}{m}{n}\selectfont\color{color_217499}base all'interfaccia di provenienza e dei parametri contenuti nell'header IP e TCP. Ha 3 chain }
\put(59.8,-692.311){\fontsize{12}{1}\usefont{T1}{cmr}{m}{n}\selectfont\color{color_217499}predefinite, ed è possibile definirne ulteriori:}
\put(59.8,-713.111){\fontsize{12}{1}\usefont{T1}{cmr}{m}{n}\selectfont\color{color_29791}◦INPUT:contiene le regole di filtraggio da usare sui pacchetti in arrivo al firewall }
\put(77.8,-726.911){\fontsize{12}{1}\usefont{T1}{cmr}{m}{n}\selectfont\color{color_217499}(destinati all'utente locale). (quali pacchetti destinati a processi locali possono raggiungerli)}
\end{picture}
\newpage
\begin{tikzpicture}[overlay]\path(0pt,0pt);\end{tikzpicture}
\begin{picture}(-5,0)(2.5,0)
\put(59.8,-85.01099){\fontsize{12}{1}\usefont{T1}{cmr}{m}{n}\selectfont\color{color_29791}◦OUTPUT:contiene le regole da usare sui pacchetti in uscita dal firewall (originati }
\put(77.8,-98.81097){\fontsize{12}{1}\usefont{T1}{cmr}{m}{n}\selectfont\color{color_217499}dall'host locale) (quali pacchetti generati da processi locali possono raggiungere la }
\put(77.8,-112.611){\fontsize{12}{1}\usefont{T1}{cmr}{m}{n}\selectfont\color{color_217499}destinazione)}
\put(59.8,-133.411){\fontsize{12}{1}\usefont{T1}{cmr}{m}{n}\selectfont\color{color_29791}◦FORWARD:contiene le regole da usare sui pacchetti in transito nel firewall (inoltrati tra }
\put(77.8,-147.211){\fontsize{12}{1}\usefont{T1}{cmr}{m}{n}\selectfont\color{color_217499}interfacce diverse) (quali pacchetti destinati a destinazioni esterne possono passare)}
\put(59.8,-168.011){\fontsize{12}{1}\usefont{T1}{cmr}{m}{n}\selectfont\color{color_29791}◦Regole della tabella FILTER:quando un pacchetto viene processato da una chain, è }
\put(77.8,-181.811){\fontsize{12}{1}\usefont{T1}{cmr}{m}{n}\selectfont\color{color_217499}soggetto alle regole specificate in essa, secondo l'ordine di inserimento}
\end{picture}
\begin{tikzpicture}[overlay]
\path(0pt,0pt);
\draw[color_217499,line width=0.7pt]
(267.4pt, -182.9109pt) -- (417.8pt, -182.9109pt)
;
\end{tikzpicture}
\begin{picture}(-5,0)(2.5,0)
\put(417.8,-181.811){\fontsize{12}{1}\usefont{T1}{cmr}{m}{n}\selectfont\color{color_217499}. Una regola può }
\put(77.8,-195.611){\fontsize{12}{1}\usefont{T1}{cmr}{m}{n}\selectfont\color{color_217499}decidere se scartare (DROP), rifiutare esplicitamente (REJECT) o accettare (ACCEPT) un }
\put(77.8,-209.411){\fontsize{12}{1}\usefont{T1}{cmr}{m}{n}\selectfont\color{color_217499}pacchetto in base a interfaccia coinvolta, IP source/dest, protocollo (TCP, UDP, ICMP), }
\put(77.8,-223.211){\fontsize{12}{1}\usefont{T1}{cmr}{m}{n}\selectfont\color{color_217499}porta source/dst, tipo di messaggio ICMP. Se un pacchetto non soddisfa nessuna regola, }
\put(77.8,-237.011){\fontsize{12}{1}\usefont{T1}{cmr}{m}{n}\selectfont\color{color_217499}viene applicata la regola di default, o policy, di quella chain. }
\put(59.8,-257.811){\fontsize{12}{1}\usefont{T1}{cmr}{m}{n}\selectfont\color{color_217499}◦Opzioni per gestione della tabella FILTER:}
\put(77.8,-278.611){\fontsize{12}{1}\usefont{T1}{cmr}{m}{n}\selectfont\color{color_29791}▪iptables -L [-nv –line-num]per visualizzare le regole attualmente in uso da ogni}
\put(95.8,-292.411){\fontsize{12}{1}\usefont{T1}{cmr}{m}{n}\selectfont\color{color_217499}chain della tabella filter. -n risultato numerico (non c'è reverse lookup del nome), -v }
\put(95.8,-306.211){\fontsize{12}{1}\usefont{T1}{cmr}{m}{n}\selectfont\color{color_217499}verboso}
\put(77.8,-327.011){\fontsize{12}{1}\usefont{T1}{cmr}{m}{n}\selectfont\color{color_29791}▪iptables -L <chain>visualizza regole in uso da una chain specifica}
\put(77.8,-347.811){\fontsize{12}{1}\usefont{T1}{cmr}{m}{n}\selectfont\color{color_29791}▪iptables -P <chain> <policy>imposta la policy di default di una chain}
\put(77.8,-368.611){\fontsize{12}{1}\usefont{T1}{cmr}{m}{n}\selectfont\color{color_29791}▪iptables -A <chain> <rule-specs> -j <policy>aggiunge una regola in coda}
\end{picture}
\begin{tikzpicture}[overlay]
\path(0pt,0pt);
\draw[color_217499,line width=0.7pt]
(444.3pt, -369.711pt) -- (479.3pt, -369.711pt)
;
\end{tikzpicture}
\begin{picture}(-5,0)(2.5,0)
\put(479.3,-368.611){\fontsize{12}{1}\usefont{T1}{cmr}{m}{n}\selectfont\color{color_217499} ad una }
\put(95.8,-382.411){\fontsize{12}{1}\usefont{T1}{cmr}{m}{n}\selectfont\color{color_217499}chain. }
\put(95.8,-403.211){\fontsize{12}{1}\usefont{T1}{cmr}{m}{n}\selectfont\color{color_217499}•<chain>è INPUT | OUTPUT | FORWARD | …}
\put(95.8,-424.011){\fontsize{12}{1}\usefont{T1}{cmr}{m}{n}\selectfont\color{color_217499}•<policy>è ACCEPT | DROP | REJECT | … (REJECT non è ammessa come }
\put(113.8,-437.811){\fontsize{12}{1}\usefont{T1}{cmr}{m}{n}\selectfont\color{color_217499}policy di default)}
\put(113.8,-458.611){\fontsize{12}{1}\usefont{T1}{cmr}{m}{n}\selectfont\color{color_217499}◦policy LOG indica al kernel di loggare i dettagli del pacchetto. Opzioni}
\put(131.8,-479.411){\fontsize{12}{1}\usefont{T1}{cmr}{m}{n}\selectfont\color{color_217499}▪--log-level <priority>indica una priority}
\put(131.8,-500.211){\fontsize{12}{1}\usefont{T1}{cmr}{m}{n}\selectfont\color{color_217499}▪--log-prefix <prefisso>indica un prefisso}
\put(131.8,-521.011){\fontsize{12}{1}\usefont{T1}{cmr}{m}{n}\selectfont\color{color_217499}▪--log-uidinserisce user-i???}
\put(113.8,-541.811){\fontsize{12}{1}\usefont{T1}{cmr}{m}{n}\selectfont\color{color_217499}◦ad ogni regola sono associati due contatori che vengono incrementati ogni volta }
\put(131.8,-555.611){\fontsize{12}{1}\usefont{T1}{cmr}{m}{n}\selectfont\color{color_217499}che un pacchetto fa "match" (un contatore di pacchetti, uno di byte }
\put(131.8,-569.411){\fontsize{12}{1}\usefont{T1}{cmr}{m}{n}\selectfont\color{color_217499}cumulativamente trasportati da essi). Una regola senza policy permette di }
\put(131.8,-583.211){\fontsize{12}{1}\usefont{T1}{cmr}{m}{n}\selectfont\color{color_217499}conteggiare il traffico con alcune caratteristiche senza interferire con transito }
\put(131.8,-597.011){\fontsize{12}{1}\usefont{T1}{cmr}{m}{n}\selectfont\color{color_217499}pacchetti}
\put(95.8,-617.811){\fontsize{12}{1}\usefont{T1}{cmr}{m}{n}\selectfont\color{color_217499}•<rule specs> specifica su quali pacchetti applicare la regola}
\put(77.8,-638.611){\fontsize{12}{1}\usefont{T1}{cmr}{m}{n}\selectfont\color{color_29791}▪iptables -I <chain> <N> <rule specs> -j <policy>inserisce regola in posizione }
\end{picture}
\begin{tikzpicture}[overlay]
\path(0pt,0pt);
\draw[color_217499,line width=0.7pt]
(457.4pt, -639.711pt) -- (518.7pt, -639.711pt)
;
\end{tikzpicture}
\begin{picture}(-5,0)(2.5,0)
\put(95.8,-652.411){\fontsize{12}{1}\usefont{T1}{cmr}{m}{n}\selectfont\color{color_217499}N nella chain indicata}
\end{picture}
\begin{tikzpicture}[overlay]
\path(0pt,0pt);
\draw[color_217499,line width=0.7pt]
(95.8pt, -653.511pt) -- (200.7pt, -653.511pt)
;
\end{tikzpicture}
\begin{picture}(-5,0)(2.5,0)
\put(77.8,-673.211){\fontsize{12}{1}\usefont{T1}{cmr}{m}{n}\selectfont\color{color_29791}▪iptables -R <chain> <N> <rule specs> -j <policy> sostituisce regola in posizione}
\end{picture}
\begin{tikzpicture}[overlay]
\path(0pt,0pt);
\draw[color_217499,line width=0.7pt]
(465.4pt, -674.311pt) -- (523.7pt, -674.311pt)
;
\end{tikzpicture}
\begin{picture}(-5,0)(2.5,0)
\put(95.8,-687.011){\fontsize{12}{1}\usefont{T1}{cmr}{m}{n}\selectfont\color{color_217499}N nella chain indicata}
\end{picture}
\begin{tikzpicture}[overlay]
\path(0pt,0pt);
\draw[color_217499,line width=0.7pt]
(95.8pt, -688.111pt) -- (200.7pt, -688.111pt)
;
\end{tikzpicture}
\begin{picture}(-5,0)(2.5,0)
\put(77.8,-707.811){\fontsize{12}{1}\usefont{T1}{cmr}{m}{n}\selectfont\color{color_29791}▪iptables -D <chain> <N>elimina la regola in posizione N di una chain}
\put(77.8,-728.611){\fontsize{12}{1}\usefont{T1}{cmr}{m}{n}\selectfont\color{color_29791}▪iptables -F <chain>flush di tutte le regole}
\end{picture}
\begin{tikzpicture}[overlay]
\path(0pt,0pt);
\draw[color_217499,line width=0.7pt]
(273.1pt, -729.711pt) -- (378.4pt, -729.711pt)
;
\end{tikzpicture}
\begin{picture}(-5,0)(2.5,0)
\put(378.4,-728.611){\fontsize{12}{1}\usefont{T1}{cmr}{m}{n}\selectfont\color{color_217499} di una specifica chain, o }
\put(95.8,-742.411){\fontsize{12}{1}\usefont{T1}{cmr}{m}{n}\selectfont\color{color_217499}omettendo<chain>, da tutte quante (non agisce su policy di default}
\end{picture}
\begin{tikzpicture}[overlay]
\path(0pt,0pt);
\draw[color_217499,line width=0.7pt]
(269.6pt, -743.511pt) -- (415.9pt, -743.511pt)
;
\end{tikzpicture}
\begin{picture}(-5,0)(2.5,0)
\put(415.9,-742.411){\fontsize{12}{1}\usefont{T1}{cmr}{m}{n}\selectfont\color{color_217499})}
\put(59.8,-763.211){\fontsize{12}{1}\usefont{T1}{cmr}{m}{n}\selectfont\color{color_29791}◦Specifica di regole <rule specs>}
\end{picture}
\newpage
\begin{tikzpicture}[overlay]\path(0pt,0pt);\end{tikzpicture}
\begin{picture}(-5,0)(2.5,0)
\put(77.8,-85.01099){\fontsize{12}{1}\usefont{T1}{cmr}{m}{n}\selectfont\color{color_29791}▪-i <interface> e -o <interface>specificano interfaccia di ingresso/uscita }
\put(77.8,-105.811){\fontsize{12}{1}\usefont{T1}{cmr}{m}{n}\selectfont\color{color_29791}▪-s <address>/<netmask> e -d <address>/<netmask>specificano IP (host o rete) di }
\put(95.8,-119.611){\fontsize{12}{1}\usefont{T1}{cmr}{m}{n}\selectfont\color{color_217499}origine/destinazione (netmask può essere in formato decimale puntato o CIDR}
\put(77.8,-140.411){\fontsize{12}{1}\usefont{T1}{cmr}{m}{n}\selectfont\color{color_29791}▪-p <tcp|udp|icmp|...>specifica il protocollo, /etc/protocols contiene lista }
\put(95.8,-154.211){\fontsize{12}{1}\usefont{T1}{cmr}{m}{n}\selectfont\color{color_217499}supportati}
\put(77.8,-175.011){\fontsize{12}{1}\usefont{T1}{cmr}{m}{n}\selectfont\color{color_29791}▪--sport <port> e --dport <port> specifica porta di origine/destinazione (/etc/services }
\put(95.8,-188.811){\fontsize{12}{1}\usefont{T1}{cmr}{m}{n}\selectfont\color{color_217499}contiene elenco porte corrispondenti ai servizi)}
\put(77.8,-209.611){\fontsize{12}{1}\usefont{T1}{cmr}{m}{n}\selectfont\color{color_29791}▪--icmp-type <typename>specifica tipo di messaggio ICMP (iptables -p icmp -h }
\put(95.8,-223.411){\fontsize{12}{1}\usefont{T1}{cmr}{m}{n}\selectfont\color{color_217499}mostra elenco tipi riconosciuti da IPTABLES)}
\put(77.8,-244.211){\fontsize{12}{1}\usefont{T1}{cmr}{m}{n}\selectfont\color{color_29791}▪--mac-source <MAC\_address>specifica MAC di origine}
\put(77.8,-265.011){\fontsize{12}{1}\usefont{T1}{cmr}{m}{n}\selectfont\color{color_29791}▪! per negare un'opzione (es. ! -s <address>/<netmask>)}
\put(77.8,-285.811){\fontsize{12}{1}\usefont{T1}{cmr}{m}{n}\selectfont\color{color_29791}▪-m state --state <NEW|ESTABLISHED>per specificare pacchetti di }
\put(95.8,-299.611){\fontsize{12}{1}\usefont{T1}{cmr}{m}{n}\selectfont\color{color_217499}connessioni TCP o flussi UDP nuovi o già attivi (stateful packet filter). Stati di }
\put(95.8,-313.411){\fontsize{12}{1}\usefont{T1}{cmr}{m}{n}\selectfont\color{color_217499}flusso/connessioni definiti da IPTABLES (conntrack):}
\put(95.8,-334.211){\fontsize{12}{1}\usefont{T1}{cmr}{m}{n}\selectfont\color{color_29791}•NEW:generato da un pacchetto appartenente a un flusso/connessione non }
\put(113.8,-348.011){\fontsize{12}{1}\usefont{T1}{cmr}{m}{n}\selectfont\color{color_217499}presente nella tabella conntrack}
\put(95.8,-368.811){\fontsize{12}{1}\usefont{T1}{cmr}{m}{n}\selectfont\color{color_29791}•ESTABLISHED:associato a flussi/connessioni dei quali sono stati già accettati }
\put(113.8,-382.611){\fontsize{12}{1}\usefont{T1}{cmr}{m}{n}\selectfont\color{color_217499}pacchetti precedenti, in entrambe le direzioni}
\put(95.8,-403.411){\fontsize{12}{1}\usefont{T1}{cmr}{m}{n}\selectfont\color{color_29791}•RELATED:associato a flussi/conness. Non  ESTABLISHED ma che sono }
\put(113.8,-417.211){\fontsize{12}{1}\usefont{T1}{cmr}{m}{n}\selectfont\color{color_217499}correlati con flussi/conness. ESTABLISHED (es. connessioni FTP control e FTP }
\put(113.8,-431.011){\fontsize{12}{1}\usefont{T1}{cmr}{m}{n}\selectfont\color{color_217499}data)}
\put(95.8,-451.811){\fontsize{12}{1}\usefont{T1}{cmr}{m}{n}\selectfont\color{color_29791}•INVALID:associato a pacchetti con stato non tracciabile}
\put(95.8,-472.611){\fontsize{12}{1}\usefont{T1}{cmr}{m}{n}\selectfont\color{color_29791}•UNTRACKED:associato a pacchetti non soggetti alla modalità stateful (tabella}
\put(113.8,-486.411){\fontsize{12}{1}\usefont{T1}{cmr}{b}{n}\selectfont\color{color_217499}raw)}
\put(95.8,-507.211){\fontsize{12}{1}\usefont{T1}{cmr}{m}{n}\selectfont\color{color_29791}•conntrack -Lcomando per visualizzare stato di flussi/conness.}
\put(41.8,-528.011){\fontsize{12}{1}\usefont{T1}{cmr}{m}{n}\selectfont\color{color_217499}•Tabella NAT:implementa le funzionalità di NAT in IPTABLES. Ha quattro chain }
\put(59.8,-541.811){\fontsize{12}{1}\usefont{T1}{cmr}{m}{n}\selectfont\color{color_217499}predefinite}
\put(59.8,-562.611){\fontsize{12}{1}\usefont{T1}{cmr}{m}{n}\selectfont\color{color_217499}◦PREROUTING:contiene regole da usare prima dell'instradamento per sostituire }
\put(77.8,-576.411){\fontsize{12}{1}\usefont{T1}{cmr}{m}{n}\selectfont\color{color_217499}indirizzo destinazione dei pacchetti (policy = Destination NAT o DNAT)}
\put(59.8,-597.211){\fontsize{12}{1}\usefont{T1}{cmr}{m}{n}\selectfont\color{color_217499}◦POSTROUTING:contiene regole da usare dopo l'instradamento per sostituire l'indirizzo }
\put(77.8,-611.011){\fontsize{12}{1}\usefont{T1}{cmr}{m}{n}\selectfont\color{color_217499}di origine dei pacchetti (policy = Source NAT o SNAT)}
\put(59.8,-631.811){\fontsize{12}{1}\usefont{T1}{cmr}{m}{n}\selectfont\color{color_217499}◦OUTPUT/INPUT:contiene regole da usare per sostituire l'indirizzo di pacchetti }
\put(77.8,-645.611){\fontsize{12}{1}\usefont{T1}{cmr}{m}{n}\selectfont\color{color_217499}generati/ricevuti localmente }
\put(77.8,-666.411){\fontsize{12}{1}\usefont{T1}{cmr}{m}{n}\selectfont\color{color_217499}La policy ACCEPT indica assenza di conversione, la policy MASQUERADE indica }
\put(77.8,-680.211){\fontsize{12}{1}\usefont{T1}{cmr}{m}{n}\selectfont\color{color_217499}conversione implicita dell'IP assegnato all'interfaccia di uscita}
\put(59.8,-701.011){\fontsize{12}{1}\usefont{T1}{cmr}{m}{n}\selectfont\color{color_29791}◦Opzioni per gestione della tabella NAT:}
\put(77.8,-721.811){\fontsize{12}{1}\usefont{T1}{cmr}{m}{n}\selectfont\color{color_29791}▪iptables -t nat -L [-nv --line-num]visualizzare regole attualmente in uso da ogni chain}
\put(95.8,-735.611){\fontsize{12}{1}\usefont{T1}{cmr}{m}{n}\selectfont\color{color_217499}della tabella NAT}
\put(77.8,-756.411){\fontsize{12}{1}\usefont{T1}{cmr}{m}{n}\selectfont\color{color_29791}▪iptables -t nat -L <chain>visualizza regole attualmente in uso da una chain specifica}
\end{picture}
\newpage
\begin{tikzpicture}[overlay]\path(0pt,0pt);\end{tikzpicture}
\begin{picture}(-5,0)(2.5,0)
\put(77.8,-85.01099){\fontsize{12}{1}\usefont{T1}{cmr}{m}{n}\selectfont\color{color_29791}▪iptables -t nat -A<chain> <rule-specs> -j <policy>aggiunge regola in coda ad }
\put(95.8,-98.81097){\fontsize{12}{1}\usefont{T1}{cmr}{m}{n}\selectfont\color{color_217499}una chain}
\put(95.8,-119.611){\fontsize{12}{1}\usefont{T1}{cmr}{m}{n}\selectfont\color{color_29791}•<chain> è POSTROUTING | PREROUTING | OUTPUT | INPUT | …}
\put(95.8,-140.411){\fontsize{12}{1}\usefont{T1}{cmr}{m}{n}\selectfont\color{color_29791}•<policy>è ACCEPT | MASQUERADE | SNAT --to-source <addr> | DNAT --}
\put(113.8,-154.211){\fontsize{12}{1}\usefont{T1}{cmr}{m}{n}\selectfont\color{color_217499}to-destination <addr> }
\put(113.8,-175.011){\fontsize{12}{1}\usefont{T1}{cmr}{m}{n}\selectfont\color{color_29791}◦<addr>è  <address> | <address>:<port>}
\put(95.8,-195.811){\fontsize{12}{1}\usefont{T1}{cmr}{m}{n}\selectfont\color{color_29791}•<rule specs>uguale a tabella FILTER}
\put(41.8,-213.811){\fontsize{9}{1}\usefont{T1}{cmr}{m}{n}\selectfont\color{color_29791}•Case study (ESEMPIO config firewall e NAT) }
\put(59.8,-231.111){\fontsize{9}{1}\usefont{T1}{cmr}{m}{n}\selectfont\color{color_29791}◦Tutte le LAN aziendali che usano IP pubblici devono essere accessibili senza restrizioni da parte di connessioni provenienti}
\put(77.8,-241.511){\fontsize{9}{1}\usefont{T1}{cmr}{m}{n}\selectfont\color{color_217499}da altre LAN aziendali. [1]}
\put(59.8,-258.811){\fontsize{9}{1}\usefont{T1}{cmr}{m}{n}\selectfont\color{color_29791}◦I server nella LAN 1 devono essere accessibili da parte di connessioni non provenienti da LAN aziendali, solo se dirette }
\put(77.8,-269.211){\fontsize{9}{1}\usefont{T1}{cmr}{m}{n}\selectfont\color{color_217499}verso i servizi HTTP e HTTPS. [2]}
\put(59.8,-286.511){\fontsize{9}{1}\usefont{T1}{cmr}{m}{n}\selectfont\color{color_29791}◦Almeno un host della LAN 2 deve poter essere raggiungibile via SSH dalla LAN 3. [3]}
\put(59.8,-303.911){\fontsize{9}{1}\usefont{T1}{cmr}{m}{n}\selectfont\color{color_29791}◦Le altre LAN aziendali non devono essere accessibili da connessioni originate dall'esterno, ma tutti i loro host e server }
\put(77.8,-314.211){\fontsize{9}{1}\usefont{T1}{cmr}{m}{n}\selectfont\color{color_217499}devono poter accedere all'esterno senza limitazioni. [4]}
\put(77.8,-331.611){\fontsize{9}{1}\usefont{T1}{cmr}{m}{n}\selectfont\color{color_29791}▪LAN 1: config firewall}
\put(95.8,-348.911){\fontsize{9}{1}\usefont{T1}{cmr}{m}{n}\selectfont\color{color_29791}•iptables -P FORWARD DROP\#policy di default}
\put(113.8,-366.311){\fontsize{9}{1}\usefont{T1}{cmr}{m}{it}\selectfont\color{color_217499}iptables -A FORWARD -i eth0 -m state --state NEW -j ACCEPT \#accetta pacchetti per nuove connessioni su eth0}
\put(113.8,-376.611){\fontsize{9}{1}\usefont{T1}{cmr}{b}{n}\selectfont\color{color_217499}[4]}
\put(113.8,-394.011){\fontsize{9}{1}\usefont{T1}{cmr}{m}{it}\selectfont\color{color_217499}iptables -A FORWARD -i ppp1 -s 192.168.0.0/24 -m state --state NEW -j ACCEPT \#accetta nuove connessioni su}
\put(113.8,-404.311){\fontsize{9}{1}\usefont{T1}{cmr}{m}{it}\selectfont\color{color_217499}ppp1 provenienti da LAN2 [1]}
\put(113.8,-421.711){\fontsize{9}{1}\usefont{T1}{cmr}{m}{it}\selectfont\color{color_217499}iptables -A FORWARD -i ppp0 -s 155.108.131.0/24 -m state --state NEW -j ACCEPT \#accetta nuove connessioni}
\put(113.8,-432.011){\fontsize{9}{1}\usefont{T1}{cmr}{m}{it}\selectfont\color{color_217499}su ppp0 provenienti da LAN3 [1]}
\put(113.8,-449.411){\fontsize{9}{1}\usefont{T1}{cmr}{m}{it}\selectfont\color{color_217499}iptables -A FORWARD -i ppp0 -d 87.15.12.0/24 -p tcp --dport 80 -m state --state NEW -j ACCEPT \#accetta }
\put(113.8,-459.711){\fontsize{9}{1}\usefont{T1}{cmr}{m}{it}\selectfont\color{color_217499}nuove connessioni su ppp0 verso i server (dest LAN 1), a patto che siano prot TCP verso porta 80 (servizio }
\put(113.8,-470.111){\fontsize{9}{1}\usefont{T1}{cmr}{m}{it}\selectfont\color{color_217499}HTTP) [2]}
\put(113.8,-487.411){\fontsize{9}{1}\usefont{T1}{cmr}{m}{it}\selectfont\color{color_217499}iptables -A FORWARD -i ppp0 -d 87.15.12.0/24 -p tcp --dport 443 -m state --state NEW -j ACCEPT \#accetta }
\put(113.8,-497.811){\fontsize{9}{1}\usefont{T1}{cmr}{m}{it}\selectfont\color{color_217499}nuove connessioni su ppp0 verso i server (dest LAN1), a patto che siano prot TCP verso porta 443 (servizio }
\put(113.8,-508.111){\fontsize{9}{1}\usefont{T1}{cmr}{m}{it}\selectfont\color{color_217499}HTTPS) [2]}
\put(113.8,-525.511){\fontsize{9}{1}\usefont{T1}{cmr}{m}{it}\selectfont\color{color_217499}iptables -I FORWARD 1 -m state --state ESTABLISHED -j ACCEPT \#accetta pacchetti di connessioni già }
\put(113.8,-535.811){\fontsize{9}{1}\usefont{T1}{cmr}{m}{it}\selectfont\color{color_217499}stabilite, pone regola in posizione 1}
\put(77.8,-553.211){\fontsize{9}{1}\usefont{T1}{cmr}{m}{n}\selectfont\color{color_29791}▪LAN 2: config firewall e NAT}
\put(95.8,-570.511){\fontsize{9}{1}\usefont{T1}{cmr}{m}{n}\selectfont\color{color_29791}•iptables -P FORWARD DROP\#policy di default}
\put(113.8,-587.911){\fontsize{9}{1}\usefont{T1}{cmr}{m}{it}\selectfont\color{color_217499}iptables -A FORWARD -i eth0 -m state --state NEW -j ACCEPT \#accetta nuove connessioni su eth0 [4]}
\put(113.8,-605.211){\fontsize{9}{1}\usefont{T1}{cmr}{m}{it}\selectfont\color{color_217499}iptables -A FORWARD -i ppp1 -s 87.15.12.0/24 -m state --state NEW -j ACCEPT \#accetta nuove connessioni su }
\put(113.8,-615.611){\fontsize{9}{1}\usefont{T1}{cmr}{m}{it}\selectfont\color{color_217499}ppp1 provenienti da LAN1 [1]}
\put(113.8,-632.911){\fontsize{9}{1}\usefont{T1}{cmr}{m}{it}\selectfont\color{color_217499}iptables -A FORWARD -i ppp0 -s 155.108.131.0/24 -m state --state NEW -j ACCEPT \#accetta su ppp0 pacchetti }
\put(113.8,-643.311){\fontsize{9}{1}\usefont{T1}{cmr}{m}{it}\selectfont\color{color_217499}per nuove connessioni provenienti da LAN3 [1]}
\put(113.8,-660.611){\fontsize{9}{1}\usefont{T1}{cmr}{m}{it}\selectfont\color{color_217499}iptables -I FORWARD 1 -m state --state ESTABLISHED -j ACCEPT \#accetta pacchetti di connessioni già }
\put(113.8,-671.011){\fontsize{9}{1}\usefont{T1}{cmr}{m}{it}\selectfont\color{color_217499}stabilite, pone regola in posizione 1}
\put(113.8,-688.311){\fontsize{9}{1}\usefont{T1}{cmr}{m}{it}\selectfont\color{color_217499}iptables -t nat -A POSTROUTING -s 192.168.0.0/24 -o ppp0 -j SNAT --to-source 87.4.8.21 \# sostituisce indirizzo}
\put(113.8,-698.711){\fontsize{9}{1}\usefont{T1}{cmr}{m}{it}\selectfont\color{color_217499}di provenienza dei pacchetti (originati in LAN2) in uscita su ppp0 con "87.4.8.21" [3]}
\put(113.8,-716.011){\fontsize{9}{1}\usefont{T1}{cmr}{m}{it}\selectfont\color{color_217499}iptables -t nat -A PREROUTING -i ppp0 -d 87.4.8.21 -p tcp --dport 2222 -j DNAT --to-destination }
\put(113.8,-726.411){\fontsize{9}{1}\usefont{T1}{cmr}{m}{it}\selectfont\color{color_217499}192.168.0.1:22 \# sostituisce indirizzo di destinazione dei pacchetti in ingresso su ppp0 (destinati a "87.4.8.21"), }
\put(113.8,-736.711){\fontsize{9}{1}\usefont{T1}{cmr}{m}{it}\selectfont\color{color_217499}se hanno prot. TCP e porta 2222, con "192.168.0.1:22" (servizio SSH su host 192.168.0.1) [3]}
\put(77.8,-754.111){\fontsize{9}{1}\usefont{T1}{cmr}{m}{n}\selectfont\color{color_29791}▪LAN 3: config firewall}
\put(95.8,-771.411){\fontsize{9}{1}\usefont{T1}{cmr}{m}{n}\selectfont\color{color_29791}•iptables -P FORWARD DROP\#policy di default}
\end{picture}
\newpage
\begin{tikzpicture}[overlay]\path(0pt,0pt);\end{tikzpicture}
\begin{picture}(-5,0)(2.5,0)
\put(113.8,-82.211){\fontsize{9}{1}\usefont{T1}{cmr}{m}{it}\selectfont\color{color_217499}iptables -A FORWARD -i eth0 -m state --state NEW -j ACCEPT \#accetta nuove connessioni su eth0 [4]}
\put(113.8,-99.51099){\fontsize{9}{1}\usefont{T1}{cmr}{m}{it}\selectfont\color{color_217499}iptables -A FORWARD -i ppp0 -s 87.15.12.0/24 -m state --state NEW -j ACCEPT \#accetta nuove connessioni su }
\put(113.8,-109.911){\fontsize{9}{1}\usefont{T1}{cmr}{m}{it}\selectfont\color{color_217499}ppp0 provenienti da LAN1 [1]}
\put(113.8,-127.211){\fontsize{9}{1}\usefont{T1}{cmr}{m}{it}\selectfont\color{color_217499}iptables -A FORWARD -i ppp0 -s 87.4.8.21 -m state --state NEW -j ACCEPT \#accetta nuove connessioni su ppp0}
\put(113.8,-137.611){\fontsize{9}{1}\usefont{T1}{cmr}{m}{it}\selectfont\color{color_217499}provenienti da "87.4.8.21" [3]}
\put(113.8,-154.911){\fontsize{9}{1}\usefont{T1}{cmr}{m}{it}\selectfont\color{color_217499}iptables -I FORWARD 1 -m state --state ESTABLISHED -j ACCEPT \#accetta pacchetti di connessioni già }
\put(113.8,-168.111){\fontsize{9}{1}\usefont{T1}{cmr}{m}{it}\selectfont\color{color_217499}stabilite, pone regola in posizione 1}
\put(41.8,-204.111){\fontsize{17.5}{1}\usefont{T1}{cmr}{b}{n}\selectfont\color{color_29791}SNMP}
\put(41.8,-225.011){\fontsize{12}{1}\usefont{T1}{cmr}{m}{n}\selectfont\color{color_29791}La gestione di apparati in rete è problematica in quanto ognuno di questi dispone, per }
\put(41.8,-238.811){\fontsize{12}{1}\usefont{T1}{cmr}{m}{n}\selectfont\color{color_29791}configurazione e monitoraggio, di strumenti proprietari, quindi c'è necessità di interfacciarsi con }
\put(41.8,-252.611){\fontsize{12}{1}\usefont{T1}{cmr}{m}{n}\selectfont\color{color_29791}linguaggi diversi a seconda che si voglia gestire ad esempio computer di SO diversi, apparati di rete}
\put(41.8,-266.411){\fontsize{12}{1}\usefont{T1}{cmr}{m}{n}\selectfont\color{color_29791}di produttori diversi. Per la configurazione persistente, i tool sono inevitabili (anche se oggi il }
\put(41.8,-280.211){\fontsize{12}{1}\usefont{T1}{cmr}{m}{n}\selectfont\color{color_29791}mondo del software-defined network fa sì che si usino apparecchi generici per la costruzione delle }
\put(41.8,-294.011){\fontsize{12}{1}\usefont{T1}{cmr}{m}{n}\selectfont\color{color_29791}reti, e le regole vengono decise da un control plane (a livello più alto); facendo sì che ci siano }
\put(41.8,-307.811){\fontsize{12}{1}\usefont{T1}{cmr}{m}{n}\selectfont\color{color_29791}protocolli standard) ma per il monitoraggio di base sono un ostacolo all'automazione (infatti spesso }
\put(41.8,-321.611){\fontsize{12}{1}\usefont{T1}{cmr}{m}{n}\selectfont\color{color_29791}si ricorre al wrapping degli strumenti proprietari per avere un unico linguaggio per interfacciarsi }
\put(41.8,-335.411){\fontsize{12}{1}\usefont{T1}{cmr}{m}{n}\selectfont\color{color_29791}con vari di questi). Ci sono inoltre svantaggi di sicurezza nei protocolli di accesso generici (es. }
\put(41.8,-349.211){\fontsize{12}{1}\usefont{T1}{cmr}{m}{n}\selectfont\color{color_29791}TELNET offre canale non sicuro, shell interattiva lascia disponibili troppe funzionalità). La risposta}
\put(41.8,-363.011){\fontsize{12}{1}\usefont{T1}{cmr}{m}{n}\selectfont\color{color_29791}a questi problemi è SNMP (Simple Network Management Protocol), standard molto diffuso per }
\put(41.8,-376.811){\fontsize{12}{1}\usefont{T1}{cmr}{m}{n}\selectfont\color{color_29791}sorvegliare grande quantità di macchine con approccio centralizzato, invece di usare comandi locali.}
\put(41.8,-390.611){\fontsize{12}{1}\usefont{T1}{cmr}{m}{n}\selectfont\color{color_29791}Si desiderava un protocollo che di mestiere facesse solo il management degli apparati connessi in }
\put(41.8,-404.411){\fontsize{12}{1}\usefont{T1}{cmr}{m}{n}\selectfont\color{color_29791}rete: "Simple" perché basilare, con due scopi: }
\put(59.8,-425.211){\fontsize{12}{1}\usefont{T1}{cmr}{m}{n}\selectfont\color{color_29791}•Essere meno invasivo possibile, aspetto di sicurezza: certe cose non si possono fare quindi }
\put(77.8,-439.011){\fontsize{12}{1}\usefont{T1}{cmr}{m}{n}\selectfont\color{color_29791}certi errori non si possono fare, anche volendo}
\put(59.8,-459.811){\fontsize{12}{1}\usefont{T1}{cmr}{m}{n}\selectfont\color{color_29791}•Costi: ogni genere di apparato di rete (telecamere, stampanti...) oggi disponibile a basso }
\put(77.8,-473.611){\fontsize{12}{1}\usefont{T1}{cmr}{m}{n}\selectfont\color{color_29791}costo, quindi senza un microprocessore con SO. La semplicità deve consentire di scrivere }
\put(77.8,-487.411){\fontsize{12}{1}\usefont{T1}{cmr}{m}{n}\selectfont\color{color_29791}firmware sui microcontrollori senza aumentare i costi di produzione.}
\put(41.8,-508.211){\fontsize{12}{1}\usefont{T1}{cmr}{m}{n}\selectfont\color{color_29791}SNMP standardizza:}
\put(59.8,-529.011){\fontsize{12}{1}\usefont{T1}{cmr}{m}{n}\selectfont\color{color_29791}•il modello dei dati, per cui qualunque proprietà di rete mi interessi identificare, viene }
\put(77.8,-542.811){\fontsize{12}{1}\usefont{T1}{cmr}{m}{n}\selectfont\color{color_29791}rappresentata come oggetto. Es. posizione di un'apparato: potrebbe essere utile sapere dove }
\put(77.8,-556.611){\fontsize{12}{1}\usefont{T1}{cmr}{m}{n}\selectfont\color{color_29791}si trova fisicamente, se arriva un notifica in ufficio che apparato sta andando a fuoco. }
\put(77.8,-577.411){\fontsize{12}{1}\usefont{T1}{cmr}{m}{n}\selectfont\color{color_29791}◦La proprietà ha ricevuto nome univoco (1.3.6.1.2.1.1.6) e ha ricevuto una sintassi }
\put(95.8,-591.211){\fontsize{12}{1}\usefont{T1}{cmr}{m}{n}\selectfont\color{color_29791}standard (stringa di non più di 255 caratteri). Sintassi e nome univoco possono essere }
\put(95.8,-605.011){\fontsize{12}{1}\usefont{T1}{cmr}{m}{n}\selectfont\color{color_29791}letti online, la definizione formale è parte della dotazione di SNMP.}
\put(59.8,-625.811){\fontsize{12}{1}\usefont{T1}{cmr}{m}{n}\selectfont\color{color_29791}•Il modello di interazione: questi dati dove stanno?Sull'apparato; questi dati chi li }
\put(77.8,-639.611){\fontsize{12}{1}\usefont{T1}{cmr}{m}{n}\selectfont\color{color_29791}usa?il gestore, che sta nel suo ufficio e controlla centinaia di apparati. Ci vuole un }
\put(77.8,-653.411){\fontsize{12}{1}\usefont{T1}{cmr}{m}{n}\selectfont\color{color_29791}protocollo per recuperare queste informazioni, ed SNMP ne offre uno applicativo veicolato }
\put(77.8,-667.211){\fontsize{12}{1}\usefont{T1}{cmr}{m}{n}\selectfont\color{color_29791}su UDP che gestisce la comunicazione tra dispositivi ed entità che li gestisce}
\put(41.8,-697.011){\fontsize{14.1}{1}\usefont{T1}{cmr}{b}{n}\selectfont\color{color_29791}Modello dei dati: OID e MIB}
\put(41.8,-717.211){\fontsize{12}{1}\usefont{T1}{cmr}{m}{n}\selectfont\color{color_29791}Alla base di SNMP è un modello generico per inquadrare qualsiasi oggetto concreto, proprietà di un}
\put(41.8,-731.011){\fontsize{12}{1}\usefont{T1}{cmr}{m}{n}\selectfont\color{color_29791}oggetto, o concetto astratto: ognuno di questi è identificato univocamente da un OID (Object }
\put(41.8,-744.811){\fontsize{12}{1}\usefont{T1}{cmr}{m}{n}\selectfont\color{color_29791}IDentifier). Il modello è stato concepito dall'ITU, con l'obiettivo di fare una tassonomia di qualsiasi }
\put(41.8,-758.611){\fontsize{12}{1}\usefont{T1}{cmr}{m}{n}\selectfont\color{color_29791}proprietà/oggetto che fa parte del mondo delle telecomunicazioni. La gerarchia parte da un radice }
\end{picture}
\newpage
\begin{tikzpicture}[overlay]\path(0pt,0pt);\end{tikzpicture}
\begin{picture}(-5,0)(2.5,0)
\put(41.8,-85.01099){\fontsize{12}{1}\usefont{T1}{cmr}{m}{n}\selectfont\color{color_29791}anonima ("."), da cui discendono tre archi (0 ITU-T, 1 ISO, 2 joint-iso-itu-t). Dalla radice alla }
\put(41.8,-98.81097){\fontsize{12}{1}\usefont{T1}{cmr}{m}{n}\selectfont\color{color_29791}foglia, si ha sequenza di numeri che porta all'oggetto: ogni nodo ha id numerico e simbolico (es. }
\put(41.8,-112.611){\fontsize{12}{1}\usefont{T1}{cmr}{m}{n}\selectfont\color{color_29791}1.3.6.1 == iso.identified-organization.dod.internet). La sequenza dei numeri è propriamente la }
\put(41.8,-126.411){\fontsize{12}{1}\usefont{T1}{cmr}{m}{n}\selectfont\color{color_29791}tassonomia delle proprietà e non ne individua un valore (qualsiasi, colore di tubo per gas pericoloso,}
\put(41.8,-140.211){\fontsize{12}{1}\usefont{T1}{cmr}{m}{n}\selectfont\color{color_29791}numero di bulloni passati da settore produzione…): quando dico 1.3.6.xxx non sto dicendo che }
\put(41.8,-154.011){\fontsize{12}{1}\usefont{T1}{cmr}{m}{n}\selectfont\color{color_29791}conto quanti pacchetti passano, sto dando un nome alla proprietà di quanti pacchetti passano.}
\put(41.8,-174.811){\fontsize{12}{1}\usefont{T1}{cmr}{m}{n}\selectfont\color{color_29791}SNMP è sotto internet, che a sua volta è sotto dod (department of defence) essendo questo stato il }
\put(41.8,-188.611){\fontsize{12}{1}\usefont{T1}{cmr}{m}{n}\selectfont\color{color_29791}principale finanziatore di internet: iso.identified-organization.dod.internet...}
\put(41.8,-209.411){\fontsize{12}{1}\usefont{T1}{cmr}{m}{n}\selectfont\color{color_29791}...internet.mgmt.mib.system.syslocationquindi 1.3.6.1.2.1.1.6è il nome dato all'oggetto }
\put(41.8,-223.211){\fontsize{12}{1}\usefont{T1}{cmr}{m}{n}\selectfont\color{color_29791}"posizione dell'apparato", e la descrizione di questo oggetto contiene la sintassi della proprietà, oltre}
\put(41.8,-237.011){\fontsize{12}{1}\usefont{T1}{cmr}{m}{n}\selectfont\color{color_29791}allo stato… Stesso discorso per ogni proprietà che mi interessi codificare. }
\put(41.8,-257.811){\fontsize{12}{1}\usefont{T1}{cmr}{m}{n}\selectfont\color{color_29791}Viene detto MIB (Managed Information Base), la collezione degli oggetti gestiti da un apparato o }
\put(41.8,-271.611){\fontsize{12}{1}\usefont{T1}{cmr}{m}{n}\selectfont\color{color_29791}da un sistema di monitoraggio. E' la collezione degli oggetti che voglio avere a disposizione per }
\put(41.8,-285.411){\fontsize{12}{1}\usefont{T1}{cmr}{m}{n}\selectfont\color{color_29791}descrivere le proprietà di un dispositivo: }
\put(59.8,-306.211){\fontsize{12}{1}\usefont{T1}{cmr}{m}{n}\selectfont\color{color_29791}•da un lato è una collezione teorica (es. dizionario, che contiene "tutte" le parole) che mi dice}
\put(77.8,-320.011){\fontsize{12}{1}\usefont{T1}{cmr}{m}{n}\selectfont\color{color_29791}come sono identificati (con stringa di numeri) e definiti (con la loro sintassi) anche tutti gli }
\put(77.8,-333.811){\fontsize{12}{1}\usefont{T1}{cmr}{m}{n}\selectfont\color{color_29791}oggetti del mondo (il MIB globale contiene identificazione e definizione di tutte le proprietà }
\put(77.8,-347.611){\fontsize{12}{1}\usefont{T1}{cmr}{m}{n}\selectfont\color{color_29791}possibili!); quindi descrizione dell'intero albero globale degli OID;}
\put(59.8,-368.411){\fontsize{12}{1}\usefont{T1}{cmr}{m}{n}\selectfont\color{color_29791}•dall'altro lato, comprato un apparato, mi chiedo qual è il MIB che descrive proprietà di }
\put(77.8,-382.211){\fontsize{12}{1}\usefont{T1}{cmr}{m}{n}\selectfont\color{color_29791}quell'apparato: sarà un sottoalbero di quello globale, fatto di pezzetti di quello globale. Le }
\put(77.8,-396.011){\fontsize{12}{1}\usefont{T1}{cmr}{m}{n}\selectfont\color{color_29791}proprietà dell'apparato potrebbero anche non essere direttamente nel MIB, perché }
\put(77.8,-409.811){\fontsize{12}{1}\usefont{T1}{cmr}{m}{n}\selectfont\color{color_29791}potrebbero essere generate dinamicamente: nel MIB globale ci sarà scritto allora che c'è una }
\put(77.8,-423.611){\fontsize{12}{1}\usefont{T1}{cmr}{m}{n}\selectfont\color{color_29791}tabella con i dati. Quando vado a controllare contenuto MIB del mio dispositivo, mi accorgo}
\put(77.8,-437.411){\fontsize{12}{1}\usefont{T1}{cmr}{m}{n}\selectfont\color{color_29791}che ci saranno più proprietà di quelle prevedibili: nel MIB ci sono le colonne definibili, ma }
\put(77.8,-451.211){\fontsize{12}{1}\usefont{T1}{cmr}{m}{n}\selectfont\color{color_29791}non le righe possibili! }
\put(41.8,-472.011){\fontsize{12}{1}\usefont{T1}{cmr}{m}{n}\selectfont\color{color_29791}In sostanza un MIB è un catalogo che associa ad ogni oggetto un OID, una sintassi (tipo di dato)  }
\put(41.8,-485.811){\fontsize{12}{1}\usefont{T1}{cmr}{m}{n}\selectfont\color{color_29791}una codifica (descrive la rappresentazione materiale, per rendere possibile la comunicazione tra }
\put(41.8,-499.611){\fontsize{12}{1}\usefont{T1}{cmr}{m}{n}\selectfont\color{color_29791}architetture diverse); e formalmente usa linguaggio SMIv2, sottoinsieme di ASN.1}
\put(41.8,-520.411){\fontsize{12}{1}\usefont{T1}{cmr}{m}{n}\selectfont\color{color_29791}Tutte le volte che viene fuori un apparato, questo mi metterà a disposizione le sue proprietà }
\put(41.8,-534.211){\fontsize{12}{1}\usefont{T1}{cmr}{m}{n}\selectfont\color{color_29791}attraverso SNMP, mediante un sottoinsieme di MIB, consultabile online. Altro caso è quello in cui è}
\put(41.8,-548.011){\fontsize{12}{1}\usefont{T1}{cmr}{m}{n}\selectfont\color{color_29791}il venditore a dover dare informazioni (che non si troveranno sul sito globale): esistono aziende }
\put(41.8,-561.811){\fontsize{12}{1}\usefont{T1}{cmr}{m}{n}\selectfont\color{color_29791}private che hanno comprato un sottoalbero, in cui possono fare cosa vogliono senza chiedere }
\put(41.8,-575.611){\fontsize{12}{1}\usefont{T1}{cmr}{m}{n}\selectfont\color{color_29791}all'ISO. Es. venditore di telecamere IP si inventa nuovo modo di commutare tra visione }
\put(41.8,-589.411){\fontsize{12}{1}\usefont{T1}{cmr}{m}{n}\selectfont\color{color_29791}diurna/notturna; la proprietà che permette di interrogare la telecamera per sapere quale mode sta }
\put(41.8,-603.211){\fontsize{12}{1}\usefont{T1}{cmr}{m}{n}\selectfont\color{color_29791}usando sarà codificata e interrogabile via SNMP, ma il modo in cui interrogarla dovrà esserci detto }
\put(41.8,-617.011){\fontsize{12}{1}\usefont{T1}{cmr}{m}{n}\selectfont\color{color_29791}dal venditore, non sarà sul sito globale delle proprietà. Il modulo MIB relativo a un certo set di }
\put(41.8,-630.811){\fontsize{12}{1}\usefont{T1}{cmr}{m}{n}\selectfont\color{color_29791}informazioni deve essere noto a chi vuole interrogare il dispositivo, per sapere quali informazioni }
\put(41.8,-644.611){\fontsize{12}{1}\usefont{T1}{cmr}{m}{n}\selectfont\color{color_29791}sono disponibili e come interpretarle. }
\put(41.8,-665.411){\fontsize{12}{1}\usefont{T1}{cmr}{m}{n}\selectfont\color{color_29791}Le due sintassi dati supportate sono SMIv1 e v2}
\end{picture}
\begin{tikzpicture}[overlay]
\path(0pt,0pt);
\draw[color_29791,line width=0.7pt]
(260.4pt, -666.511pt) -- (272.4pt, -666.511pt)
;
\end{tikzpicture}
\begin{picture}(-5,0)(2.5,0)
\put(272.4,-665.411){\fontsize{12}{1}\usefont{T1}{cmr}{m}{n}\selectfont\color{color_29791}, quest'ultima aggiunge alcune migliorie }
\put(41.8,-679.211){\fontsize{12}{1}\usefont{T1}{cmr}{m}{n}\selectfont\color{color_29791}(sottolineate):}
\put(59.8,-700.011){\fontsize{12}{1}\usefont{T1}{cmr}{m}{n}\selectfont\color{color_29791}•Tipi di dato semplici:}
\put(77.8,-720.811){\fontsize{12}{1}\usefont{T1}{cmr}{m}{n}\selectfont\color{color_29791}◦interi a 32 bit con segno}
\put(77.8,-741.611){\fontsize{12}{1}\usefont{T1}{cmr}{m}{n}\selectfont\color{color_29791}◦stringhe di byte (lunghezza max. 65535)}
\put(77.8,-762.411){\fontsize{12}{1}\usefont{T1}{cmr}{m}{n}\selectfont\color{color_29791}◦OID}
\end{picture}
\newpage
\begin{tikzpicture}[overlay]\path(0pt,0pt);\end{tikzpicture}
\begin{picture}(-5,0)(2.5,0)
\put(59.8,-85.01099){\fontsize{12}{1}\usefont{T1}{cmr}{m}{n}\selectfont\color{color_29791}•Tipi di dato application-wide:}
\put(77.8,-105.811){\fontsize{12}{1}\usefont{T1}{cmr}{m}{n}\selectfont\color{color_29791}◦network addresses: come IPv4, come generiche stringhe di byte (anche IPv6, in v2)}
\end{picture}
\begin{tikzpicture}[overlay]
\path(0pt,0pt);
\draw[color_29791,line width=0.7pt]
(248.1pt, -106.9109pt) -- (496.3pt, -106.9109pt)
;
\end{tikzpicture}
\begin{picture}(-5,0)(2.5,0)
\put(77.8,-126.611){\fontsize{12}{1}\usefont{T1}{cmr}{m}{n}\selectfont\color{color_29791}◦counters: interi a 32/64}
\end{picture}
\begin{tikzpicture}[overlay]
\path(0pt,0pt);
\draw[color_29791,line width=0.7pt]
(194.8pt, -127.711pt) -- (206.8pt, -127.711pt)
;
\end{tikzpicture}
\begin{picture}(-5,0)(2.5,0)
\put(206.8,-126.611){\fontsize{12}{1}\usefont{T1}{cmr}{m}{n}\selectfont\color{color_29791} bit positivi e crescenti (a differenza degli interi possono solo }
\put(95.8,-140.411){\fontsize{12}{1}\usefont{T1}{cmr}{m}{n}\selectfont\color{color_29791}salire), rollover a zero}
\put(77.8,-161.211){\fontsize{12}{1}\usefont{T1}{cmr}{m}{n}\selectfont\color{color_29791}◦gauges: interi non negativi con limiti minimo e massimo}
\put(77.8,-182.011){\fontsize{12}{1}\usefont{T1}{cmr}{m}{n}\selectfont\color{color_29791}◦time ticks: centesimi di secondo trascorsi da un dato evento}
\put(77.8,-202.811){\fontsize{12}{1}\usefont{T1}{cmr}{m}{n}\selectfont\color{color_29791}◦opaques: stringhe arbitrarie senza controllo di sintassi}
\put(77.8,-223.611){\fontsize{12}{1}\usefont{T1}{cmr}{m}{n}\selectfont\color{color_29791}◦integers e unsigned integers: ridefiniscono gli interi per avere precisione arbitraria  }
\end{picture}
\begin{tikzpicture}[overlay]
\path(0pt,0pt);
\draw[color_29791,line width=0.7pt]
(95.7pt, -224.711pt) -- (489.5pt, -224.711pt)
;
\end{tikzpicture}
\begin{picture}(-5,0)(2.5,0)
\put(77.8,-244.411){\fontsize{12}{1}\usefont{T1}{cmr}{m}{n}\selectfont\color{color_29791}◦bit strings: stringhe di bit singolarmente identificati}
\put(41.8,-265.211){\fontsize{12}{1}\usefont{T1}{cmr}{m}{n}\selectfont\color{color_29791}Le uniche strutture dati supportate sono scalari e tabelle (array bidimensionali). Avremo 3 varianti }
\put(41.8,-279.011){\fontsize{12}{1}\usefont{T1}{cmr}{m}{n}\selectfont\color{color_29791}sintattiche dell'OID: }
\put(59.8,-299.811){\fontsize{12}{1}\usefont{T1}{cmr}{m}{n}\selectfont\color{color_29791}•l'OID rappresenta in astratto la proprietà (nodo dell'albero): es. quando dico (1.3.6.1.2.1.1.5)}
\put(77.8,-313.611){\fontsize{12}{1}\usefont{T1}{cmr}{m}{n}\selectfont\color{color_29791}sto intendendo la proprietà "nome dell'host"}
\put(59.8,-334.411){\fontsize{12}{1}\usefont{T1}{cmr}{m}{n}\selectfont\color{color_29791}•caso scalare: alla proprietà è associato un valore scalare (per questo computer qui, qual è }
\put(77.8,-348.211){\fontsize{12}{1}\usefont{T1}{cmr}{m}{n}\selectfont\color{color_29791}valore associato a 1.3.6.1.2.1.1.5 ?) Per esprimere il concetto di VALORE della proprietà si }
\put(77.8,-362.011){\fontsize{12}{1}\usefont{T1}{cmr}{m}{n}\selectfont\color{color_29791}aggiunge zero all'identificativo: quindi per interrogare il dispositivo, si scrive "nome }
\put(77.8,-375.811){\fontsize{12}{1}\usefont{T1}{cmr}{m}{n}\selectfont\color{color_29791}proprietà.0", che sarà OID per letture/scritture dispositivo}
\put(59.8,-396.611){\fontsize{12}{1}\usefont{T1}{cmr}{m}{n}\selectfont\color{color_29791}•caso tabella: alla proprietà è associata una tabella, per indicarne un valore si aggiunge}
\put(77.8,-410.411){\fontsize{12}{1}\usefont{T1}{cmr}{m}{it}\selectfont\color{color_29791}.colonna.riga per leggere valore di una cella. Es. se voglio sapere la terza proprietà della }
\put(77.8,-424.211){\fontsize{12}{1}\usefont{T1}{cmr}{m}{n}\selectfont\color{color_29791}seconda interfaccia di rete (terza colonna, seconda riga) scrivo "nome proprietà.3.2" - che }
\put(77.8,-438.011){\fontsize{12}{1}\usefont{T1}{cmr}{m}{n}\selectfont\color{color_29791}sarà OID per letture/scritture dispositivo, (non potremo mai leggere una tabella intera e }
\put(77.8,-451.811){\fontsize{12}{1}\usefont{T1}{cmr}{m}{n}\selectfont\color{color_29791}farcela restituire, dovremo sempre ottenere uno scalare)}
\put(41.8,-472.611){\fontsize{12}{1}\usefont{T1}{cmr}{m}{n}\selectfont\color{color_29791}MIB notevoli sono:}
\put(59.8,-493.411){\fontsize{12}{1}\usefont{T1}{cmr}{m}{n}\selectfont\color{color_29791}•MIB-2: e' il sottoinsieme che inizia in 1.3.6.1.2.1; includeva tutti i dati essenziali agli }
\put(77.8,-507.211){\fontsize{12}{1}\usefont{T1}{cmr}{m}{n}\selectfont\color{color_29791}apparati di rete. Oggi è un po' obsoleto, è stato modularizzato (diviso in TCP-MIB, UDP-}
\put(77.8,-521.011){\fontsize{12}{1}\usefont{T1}{cmr}{m}{n}\selectfont\color{color_29791}MIB ,IP-MIB, IF-MIB; e i suoi pezzi delegati, così che possano essere più gestibili, }
\put(77.8,-534.811){\fontsize{12}{1}\usefont{T1}{cmr}{m}{n}\selectfont\color{color_29791}riscrivendone solo una parte). Esempi a noi utili di managed object del MIB-2 sono}
\put(77.8,-555.611){\fontsize{12}{1}\usefont{T1}{cmr}{m}{n}\selectfont\color{color_29791}◦ Il gruppo system 1.3.6.1.2.1.1, usato per memorizzare dentro l'apparato stesso}
\put(95.8,-576.411){\fontsize{12}{1}\usefont{T1}{cmr}{m}{n}\selectfont\color{color_29791}▪sysDescr(1), descrizione libera}
\put(95.8,-597.211){\fontsize{12}{1}\usefont{T1}{cmr}{m}{n}\selectfont\color{color_29791}▪sysContact(4), persona responsabile (email,telefono)}
\put(95.8,-618.011){\fontsize{12}{1}\usefont{T1}{cmr}{m}{n}\selectfont\color{color_29791}▪sysName(5), nome apparato}
\put(95.8,-638.811){\fontsize{12}{1}\usefont{T1}{cmr}{m}{n}\selectfont\color{color_29791}▪sysLocation(6), posizione apparato}
\put(77.8,-659.611){\fontsize{12}{1}\usefont{T1}{cmr}{m}{n}\selectfont\color{color_29791}◦Il gruppo IP 1.3.6.1.2.1.4, formato da 19 scalari + 4 tabelle che descrivono com'è }
\put(95.8,-673.411){\fontsize{12}{1}\usefont{T1}{cmr}{m}{n}\selectfont\color{color_29791}configurato stack IP dell'apparato (es. qual è TTL di pacchetti prodotti); usato per }
\put(95.8,-687.211){\fontsize{12}{1}\usefont{T1}{cmr}{m}{n}\selectfont\color{color_29791}memorizzare parametri generali del protocollo e tabelle con parametri specifici di ogni }
\put(95.8,-701.011){\fontsize{12}{1}\usefont{T1}{cmr}{m}{n}\selectfont\color{color_29791}interfaccia, regola di routing, mappatura MAC. Es. tabella ifEntry (1.3.6.1.2.1.2.2.1),}
\put(95.8,-714.811){\fontsize{12}{1}\usefont{T1}{cmr}{m}{n}\selectfont\color{color_29791}dove per ogni interfaccia sul sistema, ci sarà una riga con queste info (colonne della }
\put(95.8,-728.611){\fontsize{12}{1}\usefont{T1}{cmr}{m}{n}\selectfont\color{color_29791}tabella)}
\put(95.8,-749.411){\fontsize{12}{1}\usefont{T1}{cmr}{m}{n}\selectfont\color{color_29791}▪ifindex(1) indice, numero ordinale dell'interfaccia}
\put(95.8,-770.211){\fontsize{12}{1}\usefont{T1}{cmr}{m}{n}\selectfont\color{color_29791}▪ifDescr(2) descrizione}
\end{picture}
\newpage
\begin{tikzpicture}[overlay]\path(0pt,0pt);\end{tikzpicture}
\begin{picture}(-5,0)(2.5,0)
\put(95.8,-85.01099){\fontsize{12}{1}\usefont{T1}{cmr}{m}{n}\selectfont\color{color_29791}▪ifType (3) tipo}
\put(95.8,-105.811){\fontsize{12}{1}\usefont{T1}{cmr}{m}{n}\selectfont\color{color_29791}▪ifMtu (4)}
\put(95.8,-126.611){\fontsize{12}{1}\usefont{T1}{cmr}{m}{n}\selectfont\color{color_29791}▪ifSpeed (5) velocità}
\put(95.8,-147.411){\fontsize{12}{1}\usefont{T1}{cmr}{m}{n}\selectfont\color{color_29791}▪ifPhysAddress (6) indirizzo MAC}
\put(95.8,-168.211){\fontsize{12}{1}\usefont{T1}{cmr}{m}{n}\selectfont\color{color_29791}▪ifOutOctets (16) quanti pacchetti sono usciti dall'interfaccia}
\put(59.8,-189.011){\fontsize{12}{1}\usefont{T1}{cmr}{m}{n}\selectfont\color{color_29791}•PEN (Private Enterprise Numbers): il sottoalbero su 1.3.6.1.4.1 è dedicato a moduli specifici}
\put(77.8,-202.811){\fontsize{12}{1}\usefont{T1}{cmr}{m}{n}\selectfont\color{color_29791}richiesti da enti privati (possono essere richiesti gratuitamente, l'elenco è consultabile }
\put(77.8,-216.611){\fontsize{12}{1}\usefont{T1}{cmr}{m}{n}\selectfont\color{color_29791}online). Due PEN sono importanti per monitoraggio sistemi operativi, entrambi estendono il}
\put(77.8,-230.411){\fontsize{12}{1}\usefont{T1}{cmr}{m}{n}\selectfont\color{color_29791}MIB con oggetti generati dinamicamente:}
\put(77.8,-251.211){\fontsize{12}{1}\usefont{T1}{cmr}{m}{n}\selectfont\color{color_29791}◦UCD-SNMP (1.3.6.1.4.1.2021), sotto University of California Davids:in questo }
\put(95.8,-265.011){\fontsize{12}{1}\usefont{T1}{cmr}{m}{n}\selectfont\color{color_29791}modulo hanno definito tutte le proprietà tipiche di un sistema operativo. Quindi tutta la }
\put(95.8,-278.811){\fontsize{12}{1}\usefont{T1}{cmr}{m}{n}\selectfont\color{color_29791}parte di rete sta in MIB2; qui tutte le proprietà di un sistema (stato dischi, memoria, }
\put(95.8,-292.611){\fontsize{12}{1}\usefont{T1}{cmr}{m}{n}\selectfont\color{color_29791}processi, carico, log).}
\put(77.8,-313.411){\fontsize{12}{1}\usefont{T1}{cmr}{m}{n}\selectfont\color{color_29791}◦NET-SNMP-EXTEND-MIB (1.3.6.1.4.1.8072) è l'output della direttiva extend, qui }
\put(95.8,-327.211){\fontsize{12}{1}\usefont{T1}{cmr}{m}{n}\selectfont\color{color_29791}possono essere fatti comparire dinamicamente nuovi oggetti (OID generato sul }
\put(95.8,-341.011){\fontsize{12}{1}\usefont{T1}{cmr}{m}{n}\selectfont\color{color_29791}momento) in base all'output di script di un managed object. Utile perché posso fare }
\put(95.8,-354.811){\fontsize{12}{1}\usefont{T1}{cmr}{m}{n}\selectfont\color{color_29791}script che conta es. quante volte utente ha aperto un file; e poi dico ad SNMP di }
\put(95.8,-368.611){\fontsize{12}{1}\usefont{T1}{cmr}{m}{n}\selectfont\color{color_29791}estendere questo MIB, ovvero dico al sistema di generare un nuovo oggetto (con nuovo }
\put(95.8,-382.411){\fontsize{12}{1}\usefont{T1}{cmr}{m}{n}\selectfont\color{color_29791}MIB) che verrà posto sotto questo nodo di partenza (1.3.6.1.4.1.8072.xxxxxxx) ed }
\put(95.8,-396.211){\fontsize{12}{1}\usefont{T1}{cmr}{m}{n}\selectfont\color{color_29791}essendo chiaramente uno scalare, accederò accodando al valore con 0}
\put(41.8,-426.011){\fontsize{14.1}{1}\usefont{T1}{cmr}{b}{n}\selectfont\color{color_29791}Modello di interazione e protocolli}
\put(41.8,-446.211){\fontsize{12}{1}\usefont{T1}{cmr}{m}{n}\selectfont\color{color_29791}I managed object sono le varie proprietà di un dispositivo, come finora descritte. Il dispositivo }
\put(41.8,-460.011){\fontsize{12}{1}\usefont{T1}{cmr}{m}{n}\selectfont\color{color_29791}prende il nome di network element (l'oggetto fisico connesso alla rete); su questo è in esecuzione }
\put(41.8,-473.811){\fontsize{12}{1}\usefont{T1}{cmr}{m}{n}\selectfont\color{color_29791}un agent. L'agent è un componente software o firmware che accede a memoria e registri dei }
\put(41.8,-487.611){\fontsize{12}{1}\usefont{T1}{cmr}{m}{n}\selectfont\color{color_29791}dispositivi fisici per renderne visibili i contenuti come managed object; fa da interfaccia tra }
\put(41.8,-501.411){\fontsize{12}{1}\usefont{T1}{cmr}{m}{n}\selectfont\color{color_29791}protocollo SNMP e strato di configurazione/funzionamento del network element, quindi espone via }
\put(41.8,-515.211){\fontsize{12}{1}\usefont{T1}{cmr}{m}{n}\selectfont\color{color_29791}rete una possibilità di comunicare con linguaggio standard. Il componente che si occupa di }
\put(41.8,-529.011){\fontsize{12}{1}\usefont{T1}{cmr}{m}{n}\selectfont\color{color_29791}interrogare gli agent (e organizzare informazioni ricevute) è detto manager: è l'altra componente }
\put(41.8,-542.811){\fontsize{12}{1}\usefont{T1}{cmr}{m}{n}\selectfont\color{color_29791}della comunicazione, fa tipicamente parte di un NMS (Network Management System) ma non }
\put(41.8,-556.611){\fontsize{12}{1}\usefont{T1}{cmr}{m}{n}\selectfont\color{color_29791}necessariamente: noi in laboratorio useremo comandi che producono un risultato, senza essere }
\put(41.8,-570.411){\fontsize{12}{1}\usefont{T1}{cmr}{m}{n}\selectfont\color{color_29791}integrati in un sofisticato sistema di monitoraggio che produce grafici, tabelle.}
\put(41.8,-591.211){\fontsize{12}{1}\usefont{T1}{cmr}{m}{n}\selectfont\color{color_29791}Ci sono somiglianze tra modello di interazione manager-agent e C/S: manager sarebbe client e }
\put(41.8,-605.011){\fontsize{12}{1}\usefont{T1}{cmr}{m}{n}\selectfont\color{color_29791}agent sarebbe server, ma con risorse hw e numerosità invertite: su internet pensiamo a pochi server }
\put(41.8,-618.811){\fontsize{12}{1}\usefont{T1}{cmr}{m}{n}\selectfont\color{color_29791}molto potenti, per servire tanti client leggeri; nel manager-agent è il contrario. UN solo manager }
\put(41.8,-632.611){\fontsize{12}{1}\usefont{T1}{cmr}{m}{n}\selectfont\color{color_29791}(robusto per raccogliere info da molti dispositivi, di frequente, tenerle in memoria, elaborare }
\put(41.8,-646.411){\fontsize{12}{1}\usefont{T1}{cmr}{m}{n}\selectfont\color{color_29791}statistiche, generare grafiche) da cui interrogo una grande quantità di agent (aggeggi molto }
\put(41.8,-660.211){\fontsize{12}{1}\usefont{T1}{cmr}{m}{n}\selectfont\color{color_29791}semplici, poche migliaia di righe di codice implementato in firmware). Inoltre c'è anche differenza }
\put(41.8,-674.011){\fontsize{12}{1}\usefont{T1}{cmr}{m}{n}\selectfont\color{color_29791}perché a volte l'agent prende iniziativa per comunicare con manager, ad es. quando c'è informazione}
\put(41.8,-687.811){\fontsize{12}{1}\usefont{T1}{cmr}{m}{n}\selectfont\color{color_29791}critica.}
\put(41.8,-708.611){\fontsize{12}{1}\usefont{T1}{cmr}{m}{n}\selectfont\color{color_29791}Il modello di interazione base è a polling, e un giro di polling dura un certo tempo: se c'è un }
\put(41.8,-722.411){\fontsize{12}{1}\usefont{T1}{cmr}{m}{n}\selectfont\color{color_29791}apparato critico per il quale non posso aspettare quel certo tempo, l'interazione deve essere fatta ad }
\put(41.8,-736.211){\fontsize{12}{1}\usefont{T1}{cmr}{m}{n}\selectfont\color{color_29791}interrupt (nel gergo di SNMP, trap). Una trap è una notifica asincrona inviata da un agent senza }
\put(41.8,-750.011){\fontsize{12}{1}\usefont{T1}{cmr}{m}{n}\selectfont\color{color_29791}aspettare di essere interrogato dal manager. Se il modello tipico a polling invece prevede che }
\put(41.8,-763.811){\fontsize{12}{1}\usefont{T1}{cmr}{m}{n}\selectfont\color{color_29791}manager invia richiesta e riceve risposta (GET/SET), in SNMPv1 non è prevista risposta da parte }
\end{picture}
\newpage
\begin{tikzpicture}[overlay]\path(0pt,0pt);\end{tikzpicture}
\begin{picture}(-5,0)(2.5,0)
\put(41.8,-85.01099){\fontsize{12}{1}\usefont{T1}{cmr}{m}{n}\selectfont\color{color_29791}del manager alla trap. E' prevista la possibilità di creare una sorta di proxy (il master agent), che si }
\put(41.8,-98.81097){\fontsize{12}{1}\usefont{T1}{cmr}{m}{n}\selectfont\color{color_29791}comporta come agent nei confronti di un manager e come manager nei confronti di altri agent: }
\put(41.8,-112.611){\fontsize{12}{1}\usefont{T1}{cmr}{m}{n}\selectfont\color{color_29791}richieste a due strati sono utili per gestione di apparati distribuiti geograficamente; così che il }
\put(41.8,-126.411){\fontsize{12}{1}\usefont{T1}{cmr}{m}{n}\selectfont\color{color_29791}numero di agent da interrogare per ogni giro di polling sia minore (suddivisi tra più master agent); }
\put(41.8,-140.211){\fontsize{12}{1}\usefont{T1}{cmr}{m}{n}\selectfont\color{color_29791}va tenuto conto che oltre alla latenza del polling si aggiunge anche la latenza delle interrogazioni }
\put(41.8,-154.011){\fontsize{12}{1}\usefont{T1}{cmr}{m}{n}\selectfont\color{color_29791}fatte dai master agent. }
\put(41.8,-174.811){\fontsize{12}{1}\usefont{T1}{cmr}{b}{n}\selectfont\color{color_29791}SNMP è un protocollo a livello applicativo, trasportato su UDP, con agent in ascolto su porta 161 e }
\put(41.8,-188.611){\fontsize{12}{1}\usefont{T1}{cmr}{m}{n}\selectfont\color{color_29791}manager su porta 162 per ricevere le trap (rispettive versioni sicure basate su TLS, sono su porta }
\put(41.8,-202.411){\fontsize{12}{1}\usefont{T1}{cmr}{m}{n}\selectfont\color{color_29791}10161/10162). Ci sono più versioni (v1,v2,v2c,v3), noi useremo la 1 per semplicità anche se oggi }
\put(41.8,-216.211){\fontsize{12}{1}\usefont{T1}{cmr}{m}{n}\selectfont\color{color_29791}tutti i dispositivi supportano la 3. Tutte le versioni sono accomunate dalla struttura del pacchetto }
\put(41.8,-230.011){\fontsize{12}{1}\usefont{T1}{cmr}{m}{n}\selectfont\color{color_29791}PDU (Protocol Data Unit, diviso in: versione, community, PDU-type, request-id, error-status, error-}
\put(41.8,-243.811){\fontsize{12}{1}\usefont{T1}{cmr}{m}{n}\selectfont\color{color_29791}index, variable bindings).}
\put(41.8,-264.611){\fontsize{12}{1}\usefont{T1}{cmr}{m}{n}\selectfont\color{color_29791}Ci sono più tipi di PDU (quelli sottolineati sono solo in v2 e v3):}
\put(59.8,-285.411){\fontsize{12}{1}\usefont{T1}{cmr}{m}{n}\selectfont\color{color_29791}•Le 4 possibili richieste che un manager può fare ad un agent}
\put(77.8,-306.211){\fontsize{12}{1}\usefont{T1}{cmr}{m}{n}\selectfont\color{color_29791}◦GetRequestrichiesta base che un manager fa ad un agent per sapere quanto vale un }
\put(95.8,-320.011){\fontsize{12}{1}\usefont{T1}{cmr}{m}{n}\selectfont\color{color_29791}OID; richiede il valore associato al managed object}
\put(77.8,-340.811){\fontsize{12}{1}\usefont{T1}{cmr}{m}{n}\selectfont\color{color_29791}◦SetRequest richiede di settare il valore associato ad un managed object (es. se sposto }
\put(95.8,-354.611){\fontsize{12}{1}\usefont{T1}{cmr}{m}{n}\selectfont\color{color_29791}apparato da secondo a terzo piano, via SNMP vado a modificare la proprietà }
\put(95.8,-368.411){\fontsize{12}{1}\usefont{T1}{cmr}{m}{n}\selectfont\color{color_29791}systemLocation)}
\put(77.8,-389.211){\fontsize{12}{1}\usefont{T1}{cmr}{m}{n}\selectfont\color{color_29791}◦GetNextRequestrichiede di scoprire qual è OID del managed object successivo a }
\put(95.8,-403.011){\fontsize{12}{1}\usefont{T1}{cmr}{m}{n}\selectfont\color{color_29791}quello specificato. Come faccio a sapere sul manager qual è identificativo di un oggetto }
\put(95.8,-416.811){\fontsize{12}{1}\usefont{T1}{cmr}{m}{n}\selectfont\color{color_29791}che agent ha creato dinamicamente (perché è stato istruito così)? Managed non può certo}
\put(95.8,-430.611){\fontsize{12}{1}\usefont{T1}{cmr}{m}{n}\selectfont\color{color_29791}chiedere tutto gli OID possibili. }
\put(95.8,-451.411){\fontsize{12}{1}\usefont{T1}{cmr}{m}{n}\selectfont\color{color_29791}▪Dettagli: Getnextrequest implementa la visita dell'albero: questa si può fare in }
\put(113.8,-465.211){\fontsize{12}{1}\usefont{T1}{cmr}{m}{n}\selectfont\color{color_29791}diversi modi. A partire dalla radice, si può sapere quali nodi sono attaccati; poi }
\put(113.8,-479.011){\fontsize{12}{1}\usefont{T1}{cmr}{m}{n}\selectfont\color{color_29791}scendo in un nodo e chiedo quali sono sotto questo; a questo punto posso }
\put(131.8,-499.811){\fontsize{12}{1}\usefont{T1}{cmr}{m}{n}\selectfont\color{color_29791}1.o restare al primo livello e chiedere gli altri attaccati alla radice, scoprendo }
\put(149.8,-513.611){\fontsize{12}{1}\usefont{T1}{cmr}{m}{n}\selectfont\color{color_29791}quali sono tutti i nodi del secondo prima di scendere e scoprire tutto il terzo;}
\put(131.8,-534.411){\fontsize{12}{1}\usefont{T1}{cmr}{m}{n}\selectfont\color{color_29791}2.oppure scendere al primo nodo di questo livello e continuare a scendere }
\put(149.8,-548.211){\fontsize{12}{1}\usefont{T1}{cmr}{m}{n}\selectfont\color{color_29791}finché non trovo una foglia, prima di risalire ed esplorare orizzontalmente}
\put(113.8,-569.011){\fontsize{12}{1}\usefont{T1}{cmr}{m}{n}\selectfont\color{color_29791}•GetNextRequest permette di dire: caro agent, so che gestisci OID con ident. }
\put(131.8,-582.811){\fontsize{12}{1}\usefont{T1}{cmr}{m}{n}\selectfont\color{color_29791}1.23.12.3; qual è il prossimo? L'agent (che gestisce il proprio MIB) sa qual è }
\put(131.8,-596.611){\fontsize{12}{1}\usefont{T1}{cmr}{m}{n}\selectfont\color{color_29791}l'oggetto successivo, e risponderà con qual è il prossimo nodo. A questo punto il }
\put(131.8,-610.411){\fontsize{12}{1}\usefont{T1}{cmr}{m}{n}\selectfont\color{color_29791}manager può porre la stessa richiesta al nodo che gli è stato dato come risposta, }
\put(131.8,-624.211){\fontsize{12}{1}\usefont{T1}{cmr}{m}{n}\selectfont\color{color_29791}scendendo in profondità o meno; finché non esiste più un next: allora manager }
\put(131.8,-638.011){\fontsize{12}{1}\usefont{T1}{cmr}{m}{n}\selectfont\color{color_29791}avrà quindi scoperto tutti i nodi che fanno parte dell'albero.}
\put(77.8,-658.811){\fontsize{12}{1}\usefont{T1}{cmr}{m}{n}\selectfont\color{color_29791}◦GetbulkRequest  }
\end{picture}
\begin{tikzpicture}[overlay]
\path(0pt,0pt);
\draw[color_29791,line width=0.7pt]
(95.7pt, -659.911pt) -- (179.1pt, -659.911pt)
;
\end{tikzpicture}
\begin{picture}(-5,0)(2.5,0)
\put(202.2,-658.811){\fontsize{12}{1}\usefont{T1}{cmr}{m}{n}\selectfont\color{color_29791}da v2 in poi è versione ottimizzata di GetNextRequest: si evita di }
\put(95.8,-672.611){\fontsize{12}{1}\usefont{T1}{cmr}{m}{n}\selectfont\color{color_29791}avere spedizione pacchetto -> elaborazione -> pacchetto di ritorno; con getbulk si }
\put(95.8,-686.411){\fontsize{12}{1}\usefont{T1}{cmr}{m}{n}\selectfont\color{color_29791}richiede di recuperare tutti}
\end{picture}
\begin{tikzpicture}[overlay]
\path(0pt,0pt);
\draw[color_29791,line width=0.7pt]
(203.4pt, -687.511pt) -- (222.7pt, -687.511pt)
;
\end{tikzpicture}
\begin{picture}(-5,0)(2.5,0)
\put(222.8,-686.411){\fontsize{12}{1}\usefont{T1}{cmr}{m}{n}\selectfont\color{color_29791} gli oggetti successivi a quello indicato, fino a riempire il }
\put(95.8,-700.211){\fontsize{12}{1}\usefont{T1}{cmr}{m}{n}\selectfont\color{color_29791}pacchetto UDP.}
\put(59.8,-721.011){\fontsize{12}{1}\usefont{T1}{cmr}{m}{n}\selectfont\color{color_29791}•Poi ci sono le 4 risposte possibili dall'agent al manager:}
\put(77.8,-741.811){\fontsize{12}{1}\usefont{T1}{cmr}{m}{n}\selectfont\color{color_29791}◦Trapnotifica asincrona dall'agent al manager}
\end{picture}
\newpage
\begin{tikzpicture}[overlay]\path(0pt,0pt);\end{tikzpicture}
\begin{picture}(-5,0)(2.5,0)
\put(77.8,-85.01099){\fontsize{12}{1}\usefont{T1}{cmr}{m}{n}\selectfont\color{color_29791}◦InformRequest  }
\end{picture}
\begin{tikzpicture}[overlay]
\path(0pt,0pt);
\draw[color_29791,line width=0.7pt]
(95.7pt, -86.11096pt) -- (173.7pt, -86.11096pt)
;
\end{tikzpicture}
\begin{picture}(-5,0)(2.5,0)
\put(202.2,-85.01099){\fontsize{12}{1}\usefont{T1}{cmr}{m}{n}\selectfont\color{color_29791}da v2, come sopra ma con conferma di ricezione: richiedere }
\put(95.8,-98.81097){\fontsize{12}{1}\usefont{T1}{cmr}{m}{n}\selectfont\color{color_29791}risposta di conferma non è aggiunta da poco, specie perché SNMP è su UDP (non ci si }
\put(95.8,-112.611){\fontsize{12}{1}\usefont{T1}{cmr}{m}{n}\selectfont\color{color_29791}accorge di perdita pacchetti). Ma non avrebbe senso pensare di tirare su connessione }
\put(95.8,-126.411){\fontsize{12}{1}\usefont{T1}{cmr}{m}{n}\selectfont\color{color_29791}TCP: in generale, ogni volta che abbiamo protocollo req-resp con dati piccoli non ha }
\put(95.8,-140.211){\fontsize{12}{1}\usefont{T1}{cmr}{m}{n}\selectfont\color{color_29791}senso avere overhead degli handshake.}
\put(77.8,-161.011){\fontsize{12}{1}\usefont{T1}{cmr}{m}{n}\selectfont\color{color_29791}◦Response è usata dall'agent per rispondere a Request (o dal manager per rispondere }
\put(95.8,-174.811){\fontsize{12}{1}\usefont{T1}{cmr}{m}{n}\selectfont\color{color_29791}a trap/inform)}
\put(77.8,-195.611){\fontsize{12}{1}\usefont{T1}{cmr}{m}{n}\selectfont\color{color_29791}◦Report  }
\end{picture}
\begin{tikzpicture}[overlay]
\path(0pt,0pt);
\draw[color_29791,line width=0.7pt]
(95.7pt, -196.711pt) -- (131.7pt, -196.711pt)
;
\end{tikzpicture}
\begin{picture}(-5,0)(2.5,0)
\put(166.7,-195.611){\fontsize{12}{1}\usefont{T1}{cmr}{m}{n}\selectfont\color{color_29791}solo v3, usata per comunicazione inter-engine, principalmente per }
\put(95.8,-209.411){\fontsize{12}{1}\usefont{T1}{cmr}{m}{n}\selectfont\color{color_29791}segnalare problemi con elaborazione messaggi ricevuti}
\put(41.8,-230.211){\fontsize{12}{1}\usefont{T1}{cmr}{m}{n}\selectfont\color{color_29791}I protocolli v1 e v2c offrono il concetto di community, ovvero un' etichetta per stabilire il livello di }
\put(41.8,-244.011){\fontsize{12}{1}\usefont{T1}{cmr}{m}{n}\selectfont\color{color_29791}fiducia tra manager e agent. Ci sono 3 livelli di autorizzazione: una che può solo leggere info (read-}
\put(41.8,-257.811){\fontsize{12}{1}\usefont{T1}{cmr}{m}{n}\selectfont\color{color_29791}only), una che può fare le set (read-write), una autorizzata a ricevere le trap (trap). Ogni livello è }
\put(41.8,-271.611){\fontsize{12}{1}\usefont{T1}{cmr}{m}{n}\selectfont\color{color_29791}identificato da una stringa che fa sia da nome che da password (es. se so che la read-write }
\put(41.8,-285.411){\fontsize{12}{1}\usefont{T1}{cmr}{m}{n}\selectfont\color{color_29791}community di un agent si chiama private, questo è tutto ciò che mi serve per avere accesso R/W al }
\put(41.8,-299.211){\fontsize{12}{1}\usefont{T1}{cmr}{m}{n}\selectfont\color{color_29791}suo MIB). Questo mette in evidenza una debolezza di sicurezza: se non usiamo versione su TLS di }
\put(41.8,-313.011){\fontsize{12}{1}\usefont{T1}{cmr}{m}{n}\selectfont\color{color_29791}SNMP, i pacchetti sono in chiaro; quindi qualcuno sulla rete che può vedere i pacchetti, può }
\put(41.8,-326.811){\fontsize{12}{1}\usefont{T1}{cmr}{m}{n}\selectfont\color{color_29791}scoprire la community per poi fare liberamente query SNMP. Soluzione per essere sicuri che }
\put(41.8,-340.611){\fontsize{12}{1}\usefont{T1}{cmr}{m}{n}\selectfont\color{color_29791}l'infrastruttura di gestione non sia attaccata è o usare le versioni cifrate, o raddoppiare l'infrastruttura}
\put(41.8,-354.411){\fontsize{12}{1}\usefont{T1}{cmr}{m}{n}\selectfont\color{color_29791}di rete per separare quella di gestione da quella operativa.}
\put(41.8,-375.211){\fontsize{12}{1}\usefont{T1}{cmr}{m}{n}\selectfont\color{color_29791}SNMPv1 è limitato a 32bit e ha gestione errori minimale; SNMPv2 aggiunge:}
\put(59.8,-396.011){\fontsize{12}{1}\usefont{T1}{cmr}{m}{n}\selectfont\color{color_29791}•nuovi data type a 64bit}
\put(59.8,-416.811){\fontsize{12}{1}\usefont{T1}{cmr}{m}{n}\selectfont\color{color_29791}•comandi PDU GetBulk e Inform (per gestione interi sottoalberi e trap con risposta)}
\put(59.8,-437.611){\fontsize{12}{1}\usefont{T1}{cmr}{m}{n}\selectfont\color{color_29791}•party-based security system, molto macchinoso e di scarsa adozione, tanto da portare alla }
\put(77.8,-451.411){\fontsize{12}{1}\usefont{T1}{cmr}{m}{n}\selectfont\color{color_29791}creazione di due sotto-versioni:}
\put(77.8,-472.211){\fontsize{12}{1}\usefont{T1}{cmr}{m}{n}\selectfont\color{color_29791}◦SNMPv2c (c per community, è v2 senza party-based security, quindi stessa sicurezza di }
\put(95.8,-486.011){\fontsize{12}{1}\usefont{T1}{cmr}{m}{n}\selectfont\color{color_29791}v1)}
\put(77.8,-506.811){\fontsize{12}{1}\usefont{T1}{cmr}{m}{n}\selectfont\color{color_29791}◦SNMPv2u (user based, tentativo di migliorare la sicurezza rispetto a v1 senza }
\put(95.8,-520.611){\fontsize{12}{1}\usefont{T1}{cmr}{m}{n}\selectfont\color{color_29791}complicarlo, poi usato anche in v3).}
\put(41.8,-541.411){\fontsize{12}{1}\usefont{T1}{cmr}{m}{n}\selectfont\color{color_29791}SNMPv3 è una vera rivoluzione dello standard, in cui vecchi comportamenti sono ridefiniti con }
\put(41.8,-555.211){\fontsize{12}{1}\usefont{T1}{cmr}{m}{n}\selectfont\color{color_29791}termini totalmente diversi (niente più "manager" e "agent", unificati come entities). Le entity sono}
\put(59.8,-576.011){\fontsize{12}{1}\usefont{T1}{cmr}{m}{n}\selectfont\color{color_29791}•un engine: componente software dedicato a invio/ricezione messaggi, estrarre i dati e gestire}
\put(77.8,-589.811){\fontsize{12}{1}\usefont{T1}{cmr}{m}{n}\selectfont\color{color_29791}la sicurezza}
\put(59.8,-610.611){\fontsize{12}{1}\usefont{T1}{cmr}{m}{n}\selectfont\color{color_29791}•più application: implementano algoritmo che permette di confezionare la risposta giusta}
\put(41.8,-631.411){\fontsize{12}{1}\usefont{T1}{cmr}{m}{n}\selectfont\color{color_29791}Dal punto di vista della sicurezza, per gli utenti si segue il modello User-based Security Model }
\put(41.8,-645.211){\fontsize{12}{1}\usefont{T1}{cmr}{m}{n}\selectfont\color{color_29791}(USM, in cui gli utenti sono autenticati con password protette da HMAC e il canale è cifrato). Per le}
\put(41.8,-659.011){\fontsize{12}{1}\usefont{T1}{cmr}{m}{n}\selectfont\color{color_29791}informazioni il modello View-based Access Control Model (VACM) è a metà strada tra un DB e un }
\put(41.8,-672.811){\fontsize{12}{1}\usefont{T1}{cmr}{m}{n}\selectfont\color{color_29791}sistema operativo: gli utenti sono mappati in gruppi (che permette gestione in modo semplice, }
\put(41.8,-686.611){\fontsize{12}{1}\usefont{T1}{cmr}{m}{n}\selectfont\color{color_29791}aggiungendo nuovi utenti a gruppi gestiti) e le porzioni di MIB sono dette viste, permettendo la }
\put(41.8,-700.411){\fontsize{12}{1}\usefont{T1}{cmr}{m}{n}\selectfont\color{color_29791}definizione di matrici di controllo accessi: queste possono dire che l'intero sottoalbero che si genera }
\put(41.8,-714.211){\fontsize{12}{1}\usefont{T1}{cmr}{m}{n}\selectfont\color{color_29791}in un dato nodo è gestibile da un dato gruppo (sia in lettura che scrittura).}
\put(41.8,-735.011){\fontsize{12}{1}\usefont{T1}{cmr}{m}{n}\selectfont\color{color_29791}Comandi SNMP e info sui file di config (extra info):}
\put(59.8,-755.811){\fontsize{12}{1}\usefont{T1}{cmr}{m}{n}\selectfont\color{color_29791}•Lato manager}
\end{picture}
\newpage
\begin{tikzpicture}[overlay]\path(0pt,0pt);\end{tikzpicture}
\begin{picture}(-5,0)(2.5,0)
\put(77.8,-85.01099){\fontsize{12}{1}\usefont{T1}{cmr}{m}{n}\selectfont\color{color_29791}◦snmpgetrecupero di un solo oggetto}
\put(95.8,-105.811){\fontsize{12}{1}\usefont{T1}{cmr}{m}{n}\selectfont\color{color_29791}▪es. snmpget -v 1 -c public 192.168.56.203 .1.3.6.1.2.1.1.6.0}
\put(77.8,-126.611){\fontsize{12}{1}\usefont{T1}{cmr}{m}{n}\selectfont\color{color_29791}◦snmpsetimpostazione del valore di un oggetto}
\put(95.8,-147.411){\fontsize{12}{1}\usefont{T1}{cmr}{m}{n}\selectfont\color{color_29791}▪es.snmpset -v 1 -c supercom 192.168.56.203 .1.3.6.1.2.1.1.6.0 s "proprio qui"}
\put(113.8,-168.211){\fontsize{12}{1}\usefont{T1}{cmr}{m}{n}\selectfont\color{color_29791}•s indica stringa da inserire nella proprietà, altri type possibili sono}
\put(113.8,-189.011){\fontsize{12}{1}\usefont{T1}{cmr}{m}{n}\selectfont\color{color_29791}•i integer, u unsigned, a ipaddress…vedi man snmpset}
\put(77.8,-209.811){\fontsize{12}{1}\usefont{T1}{cmr}{m}{n}\selectfont\color{color_29791}◦snmpwalkusa ricorsivamente la PDU getNext per navigare un intero sottoalbero del }
\put(95.8,-223.611){\fontsize{12}{1}\usefont{T1}{cmr}{m}{n}\selectfont\color{color_29791}MIB}
\put(95.8,-244.411){\fontsize{12}{1}\usefont{T1}{cmr}{m}{n}\selectfont\color{color_144481}▪Fare walk di un OID per scalare (es. .1.3.6.1.2.1.1.6.0 sysLocation) ha lo stesso }
\put(113.8,-258.211){\fontsize{12}{1}\usefont{T1}{cmr}{m}{n}\selectfont\color{color_144481}risultato di una get, MA costa molti più pacchetti (agent fa esplorazione dell'albero, }
\put(113.8,-272.011){\fontsize{12}{1}\usefont{T1}{cmr}{m}{n}\selectfont\color{color_144481}invece di richiedere direttamente proprietà che ci interessa). Importante non usare }
\put(113.8,-285.811){\fontsize{12}{1}\usefont{T1}{cmr}{m}{n}\selectfont\color{color_144481}una walk quando basta una get}
\put(77.8,-306.611){\fontsize{12}{1}\usefont{T1}{cmr}{m}{n}\selectfont\color{color_29791}◦Questi comandi hanno molti parametri in comune, le man page di ognuno documentano }
\put(95.8,-320.411){\fontsize{12}{1}\usefont{T1}{cmr}{m}{n}\selectfont\color{color_29791}solo le opzioni specifiche. man snmpcmd documenta quelle comuni, essenziali in ogni }
\put(95.8,-334.211){\fontsize{12}{1}\usefont{T1}{cmr}{m}{n}\selectfont\color{color_29791}invocazione: }
\put(95.8,-355.011){\fontsize{12}{1}\usefont{T1}{cmr}{m}{n}\selectfont\color{color_29791}▪-vversione}
\put(95.8,-375.811){\fontsize{12}{1}\usefont{T1}{cmr}{m}{n}\selectfont\color{color_29791}▪-ccommunity}
\put(95.8,-396.611){\fontsize{12}{1}\usefont{T1}{cmr}{m}{n}\selectfont\color{color_29791}▪indirizzo del network element}
\put(95.8,-417.411){\fontsize{12}{1}\usefont{T1}{cmr}{m}{n}\selectfont\color{color_29791}▪OID del managed object}
\put(95.8,-438.211){\fontsize{12}{1}\usefont{T1}{cmr}{m}{n}\selectfont\color{color_29791}▪Quindi linea base è snmpXXX -v 1 -c <nomeCommunity> <IPelement> <OID>}
\put(113.8,-459.011){\fontsize{12}{1}\usefont{T1}{cmr}{m}{n}\selectfont\color{color_29791}•Per walk e get basta community read-only}
\put(95.8,-479.811){\fontsize{12}{1}\usefont{T1}{cmr}{m}{n}\selectfont\color{color_29791}▪Nel man ci sono opzioni con -O utili per formattare output:}
\put(113.8,-500.611){\fontsize{12}{1}\usefont{T1}{cmr}{m}{n}\selectfont\color{color_29791}•Opzione -On mostra gli OID in output in formato numerico (meno leggibile ma }
\put(131.8,-514.411){\fontsize{12}{1}\usefont{T1}{cmr}{m}{n}\selectfont\color{color_29791}più processabile) invece che testuale}
\put(77.8,-535.211){\fontsize{12}{1}\usefont{T1}{cmr}{m}{n}\selectfont\color{color_144481}◦Il file etc/snmp/snmp.conf contiene alcuni parametri per configurare il comportamento di}
\put(95.8,-549.011){\fontsize{12}{1}\usefont{T1}{cmr}{m}{n}\selectfont\color{color_144481}default dei tool manager}
\put(95.8,-569.811){\fontsize{12}{1}\usefont{T1}{cmr}{m}{n}\selectfont\color{color_144481}▪commentando la riga "mibs : " abilitiamo l'uso dei MIBs (essendo questi disabilitati }
\put(113.8,-583.611){\fontsize{12}{1}\usefont{T1}{cmr}{m}{n}\selectfont\color{color_144481}di default per questioni di licenza)}
\put(59.8,-604.411){\fontsize{12}{1}\usefont{T1}{cmr}{m}{n}\selectfont\color{color_29791}•Lato agent}
\put(77.8,-625.211){\fontsize{12}{1}\usefont{T1}{cmr}{m}{n}\selectfont\color{color_29791}◦Gli agent possono essere in firmware, ma noi vedremo come funziona un agent software }
\put(95.8,-639.011){\fontsize{12}{1}\usefont{T1}{cmr}{m}{n}\selectfont\color{color_29791}(che gira all'intero del SO). In questo caso l'agent è il demone snmpd}
\put(77.8,-659.811){\fontsize{12}{1}\usefont{T1}{cmr}{m}{n}\selectfont\color{color_29791}◦systemctl start snmpd lancia il demone secondo la configurazione attuale}
\put(77.8,-680.611){\fontsize{12}{1}\usefont{T1}{cmr}{m}{n}\selectfont\color{color_144481}◦Il file etc/snmp/snmpd.conf configura l'agent:}
\put(95.8,-701.411){\fontsize{12}{1}\usefont{T1}{cmr}{m}{n}\selectfont\color{color_144481}▪sostituendo agentAddress udp:127.0.0.1:161 con agentAddress udp:161 poniamo }
\put(113.8,-715.211){\fontsize{12}{1}\usefont{T1}{cmr}{m}{n}\selectfont\color{color_144481}agent in ascolto su porta 161 per qualsiasi indirizzo (?) (è l'indirizzo su cui si }
\put(113.8,-729.011){\fontsize{12}{1}\usefont{T1}{cmr}{m}{n}\selectfont\color{color_144481}mette in ascolto l'agent (può essere elenco di indirizzi) agentaddress 127.0.0.1,[::1] }
\put(113.8,-742.811){\fontsize{12}{1}\usefont{T1}{cmr}{m}{n}\selectfont\color{color_144481}indica sia IPv4 che IPv6 che l'agent ascolta su localhost. A default agent ascolta }
\put(113.8,-756.611){\fontsize{12}{1}\usefont{T1}{cmr}{m}{n}\selectfont\color{color_144481}traffico su porta 161 da qualsiasi interfaccia).}
\end{picture}
\newpage
\begin{tikzpicture}[overlay]\path(0pt,0pt);\end{tikzpicture}
\begin{picture}(-5,0)(2.5,0)
\put(95.8,-85.01099){\fontsize{12}{1}\usefont{T1}{cmr}{m}{n}\selectfont\color{color_144481}▪direttive rocommunity public e rwcommunity supercom definiscono 2 community }
\put(113.8,-98.81097){\fontsize{12}{1}\usefont{T1}{cmr}{m}{n}\selectfont\color{color_144481}separate, una per la lettura e una per la scrittura: comm di lettura si chiama public, }
\put(113.8,-112.611){\fontsize{12}{1}\usefont{T1}{cmr}{m}{n}\selectfont\color{color_144481}comm di scrittura si chiama supercom}
\put(95.8,-133.411){\fontsize{12}{1}\usefont{T1}{cmr}{m}{n}\selectfont\color{color_144481}▪può settare alcuni managed object di sistema (es. sysLocation, sysContact): questi }
\end{picture}
\begin{tikzpicture}[overlay]
\path(0pt,0pt);
\draw[color_144481,line width=0.7pt]
(477.7pt, -134.511pt) -- (511.4pt, -134.511pt)
;
\end{tikzpicture}
\begin{picture}(-5,0)(2.5,0)
\put(113.8,-147.211){\fontsize{12}{1}\usefont{T1}{cmr}{b}{n}\selectfont\color{color_144481}oggetti sono RW SOLO SE non sono settati nel file di config! Altrimenti read-}
\end{picture}
\begin{tikzpicture}[overlay]
\path(0pt,0pt);
\draw[color_144481,line width=0.7pt]
(113.8pt, -148.311pt) -- (509.9pt, -148.311pt)
;
\end{tikzpicture}
\begin{picture}(-5,0)(2.5,0)
\put(113.8,-161.011){\fontsize{12}{1}\usefont{T1}{cmr}{b}{n}\selectfont\color{color_144481}only.}
\end{picture}
\begin{tikzpicture}[overlay]
\path(0pt,0pt);
\draw[color_144481,line width=0.7pt]
(113.8pt, -162.111pt) -- (138.1pt, -162.111pt)
;
\end{tikzpicture}
\begin{picture}(-5,0)(2.5,0)
\put(95.8,-181.811){\fontsize{12}{1}\usefont{T1}{cmr}{m}{n}\selectfont\color{color_29791}▪È possibile inserire direttive di monitoraggio dei parametri base (carico macchina, }
\put(113.8,-195.611){\fontsize{12}{1}\usefont{T1}{cmr}{m}{n}\selectfont\color{color_29791}uso dei dischi, presenza di processi), estensione UCD-SNMPdoc athttp://net-}
\put(113.8,-209.411){\fontsize{12}{1}\usefont{T1}{cmr}{m}{n}\selectfont\color{color_29791}snmp.sourceforge.net/docs/mibs/ucdavis.html}
\put(113.8,-230.211){\fontsize{12}{1}\usefont{T1}{cmr}{m}{n}\selectfont\color{color_29791}•load [max-1] [max-5] [max-15]}
\put(131.8,-251.011){\fontsize{12}{1}\usefont{T1}{cmr}{m}{n}\selectfont\color{color_29791}◦tabella .1.3.6.1.4.1.2021.10 }
\put(131.8,-271.811){\fontsize{12}{1}\usefont{T1}{cmr}{m}{n}\selectfont\color{color_29791}◦i tre valori indicano il massimo tollerabile negli ultimi 1-5-15 min: tre righe }
\put(149.8,-285.611){\fontsize{12}{1}\usefont{T1}{cmr}{m}{n}\selectfont\color{color_29791}(carico negli ultimi 1-5-15- min); le colonne hanno il carico effettivo, e il flag}
\put(149.8,-299.411){\fontsize{12}{1}\usefont{T1}{cmr}{m}{n}\selectfont\color{color_29791}di superamento delle rispettive soglie.}
\put(149.8,-320.211){\fontsize{12}{1}\usefont{T1}{cmr}{m}{n}\selectfont\color{color_29791}▪scrivendo load 0.1nel file conf e riavviando il demone, all'interno }
\put(167.8,-334.011){\fontsize{12}{1}\usefont{T1}{cmr}{m}{n}\selectfont\color{color_29791}della tabella indicata ( .1.3.6.1.4.1.2021.10  )  dovrebbe comparire lo stato}
\put(167.8,-347.811){\fontsize{12}{1}\usefont{T1}{cmr}{m}{n}\selectfont\color{color_29791}del carico di sistema (si indica che il limite è 0 nell'ultimo minuto). }
\put(149.8,-368.611){\fontsize{12}{1}\usefont{T1}{cmr}{m}{n}\selectfont\color{color_29791}▪walk su OID della tabella(  snmpwalk -v 1 -c public }
\put(167.8,-382.411){\fontsize{12}{1}\usefont{T1}{cmr}{b}{it}\selectfont\color{color_29791}192.168.56.203 .1.3.6.1.4.1.2021.10 | less  ) restituisce tutti gli }
\put(167.8,-396.211){\fontsize{12}{1}\usefont{T1}{cmr}{m}{n}\selectfont\color{color_29791}elementi come lista unidimensionale, come se fosse una tabella }
\put(167.8,-410.011){\fontsize{12}{1}\usefont{T1}{cmr}{m}{n}\selectfont\color{color_29791}serializzata. Vediamo output con OID che si ripetono, e non finiscono in }
\put(167.8,-423.811){\fontsize{12}{1}\usefont{T1}{cmr}{m}{n}\selectfont\color{color_29791}0: i nomi prima degli OID numerici sono i nomi delle colonne.}
\put(149.8,-444.611){\fontsize{12}{1}\usefont{T1}{cmr}{m}{n}\selectfont\color{color_29791}▪la prima colonna contiene il numero delle righe (sono 3 perché load offre }
\put(167.8,-458.411){\fontsize{12}{1}\usefont{T1}{cmr}{m}{n}\selectfont\color{color_29791}3 risultati, una riga per 1, una riga per 5, una riga per 15)}
\put(167.8,-479.211){\fontsize{12}{1}\usefont{T1}{cmr}{m}{it}\selectfont\color{color_29791}UCD-SNMP-MIB::laErrorFlag.1 = INTEGER: error(1)}
\put(149.8,-500.011){\fontsize{12}{1}\usefont{T1}{cmr}{m}{n}\selectfont\color{color_29791}▪flag di errore viene settato perché abbiamo superato la soglia di 0.1 }
\put(167.8,-513.811){\fontsize{12}{1}\usefont{T1}{cmr}{m}{n}\selectfont\color{color_29791}nell'ultimo minuto (carico simulato con find / )}
\put(149.8,-534.611){\fontsize{12}{1}\usefont{T1}{cmr}{m}{n}\selectfont\color{color_29791}▪Leggere il flag permette di usare una definizione più astratta (il manager }
\put(167.8,-548.411){\fontsize{12}{1}\usefont{T1}{cmr}{m}{n}\selectfont\color{color_29791}non deve sapere quanto è il limite)}
\put(149.8,-569.211){\fontsize{12}{1}\usefont{T1}{cmr}{m}{n}\selectfont\color{color_29791}▪per usi in script, utile OID  }
\end{picture}
\begin{tikzpicture}[overlay]
\path(0pt,0pt);
\draw[color_29791,line width=0.7pt]
(167.7pt, -570.311pt) -- (293.7pt, -570.311pt)
;
\end{tikzpicture}
\begin{picture}(-5,0)(2.5,0)
\put(293.7,-569.211){\fontsize{12}{1}\usefont{T1}{cmr}{m}{n}\selectfont\color{color_29791}       }
\end{picture}
\begin{tikzpicture}[overlay]
\path(0pt,0pt);
\draw[color_29791,line width=0.7pt]
(293.7pt, -570.311pt) -- (309.5pt, -570.311pt)
;
\end{tikzpicture}
\begin{picture}(-5,0)(2.5,0)
\put(309.5,-569.211){\fontsize{12}{1}\usefont{T1}{cmr}{m}{n}\selectfont\color{color_29791}             }
\end{picture}
\begin{tikzpicture}[overlay]
\path(0pt,0pt);
\draw[color_29791,line width=0.7pt]
(309.5pt, -570.311pt) -- (345pt, -570.311pt)
;
\end{tikzpicture}
\begin{picture}(-5,0)(2.5,0)
\put(345.1,-569.211){\fontsize{12}{1}\usefont{T1}{cmr}{m}{it}\selectfont\color{color_29791}UCD-SNMP-MIB::laLoadInt.1 =   }
\end{picture}
\begin{tikzpicture}[overlay]
\path(0pt,0pt);
\draw[color_29791,line width=0.7pt]
(345pt, -570.311pt) -- (509.5pt, -570.311pt)
;
\end{tikzpicture}
\begin{picture}(-5,0)(2.5,0)
\put(167.8,-583.011){\fontsize{12}{1}\usefont{T1}{cmr}{m}{it}\selectfont\color{color_29791}INTEGER: 30  }
\end{picture}
\begin{tikzpicture}[overlay]
\path(0pt,0pt);
\draw[color_29791,line width=0.7pt]
(167.7pt, -584.111pt) -- (236.1pt, -584.111pt)
;
\end{tikzpicture}
\begin{picture}(-5,0)(2.5,0)
\put(238.6,-583.011){\fontsize{12}{1}\usefont{T1}{cmr}{m}{n}\selectfont\color{color_29791}             }
\end{picture}
\begin{tikzpicture}[overlay]
\path(0pt,0pt);
\draw[color_29791,line width=0.7pt]
(238.6pt, -584.111pt) -- (274.1pt, -584.111pt)
;
\end{tikzpicture}
\begin{picture}(-5,0)(2.5,0)
\put(274.2,-583.011){\fontsize{12}{1}\usefont{T1}{cmr}{m}{n}\selectfont\color{color_29791}che fornisce intero già moltiplicato per 100, ci   }
\end{picture}
\begin{tikzpicture}[overlay]
\path(0pt,0pt);
\draw[color_29791,line width=0.7pt]
(274.1pt, -584.111pt) -- (499pt, -584.111pt)
;
\end{tikzpicture}
\begin{picture}(-5,0)(2.5,0)
\put(167.8,-596.811){\fontsize{12}{1}\usefont{T1}{cmr}{m}{n}\selectfont\color{color_29791}toglie di mezzo il .}
\end{picture}
\begin{tikzpicture}[overlay]
\path(0pt,0pt);
\draw[color_29791,line width=0.7pt]
(167.8pt, -597.911pt) -- (257.4pt, -597.911pt)
;
\end{tikzpicture}
\begin{picture}(-5,0)(2.5,0)
\put(113.8,-617.611){\fontsize{12}{1}\usefont{T1}{cmr}{m}{n}\selectfont\color{color_29791}•disk [partizione] [minfree|minfree\%]}
\put(131.8,-638.411){\fontsize{12}{1}\usefont{T1}{cmr}{m}{n}\selectfont\color{color_29791}◦tabella .1.3.6.1.4.1.2021.9}
\put(131.8,-659.211){\fontsize{12}{1}\usefont{T1}{cmr}{m}{n}\selectfont\color{color_29791}◦una riga per ogni partizione messa sotto controllo da una direttiva disk}
\put(131.8,-680.011){\fontsize{12}{1}\usefont{T1}{cmr}{m}{n}\selectfont\color{color_29791}◦colonne hanno tutti i dettagli della partizione e flag di spazio sotto il minimo}
\put(113.8,-700.811){\fontsize{12}{1}\usefont{T1}{cmr}{m}{n}\selectfont\color{color_29791}•proc [nomeprocesso] [maxnum [minnum]]}
\put(131.8,-721.611){\fontsize{12}{1}\usefont{T1}{cmr}{m}{n}\selectfont\color{color_29791}◦tabella.1.3.6.1.4.1.2021.2}
\put(131.8,-742.411){\fontsize{12}{1}\usefont{T1}{cmr}{m}{n}\selectfont\color{color_29791}◦una riga per ogni processo messo sotto controllo da una direttiva proc }
\put(131.8,-763.211){\fontsize{12}{1}\usefont{T1}{cmr}{m}{n}\selectfont\color{color_29791}◦colonne hanno numero di istanze, flag di superamento delle soglie}
\end{picture}
\newpage
\begin{tikzpicture}[overlay]\path(0pt,0pt);\end{tikzpicture}
\begin{picture}(-5,0)(2.5,0)
\put(149.8,-85.01099){\fontsize{12}{1}\usefont{T1}{cmr}{m}{n}\selectfont\color{color_29791}▪non specificando minimo e massimo si hanno default di 1 – illimitato}
\put(149.8,-105.811){\fontsize{12}{1}\usefont{T1}{cmr}{m}{n}\selectfont\color{color_144481}▪es.aggiungiamo proc sshd al file di configurazione: }
\put(167.8,-119.611){\fontsize{12}{1}\usefont{T1}{cmr}{m}{n}\selectfont\color{color_144481}facendo walk della tabella indicata, si ha una tabella con una sola riga }
\put(167.8,-133.411){\fontsize{12}{1}\usefont{T1}{cmr}{m}{n}\selectfont\color{color_144481}(perché abbiamo messo sotto controllo un solo processo). }
\put(167.8,-154.211){\fontsize{12}{1}\usefont{T1}{cmr}{m}{n}\selectfont\color{color_29791}•UCD-SNMP-MIB::prCount.1 = INTEGER: 3è il }
\put(185.8,-168.011){\fontsize{12}{1}\usefont{T1}{cmr}{m}{n}\selectfont\color{color_144481}numero di processi sshd running, flag errore non è settato essendo }
\put(185.8,-181.811){\fontsize{12}{1}\usefont{T1}{cmr}{m}{n}\selectfont\color{color_144481}sotto la soglia max (non avendola specificata, è illimitata)}
\put(167.8,-202.611){\fontsize{12}{1}\usefont{T1}{cmr}{m}{n}\selectfont\color{color_144481}•POI se aggiungiamo riga proc atd e modifichimo la prima }
\put(185.8,-216.411){\fontsize{12}{1}\usefont{T1}{cmr}{m}{n}\selectfont\color{color_144481}in proc ssh 2 1la tabella diventa di due righe (una per }
\put(185.8,-230.211){\fontsize{12}{1}\usefont{T1}{cmr}{m}{n}\selectfont\color{color_144481}ogni processo) e otteniamo flag settatoUCD-SNMP-}
\put(185.8,-244.011){\fontsize{12}{1}\usefont{T1}{cmr}{m}{it}\selectfont\color{color_144481}MIB::prErrMessage.1 = STRING: Too many sshd running (\# = 3) }
\put(185.8,-257.811){\fontsize{12}{1}\usefont{T1}{cmr}{m}{n}\selectfont\color{color_144481}perché minimo è 1 e massimo è 2}
\put(95.8,-278.611){\fontsize{12}{1}\usefont{T1}{cmr}{m}{n}\selectfont\color{color_29791}▪È possibile inserire direttive per eseguire codice il cui output è reso accessibile come }
\put(113.8,-292.411){\fontsize{12}{1}\usefont{T1}{cmr}{m}{n}\selectfont\color{color_29791}managed object. Estensione NET-EXTEND:doc at}
\put(149.3,-306.211){\fontsize{12}{1}\usefont{T1}{cmr}{m}{n}\selectfont\color{color_29791}http://www.oidview.com/mibs/8072/NET-SNMP-EXTEND-MIB.html}
\put(113.8,-327.011){\fontsize{12}{1}\usefont{T1}{cmr}{m}{n}\selectfont\color{color_29791}•tabella NET-SNMP-EXTEND-MIB::.1.3.6.1.4.1.8072}
\put(131.8,-347.811){\fontsize{12}{1}\usefont{T1}{cmr}{m}{n}\selectfont\color{color_29791}◦snmpwalk -v 1 -c public 192.168.56.203 .1.3.6.1.4.1.8072 | less per }
\put(149.8,-361.611){\fontsize{12}{1}\usefont{T1}{cmr}{m}{n}\selectfont\color{color_29791}visualizzare intera tabella}
\put(113.8,-382.411){\fontsize{12}{1}\usefont{T1}{cmr}{m}{n}\selectfont\color{color_29791}•righe con nome = etichetta della direttiva extend-shetichetta di una riga }
\put(131.8,-396.211){\fontsize{12}{1}\usefont{T1}{cmr}{m}{n}\selectfont\color{color_29791}sarà il nome simbolico che ho dato alla mia direttiva extend}
\put(113.8,-417.011){\fontsize{12}{1}\usefont{T1}{cmr}{m}{n}\selectfont\color{color_29791}•diverse colonne, la più comune è nsExtendOutputFullmostra l'output}
\put(131.8,-430.811){\fontsize{12}{1}\usefont{T1}{cmr}{m}{n}\selectfont\color{color_29791}intero dello script associato alla direttiva}
\put(131.8,-451.611){\fontsize{12}{1}\usefont{T1}{cmr}{m}{n}\selectfont\color{color_29791}◦snmpwalk -v 1 -c public 192.168.56.203 .1.3.6.1.4.1.8072 | grep }
\put(149.8,-465.411){\fontsize{12}{1}\usefont{T1}{cmr}{b}{it}\selectfont\color{color_29791}"nsExtendOutputFull" | less}
\put(113.8,-486.211){\fontsize{12}{1}\usefont{T1}{cmr}{m}{n}\selectfont\color{color_29791}•formato sarà extend-sh <etichetta> <riga di comando>}
\put(167.3,-500.011){\fontsize{12}{1}\usefont{T1}{cmr}{m}{n}\selectfont\color{color_29791}(ovviamente riga di comando potrà avere pipe e quant'altro)}
\put(113.8,-520.811){\fontsize{12}{1}\usefont{T1}{cmr}{m}{n}\selectfont\color{color_144481}•es.aggiungiamoextend-sh test1 echo HelloWorldal file }
\put(131.8,-534.611){\fontsize{12}{1}\usefont{T1}{cmr}{m}{n}\selectfont\color{color_144481}conf. OID corrispondente sarà NET-SNMP-EXTEND-}
\put(131.8,-548.411){\fontsize{12}{1}\usefont{T1}{cmr}{m}{it}\selectfont\color{color_144481}MIB::nsExtendOutputFull."test1"(notare i doppi apici, indicano }
\end{picture}
\begin{tikzpicture}[overlay]
\path(0pt,0pt);
\draw[color_144481,line width=0.7pt]
(344.5pt, -549.511pt) -- (501pt, -549.511pt)
;
\end{tikzpicture}
\begin{picture}(-5,0)(2.5,0)
\put(131.8,-562.211){\fontsize{12}{1}\usefont{T1}{cmr}{b}{n}\selectfont\color{color_144481}all'agent che è un nome da risolvere, non un segmento di OID standard – }
\end{picture}
\begin{tikzpicture}[overlay]
\path(0pt,0pt);
\draw[color_144481,line width=0.7pt]
(131.8pt, -563.311pt) -- (507.3pt, -563.311pt)
;
\end{tikzpicture}
\begin{picture}(-5,0)(2.5,0)
\put(131.8,-576.011){\fontsize{12}{1}\usefont{T1}{cmr}{b}{n}\selectfont\color{color_144481}attenzione all'espansione bash, devono arrivare al comando snmp!). }
\end{picture}
\begin{tikzpicture}[overlay]
\path(0pt,0pt);
\draw[color_144481,line width=0.7pt]
(131.8pt, -577.111pt) -- (480.9pt, -577.111pt)
;
\end{tikzpicture}
\begin{picture}(-5,0)(2.5,0)
\put(131.8,-589.811){\fontsize{12}{1}\usefont{T1}{cmr}{b}{n}\selectfont\color{color_144481}Importante usare path assoluto per i comandi inseriti}
\end{picture}
\begin{tikzpicture}[overlay]
\path(0pt,0pt);
\draw[color_144481,line width=0.7pt]
(131.8pt, -590.911pt) -- (404.4pt, -590.911pt)
;
\end{tikzpicture}
\begin{picture}(-5,0)(2.5,0)
\put(113.8,-610.611){\fontsize{12}{1}\usefont{T1}{cmr}{m}{n}\selectfont\color{color_144481}•Quando agent riceve una getRequest all'OID corrispondente, esegue il comando e}
\put(131.8,-624.411){\fontsize{12}{1}\usefont{T1}{cmr}{m}{n}\selectfont\color{color_144481}restituisce l'output nella Response}
\put(113.8,-645.211){\fontsize{12}{1}\usefont{T1}{cmr}{m}{n}\selectfont\color{color_29791}•Per eseguire comandi privilegiati  }
\end{picture}
\begin{tikzpicture}[overlay]
\path(0pt,0pt);
\draw[color_29791,line width=0.7pt]
(131.7pt, -646.311pt) -- (300.2pt, -646.311pt)
;
\end{tikzpicture}
\begin{picture}(-5,0)(2.5,0)
\put(300.4,-645.211){\fontsize{12}{1}\usefont{T1}{cmr}{m}{n}\selectfont\color{color_29791}, dovremo autorizzare utente dell'agent (nelle }
\put(131.8,-659.011){\fontsize{12}{1}\usefont{T1}{cmr}{m}{n}\selectfont\color{color_29791}VM 9CFU utente si chiama snmp) nel file sudoers, ad eseguire quello specifico }
\end{picture}
\begin{tikzpicture}[overlay]
\path(0pt,0pt);
\draw[color_29791,line width=0.7pt]
(436.4pt, -660.111pt) -- (515.7pt, -660.111pt)
;
\end{tikzpicture}
\begin{picture}(-5,0)(2.5,0)
\put(131.8,-672.811){\fontsize{12}{1}\usefont{T1}{cmr}{m}{n}\selectfont\color{color_29791}comando.}
\end{picture}
\begin{tikzpicture}[overlay]
\path(0pt,0pt);
\draw[color_29791,line width=0.7pt]
(131.8pt, -673.911pt) -- (178.8pt, -673.911pt)
;
\end{tikzpicture}
\begin{picture}(-5,0)(2.5,0)
\put(178.8,-672.811){\fontsize{12}{1}\usefont{T1}{cmr}{m}{n}\selectfont\color{color_29791} }
\put(131.8,-693.611){\fontsize{12}{1}\usefont{T1}{cmr}{m}{n}\selectfont\color{color_29791}◦Con visudo aggiungiamo al file /etc/sudoers riga }
\put(167.8,-714.411){\fontsize{12}{1}\usefont{T1}{cmr}{b}{it}\selectfont\color{color_29791}snmp ALL=NOPASSWD:/absolutePathComando -opzioni}
\put(131.8,-735.211){\fontsize{12}{1}\usefont{T1}{cmr}{m}{n}\selectfont\color{color_29791}◦nel file /etc/snmp/snmpd.conf, direttiva sarà}
\put(149.8,-756.011){\fontsize{12}{1}\usefont{T1}{cmr}{b}{it}\selectfont\color{color_29791}extend-sh <etichetta> /usr/bin/sudo <riga di comando>}
\end{picture}
\newpage
\begin{tikzpicture}[overlay]\path(0pt,0pt);\end{tikzpicture}
\begin{picture}(-5,0)(2.5,0)
\put(131.8,-85.01099){\fontsize{12}{1}\usefont{T1}{cmr}{m}{n}\selectfont\color{color_144481}◦es.sull'agent }
\put(149.8,-105.811){\fontsize{12}{1}\usefont{T1}{cmr}{m}{n}\selectfont\color{color_144481}▪file /etc/sudoerssnmp    ALL=NOPASSWD:/sbin/iptables -vnL}
\put(167.8,-126.611){\fontsize{12}{1}\usefont{T1}{cmr}{m}{n}\selectfont\color{color_144481}•si può testare con}
\put(185.8,-147.411){\fontsize{12}{1}\usefont{T1}{cmr}{m}{n}\selectfont\color{color_144481}◦\# las@Router:~\$ sudo -H -u snmp /bin/bash}
\put(203.8,-168.211){\fontsize{12}{1}\usefont{T1}{cmr}{m}{n}\selectfont\color{color_144481}\# snmp@Router:/home/las\$ sudo /sbin/iptables -vnL }
\put(149.8,-189.011){\fontsize{12}{1}\usefont{T1}{cmr}{m}{n}\selectfont\color{color_144481}▪file /etc/snmp/snmpd.conf extend-sh ipt /usr/bin/sudo /sbin/iptables -}
\put(167.8,-202.811){\fontsize{12}{1}\usefont{T1}{cmr}{m}{n}\selectfont\color{color_144481}vnL}
\put(41.8,-223.611){\fontsize{12}{1}\usefont{T1}{cmr}{b}{n}\selectfont\color{color_29791}DETTAGLI: SNMP non è molto comprensibile, è efficiente: risale a quando i byte avevano un alto}
\put(41.8,-237.411){\fontsize{12}{1}\usefont{T1}{cmr}{m}{n}\selectfont\color{color_29791}costo. Oggi nuovi standard, come YANG, sono indipendenti da protocolli di trasporto, offrendo }
\put(41.8,-251.211){\fontsize{12}{1}\usefont{T1}{cmr}{m}{n}\selectfont\color{color_29791}strutture dati definite in stile XML. Ci sono nuovi protocolli che introducono 2 grandi novità:}
\put(41.8,-272.011){\fontsize{12}{1}\usefont{T1}{cmr}{m}{n}\selectfont\color{color_29791}1) transazionali: modifiche da remoto garantendo o che sono riuscito a fare tutto quello che serve, o }
\put(41.8,-285.811){\fontsize{12}{1}\usefont{T1}{cmr}{m}{n}\selectfont\color{color_29791}che non è cambiato niente }
\put(41.8,-306.611){\fontsize{12}{1}\usefont{T1}{cmr}{m}{n}\selectfont\color{color_29791}2) non più comunicazioni C/S, ma si usano piattaforme di comunicazione per realizzare modelli }
\put(41.8,-320.411){\fontsize{12}{1}\usefont{T1}{cmr}{m}{n}\selectfont\color{color_29791}pub/sub; con maggiore flessibilità per chi vuole ricevere date notifiche (divise per argomento e }
\put(41.8,-334.211){\fontsize{12}{1}\usefont{T1}{cmr}{m}{n}\selectfont\color{color_29791}contenuto, non per topologia dell'apparato)}
\put(41.8,-370.211){\fontsize{17.5}{1}\usefont{T1}{cmr}{b}{n}\selectfont\color{color_29791}LDAP}
\put(41.8,-391.111){\fontsize{12}{1}\usefont{T1}{cmr}{m}{n}\selectfont\color{color_29791}Per configurare un sistema sono necessari diversi file di configurazione, e in una rete con molte }
\put(41.8,-404.911){\fontsize{12}{1}\usefont{T1}{cmr}{m}{n}\selectfont\color{color_29791}macchine sarebbe necessario cambiare centinaia di file per compiere coerentemente operazioni }
\put(41.8,-418.711){\fontsize{12}{1}\usefont{T1}{cmr}{m}{n}\selectfont\color{color_29791}come l'aggiunta di un utente, di un host; la modifica di un parametro di un demone. Da qui nasce }
\put(41.8,-432.511){\fontsize{12}{1}\usefont{T1}{cmr}{m}{n}\selectfont\color{color_29791}l'esigenza di condividere queste informazioni in maniera centralizzata superando l'approccio a file: }
\put(41.8,-446.311){\fontsize{12}{1}\usefont{T1}{cmr}{m}{n}\selectfont\color{color_29791}LDAP è una directory centralizzata dove si possono conservare informazioni che servono a tutti i }
\put(41.8,-460.111){\fontsize{12}{1}\usefont{T1}{cmr}{m}{n}\selectfont\color{color_29791}calcolatori della rete. (Abbiamo già visto Vagrant e Ansible per configuration as code, ma essendo }
\end{picture}
\begin{tikzpicture}[overlay]
\path(0pt,0pt);
\draw[color_29791,line width=0.7pt]
(145.8pt, -456.711pt) -- (515.8pt, -456.711pt)
;
\end{tikzpicture}
\begin{picture}(-5,0)(2.5,0)
\put(41.8,-473.911){\fontsize{12}{1}\usefont{T1}{cmr}{m}{n}\selectfont\color{color_29791}pensati per essere usati in momenti specifici del ciclo di vita di una VM -si pensi a boot e }
\end{picture}
\begin{tikzpicture}[overlay]
\path(0pt,0pt);
\draw[color_29791,line width=0.7pt]
(41.8pt, -470.511pt) -- (473.1pt, -470.511pt)
;
\end{tikzpicture}
\begin{picture}(-5,0)(2.5,0)
\put(41.8,-487.711){\fontsize{12}{1}\usefont{T1}{cmr}{m}{n}\selectfont\color{color_29791}provisioning- non tutti i parametri sono impostabili con questi: ad esempio risoluzione dei nomi e }
\end{picture}
\begin{tikzpicture}[overlay]
\path(0pt,0pt);
\draw[color_29791,line width=0.7pt]
(41.8pt, -484.311pt) -- (513.3pt, -484.311pt)
;
\end{tikzpicture}
\begin{picture}(-5,0)(2.5,0)
\put(41.8,-501.511){\fontsize{12}{1}\usefont{T1}{cmr}{m}{n}\selectfont\color{color_29791}lista utenti, va bene che siano impostate già al provisioning; ma se poi c'è bisogno di creare un }
\end{picture}
\begin{tikzpicture}[overlay]
\path(0pt,0pt);
\draw[color_29791,line width=0.7pt]
(41.8pt, -498.111pt) -- (497.4pt, -498.111pt)
;
\end{tikzpicture}
\begin{picture}(-5,0)(2.5,0)
\put(41.8,-515.311){\fontsize{12}{1}\usefont{T1}{cmr}{m}{n}\selectfont\color{color_29791}sistema multiutente dove vanno modificati molte volte in un giorno, ci vuole un protocollo in tempo}
\end{picture}
\begin{tikzpicture}[overlay]
\path(0pt,0pt);
\draw[color_29791,line width=0.7pt]
(41.8pt, -511.911pt) -- (522.3pt, -511.911pt)
;
\end{tikzpicture}
\begin{picture}(-5,0)(2.5,0)
\put(41.8,-529.111){\fontsize{12}{1}\usefont{T1}{cmr}{m}{n}\selectfont\color{color_29791}reale, non basta il provisioning). }
\end{picture}
\begin{tikzpicture}[overlay]
\path(0pt,0pt);
\draw[color_29791,line width=0.7pt]
(41.8pt, -525.711pt) -- (200.1pt, -525.711pt)
;
\end{tikzpicture}
\begin{picture}(-5,0)(2.5,0)
\put(41.8,-549.911){\fontsize{12}{1}\usefont{T1}{cmr}{m}{n}\selectfont\color{color_29791}Nel corso abbiamo visto un minimo di sistemi di config. Management, come Vagrant e Ansible, }
\end{picture}
\begin{tikzpicture}[overlay]
\path(0pt,0pt);
\draw[color_29791,line width=0.7pt]
(41.8pt, -546.511pt) -- (504.1pt, -546.511pt)
;
\end{tikzpicture}
\begin{picture}(-5,0)(2.5,0)
\put(41.8,-563.711){\fontsize{12}{1}\usefont{T1}{cmr}{m}{n}\selectfont\color{color_29791}pensati per fornire modelli in grado di descrivere lo stato di configurazione per un determinato }
\end{picture}
\begin{tikzpicture}[overlay]
\path(0pt,0pt);
\draw[color_29791,line width=0.7pt]
(41.8pt, -560.311pt) -- (498.6pt, -560.311pt)
;
\end{tikzpicture}
\begin{picture}(-5,0)(2.5,0)
\put(41.8,-577.511){\fontsize{12}{1}\usefont{T1}{cmr}{m}{n}\selectfont\color{color_29791}ruolo -VM, servizio...- e di istanziare il modello sulla base delle specifiche del target (hostname, ip, }
\end{picture}
\begin{tikzpicture}[overlay]
\path(0pt,0pt);
\draw[color_29791,line width=0.7pt]
(41.8pt, -574.111pt) -- (522pt, -574.111pt)
;
\end{tikzpicture}
\begin{picture}(-5,0)(2.5,0)
\put(41.8,-591.311){\fontsize{12}{1}\usefont{T1}{cmr}{m}{n}\selectfont\color{color_29791}layout dischi);}
\end{picture}
\begin{tikzpicture}[overlay]
\path(0pt,0pt);
\draw[color_29791,line width=0.7pt]
(41.8pt, -587.911pt) -- (110.8pt, -587.911pt)
;
\end{tikzpicture}
\begin{picture}(-5,0)(2.5,0)
\put(110.8,-591.311){\fontsize{12}{1}\usefont{T1}{cmr}{m}{n}\selectfont\color{color_29791} LDAP è in grado di fornire sia servizi di templating e configuration management }
\put(41.8,-605.111){\fontsize{12}{1}\usefont{T1}{cmr}{m}{n}\selectfont\color{color_217499}(config management ovvero fornire modelli per descrivere stato di configurazione per un ruolo -}
\put(41.8,-618.911){\fontsize{12}{1}\usefont{T1}{cmr}{m}{n}\selectfont\color{color_217499}VM, servizio...-, e istanziazione del modello sulla base delle specifiche del target -hostname, ip, }
\put(41.8,-632.711){\fontsize{12}{1}\usefont{T1}{cmr}{m}{n}\selectfont\color{color_217499}layout dischi...-), che autenticazione centralizzata.}
\put(41.8,-653.511){\fontsize{12}{1}\usefont{T1}{cmr}{m}{n}\selectfont\color{color_29791}Per fornire autenticazione centralizzata è necessario specificare due aspetti:}
\put(59.8,-674.311){\fontsize{12}{1}\usefont{T1}{cmr}{m}{n}\selectfont\color{color_29791}•Come scegliere localmente la sorgente dei dati:}
\put(77.8,-695.111){\fontsize{12}{1}\usefont{T1}{cmr}{m}{n}\selectfont\color{color_29791}◦NSS è un sistema generale per selezionare i servizi di nomi, abbiamo già visto che }
\put(95.8,-708.911){\fontsize{12}{1}\usefont{T1}{cmr}{m}{n}\selectfont\color{color_29791}permette di scegliere quale fonte usare per individuare concretamente un DB indicato }
\put(95.8,-722.711){\fontsize{12}{1}\usefont{T1}{cmr}{m}{n}\selectfont\color{color_29791}come astratto. NSS si aspetta di trovare libnss-ldap.so per sapere come interfacciarsi con}
\put(95.8,-736.511){\fontsize{12}{1}\usefont{T1}{cmr}{m}{n}\selectfont\color{color_29791}LDAP: è una libreria che fa da commutatore, sa trovare le implementazioni concrete che}
\put(95.8,-750.311){\fontsize{12}{1}\usefont{T1}{cmr}{m}{n}\selectfont\color{color_29791}servono, grazie a file di configurazione di NSS. }
\end{picture}
\newpage
\begin{tikzpicture}[overlay]\path(0pt,0pt);\end{tikzpicture}
\begin{picture}(-5,0)(2.5,0)
\put(77.8,-85.01099){\fontsize{12}{1}\usefont{T1}{cmr}{m}{n}\selectfont\color{color_29791}◦PAM è un sistema modulare per implementare authentication modules. Permette di }
\put(95.8,-98.81097){\fontsize{12}{1}\usefont{T1}{cmr}{m}{n}\selectfont\color{color_29791}implementare in modo flessibile le politiche di autenticazione (es. interfacciamento con }
\put(95.8,-112.611){\fontsize{12}{1}\usefont{T1}{cmr}{m}{n}\selectfont\color{color_29791}lettore impronte digitali, scrittura di algoritmo complesso per valutare molti parametri }
\put(95.8,-126.411){\fontsize{12}{1}\usefont{T1}{cmr}{m}{n}\selectfont\color{color_29791}sull'utente da autenticare...), con un sistema simile ad NSS: ha un commutatore che va a }
\put(95.8,-140.211){\fontsize{12}{1}\usefont{T1}{cmr}{m}{n}\selectfont\color{color_29791}cercare le implementazioni dei metodi astratti descritti in un file di configurazione }
\put(95.8,-154.011){\fontsize{12}{1}\usefont{T1}{cmr}{m}{n}\selectfont\color{color_29791}(esempio: account required pam\_nologin.so lo shared }
\put(95.8,-167.811){\fontsize{12}{1}\usefont{T1}{cmr}{m}{n}\selectfont\color{color_29791}object implementa in binario la funzione necessaria a gestire quel caso). Le librerie }
\put(95.8,-181.611){\fontsize{12}{1}\usefont{T1}{cmr}{m}{n}\selectfont\color{color_29791}modulari possono essere usate in maniera flessibile dalle applicazioni (da una parte ho }
\put(95.8,-195.411){\fontsize{12}{1}\usefont{T1}{cmr}{m}{n}\selectfont\color{color_29791}librerie che implementano tutte le politiche, dall'altra le app che ne hanno bisogno; in }
\put(95.8,-209.211){\fontsize{12}{1}\usefont{T1}{cmr}{m}{n}\selectfont\color{color_29791}mezzo c'è file di configurazione che dice quale libreria usare per una data applicazione). }
\put(95.8,-223.011){\fontsize{12}{1}\usefont{T1}{cmr}{m}{n}\selectfont\color{color_29791}C'è una libreria libpam-ldap.so che permette a PAM di interfacciarsi con LDAP.}
\put(59.8,-243.811){\fontsize{12}{1}\usefont{T1}{cmr}{m}{n}\selectfont\color{color_29791}•Come trasferire i dati dal server centrale ai client}
\put(77.8,-264.611){\fontsize{12}{1}\usefont{T1}{cmr}{m}{n}\selectfont\color{color_29791}◦LDAP è il protocollo di distribuzione delle informazioni. Attraverso NSS ottengo una }
\put(95.8,-278.411){\fontsize{12}{1}\usefont{T1}{cmr}{m}{n}\selectfont\color{color_29791}lista di utenti in modo configurabile, poi PAM fornisce i metodi per l'autenticazione: per }
\put(95.8,-292.211){\fontsize{12}{1}\usefont{T1}{cmr}{m}{n}\selectfont\color{color_29791}LDAP sarà necessario che funzionino entrambi, e che ci sia lo strato intermedio delle }
\put(95.8,-306.011){\fontsize{12}{1}\usefont{T1}{cmr}{m}{n}\selectfont\color{color_29791}librerie libnss-ldap (per usare db remoto LDAP come sorgente di nomi utenti, gruppi, }
\put(95.8,-319.811){\fontsize{12}{1}\usefont{T1}{cmr}{m}{n}\selectfont\color{color_29791}host…) e libpam-ldap (per accedere a db remoto LDAP per autenticare utenti e ricavare }
\put(95.8,-333.611){\fontsize{12}{1}\usefont{T1}{cmr}{m}{n}\selectfont\color{color_29791}regole di autenticazione).}
\put(41.8,-354.411){\fontsize{12}{1}\usefont{T1}{cmr}{m}{n}\selectfont\color{color_29791}LDAP implementa un directory service, ovvero un database specializzato basato sul modello }
\put(41.8,-368.211){\fontsize{12}{1}\usefont{T1}{cmr}{m}{n}\selectfont\color{color_29791}dell'elenco telefonico: permette di cercare informazioni in base ad attributi, ma è importante che sia }
\put(41.8,-382.011){\fontsize{12}{1}\usefont{T1}{cmr}{m}{n}\selectfont\color{color_29791}specializzato e non generico perché}
\put(59.8,-402.811){\fontsize{12}{1}\usefont{T1}{cmr}{m}{n}\selectfont\color{color_29791}•fornisce grande velocità di lettura ma minore velocità di scrittura. Come per elenco del }
\put(77.8,-416.611){\fontsize{12}{1}\usefont{T1}{cmr}{m}{n}\selectfont\color{color_29791}telefono, si prevedono molte letture e poche scritture (es. in azienda, ogni mese assumo 1 }
\put(77.8,-430.411){\fontsize{12}{1}\usefont{T1}{cmr}{m}{n}\selectfont\color{color_29791}persona, ma ogni giorno 1000 fanno accesso e devono autenticarsi).}
\put(59.8,-451.211){\fontsize{12}{1}\usefont{T1}{cmr}{m}{n}\selectfont\color{color_29791}•E' un DB gerarchico, al suo interno è diviso in diverse parti (es. ogni nazione ha il suo }
\put(77.8,-465.011){\fontsize{12}{1}\usefont{T1}{cmr}{m}{n}\selectfont\color{color_29791}namespace (o anche ogni regione, in italia)). La gerarchia ha pregi e difetti:}
\put(77.8,-485.811){\fontsize{12}{1}\usefont{T1}{cmr}{m}{n}\selectfont\color{color_29791}◦Pregi:}
\put(95.8,-506.611){\fontsize{12}{1}\usefont{T1}{cmr}{m}{n}\selectfont\color{color_29791}▪se ho modo semplice di immaginare la posizione dell'informazione; questa diventa }
\put(113.8,-520.411){\fontsize{12}{1}\usefont{T1}{cmr}{m}{n}\selectfont\color{color_29791}criterio di ricerca iperefficiente; navigare una gerarchia è semplicissimo.}
\put(95.8,-541.211){\fontsize{12}{1}\usefont{T1}{cmr}{m}{n}\selectfont\color{color_29791}▪molto facile creare delle deleghe nella gestione di sottoalberi: nessun problema di }
\put(113.8,-555.011){\fontsize{12}{1}\usefont{T1}{cmr}{m}{n}\selectfont\color{color_29791}collisione, i nomi devono essere univoci globalmente solo nel loro livello di }
\put(113.8,-568.811){\fontsize{12}{1}\usefont{T1}{cmr}{m}{n}\selectfont\color{color_29791}sottoalbero.}
\put(77.8,-589.611){\fontsize{12}{1}\usefont{T1}{cmr}{m}{n}\selectfont\color{color_29791}◦Svantaggi:}
\put(95.8,-610.411){\fontsize{12}{1}\usefont{T1}{cmr}{m}{n}\selectfont\color{color_29791}▪una volta individuata la struttura della gerarchia, è difficile cambiarla: si ha una }
\put(113.8,-624.211){\fontsize{12}{1}\usefont{T1}{cmr}{m}{n}\selectfont\color{color_29791}visione rigida, con il tradeoff dello scegliere a priori la vista che mi interessa, e avere}
\put(113.8,-638.011){\fontsize{12}{1}\usefont{T1}{cmr}{m}{n}\selectfont\color{color_29791}una ricerca meno efficiente se ce ne interessa un'altra dopo.}
\put(59.8,-658.811){\fontsize{12}{1}\usefont{T1}{cmr}{m}{n}\selectfont\color{color_29791}•Non è relazionale, si possono evitare le transazioni e si può usare meccanismo di locking }
\put(77.8,-672.611){\fontsize{12}{1}\usefont{T1}{cmr}{m}{n}\selectfont\color{color_29791}semplice, perché le scritture sono rare e quindi che ce ne siano due nello stesso momento è }
\put(77.8,-686.411){\fontsize{12}{1}\usefont{T1}{cmr}{m}{n}\selectfont\color{color_29791}difficile; un'ipotetica collisione data da modifica dello stesso dato, andrà gestita a mano, il }
\put(77.8,-700.211){\fontsize{12}{1}\usefont{T1}{cmr}{m}{n}\selectfont\color{color_29791}protocollo non se ne occupa.}
\put(59.8,-721.011){\fontsize{12}{1}\usefont{T1}{cmr}{m}{n}\selectfont\color{color_29791}•Deve essere altamente disponibile e quindi poter essere facilmente replicato; avere molte }
\put(77.8,-734.811){\fontsize{12}{1}\usefont{T1}{cmr}{m}{n}\selectfont\color{color_29791}copie è importante.}
\put(41.8,-755.611){\fontsize{12}{1}\usefont{T1}{cmr}{m}{n}\selectfont\color{color_29791}LDAP esiste da 30 anni, usato su Internet per i sistemi UNIX-like (prima UNIX era fortemente }
\put(41.8,-769.411){\fontsize{12}{1}\usefont{T1}{cmr}{m}{n}\selectfont\color{color_29791}multiutente, poi è diventato principalmente server) e da quando Windows è diventato il principale }
\end{picture}
\newpage
\begin{tikzpicture}[overlay]\path(0pt,0pt);\end{tikzpicture}
\begin{picture}(-5,0)(2.5,0)
\put(41.8,-85.01099){\fontsize{12}{1}\usefont{T1}{cmr}{m}{n}\selectfont\color{color_29791}sistema client per reti professionali, ha implementato una sua versione di LDAP sotto Active }
\put(41.8,-98.81097){\fontsize{12}{1}\usefont{T1}{cmr}{m}{n}\selectfont\color{color_29791}Directory.}
\put(41.8,-128.611){\fontsize{14.1}{1}\usefont{T1}{cmr}{b}{n}\selectfont\color{color_29791}Modello dei dati}
\put(41.8,-148.811){\fontsize{12}{1}\usefont{T1}{cmr}{m}{n}\selectfont\color{color_29791}LDAP non si preoccupa di come vogliamo memorizzare la base di dati (noSQL, SQL, file di dati }
\put(41.8,-162.611){\fontsize{12}{1}\usefont{T1}{cmr}{m}{n}\selectfont\color{color_29791}complessivo, un file per ogni record…) la cosa importante è che ci sia un modulo che LDAP possa }
\put(41.8,-176.411){\fontsize{12}{1}\usefont{T1}{cmr}{m}{n}\selectfont\color{color_29791}usare per accedere fisicamente ai dati, DBI (DataBase Interface) è un'interfaccia per accedere al }
\put(41.8,-190.211){\fontsize{12}{1}\usefont{T1}{cmr}{m}{n}\selectfont\color{color_29791}backend più adatto (Si pone tra LDAP Server e driver file/SQL...). DBI gestisce una serie di entry, }
\put(41.8,-204.011){\fontsize{12}{1}\usefont{T1}{cmr}{m}{n}\selectfont\color{color_29791}anche dette object; ogni entry ha una collezione di attribute, ognuno dei quali ha un type e può }
\put(41.8,-217.811){\fontsize{12}{1}\usefont{T1}{cmr}{m}{n}\selectfont\color{color_29791}avere diversi value (diversamente da un DB tradizionale, dove ogni cella ha un solo valore; in DBI, }
\put(41.8,-231.611){\fontsize{12}{1}\usefont{T1}{cmr}{m}{n}\selectfont\color{color_29791}una entry che rappresenta un utente che può avere attributo "telefono", può avere più valori per }
\put(41.8,-245.411){\fontsize{12}{1}\usefont{T1}{cmr}{m}{n}\selectfont\color{color_29791}questo attributo). Ogni entry ha un identificativo gerarchico detto DN (Distinguished Name)}
\put(41.8,-259.211){\fontsize{12}{1}\usefont{T1}{cmr}{m}{n}\selectfont\color{color_29791}(gerachico, come gli OID per SNMP).}
\put(41.8,-280.011){\fontsize{12}{1}\usefont{T1}{cmr}{m}{n}\selectfont\color{color_29791}Le entry sono legate in gerarchia, il DN dice quali rami seguire per arrivare al nodo che mi }
\put(41.8,-293.811){\fontsize{12}{1}\usefont{T1}{cmr}{m}{n}\selectfont\color{color_29791}interessa. La struttura risultante è chiamata DIT (Dir Information Tree) e rappresenta il modo in }
\put(41.8,-307.611){\fontsize{12}{1}\usefont{T1}{cmr}{m}{n}\selectfont\color{color_29791}cui viene esposta la base dati gestita dalla DBI: questa dice come accedere alle entry (che possono }
\put(41.8,-321.411){\fontsize{12}{1}\usefont{T1}{cmr}{m}{n}\selectfont\color{color_29791}essere in qualsiasi formato);  ma dal lato di chi usa LDAP le entry risultano ordinatamente }
\put(41.8,-335.211){\fontsize{12}{1}\usefont{T1}{cmr}{m}{n}\selectfont\color{color_29791}organizzate in un DIT. Quindi i dati possono avere una rappresentazione fisica arbitraria (es. anche }
\put(41.8,-349.011){\fontsize{12}{1}\usefont{T1}{cmr}{m}{n}\selectfont\color{color_29791}come file flat, oppure su disco i dati sono visti come blocchi); ma da un punto di vista logico, }
\put(41.8,-362.811){\fontsize{12}{1}\usefont{T1}{cmr}{m}{n}\selectfont\color{color_29791}mediante LDAP si possono vedere come ordinati gerarchicamente, come un albero. }
\put(41.8,-392.611){\fontsize{14.1}{1}\usefont{T1}{cmr}{b}{n}\selectfont\color{color_29791}Replica}
\put(41.8,-412.811){\fontsize{12}{1}\usefont{T1}{cmr}{m}{n}\selectfont\color{color_29791}Essendo un servizio essenziale, LDAP prevede nativamente meccanismi per l'alta disponibilità. }
\put(41.8,-426.611){\fontsize{12}{1}\usefont{T1}{cmr}{m}{n}\selectfont\color{color_29791}Implementare la ridondanza è semplice, siccome scritture sono rare, (e quelle incompatibili }
\put(41.8,-440.411){\fontsize{12}{1}\usefont{T1}{cmr}{m}{n}\selectfont\color{color_29791}eccezionali). Il modello più usato è quello multimaster, in cui si copia il DB su vari server e si }
\put(41.8,-454.211){\fontsize{12}{1}\usefont{T1}{cmr}{m}{n}\selectfont\color{color_29791}permette ad ognuno sia accesso in lettura che in scrittura; con protocolli che controllano }
\put(41.8,-468.011){\fontsize{12}{1}\usefont{T1}{cmr}{m}{n}\selectfont\color{color_29791}periodicamente le inconsistenze tra i server e valutano se propagare modifiche da uno all'altro, o }
\put(41.8,-481.811){\fontsize{12}{1}\usefont{T1}{cmr}{m}{n}\selectfont\color{color_29791}come gestire manualmente le inconsistenze. L'alternativa è il modello master-replica, in cui c'è un }
\put(41.8,-495.611){\fontsize{12}{1}\usefont{T1}{cmr}{m}{n}\selectfont\color{color_29791}solo server che accetta scritture dai client e molti server replica che offrono solo lettura (ha senso }
\put(41.8,-509.411){\fontsize{12}{1}\usefont{T1}{cmr}{m}{n}\selectfont\color{color_29791}avere solo letture sulle tante repliche, essendo l'operazione che è più necessario scali). I modelli }
\put(41.8,-523.211){\fontsize{12}{1}\usefont{T1}{cmr}{m}{n}\selectfont\color{color_29791}master-replica possono essere anche by referral (in cui il client diventa più complicato, perché prova}
\put(41.8,-537.011){\fontsize{12}{1}\usefont{T1}{cmr}{m}{n}\selectfont\color{color_29791}a scrivere su un server, e se incontra replica read-only; riceve informazioni sul master a cui }
\put(41.8,-550.811){\fontsize{12}{1}\usefont{T1}{cmr}{m}{n}\selectfont\color{color_29791}rivolgersi) e by chain (in cui il client è il più semplice possibile, e i server sono più complessi: i }
\put(41.8,-564.611){\fontsize{12}{1}\usefont{T1}{cmr}{m}{n}\selectfont\color{color_29791}replica fanno finta di essere scrivibili ma propagano richiesta al master, facendo da proxy).}
\put(41.8,-585.411){\fontsize{12}{1}\usefont{T1}{cmr}{m}{n}\selectfont\color{color_29791}Sezionare un DIT è vantaggioso (ad esempio in termini di prestazioni, per la località delle risorse }
\put(41.8,-599.211){\fontsize{12}{1}\usefont{T1}{cmr}{m}{n}\selectfont\color{color_29791}rispetto a sedi geograficamente distanti, o per delega amministrativa per sezioni di un'azienda) e }
\put(41.8,-613.011){\fontsize{12}{1}\usefont{T1}{cmr}{m}{n}\selectfont\color{color_29791}avere una struttura ad albero rende facile distribuire i dati, ponendo i sottoalberi in diverse }
\put(41.8,-626.811){\fontsize{12}{1}\usefont{T1}{cmr}{m}{n}\selectfont\color{color_29791}posizioni. La distibuzione può essere fatta mediante foglie LDAP che sembrano entry normali, ma }
\put(41.8,-640.611){\fontsize{12}{1}\usefont{T1}{cmr}{m}{n}\selectfont\color{color_29791}in realtà contengono puntatori alla posizione di un sottoalbero: }
\put(59.8,-661.411){\fontsize{12}{1}\usefont{T1}{cmr}{m}{n}\selectfont\color{color_29791}•dal lato del server principale, un nodo link (detto reference) permette di sapere a quale altro }
\put(77.8,-675.211){\fontsize{12}{1}\usefont{T1}{cmr}{m}{n}\selectfont\color{color_29791}server subordinato rivolgersi per scendere nella gerarchia e avere i dati del sottoalbero non }
\put(77.8,-689.011){\fontsize{12}{1}\usefont{T1}{cmr}{m}{n}\selectfont\color{color_29791}locale; (Subordinate knowledge link)}
\put(59.8,-709.811){\fontsize{12}{1}\usefont{T1}{cmr}{m}{n}\selectfont\color{color_29791}•dal lato del server che ottiene la delega, il nodo link contiene l'indicazione del server a cui }
\put(77.8,-723.611){\fontsize{12}{1}\usefont{T1}{cmr}{m}{n}\selectfont\color{color_29791}riferirsi per la parte superiore dell'albero: così il server subordinato sa che ha una sua radice }
\put(77.8,-737.411){\fontsize{12}{1}\usefont{T1}{cmr}{m}{n}\selectfont\color{color_29791}ma che non è assoluta; ed in realtà c'è un genitore a cui può rivolgersi per conoscere il resto }
\put(77.8,-751.211){\fontsize{12}{1}\usefont{T1}{cmr}{m}{n}\selectfont\color{color_29791}del DIT. (Superior knowledge link)}
\end{picture}
\newpage
\begin{tikzpicture}[overlay]\path(0pt,0pt);\end{tikzpicture}
\begin{picture}(-5,0)(2.5,0)
\put(41.8,-87.01099){\fontsize{14.1}{1}\usefont{T1}{cmr}{b}{n}\selectfont\color{color_29791}Formato entry}
\put(41.8,-107.211){\fontsize{12}{1}\usefont{T1}{cmr}{m}{n}\selectfont\color{color_29791}Una entry è una collezione di attributi. Questi non sono definiti con un nome come nei linguaggi di }
\put(41.8,-121.011){\fontsize{12}{1}\usefont{T1}{cmr}{m}{n}\selectfont\color{color_29791}programmazione, ma si dichiara direttamente il valore etichettato dal tipo (In C:int i = 1 dice tipo, }
\put(41.8,-134.811){\fontsize{12}{1}\usefont{T1}{cmr}{m}{n}\selectfont\color{color_29791}nome e valore; in LDAP variabili sono come "int 1", coppia tipo-valore). Tra gli attributi, due sono }
\put(41.8,-148.611){\fontsize{12}{1}\usefont{T1}{cmr}{m}{n}\selectfont\color{color_29791}presenti in tutte le entry: dn (distinguished name) è nome univoco della entry, objectClass (una o }
\put(41.8,-162.411){\fontsize{12}{1}\usefont{T1}{cmr}{m}{n}\selectfont\color{color_29791}più) è la classe a cui appartiene la entry. Esempio entry:}
\put(77.3,-183.211){\fontsize{12}{1}\usefont{T1}{cmr}{m}{it}\selectfont\color{color_119557}dn: dc=labammsisDN è "dc=labammsis", perché}
\end{picture}
\begin{tikzpicture}[overlay]
\path(0pt,0pt);
\draw[color_29791,line width=0.7pt]
(332.9pt, -184.311pt) -- (364.9pt, -184.311pt)
;
\end{tikzpicture}
\begin{picture}(-5,0)(2.5,0)
\put(364.9,-183.211){\fontsize{12}{1}\usefont{T1}{cmr}{m}{n}\selectfont\color{color_29791} c'è attributo dc: labammsis}
\put(77.3,-204.011){\fontsize{12}{1}\usefont{T1}{cmr}{m}{it}\selectfont\color{color_119557}objectClass: dcObject}
\put(77.3,-224.811){\fontsize{12}{1}\usefont{T1}{cmr}{m}{it}\selectfont\color{color_119557}objectClass: organization}
\put(77.3,-245.611){\fontsize{12}{1}\usefont{T1}{cmr}{m}{it}\selectfont\color{color_119557}dc: labammsis}
\put(77.3,-266.411){\fontsize{12}{1}\usefont{T1}{cmr}{m}{it}\selectfont\color{color_119557}o: università}
\put(41.8,-287.211){\fontsize{12}{1}\usefont{T1}{cmr}{m}{n}\selectfont\color{color_29791}Le parti blu sono i tipi di attributo (etichette che dicono il tipo), la parte in verde è il valore }
\put(41.8,-301.011){\fontsize{12}{1}\usefont{T1}{cmr}{m}{n}\selectfont\color{color_29791}dell'attributo. Alcuni attributi sono univoci (es. dn deve essere univoco), altri possono essere }
\put(41.8,-314.811){\fontsize{12}{1}\usefont{T1}{cmr}{m}{n}\selectfont\color{color_29791}multipli (es. attributo di tipo objectClass ha 2 valori, o se preferiamo ci sono due attributi dello }
\put(41.8,-328.611){\fontsize{12}{1}\usefont{T1}{cmr}{m}{n}\selectfont\color{color_29791}stesso tipo). Questo formato è usato per scambiare entry tra client e server e viene detto LDIF }
\put(41.8,-342.411){\fontsize{12}{1}\usefont{T1}{cmr}{m}{n}\selectfont\color{color_29791}(LDAP Interchange Format), ogni entry è sequenza di righe, e in ogni riga c'è "tipo:valore" dei }
\put(41.8,-356.211){\fontsize{12}{1}\usefont{T1}{cmr}{m}{n}\selectfont\color{color_29791}diversi valori; per mettere il DB in un solo file LDIF, basta scrivere ogni entry così e separarle con }
\put(41.8,-370.011){\fontsize{12}{1}\usefont{T1}{cmr}{m}{n}\selectfont\color{color_29791}una riga vuota.}
\put(41.8,-390.811){\fontsize{12}{1}\usefont{T1}{cmr}{m}{n}\selectfont\color{color_29791}Lo schema garantisce che ogni entry inserita sia ben formata e che tutti gli utilizzatori sappiano }
\put(41.8,-404.611){\fontsize{12}{1}\usefont{T1}{cmr}{m}{n}\selectfont\color{color_29791}come vedere l'insieme delle entry (possano vedere l'organizzazione del DB per interpretare }
\put(41.8,-418.411){\fontsize{12}{1}\usefont{T1}{cmr}{m}{n}\selectfont\color{color_29791}correttamente le informazioni); mediante un insieme di regole che descrivono i dati immagazzinati, }
\put(41.8,-432.211){\fontsize{12}{1}\usefont{T1}{cmr}{m}{n}\selectfont\color{color_29791}usando due definizioni: le objectClass (una o più per entry, specifica i tipi di attributo obbligatori e }
\put(41.8,-446.011){\fontsize{12}{1}\usefont{T1}{cmr}{m}{n}\selectfont\color{color_29791}facoltativi -e per esclusione, quelli vietati-) e gli attributeType (definiscono i tipi di dato e le regole }
\put(41.8,-459.811){\fontsize{12}{1}\usefont{T1}{cmr}{m}{n}\selectfont\color{color_29791}per compararli).}
\put(41.8,-480.611){\fontsize{12}{1}\usefont{T1}{cmr}{m}{n}\selectfont\color{color_29791}Lo schema è composto da tante definizioni di classi di oggetti; le objectClass, per definire cosa è }
\put(41.8,-494.411){\fontsize{12}{1}\usefont{T1}{cmr}{m}{n}\selectfont\color{color_29791}valido in una data classe, usano le definizioni dei tipi di attributo. Gli attributeType, per essere }
\put(41.8,-508.211){\fontsize{12}{1}\usefont{T1}{cmr}{m}{n}\selectfont\color{color_29791}definiti, si avvalgono della sintassi (es. attributeType numero di telefono, potrà dire ha sintassi di }
\put(41.8,-522.011){\fontsize{12}{1}\usefont{T1}{cmr}{m}{n}\selectfont\color{color_29791}stringa di max 12 caratteri). In LDAP; qualcuno ha definito quali sintassi esistono (volendo un }
\put(41.8,-535.811){\fontsize{12}{1}\usefont{T1}{cmr}{m}{n}\selectfont\color{color_29791}esempio sono come i tipi di dato di base, interi stringhe float booleani etc.).}
\put(41.8,-556.611){\fontsize{12}{1}\usefont{T1}{cmr}{m}{n}\selectfont\color{color_29791}Quanto appena visto è lo schema, che dice solo cosa è valido e cosa no, mentre il contenuto }
\put(41.8,-570.411){\fontsize{12}{1}\usefont{T1}{cmr}{m}{n}\selectfont\color{color_29791}effettivo deve essere modellato seguendo lo schema: ogni attributo avrà un valore, che deve essere }
\put(41.8,-584.211){\fontsize{12}{1}\usefont{T1}{cmr}{m}{n}\selectfont\color{color_29791}compatibile con la sintassi (es. non posso assegnare valore di 15 caratteri ad attributo che prevede }
\put(41.8,-598.011){\fontsize{12}{1}\usefont{T1}{cmr}{m}{n}\selectfont\color{color_29791}sintassi di 12 caratteri), quindi i valori devono rispettare i vincoli degli attributiType. Ogni attributo }
\put(41.8,-611.811){\fontsize{12}{1}\usefont{T1}{cmr}{m}{n}\selectfont\color{color_29791}deve rispettare i vincoli di numerosità imposti dall'objectClass di ogni entry (quali attributi }
\put(41.8,-625.611){\fontsize{12}{1}\usefont{T1}{cmr}{m}{n}\selectfont\color{color_29791}devono/possono essere presenti). Le objectClass seguono le definizioni dello schema del DIT.}
\put(41.8,-655.411){\fontsize{14.1}{1}\usefont{T1}{cmr}{b}{n}\selectfont\color{color_29791}Spazio dei nomi}
\put(41.8,-675.611){\fontsize{12}{1}\usefont{T1}{cmr}{m}{n}\selectfont\color{color_29791}Per ogni entry è necessario scegliere un attributo rappresentativo per individuarla. C'è libertà di }
\put(41.8,-689.411){\fontsize{12}{1}\usefont{T1}{cmr}{m}{n}\selectfont\color{color_29791}scegliere quale attributo darà nome alla entry; nell'esempio di prima si è deciso che attributo dc con }
\put(41.8,-703.211){\fontsize{12}{1}\usefont{T1}{cmr}{m}{n}\selectfont\color{color_29791}valore labammsis viene usato come nome: quindi il nome (dn , distinguished name) è }
\put(77.3,-717.011){\fontsize{12}{1}\usefont{T1}{cmr}{b}{it}\selectfont\color{color_29791}dc=labammsis perché tra gli attributi della entry c'è un attributo dc che vale }
\put(41.8,-730.811){\fontsize{12}{1}\usefont{T1}{cmr}{m}{it}\selectfont\color{color_29791}labammsis; in alternativa il nome potrebbe essere es. o=università perché c'è attributo o con valore }
\put(41.8,-744.611){\fontsize{12}{1}\usefont{T1}{cmr}{m}{it}\selectfont\color{color_29791}università. Essendo la scelta del nome arbitraria, in teoria si può scegliere che le entry siano legate }
\put(41.8,-758.411){\fontsize{12}{1}\usefont{T1}{cmr}{m}{n}\selectfont\color{color_29791}ad un nodo padre mediante nomi relativi diversi (es. alcune di nome relativo o=qualcosa e altre con }
\end{picture}
\newpage
\begin{tikzpicture}[overlay]\path(0pt,0pt);\end{tikzpicture}
\begin{picture}(-5,0)(2.5,0)
\put(41.8,-85.01099){\fontsize{12}{1}\usefont{T1}{cmr}{m}{n}\selectfont\color{color_29791}nome relativo dc=qualcosa), mischiandole, ma dopo un po' non si capisce più niente: è opportuno }
\put(41.8,-98.81097){\fontsize{12}{1}\usefont{T1}{cmr}{m}{n}\selectfont\color{color_29791}seguire linee guida per costruire una gerarchia sensata.}
\put(41.8,-119.611){\fontsize{12}{1}\usefont{T1}{cmr}{m}{n}\selectfont\color{color_29791}La stringa tipoAttributo=valoreAttributo scelta è il relative distinguished name (RDN) della entry }
\put(41.8,-133.411){\fontsize{12}{1}\usefont{T1}{cmr}{m}{n}\selectfont\color{color_29791}(quello autocontenuto che non mi dice dov'è la entry; poi devo scegliere a quale nodo già esistente }
\put(41.8,-147.211){\fontsize{12}{1}\usefont{T1}{cmr}{m}{n}\selectfont\color{color_29791}agganciarla -con il BDN-); il nome del nodo a cui agganciare la entry è il base distinguished name }
\put(41.8,-161.011){\fontsize{12}{1}\usefont{T1}{cmr}{m}{n}\selectfont\color{color_29791}(BDN). Il distinguished name (DN) è il nome univoco globalmente}
\end{picture}
\begin{tikzpicture}[overlay]
\path(0pt,0pt);
\draw[color_29791,line width=0.7pt]
(269.6pt, -162.111pt) -- (370.5pt, -162.111pt)
;
\end{tikzpicture}
\begin{picture}(-5,0)(2.5,0)
\put(370.5,-161.011){\fontsize{12}{1}\usefont{T1}{cmr}{m}{n}\selectfont\color{color_29791} ottenuto concatendando i due, }
\put(41.8,-174.811){\fontsize{12}{1}\usefont{T1}{cmr}{m}{n}\selectfont\color{color_29791}quindi DN=RDN+BDN.}
\put(41.8,-195.611){\fontsize{12}{1}\usefont{T1}{cmr}{m}{n}\selectfont\color{color_29791}Nella radice, DN e RDN coincidono. Supponiamo che voglia suddividere albero con organizational}
\put(41.8,-209.411){\fontsize{12}{1}\usefont{T1}{cmr}{m}{it}\selectfont\color{color_29791}unit, una per docenti e una per studenti: un nodo ha come RDN ou=docenti, un nodo ha RDN }
\put(41.8,-223.211){\fontsize{12}{1}\usefont{T1}{cmr}{m}{n}\selectfont\color{color_29791}ou=studenti; essendo agganciati a quello padre; il DN (nome univoco}
\end{picture}
\begin{tikzpicture}[overlay]
\path(0pt,0pt);
\draw[color_29791,line width=0.7pt]
(336.2pt, -224.311pt) -- (374.8pt, -224.311pt)
;
\end{tikzpicture}
\begin{picture}(-5,0)(2.5,0)
\put(374.8,-223.211){\fontsize{12}{1}\usefont{T1}{cmr}{m}{n}\selectfont\color{color_29791} che individua i nodi }
\put(41.8,-237.011){\fontsize{12}{1}\usefont{T1}{cmr}{m}{n}\selectfont\color{color_29791}nell'albero complessivo) è ou=docenti,dc=unibo . Nel DN l'ordine dei termini è foglia->ramo-}
\put(41.8,-250.811){\fontsize{12}{1}\usefont{T1}{cmr}{m}{n}\selectfont\color{color_29791}>radice, da sinistra a destra (quindi top level domain è a destra, come per il DNS; l'opposto del }
\put(41.8,-264.611){\fontsize{12}{1}\usefont{T1}{cmr}{m}{n}\selectfont\color{color_29791}filesystem dove la radice è la prima cosa a sinistra). Il terzo livello avrà RDN cn=mario (c'è docente}
\put(41.8,-278.411){\fontsize{12}{1}\usefont{T1}{cmr}{m}{n}\selectfont\color{color_29791}mario e studente mario): si possono avere più nodi con lo stesso RDN solo se hanno BDN diversi, }
\put(41.8,-292.211){\fontsize{12}{1}\usefont{T1}{cmr}{m}{n}\selectfont\color{color_29791}essendo il DN complessivo di ognuno a dover essere univoco.}
\put(41.8,-313.011){\fontsize{12}{1}\usefont{T1}{cmr}{m}{n}\selectfont\color{color_29791}Quindi ogni RDN deve essere localmente unico (tutte le entry connesse allo stesso BDN, quindi }
\put(41.8,-326.811){\fontsize{12}{1}\usefont{T1}{cmr}{m}{n}\selectfont\color{color_217499}univocità nel proprio livello). Nel caso in cui un singolo attributo non sia sufficiente per avere RDN}
\put(41.8,-340.611){\fontsize{12}{1}\usefont{T1}{cmr}{m}{n}\selectfont\color{color_29791}unico (ad esempio, cn common name può avere omonimie) si può:}
\put(59.8,-361.411){\fontsize{12}{1}\usefont{T1}{cmr}{m}{n}\selectfont\color{color_29791}•Introdurre id univoco, soluzione "da DB":}
\put(77.8,-382.211){\fontsize{12}{1}\usefont{T1}{cmr}{m}{n}\selectfont\color{color_29791}◦si può effettivamente aggiungere un attributo come uniqueIdentifier, ma l'idea di LDAP }
\put(95.8,-396.011){\fontsize{12}{1}\usefont{T1}{cmr}{m}{n}\selectfont\color{color_29791}è che gli attributi rappresentino qualcosa di proprio della entry (e un identificativo }
\put(95.8,-409.811){\fontsize{12}{1}\usefont{T1}{cmr}{m}{n}\selectfont\color{color_29791}progressivo non lo è). Così facendo salta il discorso della facilità di navigazione della }
\put(95.8,-423.611){\fontsize{12}{1}\usefont{T1}{cmr}{m}{n}\selectfont\color{color_29791}gerarchia, per ricerca efficiente. (esempio, per trovare un docente, associare un id }
\put(95.8,-437.411){\fontsize{12}{1}\usefont{T1}{cmr}{m}{n}\selectfont\color{color_29791}univoco fa sì che si debba fare ricerca dell'id: per trovare docente prima scendo nella sua}
\put(95.8,-451.211){\fontsize{12}{1}\usefont{T1}{cmr}{m}{n}\selectfont\color{color_29791}università, poi nel settore in cui insegna, ma se poi invece del nome devo sapere il suo }
\put(95.8,-465.011){\fontsize{12}{1}\usefont{T1}{cmr}{m}{n}\selectfont\color{color_29791}numero di serie; tanto valeva fare ricerca globale)}
\put(59.8,-485.811){\fontsize{12}{1}\usefont{T1}{cmr}{m}{n}\selectfont\color{color_29791}•Concatenare più coppie attributo-valore:}
\put(77.8,-506.611){\fontsize{12}{1}\usefont{T1}{cmr}{m}{n}\selectfont\color{color_29791}◦Soluzione adottata da LDAP, si concatenano con + più coppie attributo=valore per }
\put(95.8,-520.411){\fontsize{12}{1}\usefont{T1}{cmr}{m}{n}\selectfont\color{color_29791}ottenere RDN localmente univoco (esempio, cn=Marco+sn=Prandini come RDN, per }
\put(95.8,-534.211){\fontsize{12}{1}\usefont{T1}{cmr}{m}{n}\selectfont\color{color_29791}evitare omonimie su cn nome, si concatena con sn cognome, poi ok possono esserci }
\put(95.8,-548.011){\fontsize{12}{1}\usefont{T1}{cmr}{m}{n}\selectfont\color{color_29791}omonimie nome-cognome ma è un esempio)}
\put(41.8,-577.811){\fontsize{14.1}{1}\usefont{T1}{cmr}{b}{n}\selectfont\color{color_29791}Attributi}
\put(41.8,-598.011){\fontsize{12}{1}\usefont{T1}{cmr}{m}{n}\selectfont\color{color_29791}Un attributeType è simile a un tipo di un linguaggio di programmazione, ma a differenza di questi, }
\put(41.8,-611.811){\fontsize{12}{1}\usefont{T1}{cmr}{m}{n}\selectfont\color{color_29791}specifica separatamente:}
\put(59.8,-632.611){\fontsize{12}{1}\usefont{T1}{cmr}{m}{n}\selectfont\color{color_29791}•SYNTAX ,sintassi:Il vero e proprio tipo di dato nativo (documentazione sintassi }
\put(77.8,-646.411){\fontsize{12}{1}\usefont{T1}{cmr}{m}{n}\selectfont\color{color_29919}https://tools.ietf.org/html/rfc4517\#section-3}
\end{picture}
\begin{tikzpicture}[overlay]
\path(0pt,0pt);
\draw[color_29919,line width=0.7pt]
(77.8pt, -647.511pt) -- (288.2pt, -647.511pt)
;
\end{tikzpicture}
\begin{picture}(-5,0)(2.5,0)
\put(288.2,-646.411){\fontsize{12}{1}\usefont{T1}{cmr}{m}{n}\selectfont\color{color_29791} )}
\put(59.8,-667.211){\fontsize{12}{1}\usefont{T1}{cmr}{m}{n}\selectfont\color{color_29791}•Matching rules:Le regole per stabilire i criteri di confronto tra valori, per valutare }
\put(77.8,-681.011){\fontsize{12}{1}\usefont{T1}{cmr}{m}{n}\selectfont\color{color_29791}uguaglianza e ordinamento nelle ricerche. In un linguaggio di programmazione, se ad es. }
\put(77.8,-694.811){\fontsize{12}{1}\usefont{T1}{cmr}{m}{n}\selectfont\color{color_29791}dico che una variabile a è di tipo intero, per fare confronti verrà implementata una logica }
\put(77.8,-708.611){\fontsize{12}{1}\usefont{T1}{cmr}{m}{n}\selectfont\color{color_29791}cablata all'interno del concetto di tipo intero. In LDAP per ogni tipo di dato potrei avere }
\put(77.8,-722.411){\fontsize{12}{1}\usefont{T1}{cmr}{m}{n}\selectfont\color{color_29791}diversi modi di interpretare confronti logici:(doc }
\put(77.8,-736.211){\fontsize{12}{1}\usefont{T1}{cmr}{m}{n}\selectfont\color{color_29919}https://tools.ietf.org/html/rfc4517\#section-4}
\end{picture}
\begin{tikzpicture}[overlay]
\path(0pt,0pt);
\draw[color_29919,line width=0.7pt]
(77.8pt, -737.311pt) -- (288.2pt, -737.311pt)
;
\end{tikzpicture}
\begin{picture}(-5,0)(2.5,0)
\put(288.2,-736.211){\fontsize{12}{1}\usefont{T1}{cmr}{m}{n}\selectfont\color{color_29791} )}
\put(77.8,-757.011){\fontsize{12}{1}\usefont{T1}{cmr}{m}{n}\selectfont\color{color_29791}◦ORDERING ,ordinamento:es. stringa che rappresenta un numero, se vale 100 è}
\put(95.8,-770.811){\fontsize{12}{1}\usefont{T1}{cmr}{m}{n}\selectfont\color{color_29791}maggiore o minore di 2? dipende se interpreto con criterio lessicografico o numerico}
\end{picture}
\newpage
\begin{tikzpicture}[overlay]\path(0pt,0pt);\end{tikzpicture}
\begin{picture}(-5,0)(2.5,0)
\put(77.8,-85.01099){\fontsize{12}{1}\usefont{T1}{cmr}{m}{n}\selectfont\color{color_29791}◦EQUALITY :es. per un tipo attributo che è una stringa, si può dire che il }
\put(95.8,-98.81097){\fontsize{12}{1}\usefont{T1}{cmr}{m}{n}\selectfont\color{color_29791}confronto tra stringhe di quel tipo deve essere fatto ignorando maiusc e minusc (come }
\put(95.8,-112.611){\fontsize{12}{1}\usefont{T1}{cmr}{m}{n}\selectfont\color{color_29791}per le email, case insensitive). Per attributo email in una entry, vedrò che attributeType }
\put(95.8,-126.411){\fontsize{12}{1}\usefont{T1}{cmr}{m}{n}\selectfont\color{color_29791}email è definito come sintassi stringa, e con EQUALITY caseInsensitiveMatch (che vuol}
\put(95.8,-140.211){\fontsize{12}{1}\usefont{T1}{cmr}{m}{n}\selectfont\color{color_29791}dire, quando fai il confronto con altro valore, ignora maiusc e minusc). Per confronto tra }
\put(95.8,-154.011){\fontsize{12}{1}\usefont{T1}{cmr}{m}{n}\selectfont\color{color_29791}password, sarebbe caseSensitiveMatch}
\put(77.8,-174.811){\fontsize{12}{1}\usefont{T1}{cmr}{m}{n}\selectfont\color{color_29791}◦SUBSTRING:potrei definire che il criterio di confronto è la presenza di una }
\put(95.8,-188.611){\fontsize{12}{1}\usefont{T1}{cmr}{m}{n}\selectfont\color{color_29791}sottostringa (es. in un'azienda con centralino e numero di interno, considero match }
\put(95.8,-202.411){\fontsize{12}{1}\usefont{T1}{cmr}{m}{n}\selectfont\color{color_29791}valido le prime 6 cifre del numero di telefono, perché vuol dire che sono su numeri }
\put(95.8,-216.211){\fontsize{12}{1}\usefont{T1}{cmr}{m}{n}\selectfont\color{color_29791}interni)}
\put(59.8,-237.011){\fontsize{12}{1}\usefont{T1}{cmr}{m}{n}\selectfont\color{color_29791}•Vincoli d'uso:SINGLE-VALUEammette un solo valore di questo tipo nella }
\put(77.8,-250.811){\fontsize{12}{1}\usefont{T1}{cmr}{m}{n}\selectfont\color{color_29791}entry (es. dn è single-value),ce ne sono altri come NO-USER-MODIFICATION, USAGE...}
\put(59.8,-271.611){\fontsize{12}{1}\usefont{T1}{cmr}{m}{n}\selectfont\color{color_29791}•Eventuali dipendenze gerarchiche:con SUP <altro attributeType> il tipo eredita tutte le }
\put(77.8,-285.411){\fontsize{12}{1}\usefont{T1}{cmr}{m}{n}\selectfont\color{color_29791}proprietà del superiore, può ridefinirne alcune, e nelle ricerche per un tipo superiore }
\put(77.8,-299.211){\fontsize{12}{1}\usefont{T1}{cmr}{m}{n}\selectfont\color{color_29791}vengono restituiti tutti i valori dei tipi basati su quello (es. cellulare ha come tipo superiore }
\put(77.8,-313.011){\fontsize{12}{1}\usefont{T1}{cmr}{m}{n}\selectfont\color{color_29791}l'attributeType telefono (che è una stringa, ha ordinamento, etc) Quando creo una entry }
\put(77.8,-326.811){\fontsize{12}{1}\usefont{T1}{cmr}{m}{n}\selectfont\color{color_29791}dell'utente marco, se ci metto dentro un telefono e un cellulare, se qualcuno cerca "telefono" }
\put(77.8,-340.611){\fontsize{12}{1}\usefont{T1}{cmr}{m}{n}\selectfont\color{color_29791}verrà fuori sia telefono che cellulare: questo permette di definire in modo ordinato, }
\put(77.8,-354.411){\fontsize{12}{1}\usefont{T1}{cmr}{m}{n}\selectfont\color{color_29791}separatamente, 1 telefono, 1 cellulare e 1 numero di casa; ma di ottenere comunque tutti e }
\put(77.8,-368.211){\fontsize{12}{1}\usefont{T1}{cmr}{m}{n}\selectfont\color{color_29791}tre i risultati se si richiede "telefono", quando non si sa quale si vuole specificamente).}
\put(41.8,-389.011){\fontsize{12}{1}\usefont{T1}{cmr}{m}{n}\selectfont\color{color_29791}Esistono molti attributeType già definiti, standard per gli usi più comuni (doc. }
\put(41.8,-402.811){\fontsize{12}{1}\usefont{T1}{cmr}{m}{n}\selectfont\color{color_29919}https://oav.net/mirrors/LDAP-ObjectClasses.html}
\end{picture}
\begin{tikzpicture}[overlay]
\path(0pt,0pt);
\draw[color_29919,line width=0.7pt]
(41.8pt, -403.911pt) -- (279.6pt, -403.911pt)
;
\end{tikzpicture}
\begin{picture}(-5,0)(2.5,0)
\put(279.7,-402.811){\fontsize{12}{1}\usefont{T1}{cmr}{m}{n}\selectfont\color{color_29791} sono a questo link, sezione Attributes). Tra i più }
\put(41.8,-416.611){\fontsize{12}{1}\usefont{T1}{cmr}{m}{n}\selectfont\color{color_29791}comuni abbiamo }
\put(59.8,-437.411){\fontsize{12}{1}\usefont{T1}{cmr}{m}{n}\selectfont\color{color_29791}•cn common name, }
\put(59.8,-458.211){\fontsize{12}{1}\usefont{T1}{cmr}{m}{n}\selectfont\color{color_29791}•dc domain component, }
\put(59.8,-479.011){\fontsize{12}{1}\usefont{T1}{cmr}{m}{n}\selectfont\color{color_29791}•o organization, }
\put(59.8,-499.811){\fontsize{12}{1}\usefont{T1}{cmr}{m}{n}\selectfont\color{color_29791}•ccountry}
\put(41.8,-520.611){\fontsize{12}{1}\usefont{T1}{cmr}{m}{n}\selectfont\color{color_29791}Per definire un nuovo attributeType, si inserisce la sua descrizione nella config. del server, sotto }
\put(41.8,-534.411){\fontsize{12}{1}\usefont{T1}{cmr}{m}{n}\selectfont\color{color_29791}forma di attributo di una entry speciale (quasi mai si ridefinisce un nuovo attributeType, essendo già}
\put(41.8,-548.211){\fontsize{12}{1}\usefont{T1}{cmr}{m}{n}\selectfont\color{color_29791}diffusi molti attributeType standard):}
\put(77.3,-569.011){\fontsize{12}{1}\usefont{T1}{cmr}{m}{it}\selectfont\color{color_29791}olcAttributeTypes: ( 1000.1.1.1 NAME ( 'fn' 'filename' )→ due nomi alias}
\put(77.3,-589.811){\fontsize{12}{1}\usefont{T1}{cmr}{m}{it}\selectfont\color{color_29791}DESC 'nome del file'}
\put(77.3,-610.611){\fontsize{12}{1}\usefont{T1}{cmr}{m}{it}\selectfont\color{color_29791}EQUALITY caseExactMatch}
\put(77.3,-631.411){\fontsize{12}{1}\usefont{T1}{cmr}{m}{it}\selectfont\color{color_29791}SUBSTR caseExactSubstringsMatch}
\put(77.3,-652.211){\fontsize{12}{1}\usefont{T1}{cmr}{m}{it}\selectfont\color{color_29791}SYNTAX 1.3.6.1.4.1.1466.115.121.1.15 )→ questo è come scrivere "string": per definire }
\put(41.8,-666.011){\fontsize{12}{1}\usefont{T1}{cmr}{m}{n}\selectfont\color{color_29791}sintassi, regole di matching, si usano gli OID}
\put(77.3,-686.811){\fontsize{12}{1}\usefont{T1}{cmr}{m}{n}\selectfont\color{color_29791}Se vogliamo creare attributi nuovi, dobbiamo usare come nome un OID univoco (es. 1000 }
\put(41.8,-700.611){\fontsize{12}{1}\usefont{T1}{cmr}{m}{n}\selectfont\color{color_29791}non esiste di sicuro, essendo la radice OID 0 o 1 o 2)}
\put(41.8,-730.411){\fontsize{14.1}{1}\usefont{T1}{cmr}{b}{n}\selectfont\color{color_29791}Classi}
\put(41.8,-750.611){\fontsize{12}{1}\usefont{T1}{cmr}{m}{n}\selectfont\color{color_29791}Lo scopo essenziale delle objectClass è di indicare in una certa entry quali attributi sono obbligatori}
\put(41.8,-764.411){\fontsize{12}{1}\usefont{T1}{cmr}{m}{n}\selectfont\color{color_29791}(MUST) e quali facoltativi (MAY): quando si va a compilare una entry, si è obbligati a riempire }
\end{picture}
\newpage
\begin{tikzpicture}[overlay]\path(0pt,0pt);\end{tikzpicture}
\begin{picture}(-5,0)(2.5,0)
\put(41.8,-85.01099){\fontsize{12}{1}\usefont{T1}{cmr}{m}{n}\selectfont\color{color_29791}attributi MUST, ed è vietato riempire attributi che non siano MUST o MAY. Le classi possono }
\put(41.8,-98.81097){\fontsize{12}{1}\usefont{T1}{cmr}{m}{n}\selectfont\color{color_29791}essere di tre tipi:}
\put(59.8,-119.611){\fontsize{12}{1}\usefont{T1}{cmr}{m}{n}\selectfont\color{color_29791}•ABSTRACT servono per creare tassonomia di oggetti, ma sono troppo astratte per originare }
\put(77.8,-133.411){\fontsize{12}{1}\usefont{T1}{cmr}{m}{n}\selectfont\color{color_29791}entry (non si possono istanziare)}
\put(77.8,-154.211){\fontsize{12}{1}\usefont{T1}{cmr}{m}{n}\selectfont\color{color_29791}◦es. top è classe predefinita, astratta, che contiene solo distinguished name e object class }
\put(95.8,-168.011){\fontsize{12}{1}\usefont{T1}{cmr}{m}{n}\selectfont\color{color_29791}(dato che tutte le entry dovranno contenerli). Poi classe astratta animale/vegetale }
\put(95.8,-181.811){\fontsize{12}{1}\usefont{T1}{cmr}{m}{n}\selectfont\color{color_29791}aggiunge solo numero di cromosomi (e ancora non è istanziabile).}
\put(59.8,-202.611){\fontsize{12}{1}\usefont{T1}{cmr}{m}{n}\selectfont\color{color_29791}•STRUCTURAL servono per descrivere categorie di oggetti concreti}
\put(77.8,-223.411){\fontsize{12}{1}\usefont{T1}{cmr}{m}{n}\selectfont\color{color_29791}◦es. aggiungendo nome, razza e peso si possono specificare classi cane e gatto }
\put(95.8,-237.211){\fontsize{12}{1}\usefont{T1}{cmr}{m}{n}\selectfont\color{color_29791}(strutturali, istanziabili, che ereditano gli attributi delle classi da cui discendono)}
\put(59.8,-258.011){\fontsize{12}{1}\usefont{T1}{cmr}{m}{n}\selectfont\color{color_29791}•AUXILIARY possono descrivere collezioni di attributi aggiuntivi che arricchiscono una }
\put(77.8,-271.811){\fontsize{12}{1}\usefont{T1}{cmr}{m}{n}\selectfont\color{color_29791}entry, senza essere collegati a categorie specifiche di oggetti}
\put(77.8,-292.611){\fontsize{12}{1}\usefont{T1}{cmr}{m}{n}\selectfont\color{color_29791}◦fascicoloSanitario può estendere alcune entry di cane: per definire un'entry di classe }
\put(95.8,-306.411){\fontsize{12}{1}\usefont{T1}{cmr}{m}{it}\selectfont\color{color_29791}cane dovrò/potrò specificare nome, razza,peso;poi potrò dire che una entry cane ha }
\put(95.8,-320.211){\fontsize{12}{1}\usefont{T1}{cmr}{m}{n}\selectfont\color{color_29791}ANCHE objectClass fascicoloSanitario, e questo mi permetterà/obbligherà (may/must) }
\put(95.8,-334.011){\fontsize{12}{1}\usefont{T1}{cmr}{m}{n}\selectfont\color{color_29791}di inserire anche i dati aggiuntivi.}
\put(77.8,-354.811){\fontsize{12}{1}\usefont{T1}{cmr}{m}{n}\selectfont\color{color_29791}◦Permette di specificare, ad esempio, una classe ausiliaria che prevede attributo colore; }
\put(95.8,-368.611){\fontsize{12}{1}\usefont{T1}{cmr}{m}{n}\selectfont\color{color_29791}senza bisogno di creare due classi separate AutoConColore/AutoSenzaColore.}
\put(41.8,-389.411){\fontsize{12}{1}\usefont{T1}{cmr}{m}{n}\selectfont\color{color_29791}Ogni entry dovrà}
\end{picture}
\begin{tikzpicture}[overlay]
\path(0pt,0pt);
\draw[color_29791,line width=0.7pt]
(96.5pt, -390.511pt) -- (126.5pt, -390.511pt)
;
\end{tikzpicture}
\begin{picture}(-5,0)(2.5,0)
\put(126.5,-389.411){\fontsize{12}{1}\usefont{T1}{cmr}{m}{n}\selectfont\color{color_29791} avere UNA SOLA classe strutturale, e potrà avere più classi ausiliarie (es. marco }
\put(41.8,-403.211){\fontsize{12}{1}\usefont{T1}{cmr}{m}{n}\selectfont\color{color_29791}prandini ha objectClass strutturale docente; potrò aggiungere classe ausiliaria rettore (con }
\put(41.8,-417.011){\fontsize{12}{1}\usefont{T1}{cmr}{m}{n}\selectfont\color{color_29791}informazioni aggiuntive, ognuna di esse in MUST o MAY. La classe ausiliaria dovrà comunque }
\put(41.8,-430.811){\fontsize{12}{1}\usefont{T1}{cmr}{m}{n}\selectfont\color{color_29791}essere definita, e non potrà essere usata da sola). }
\put(41.8,-451.611){\fontsize{12}{1}\usefont{T1}{cmr}{m}{n}\selectfont\color{color_29791}Le classi ausiliarie quindi permettono di creare un modello con una sorta di ereditarietà multipla: }
\put(59.8,-472.411){\fontsize{12}{1}\usefont{T1}{cmr}{m}{n}\selectfont\color{color_29791}•Se in Java potrei avere il problema di 2 classi che definiscono attributo A con tipi diversi, }
\put(77.8,-486.211){\fontsize{12}{1}\usefont{T1}{cmr}{m}{n}\selectfont\color{color_29791}(una come intero, una come stringa. Quando istanzio oggetto che è istanza }
\put(77.8,-500.011){\fontsize{12}{1}\usefont{T1}{cmr}{m}{n}\selectfont\color{color_29791}contemporaneamente delle due classi, A è intero o stringa?) in LDAP invece attributeType }
\put(77.8,-513.811){\fontsize{12}{1}\usefont{T1}{cmr}{m}{n}\selectfont\color{color_29791}"nome" di classi docente e rettore non è definito nella entry, ma ESTERNAMENTE, nello }
\put(77.8,-527.611){\fontsize{12}{1}\usefont{T1}{cmr}{m}{n}\selectfont\color{color_29791}schema: quando mi riferisco ad un certo tipo, e' per forza QUEL tipo. }
\put(59.8,-548.411){\fontsize{12}{1}\usefont{T1}{cmr}{m}{n}\selectfont\color{color_29791}•E' come dire che in JAVA, tutte le variabili che si chiamano (con il nome) A hanno lo stesso }
\put(77.8,-562.211){\fontsize{12}{1}\usefont{T1}{cmr}{m}{n}\selectfont\color{color_29791}tipo. Questo perché in LDAP le variabili non hanno nome, appunto, ma solo tipo. Ogni }
\put(77.8,-576.011){\fontsize{12}{1}\usefont{T1}{cmr}{m}{n}\selectfont\color{color_29791}objectClass specifica attributeType che devono essere definiti nello stesso schema, non ci }
\put(77.8,-589.811){\fontsize{12}{1}\usefont{T1}{cmr}{m}{n}\selectfont\color{color_29791}sono quindi conflitti se più objectClass in una entry menzionano lo stesso attributeType.}
\put(59.8,-610.611){\fontsize{12}{1}\usefont{T1}{cmr}{m}{n}\selectfont\color{color_29791}•L'effetto è che la entry che usa le diverse classi deve}
\end{picture}
\begin{tikzpicture}[overlay]
\path(0pt,0pt);
\draw[color_29791,line width=0.7pt]
(306.3pt, -611.711pt) -- (328.9pt, -611.711pt)
;
\end{tikzpicture}
\begin{picture}(-5,0)(2.5,0)
\put(329,-610.611){\fontsize{12}{1}\usefont{T1}{cmr}{m}{n}\selectfont\color{color_29791} contenere tutti gli attributi MUST di }
\put(77.8,-624.411){\fontsize{12}{1}\usefont{T1}{cmr}{m}{n}\selectfont\color{color_29791}tutte le sue classi, e può contenere tutti gli attributi MAY. In caso di vincoli diversi per lo }
\put(77.8,-638.211){\fontsize{12}{1}\usefont{T1}{cmr}{m}{n}\selectfont\color{color_29791}stesso attributeType nelle diverse classi, vince il più stringente (un attributo sia nell'insieme }
\put(77.8,-652.011){\fontsize{12}{1}\usefont{T1}{cmr}{m}{n}\selectfont\color{color_29791}MUST che MAY, deve}
\end{picture}
\begin{tikzpicture}[overlay]
\path(0pt,0pt);
\draw[color_29791,line width=0.7pt]
(165pt, -653.111pt) -- (187.6pt, -653.111pt)
;
\end{tikzpicture}
\begin{picture}(-5,0)(2.5,0)
\put(187.6,-652.011){\fontsize{12}{1}\usefont{T1}{cmr}{m}{n}\selectfont\color{color_29791} essere presente).}
\put(41.8,-672.811){\fontsize{12}{1}\usefont{T1}{cmr}{m}{n}\selectfont\color{color_29791}Le classi possono essere definite gerarchicamente usando SUP <altra objectClass> nella }
\put(41.8,-686.611){\fontsize{12}{1}\usefont{T1}{cmr}{m}{n}\selectfont\color{color_29791}definizione. La classe inferiore eredita tutti gli attributi MUST e MAY delle superiori e può solo }
\put(41.8,-700.411){\fontsize{12}{1}\usefont{T1}{cmr}{m}{n}\selectfont\color{color_29791}aggiungerne}
\end{picture}
\begin{tikzpicture}[overlay]
\path(0pt,0pt);
\draw[color_29791,line width=0.7pt]
(41.8pt, -701.511pt) -- (101.1pt, -701.511pt)
;
\end{tikzpicture}
\begin{picture}(-5,0)(2.5,0)
\put(101.1,-700.411){\fontsize{12}{1}\usefont{T1}{cmr}{m}{n}\selectfont\color{color_29791} (e non toglierne) altri.}
\put(41.8,-721.211){\fontsize{12}{1}\usefont{T1}{cmr}{m}{n}\selectfont\color{color_29791}Esistono molte objectClass già definite, standard per gli usi più comuni (doc. }
\put(41.8,-735.011){\fontsize{12}{1}\usefont{T1}{cmr}{m}{n}\selectfont\color{color_29919}https://oav.net/mirrors/LDAP-ObjectClasses.html}
\end{picture}
\begin{tikzpicture}[overlay]
\path(0pt,0pt);
\draw[color_29919,line width=0.7pt]
(41.8pt, -736.111pt) -- (279.6pt, -736.111pt)
;
\end{tikzpicture}
\begin{picture}(-5,0)(2.5,0)
\put(279.7,-735.011){\fontsize{12}{1}\usefont{T1}{cmr}{m}{n}\selectfont\color{color_29791} ). Tra le più comuni}
\put(59.8,-755.811){\fontsize{12}{1}\usefont{T1}{cmr}{m}{n}\selectfont\color{color_29791}•organization}
\end{picture}
\newpage
\begin{tikzpicture}[overlay]\path(0pt,0pt);\end{tikzpicture}
\begin{picture}(-5,0)(2.5,0)
\put(59.8,-85.01099){\fontsize{12}{1}\usefont{T1}{cmr}{m}{n}\selectfont\color{color_29791}•person}
\put(59.8,-105.811){\fontsize{12}{1}\usefont{T1}{cmr}{m}{n}\selectfont\color{color_29791}•device}
\put(41.8,-126.611){\fontsize{12}{1}\usefont{T1}{cmr}{m}{n}\selectfont\color{color_29791}Per definire una nuova objectClass , si inserisce la sua descrizione nella config. del server, sotto }
\put(41.8,-140.411){\fontsize{12}{1}\usefont{T1}{cmr}{m}{n}\selectfont\color{color_29791}forma di attributo di una entry speciale:}
\put(77.3,-161.211){\fontsize{12}{1}\usefont{T1}{cmr}{m}{it}\selectfont\color{color_29791}olcObjectClasses: ( 1000.2.1.1 NAME ( 'dir' ) → per definire nomi univoci, si usano gli OID}
\put(77.3,-182.011){\fontsize{12}{1}\usefont{T1}{cmr}{m}{it}\selectfont\color{color_29791}DESC 'una directory'}
\put(77.3,-202.811){\fontsize{12}{1}\usefont{T1}{cmr}{m}{it}\selectfont\color{color_29791}MUST fn}
\put(77.3,-223.611){\fontsize{12}{1}\usefont{T1}{cmr}{m}{it}\selectfont\color{color_29791}MAY fs}
\put(77.3,-244.411){\fontsize{12}{1}\usefont{T1}{cmr}{m}{it}\selectfont\color{color_29791}AUXILIARY )}
\put(41.8,-265.211){\fontsize{12}{1}\usefont{T1}{cmr}{m}{n}\selectfont\color{color_35081}In LDAP, tutte le direttive di configurazione sono attributi in entry di un albero separato da quello }
\put(41.8,-279.011){\fontsize{12}{1}\usefont{T1}{cmr}{m}{n}\selectfont\color{color_35081}dei dati veri e propri: }
\put(41.8,-362.211){\fontsize{12}{1}\usefont{T1}{cmr}{m}{n}\selectfont\color{color_35081}Per riconfigurare un server LDAP si inserisce quindi una entry in formato LDIF, es. per un nuovo }
\put(41.8,-376.011){\fontsize{12}{1}\usefont{T1}{cmr}{m}{n}\selectfont\color{color_35081}schema si può creare il file filesystem.ldif :}
\put(41.8,-394.011){\fontsize{9}{1}\usefont{T1}{cmr}{m}{it}\selectfont\color{color_35081}dn: cn=filesystem,cn=schema,cn=config}
\put(41.8,-411.311){\fontsize{9}{1}\usefont{T1}{cmr}{m}{it}\selectfont\color{color_35081}objectClass: olcSchemaConfig}
\put(41.8,-428.711){\fontsize{9}{1}\usefont{T1}{cmr}{m}{it}\selectfont\color{color_35081}cn: filesystem}
\put(41.8,-446.011){\fontsize{9}{1}\usefont{T1}{cmr}{m}{it}\selectfont\color{color_35081}olcAttributeTypes: ( 1000.1.1.1 NAME ( 'fn' 'filename' )}
\put(41.8,-463.411){\fontsize{9}{1}\usefont{T1}{cmr}{m}{it}\selectfont\color{color_35081}  DESC 'nome del file'}
\put(41.8,-480.711){\fontsize{9}{1}\usefont{T1}{cmr}{m}{it}\selectfont\color{color_35081}  EQUALITY caseExactMatch}
\put(41.8,-498.111){\fontsize{9}{1}\usefont{T1}{cmr}{m}{it}\selectfont\color{color_35081} SUBSTR caseExactSubstringsMatch}
\put(41.8,-515.411){\fontsize{9}{1}\usefont{T1}{cmr}{m}{it}\selectfont\color{color_35081} SYNTAX 1.3.6.1.4.1.1466.115.121.1.15 )}
\put(41.8,-532.811){\fontsize{9}{1}\usefont{T1}{cmr}{m}{it}\selectfont\color{color_35081}olcAttributeTypes: ( 1000.1.1.2 NAME ( 'fs' 'filesize' )}
\put(41.8,-550.111){\fontsize{9}{1}\usefont{T1}{cmr}{m}{it}\selectfont\color{color_35081}  DESC 'dimensioni del file'}
\put(41.8,-567.511){\fontsize{9}{1}\usefont{T1}{cmr}{m}{it}\selectfont\color{color_35081}  EQUALITY integerMatch}
\put(41.8,-584.811){\fontsize{9}{1}\usefont{T1}{cmr}{m}{it}\selectfont\color{color_35081}  ORDERING integerOrderingMatch}
\put(41.8,-602.211){\fontsize{9}{1}\usefont{T1}{cmr}{m}{it}\selectfont\color{color_35081}  SYNTAX 1.3.6.1.4.1.1466.115.121.1.27 )}
\put(41.8,-619.511){\fontsize{9}{1}\usefont{T1}{cmr}{m}{it}\selectfont\color{color_35081}olcObjectClasses: ( 1000.2.1.1 NAME 'dir'}
\put(41.8,-636.911){\fontsize{9}{1}\usefont{T1}{cmr}{m}{it}\selectfont\color{color_35081}  DESC 'una directory'}
\put(41.8,-654.211){\fontsize{9}{1}\usefont{T1}{cmr}{m}{it}\selectfont\color{color_35081}  MUST fn}
\put(41.8,-671.611){\fontsize{9}{1}\usefont{T1}{cmr}{m}{it}\selectfont\color{color_35081}  MAY fs}
\put(41.8,-688.911){\fontsize{9}{1}\usefont{T1}{cmr}{m}{it}\selectfont\color{color_35081}  AUXILIARY )}
\put(41.8,-706.311){\fontsize{9}{1}\usefont{T1}{cmr}{m}{it}\selectfont\color{color_35081}olcObjectClasses: ( 1000.2.1.2 NAME 'file'}
\put(41.8,-723.611){\fontsize{9}{1}\usefont{T1}{cmr}{m}{it}\selectfont\color{color_35081}  DESC 'un file'}
\put(41.8,-741.011){\fontsize{9}{1}\usefont{T1}{cmr}{m}{it}\selectfont\color{color_35081}  MUST ( fn \$ fs )}
\put(41.8,-758.311){\fontsize{9}{1}\usefont{T1}{cmr}{m}{it}\selectfont\color{color_35081}  AUXILIARY )}
\end{picture}
\newpage
\begin{tikzpicture}[overlay]\path(0pt,0pt);\end{tikzpicture}
\begin{picture}(-5,0)(2.5,0)
\put(41.8,-85.01099){\fontsize{12}{1}\usefont{T1}{cmr}{m}{n}\selectfont\color{color_35081}Solitamente si usa un comando apposito per scavalcare autenticazione LDAP e agire da }
\put(41.8,-98.81097){\fontsize{12}{1}\usefont{T1}{cmr}{m}{n}\selectfont\color{color_35081}amministratori locali (admin del servizio):ldapadd -Y EXTERNAL -H ldapi:/// -f filesystem.ldif}
\put(41.8,-128.611){\fontsize{14.1}{1}\usefont{T1}{cmr}{b}{n}\selectfont\color{color_29791}Protocollo e operazioni offerte}
\put(41.8,-148.811){\fontsize{12}{1}\usefont{T1}{cmr}{m}{n}\selectfont\color{color_29791}Il protocollo LDAP è a livello applicativo e si basa su TCP a liv. Trasporto (porta 389 / 636 over }
\put(41.8,-162.611){\fontsize{12}{1}\usefont{T1}{cmr}{m}{n}\selectfont\color{color_29791}TLS), si basa su una sessione (viene autenticato l'utente che vuole effettuare la connessione). Alla }
\put(41.8,-176.411){\fontsize{12}{1}\usefont{T1}{cmr}{m}{it}\selectfont\color{color_29791}bind (che riceve risposta positiva o negativa in base al successo dell'autenticazione); segue una serie}
\put(41.8,-190.211){\fontsize{12}{1}\usefont{T1}{cmr}{m}{n}\selectfont\color{color_29791}di richieste di operazione, tutte effettuate a nome dell'utente autenticato, ognuna di queste può }
\put(41.8,-204.011){\fontsize{12}{1}\usefont{T1}{cmr}{m}{n}\selectfont\color{color_29791}essere permessa o meno in base al sistema di controllo degli accessi (in LDAP per ogni entry, per }
\put(41.8,-217.811){\fontsize{12}{1}\usefont{T1}{cmr}{m}{n}\selectfont\color{color_29791}ogni attributo… si possono specificare permessi per ogni utente). Alla fine delle operazioni c'è }
\put(41.8,-231.611){\fontsize{12}{1}\usefont{T1}{cmr}{m}{n}\selectfont\color{color_29791}l'unbind.}
\put(41.8,-252.411){\fontsize{12}{1}\usefont{T1}{cmr}{m}{n}\selectfont\color{color_29791}Quindi per ogni operazione sarà necessario specificare un bind DN : equivale all'utente con cui }
\put(41.8,-266.211){\fontsize{12}{1}\usefont{T1}{cmr}{m}{n}\selectfont\color{color_29791}autenticarsi sul server LDAP, la bind dirà alla directory chi è l'utente e la directory deciderà se }
\put(41.8,-280.011){\fontsize{12}{1}\usefont{T1}{cmr}{m}{n}\selectfont\color{color_29791}l'utente ha diritto o meno a fare l'operazione.}
\put(41.8,-300.811){\fontsize{12}{1}\usefont{T1}{cmr}{m}{n}\selectfont\color{color_29791}Le operazioni offerte sono:}
\put(59.8,-321.611){\fontsize{12}{1}\usefont{T1}{cmr}{m}{n}\selectfont\color{color_29791}•Search}
\put(77.8,-342.411){\fontsize{12}{1}\usefont{T1}{cmr}{m}{n}\selectfont\color{color_29791}◦La ricerca deve specificare: come tutte le operazioni un bind DN, poi}
\put(95.8,-363.211){\fontsize{12}{1}\usefont{T1}{cmr}{m}{n}\selectfont\color{color_29791}▪Un base DN :il punto del DIT da cui iniziare la ricerca (radice da cui }
\put(113.8,-377.011){\fontsize{12}{1}\usefont{T1}{cmr}{m}{n}\selectfont\color{color_29791}cercare)}
\put(95.8,-397.811){\fontsize{12}{1}\usefont{T1}{cmr}{m}{n}\selectfont\color{color_29791}▪Uno scope :a partire da base DN, indica quanto estendere la ricerca, con }
\put(113.8,-411.611){\fontsize{12}{1}\usefont{T1}{cmr}{m}{n}\selectfont\color{color_29791}tre valori possibili}
\put(113.8,-432.411){\fontsize{12}{1}\usefont{T1}{cmr}{m}{n}\selectfont\color{color_29791}•subdefault, indica l'intero sottoalbero}
\put(113.8,-453.211){\fontsize{12}{1}\usefont{T1}{cmr}{m}{n}\selectfont\color{color_29791}•onesolo figli diretti del base DN ( un solo livello in basso)}
\put(113.8,-474.011){\fontsize{12}{1}\usefont{T1}{cmr}{m}{n}\selectfont\color{color_29791}•basesolo il nodo base}
\put(95.8,-494.811){\fontsize{12}{1}\usefont{T1}{cmr}{m}{n}\selectfont\color{color_29791}▪Eventualmente un filtro (base e scope obbligatori, filtro è facoltativo):per }
\put(113.8,-508.611){\fontsize{12}{1}\usefont{T1}{cmr}{m}{n}\selectfont\color{color_29791}ricercare in base a contenuto della entry invece che per posizione, usando espressioni}
\put(113.8,-522.411){\fontsize{12}{1}\usefont{T1}{cmr}{m}{n}\selectfont\color{color_29791}logiche in notazione prefissa (operatore precede operandi}
\end{picture}
\begin{tikzpicture}[overlay]
\path(0pt,0pt);
\draw[color_29791,line width=0.7pt]
(258.4pt, -523.511pt) -- (389pt, -523.511pt)
;
\end{tikzpicture}
\begin{picture}(-5,0)(2.5,0)
\put(389,-522.411){\fontsize{12}{1}\usefont{T1}{cmr}{m}{n}\selectfont\color{color_29791})}
\put(77.8,-543.211){\fontsize{12}{1}\usefont{T1}{cmr}{m}{n}\selectfont\color{color_29791}◦Sintassi filtri ed esempi:(operatore(operand1)(operand2))eventualmente guarda }
\put(95.8,-557.011){\fontsize{12}{1}\usefont{T1}{cmr}{m}{n}\selectfont\color{color_29791}slide}
\put(95.8,-575.011){\fontsize{9}{1}\usefont{T1}{cmr}{m}{n}\selectfont\color{color_29791}▪<filter>::='('<filtercomp>')'}
\put(113.8,-592.311){\fontsize{9}{1}\usefont{T1}{cmr}{m}{n}\selectfont\color{color_29791}•<filtercomp>::=<and>|<or>|<not>|<item>}
\put(131.8,-609.711){\fontsize{9}{1}\usefont{T1}{cmr}{m}{n}\selectfont\color{color_29791}◦<and>::='\&'<filterlist>}
\put(131.8,-627.011){\fontsize{9}{1}\usefont{T1}{cmr}{m}{n}\selectfont\color{color_29791}◦<or>::='|'<filterlist>}
\put(131.8,-644.411){\fontsize{9}{1}\usefont{T1}{cmr}{m}{n}\selectfont\color{color_29791}◦<not>::='!'<filter>}
\put(149.8,-661.711){\fontsize{9}{1}\usefont{T1}{cmr}{m}{n}\selectfont\color{color_29791}▪<filterlist>::=<filter>|<filter><filterlist>}
\put(131.8,-679.111){\fontsize{9}{1}\usefont{T1}{cmr}{m}{n}\selectfont\color{color_29791}◦<item>::=<simple>|<present>|<substring>}
\put(149.8,-696.411){\fontsize{9}{1}\usefont{T1}{cmr}{m}{n}\selectfont\color{color_29791}▪<simple>::=<attr><filtertype><value>}
\put(149.8,-713.811){\fontsize{9}{1}\usefont{T1}{cmr}{m}{n}\selectfont\color{color_29791}▪<filtertype>::=<equal>|<approx>|<greater>|<less>}
\put(167.8,-731.111){\fontsize{9}{1}\usefont{T1}{cmr}{m}{n}\selectfont\color{color_29791}•<equal>::='='}
\put(167.8,-748.511){\fontsize{9}{1}\usefont{T1}{cmr}{m}{n}\selectfont\color{color_29791}•<approx>::='~='}
\put(167.8,-765.811){\fontsize{9}{1}\usefont{T1}{cmr}{m}{n}\selectfont\color{color_29791}•<greater>::='>='}
\end{picture}
\newpage
\begin{tikzpicture}[overlay]\path(0pt,0pt);\end{tikzpicture}
\begin{picture}(-5,0)(2.5,0)
\put(167.8,-82.211){\fontsize{9}{1}\usefont{T1}{cmr}{m}{n}\selectfont\color{color_29791}•<less>::='<='}
\put(149.8,-99.51099){\fontsize{9}{1}\usefont{T1}{cmr}{m}{n}\selectfont\color{color_29791}▪<present>::=<attr>'=*'}
\put(149.8,-116.911){\fontsize{9}{1}\usefont{T1}{cmr}{m}{n}\selectfont\color{color_29791}▪<substring>::=<attr>'='<initial><any><final>}
\put(167.8,-134.211){\fontsize{9}{1}\usefont{T1}{cmr}{m}{n}\selectfont\color{color_29791}•<initial>::=NULL|<value>}
\put(167.8,-151.611){\fontsize{9}{1}\usefont{T1}{cmr}{m}{n}\selectfont\color{color_29791}•<any>::='*'<starval>}
\put(185.8,-168.911){\fontsize{9}{1}\usefont{T1}{cmr}{m}{n}\selectfont\color{color_29791}◦<starval>::=NULL|<value>'*'<starval>}
\put(167.8,-186.311){\fontsize{9}{1}\usefont{T1}{cmr}{m}{n}\selectfont\color{color_29791}•<final>::=NULL|<value>}
\put(95.8,-206.411){\fontsize{12}{1}\usefont{T1}{cmr}{m}{n}\selectfont\color{color_29791}▪(cn=Babs Jensen)tutte le entry nello scope, che hanno come attributo cn la string}
\put(113.8,-220.211){\fontsize{12}{1}\usefont{T1}{cmr}{m}{n}\selectfont\color{color_29791}"Babs Jensen"}
\put(95.8,-241.011){\fontsize{12}{1}\usefont{T1}{cmr}{m}{n}\selectfont\color{color_29791}▪(!(cn=Tim Howes))operatore "diverso da"}
\put(95.8,-261.811){\fontsize{12}{1}\usefont{T1}{cmr}{m}{n}\selectfont\color{color_29791}▪(o= univ*of*mich*)* wildcard, qualsiasi stringa}
\put(95.8,-282.611){\fontsize{12}{1}\usefont{T1}{cmr}{m}{n}\selectfont\color{color_29791}▪(\&(objectClass=Person)(|(sn=Jensen)(cn=Babs J*)))due operandi in and }
\put(113.8,-296.411){\fontsize{12}{1}\usefont{T1}{cmr}{m}{n}\selectfont\color{color_29791}(operatore \& ), ognuno tra parentesi: il primo è valore objectClass uguale a person; il}
\put(113.8,-310.211){\fontsize{12}{1}\usefont{T1}{cmr}{m}{n}\selectfont\color{color_29791}secondo è nuovamente una composizione, ma in or (operatore |), il primo sn uguale a}
\put(113.8,-324.011){\fontsize{12}{1}\usefont{T1}{cmr}{m}{n}\selectfont\color{color_29791}jensen, il secondo cn uguale a babs j* , dove * è qualsiasi stringa}
\put(113.8,-344.811){\fontsize{12}{1}\usefont{T1}{cmr}{m}{n}\selectfont\color{color_29791}•Abbiamo quindi possibilità di comporre più espressioni, ponendole come }
\put(131.8,-358.611){\fontsize{12}{1}\usefont{T1}{cmr}{m}{n}\selectfont\color{color_29791}operandi di altre}
\put(95.8,-379.411){\fontsize{12}{1}\usefont{T1}{cmr}{m}{n}\selectfont\color{color_29791}▪(\&(|(uid=jack)(uid=jill))(objectclass=posix Account)risultato finale è unica }
\put(113.8,-393.211){\fontsize{12}{1}\usefont{T1}{cmr}{m}{n}\selectfont\color{color_29791}riga con la notazione prefissa}
\put(77.8,-414.011){\fontsize{12}{1}\usefont{T1}{cmr}{m}{n}\selectfont\color{color_29791}◦comando per ricerca:ldapsearch -x -b dc=labammsis [ -s base | one | sub ] }
\put(95.8,-427.811){\fontsize{12}{1}\usefont{T1}{cmr}{b}{n}\selectfont\color{color_29791}[filtro]}
\put(95.8,-448.611){\fontsize{12}{1}\usefont{T1}{cmr}{m}{n}\selectfont\color{color_29791}▪-s indica lo scope}
\put(95.8,-469.411){\fontsize{12}{1}\usefont{T1}{cmr}{m}{n}\selectfont\color{color_29791}▪-b indica base DN, punto di partenza della ricerca}
\put(95.8,-490.211){\fontsize{12}{1}\usefont{T1}{cmr}{m}{n}\selectfont\color{color_29791}▪opzione -h indica server a cui connettersi}
\put(113.8,-511.011){\fontsize{12}{1}\usefont{T1}{cmr}{m}{n}\selectfont\color{color_29791}•ldapsearch -x -h 192.168.56.203 -b "dc=labammsis" -s sub 'cn=marco'}
\put(95.8,-531.811){\fontsize{12}{1}\usefont{T1}{cmr}{m}{n}\selectfont\color{color_29791}▪-x attiva autenticazione base}
\put(59.8,-552.611){\fontsize{12}{1}\usefont{T1}{cmr}{m}{n}\selectfont\color{color_29791}•Compare}
\put(59.8,-573.411){\fontsize{12}{1}\usefont{T1}{cmr}{m}{n}\selectfont\color{color_29791}•Add}
\put(77.8,-594.211){\fontsize{12}{1}\usefont{T1}{cmr}{m}{n}\selectfont\color{color_29791}◦comando per aggiunta:ldapadd -x -D "cn=admin,dc=labammsis" -w }
\put(95.8,-608.011){\fontsize{12}{1}\usefont{T1}{cmr}{b}{n}\selectfont\color{color_29791}admin [ -f file\_ldif\_da\_inserire ]}
\put(77.8,-628.811){\fontsize{12}{1}\usefont{T1}{cmr}{m}{n}\selectfont\color{color_29791}◦-f specifica il file, se omesso usa stdin}
\put(77.8,-649.611){\fontsize{12}{1}\usefont{T1}{cmr}{m}{n}\selectfont\color{color_29791}◦-D attiva query autenticate, fornendo credenziali amministratore di LDAP:}
\put(95.8,-670.411){\fontsize{12}{1}\usefont{T1}{cmr}{m}{n}\selectfont\color{color_29791}▪-D <utente> ("cn=admin,dc=labammsis")}
\put(95.8,-691.211){\fontsize{12}{1}\usefont{T1}{cmr}{m}{n}\selectfont\color{color_29791}▪-w <password> ("admin")}
\put(95.8,-712.011){\fontsize{12}{1}\usefont{T1}{cmr}{m}{n}\selectfont\color{color_217499}▪Si deve sempre specificare almeno una classe strutturale nell'LDIF !!!}
\put(59.8,-732.811){\fontsize{12}{1}\usefont{T1}{cmr}{m}{n}\selectfont\color{color_29791}•Modify}
\put(77.8,-753.611){\fontsize{12}{1}\usefont{T1}{cmr}{m}{n}\selectfont\color{color_29791}◦comando per modificaldapmodify}
\end{picture}
\newpage
\begin{tikzpicture}[overlay]\path(0pt,0pt);\end{tikzpicture}
\begin{picture}(-5,0)(2.5,0)
\put(77.8,-85.01099){\fontsize{12}{1}\usefont{T1}{cmr}{m}{n}\selectfont\color{color_29791}◦Stessi parametri di ldapadd, ha più casi: si può usare per aggiunta/modifica/rimozione }
\put(95.8,-98.81097){\fontsize{12}{1}\usefont{T1}{cmr}{m}{n}\selectfont\color{color_29791}attributo}
\put(77.8,-119.611){\fontsize{12}{1}\usefont{T1}{cmr}{m}{n}\selectfont\color{color_217499}◦si usa un LDIF con attributo changetype: modifyseguito da add | replace | delete che }
\put(95.8,-133.411){\fontsize{12}{1}\usefont{T1}{cmr}{m}{n}\selectfont\color{color_217499}specificano su quale attributo agire, e poi }
\put(77.8,-154.211){\fontsize{12}{1}\usefont{T1}{cmr}{m}{n}\selectfont\color{color_217499}◦Esempi nel prontuario sul sito del corso...}
\put(59.8,-175.011){\fontsize{12}{1}\usefont{T1}{cmr}{m}{n}\selectfont\color{color_29791}•Delete}
\put(77.8,-195.811){\fontsize{12}{1}\usefont{T1}{cmr}{m}{n}\selectfont\color{color_29791}◦comando per rimozioneldapdelete -x -D "cn=admin,dc=labammsis" -w }
\put(95.8,-209.611){\fontsize{12}{1}\usefont{T1}{cmr}{b}{n}\selectfont\color{color_29791}admin DN}
\put(95.8,-230.411){\fontsize{12}{1}\usefont{T1}{cmr}{m}{n}\selectfont\color{color_29791}▪DN è distinguished name della entry da eliminare}
\put(59.8,-251.211){\fontsize{12}{1}\usefont{T1}{cmr}{m}{n}\selectfont\color{color_29791}•ModifyRDN}
\put(59.8,-272.011){\fontsize{12}{1}\usefont{T1}{cmr}{m}{n}\selectfont\color{color_217499}•Configurazione server: }
\put(77.8,-292.811){\fontsize{12}{1}\usefont{T1}{cmr}{m}{n}\selectfont\color{color_217499}◦ldapsearch -Y EXTERNAL -H ldapi:/// -b "cn=config"visualizza su localhost}
\end{picture}
\begin{tikzpicture}[overlay]
\path(0pt,0pt);
\draw[color_217499,line width=0.7pt]
(465.8pt, -293.9109pt) -- (522.8pt, -293.9109pt)
;
\end{tikzpicture}
\begin{picture}(-5,0)(2.5,0)
\put(95.8,-306.611){\fontsize{12}{1}\usefont{T1}{cmr}{m}{n}\selectfont\color{color_217499}(con meccanismo di autenticazione "EXTERNAL", ovvero sui nostri sistemi, fidandosi }
\put(95.8,-320.411){\fontsize{12}{1}\usefont{T1}{cmr}{m}{n}\selectfont\color{color_217499}dell'utente UNIX → root è autorizzato a configurare)}
\put(77.8,-341.211){\fontsize{12}{1}\usefont{T1}{cmr}{m}{n}\selectfont\color{color_217499}◦Modifica del DIT della configurazione si fa comunque con ldapadd, ldapmodify, }
\put(95.8,-355.011){\fontsize{12}{1}\usefont{T1}{cmr}{b}{it}\selectfont\color{color_217499}ldapdelete (sempre con autenticazione speciale).}
\put(77.8,-375.811){\fontsize{12}{1}\usefont{T1}{cmr}{m}{n}\selectfont\color{color_217499}◦Un'operazione che può essere necessaria è la definizione di nuove classi di oggetti e }
\put(95.8,-389.611){\fontsize{12}{1}\usefont{T1}{cmr}{m}{n}\selectfont\color{color_217499}nuovi tipi di attributo, se quelli standard non sono adatti alla modellazione del nostro }
\put(95.8,-403.411){\fontsize{12}{1}\usefont{T1}{cmr}{m}{n}\selectfont\color{color_217499}DIT. Vedi guida sul sito del corso}
\end{picture}
\end{document}